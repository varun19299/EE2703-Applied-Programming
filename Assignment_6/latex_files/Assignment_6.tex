
% Default to the notebook output style

    


% Inherit from the specified cell style.




    
\documentclass[11pt]{article}

    
    
    \usepackage[T1]{fontenc}
    % Nicer default font (+ math font) than Computer Modern for most use cases
    \usepackage{mathpazo}

    % Basic figure setup, for now with no caption control since it's done
    % automatically by Pandoc (which extracts ![](path) syntax from Markdown).
    \usepackage{graphicx}
    % We will generate all images so they have a width \maxwidth. This means
    % that they will get their normal width if they fit onto the page, but
    % are scaled down if they would overflow the margins.
    \makeatletter
    \def\maxwidth{\ifdim\Gin@nat@width>\linewidth\linewidth
    \else\Gin@nat@width\fi}
    \makeatother
    \let\Oldincludegraphics\includegraphics
    % Set max figure width to be 80% of text width, for now hardcoded.
    \renewcommand{\includegraphics}[1]{\Oldincludegraphics[width=.8\maxwidth]{#1}}
    % Ensure that by default, figures have no caption (until we provide a
    % proper Figure object with a Caption API and a way to capture that
    % in the conversion process - todo).
    \usepackage{caption}
    \DeclareCaptionLabelFormat{nolabel}{}
    \captionsetup{labelformat=nolabel}

    \usepackage{adjustbox} % Used to constrain images to a maximum size 
    \usepackage{xcolor} % Allow colors to be defined
    \usepackage{enumerate} % Needed for markdown enumerations to work
    \usepackage{geometry} % Used to adjust the document margins
    \usepackage{amsmath} % Equations
    \usepackage{amssymb} % Equations
    \usepackage{textcomp} % defines textquotesingle
    % Hack from http://tex.stackexchange.com/a/47451/13684:
    \AtBeginDocument{%
        \def\PYZsq{\textquotesingle}% Upright quotes in Pygmentized code
    }
    \usepackage{upquote} % Upright quotes for verbatim code
    \usepackage{eurosym} % defines \euro
    \usepackage[mathletters]{ucs} % Extended unicode (utf-8) support
    \usepackage[utf8x]{inputenc} % Allow utf-8 characters in the tex document
    \usepackage{fancyvrb} % verbatim replacement that allows latex
    \usepackage{grffile} % extends the file name processing of package graphics 
                         % to support a larger range 
    % The hyperref package gives us a pdf with properly built
    % internal navigation ('pdf bookmarks' for the table of contents,
    % internal cross-reference links, web links for URLs, etc.)
    \usepackage{hyperref}
    \usepackage{longtable} % longtable support required by pandoc >1.10
    \usepackage{booktabs}  % table support for pandoc > 1.12.2
    \usepackage[inline]{enumitem} % IRkernel/repr support (it uses the enumerate* environment)
    \usepackage[normalem]{ulem} % ulem is needed to support strikethroughs (\sout)
                                % normalem makes italics be italics, not underlines
    

    
    
    % Colors for the hyperref package
    \definecolor{urlcolor}{rgb}{0,.145,.698}
    \definecolor{linkcolor}{rgb}{.71,0.21,0.01}
    \definecolor{citecolor}{rgb}{.12,.54,.11}

    % ANSI colors
    \definecolor{ansi-black}{HTML}{3E424D}
    \definecolor{ansi-black-intense}{HTML}{282C36}
    \definecolor{ansi-red}{HTML}{E75C58}
    \definecolor{ansi-red-intense}{HTML}{B22B31}
    \definecolor{ansi-green}{HTML}{00A250}
    \definecolor{ansi-green-intense}{HTML}{007427}
    \definecolor{ansi-yellow}{HTML}{DDB62B}
    \definecolor{ansi-yellow-intense}{HTML}{B27D12}
    \definecolor{ansi-blue}{HTML}{208FFB}
    \definecolor{ansi-blue-intense}{HTML}{0065CA}
    \definecolor{ansi-magenta}{HTML}{D160C4}
    \definecolor{ansi-magenta-intense}{HTML}{A03196}
    \definecolor{ansi-cyan}{HTML}{60C6C8}
    \definecolor{ansi-cyan-intense}{HTML}{258F8F}
    \definecolor{ansi-white}{HTML}{C5C1B4}
    \definecolor{ansi-white-intense}{HTML}{A1A6B2}

    % commands and environments needed by pandoc snippets
    % extracted from the output of `pandoc -s`
    \providecommand{\tightlist}{%
      \setlength{\itemsep}{0pt}\setlength{\parskip}{0pt}}
    \DefineVerbatimEnvironment{Highlighting}{Verbatim}{commandchars=\\\{\}}
    % Add ',fontsize=\small' for more characters per line
    \newenvironment{Shaded}{}{}
    \newcommand{\KeywordTok}[1]{\textcolor[rgb]{0.00,0.44,0.13}{\textbf{{#1}}}}
    \newcommand{\DataTypeTok}[1]{\textcolor[rgb]{0.56,0.13,0.00}{{#1}}}
    \newcommand{\DecValTok}[1]{\textcolor[rgb]{0.25,0.63,0.44}{{#1}}}
    \newcommand{\BaseNTok}[1]{\textcolor[rgb]{0.25,0.63,0.44}{{#1}}}
    \newcommand{\FloatTok}[1]{\textcolor[rgb]{0.25,0.63,0.44}{{#1}}}
    \newcommand{\CharTok}[1]{\textcolor[rgb]{0.25,0.44,0.63}{{#1}}}
    \newcommand{\StringTok}[1]{\textcolor[rgb]{0.25,0.44,0.63}{{#1}}}
    \newcommand{\CommentTok}[1]{\textcolor[rgb]{0.38,0.63,0.69}{\textit{{#1}}}}
    \newcommand{\OtherTok}[1]{\textcolor[rgb]{0.00,0.44,0.13}{{#1}}}
    \newcommand{\AlertTok}[1]{\textcolor[rgb]{1.00,0.00,0.00}{\textbf{{#1}}}}
    \newcommand{\FunctionTok}[1]{\textcolor[rgb]{0.02,0.16,0.49}{{#1}}}
    \newcommand{\RegionMarkerTok}[1]{{#1}}
    \newcommand{\ErrorTok}[1]{\textcolor[rgb]{1.00,0.00,0.00}{\textbf{{#1}}}}
    \newcommand{\NormalTok}[1]{{#1}}
    
    % Additional commands for more recent versions of Pandoc
    \newcommand{\ConstantTok}[1]{\textcolor[rgb]{0.53,0.00,0.00}{{#1}}}
    \newcommand{\SpecialCharTok}[1]{\textcolor[rgb]{0.25,0.44,0.63}{{#1}}}
    \newcommand{\VerbatimStringTok}[1]{\textcolor[rgb]{0.25,0.44,0.63}{{#1}}}
    \newcommand{\SpecialStringTok}[1]{\textcolor[rgb]{0.73,0.40,0.53}{{#1}}}
    \newcommand{\ImportTok}[1]{{#1}}
    \newcommand{\DocumentationTok}[1]{\textcolor[rgb]{0.73,0.13,0.13}{\textit{{#1}}}}
    \newcommand{\AnnotationTok}[1]{\textcolor[rgb]{0.38,0.63,0.69}{\textbf{\textit{{#1}}}}}
    \newcommand{\CommentVarTok}[1]{\textcolor[rgb]{0.38,0.63,0.69}{\textbf{\textit{{#1}}}}}
    \newcommand{\VariableTok}[1]{\textcolor[rgb]{0.10,0.09,0.49}{{#1}}}
    \newcommand{\ControlFlowTok}[1]{\textcolor[rgb]{0.00,0.44,0.13}{\textbf{{#1}}}}
    \newcommand{\OperatorTok}[1]{\textcolor[rgb]{0.40,0.40,0.40}{{#1}}}
    \newcommand{\BuiltInTok}[1]{{#1}}
    \newcommand{\ExtensionTok}[1]{{#1}}
    \newcommand{\PreprocessorTok}[1]{\textcolor[rgb]{0.74,0.48,0.00}{{#1}}}
    \newcommand{\AttributeTok}[1]{\textcolor[rgb]{0.49,0.56,0.16}{{#1}}}
    \newcommand{\InformationTok}[1]{\textcolor[rgb]{0.38,0.63,0.69}{\textbf{\textit{{#1}}}}}
    \newcommand{\WarningTok}[1]{\textcolor[rgb]{0.38,0.63,0.69}{\textbf{\textit{{#1}}}}}
    
    
    % Define a nice break command that doesn't care if a line doesn't already
    % exist.
    \def\br{\hspace*{\fill} \\* }
    % Math Jax compatability definitions
    \def\gt{>}
    \def\lt{<}
    % Document parameters
    \title{Assignment\_6}
    
    
    

    % Pygments definitions
    
\makeatletter
\def\PY@reset{\let\PY@it=\relax \let\PY@bf=\relax%
    \let\PY@ul=\relax \let\PY@tc=\relax%
    \let\PY@bc=\relax \let\PY@ff=\relax}
\def\PY@tok#1{\csname PY@tok@#1\endcsname}
\def\PY@toks#1+{\ifx\relax#1\empty\else%
    \PY@tok{#1}\expandafter\PY@toks\fi}
\def\PY@do#1{\PY@bc{\PY@tc{\PY@ul{%
    \PY@it{\PY@bf{\PY@ff{#1}}}}}}}
\def\PY#1#2{\PY@reset\PY@toks#1+\relax+\PY@do{#2}}

\expandafter\def\csname PY@tok@w\endcsname{\def\PY@tc##1{\textcolor[rgb]{0.73,0.73,0.73}{##1}}}
\expandafter\def\csname PY@tok@c\endcsname{\let\PY@it=\textit\def\PY@tc##1{\textcolor[rgb]{0.25,0.50,0.50}{##1}}}
\expandafter\def\csname PY@tok@cp\endcsname{\def\PY@tc##1{\textcolor[rgb]{0.74,0.48,0.00}{##1}}}
\expandafter\def\csname PY@tok@k\endcsname{\let\PY@bf=\textbf\def\PY@tc##1{\textcolor[rgb]{0.00,0.50,0.00}{##1}}}
\expandafter\def\csname PY@tok@kp\endcsname{\def\PY@tc##1{\textcolor[rgb]{0.00,0.50,0.00}{##1}}}
\expandafter\def\csname PY@tok@kt\endcsname{\def\PY@tc##1{\textcolor[rgb]{0.69,0.00,0.25}{##1}}}
\expandafter\def\csname PY@tok@o\endcsname{\def\PY@tc##1{\textcolor[rgb]{0.40,0.40,0.40}{##1}}}
\expandafter\def\csname PY@tok@ow\endcsname{\let\PY@bf=\textbf\def\PY@tc##1{\textcolor[rgb]{0.67,0.13,1.00}{##1}}}
\expandafter\def\csname PY@tok@nb\endcsname{\def\PY@tc##1{\textcolor[rgb]{0.00,0.50,0.00}{##1}}}
\expandafter\def\csname PY@tok@nf\endcsname{\def\PY@tc##1{\textcolor[rgb]{0.00,0.00,1.00}{##1}}}
\expandafter\def\csname PY@tok@nc\endcsname{\let\PY@bf=\textbf\def\PY@tc##1{\textcolor[rgb]{0.00,0.00,1.00}{##1}}}
\expandafter\def\csname PY@tok@nn\endcsname{\let\PY@bf=\textbf\def\PY@tc##1{\textcolor[rgb]{0.00,0.00,1.00}{##1}}}
\expandafter\def\csname PY@tok@ne\endcsname{\let\PY@bf=\textbf\def\PY@tc##1{\textcolor[rgb]{0.82,0.25,0.23}{##1}}}
\expandafter\def\csname PY@tok@nv\endcsname{\def\PY@tc##1{\textcolor[rgb]{0.10,0.09,0.49}{##1}}}
\expandafter\def\csname PY@tok@no\endcsname{\def\PY@tc##1{\textcolor[rgb]{0.53,0.00,0.00}{##1}}}
\expandafter\def\csname PY@tok@nl\endcsname{\def\PY@tc##1{\textcolor[rgb]{0.63,0.63,0.00}{##1}}}
\expandafter\def\csname PY@tok@ni\endcsname{\let\PY@bf=\textbf\def\PY@tc##1{\textcolor[rgb]{0.60,0.60,0.60}{##1}}}
\expandafter\def\csname PY@tok@na\endcsname{\def\PY@tc##1{\textcolor[rgb]{0.49,0.56,0.16}{##1}}}
\expandafter\def\csname PY@tok@nt\endcsname{\let\PY@bf=\textbf\def\PY@tc##1{\textcolor[rgb]{0.00,0.50,0.00}{##1}}}
\expandafter\def\csname PY@tok@nd\endcsname{\def\PY@tc##1{\textcolor[rgb]{0.67,0.13,1.00}{##1}}}
\expandafter\def\csname PY@tok@s\endcsname{\def\PY@tc##1{\textcolor[rgb]{0.73,0.13,0.13}{##1}}}
\expandafter\def\csname PY@tok@sd\endcsname{\let\PY@it=\textit\def\PY@tc##1{\textcolor[rgb]{0.73,0.13,0.13}{##1}}}
\expandafter\def\csname PY@tok@si\endcsname{\let\PY@bf=\textbf\def\PY@tc##1{\textcolor[rgb]{0.73,0.40,0.53}{##1}}}
\expandafter\def\csname PY@tok@se\endcsname{\let\PY@bf=\textbf\def\PY@tc##1{\textcolor[rgb]{0.73,0.40,0.13}{##1}}}
\expandafter\def\csname PY@tok@sr\endcsname{\def\PY@tc##1{\textcolor[rgb]{0.73,0.40,0.53}{##1}}}
\expandafter\def\csname PY@tok@ss\endcsname{\def\PY@tc##1{\textcolor[rgb]{0.10,0.09,0.49}{##1}}}
\expandafter\def\csname PY@tok@sx\endcsname{\def\PY@tc##1{\textcolor[rgb]{0.00,0.50,0.00}{##1}}}
\expandafter\def\csname PY@tok@m\endcsname{\def\PY@tc##1{\textcolor[rgb]{0.40,0.40,0.40}{##1}}}
\expandafter\def\csname PY@tok@gh\endcsname{\let\PY@bf=\textbf\def\PY@tc##1{\textcolor[rgb]{0.00,0.00,0.50}{##1}}}
\expandafter\def\csname PY@tok@gu\endcsname{\let\PY@bf=\textbf\def\PY@tc##1{\textcolor[rgb]{0.50,0.00,0.50}{##1}}}
\expandafter\def\csname PY@tok@gd\endcsname{\def\PY@tc##1{\textcolor[rgb]{0.63,0.00,0.00}{##1}}}
\expandafter\def\csname PY@tok@gi\endcsname{\def\PY@tc##1{\textcolor[rgb]{0.00,0.63,0.00}{##1}}}
\expandafter\def\csname PY@tok@gr\endcsname{\def\PY@tc##1{\textcolor[rgb]{1.00,0.00,0.00}{##1}}}
\expandafter\def\csname PY@tok@ge\endcsname{\let\PY@it=\textit}
\expandafter\def\csname PY@tok@gs\endcsname{\let\PY@bf=\textbf}
\expandafter\def\csname PY@tok@gp\endcsname{\let\PY@bf=\textbf\def\PY@tc##1{\textcolor[rgb]{0.00,0.00,0.50}{##1}}}
\expandafter\def\csname PY@tok@go\endcsname{\def\PY@tc##1{\textcolor[rgb]{0.53,0.53,0.53}{##1}}}
\expandafter\def\csname PY@tok@gt\endcsname{\def\PY@tc##1{\textcolor[rgb]{0.00,0.27,0.87}{##1}}}
\expandafter\def\csname PY@tok@err\endcsname{\def\PY@bc##1{\setlength{\fboxsep}{0pt}\fcolorbox[rgb]{1.00,0.00,0.00}{1,1,1}{\strut ##1}}}
\expandafter\def\csname PY@tok@kc\endcsname{\let\PY@bf=\textbf\def\PY@tc##1{\textcolor[rgb]{0.00,0.50,0.00}{##1}}}
\expandafter\def\csname PY@tok@kd\endcsname{\let\PY@bf=\textbf\def\PY@tc##1{\textcolor[rgb]{0.00,0.50,0.00}{##1}}}
\expandafter\def\csname PY@tok@kn\endcsname{\let\PY@bf=\textbf\def\PY@tc##1{\textcolor[rgb]{0.00,0.50,0.00}{##1}}}
\expandafter\def\csname PY@tok@kr\endcsname{\let\PY@bf=\textbf\def\PY@tc##1{\textcolor[rgb]{0.00,0.50,0.00}{##1}}}
\expandafter\def\csname PY@tok@bp\endcsname{\def\PY@tc##1{\textcolor[rgb]{0.00,0.50,0.00}{##1}}}
\expandafter\def\csname PY@tok@fm\endcsname{\def\PY@tc##1{\textcolor[rgb]{0.00,0.00,1.00}{##1}}}
\expandafter\def\csname PY@tok@vc\endcsname{\def\PY@tc##1{\textcolor[rgb]{0.10,0.09,0.49}{##1}}}
\expandafter\def\csname PY@tok@vg\endcsname{\def\PY@tc##1{\textcolor[rgb]{0.10,0.09,0.49}{##1}}}
\expandafter\def\csname PY@tok@vi\endcsname{\def\PY@tc##1{\textcolor[rgb]{0.10,0.09,0.49}{##1}}}
\expandafter\def\csname PY@tok@vm\endcsname{\def\PY@tc##1{\textcolor[rgb]{0.10,0.09,0.49}{##1}}}
\expandafter\def\csname PY@tok@sa\endcsname{\def\PY@tc##1{\textcolor[rgb]{0.73,0.13,0.13}{##1}}}
\expandafter\def\csname PY@tok@sb\endcsname{\def\PY@tc##1{\textcolor[rgb]{0.73,0.13,0.13}{##1}}}
\expandafter\def\csname PY@tok@sc\endcsname{\def\PY@tc##1{\textcolor[rgb]{0.73,0.13,0.13}{##1}}}
\expandafter\def\csname PY@tok@dl\endcsname{\def\PY@tc##1{\textcolor[rgb]{0.73,0.13,0.13}{##1}}}
\expandafter\def\csname PY@tok@s2\endcsname{\def\PY@tc##1{\textcolor[rgb]{0.73,0.13,0.13}{##1}}}
\expandafter\def\csname PY@tok@sh\endcsname{\def\PY@tc##1{\textcolor[rgb]{0.73,0.13,0.13}{##1}}}
\expandafter\def\csname PY@tok@s1\endcsname{\def\PY@tc##1{\textcolor[rgb]{0.73,0.13,0.13}{##1}}}
\expandafter\def\csname PY@tok@mb\endcsname{\def\PY@tc##1{\textcolor[rgb]{0.40,0.40,0.40}{##1}}}
\expandafter\def\csname PY@tok@mf\endcsname{\def\PY@tc##1{\textcolor[rgb]{0.40,0.40,0.40}{##1}}}
\expandafter\def\csname PY@tok@mh\endcsname{\def\PY@tc##1{\textcolor[rgb]{0.40,0.40,0.40}{##1}}}
\expandafter\def\csname PY@tok@mi\endcsname{\def\PY@tc##1{\textcolor[rgb]{0.40,0.40,0.40}{##1}}}
\expandafter\def\csname PY@tok@il\endcsname{\def\PY@tc##1{\textcolor[rgb]{0.40,0.40,0.40}{##1}}}
\expandafter\def\csname PY@tok@mo\endcsname{\def\PY@tc##1{\textcolor[rgb]{0.40,0.40,0.40}{##1}}}
\expandafter\def\csname PY@tok@ch\endcsname{\let\PY@it=\textit\def\PY@tc##1{\textcolor[rgb]{0.25,0.50,0.50}{##1}}}
\expandafter\def\csname PY@tok@cm\endcsname{\let\PY@it=\textit\def\PY@tc##1{\textcolor[rgb]{0.25,0.50,0.50}{##1}}}
\expandafter\def\csname PY@tok@cpf\endcsname{\let\PY@it=\textit\def\PY@tc##1{\textcolor[rgb]{0.25,0.50,0.50}{##1}}}
\expandafter\def\csname PY@tok@c1\endcsname{\let\PY@it=\textit\def\PY@tc##1{\textcolor[rgb]{0.25,0.50,0.50}{##1}}}
\expandafter\def\csname PY@tok@cs\endcsname{\let\PY@it=\textit\def\PY@tc##1{\textcolor[rgb]{0.25,0.50,0.50}{##1}}}

\def\PYZbs{\char`\\}
\def\PYZus{\char`\_}
\def\PYZob{\char`\{}
\def\PYZcb{\char`\}}
\def\PYZca{\char`\^}
\def\PYZam{\char`\&}
\def\PYZlt{\char`\<}
\def\PYZgt{\char`\>}
\def\PYZsh{\char`\#}
\def\PYZpc{\char`\%}
\def\PYZdl{\char`\$}
\def\PYZhy{\char`\-}
\def\PYZsq{\char`\'}
\def\PYZdq{\char`\"}
\def\PYZti{\char`\~}
% for compatibility with earlier versions
\def\PYZat{@}
\def\PYZlb{[}
\def\PYZrb{]}
\makeatother


    % Exact colors from NB
    \definecolor{incolor}{rgb}{0.0, 0.0, 0.5}
    \definecolor{outcolor}{rgb}{0.545, 0.0, 0.0}



    
    % Prevent overflowing lines due to hard-to-break entities
    \sloppy 
    % Setup hyperref package
    \hypersetup{
      breaklinks=true,  % so long urls are correctly broken across lines
      colorlinks=true,
      urlcolor=urlcolor,
      linkcolor=linkcolor,
      citecolor=citecolor,
      }
    % Slightly bigger margins than the latex defaults
    
    \geometry{verbose,tmargin=1in,bmargin=1in,lmargin=1in,rmargin=1in}
    
    

    \begin{document}
    
    
    \maketitle
    
    

    
    \hypertarget{assignment_6}{%
\section{Assignment\_6}\label{assignment_6}}

\emph{EE2703: Applied Programming\\
Author: Varun Sundar, EE16B068}

Conventions 1. We are using Python 3, GCC for C 2. Underscore naming vs
Camel Case 3. PEP 25 convention style.

\hypertarget{abstract}{%
\subsection{Abstract}\label{abstract}}

This weeks assignment concerns with solving for the simulation of a
tublelight, in 1-Dimension. We make certain simplifying assumptions that
allow us to simulate easily.

\hypertarget{introduction}{%
\subsection{Introduction}\label{introduction}}

Consider electrons being emitted by the cathode with zero energy, and
accelerate in this field. When they get beyond a threshold energy
\(E_0\), they can drive atoms to excited states. The relaxation of these
atoms results in light emission. In our model, we will assume that the
relaxation is immediate. The electron loses all its energy and the
process starts again.

Those electrons reaching the anode are absorbed and lost. Each ``time
step'', an average of \(N\) electrons are introduced at the cathode. The
actual number of electrons is determined by finding the integer part of
a random number that is ``normally distributed'' (Gaussian Distributed)
with standard deviation of 2 and mean 10.

\hypertarget{code-description}{%
\subsection{Code Description}\label{code-description}}

We have wrapped the entire simulation into a class called,
\emph{Simulate\_tubelight}, which creates the universe, then vectors.
The simulation steps (which is run sequentially in a loop) is described
below:

\begin{enumerate}
\def\labelenumi{\arabic{enumi}.}
\tightlist
\item
  All electrons are constrained between \(0<x<L\), anything out of range
  is set to \(x=0,u=0\).
\item
  Displacement and Velocity are updated with time steps as:
  \(dx_i =u_i\Delta t+\frac{a(\Delta t)}{2} =u_i+0.5\).
\item
  Determine which particles have hit the anode, reinitiate them to
  \((x=0,u=0)\).
\item
  Find out which electrons are sufficiently energised, together with a
  probability of collision, determine the number of photons emmited at a
  particular position.
\item
  Assuming inelastic collision, set their velocities to zero, and random
  position.
\item
  Inject a randomly distributed number of electrons, with the given mean
  \(M\) and variance \(M_sig\).
\item
  Compile a list of existing electrons into respective vectors.
\end{enumerate}

The entire loop may be run iteratively using a class method.

    \begin{Verbatim}[commandchars=\\\{\}]
{\color{incolor}In [{\color{incolor}404}]:} \PY{k+kn}{import} \PY{n+nn}{numpy} \PY{k}{as} \PY{n+nn}{np}
          \PY{k+kn}{import} \PY{n+nn}{matplotlib}\PY{n+nn}{.}\PY{n+nn}{pyplot} \PY{k}{as} \PY{n+nn}{plt}
\end{Verbatim}


    \begin{Verbatim}[commandchars=\\\{\}]
{\color{incolor}In [{\color{incolor}407}]:} \PY{k}{class} \PY{n+nc}{Simulate\PYZus{}tubelight}\PY{p}{(}\PY{n+nb}{object}\PY{p}{)}\PY{p}{:}
              \PY{l+s+sd}{\PYZsq{}\PYZsq{}\PYZsq{}Simulate a tubelight\PYZsq{}\PYZsq{}\PYZsq{}}
              \PY{k}{def} \PY{n+nf}{\PYZus{}\PYZus{}init\PYZus{}\PYZus{}}\PY{p}{(}\PY{n+nb+bp}{self}\PY{p}{,}\PY{n}{n}\PY{o}{=}\PY{l+m+mi}{100}\PY{p}{,}\PY{n}{M}\PY{o}{=}\PY{l+m+mi}{5}\PY{p}{,}\PY{n}{nk}\PY{o}{=}\PY{l+m+mi}{1000}\PY{p}{,}\PY{n}{u0}\PY{o}{=}\PY{l+m+mi}{7}\PY{p}{,}\PY{n}{p}\PY{o}{=}\PY{l+m+mf}{0.5}\PY{p}{,}\PY{n}{Msig}\PY{o}{=}\PY{l+m+mf}{0.2}\PY{p}{)}\PY{p}{:}
                  \PY{n+nb+bp}{self}\PY{o}{.}\PY{n}{create\PYZus{}universe}\PY{p}{(}\PY{n}{n}\PY{p}{,}\PY{n}{M}\PY{p}{,}\PY{n}{nk}\PY{p}{,}\PY{n}{u0}\PY{p}{,}\PY{n}{p}\PY{p}{,}\PY{n}{Msig}\PY{p}{)}
                  \PY{n+nb+bp}{self}\PY{o}{.}\PY{n}{create\PYZus{}vectors}\PY{p}{(}\PY{p}{)}
              
              \PY{k}{def} \PY{n+nf}{create\PYZus{}universe}\PY{p}{(}\PY{n+nb+bp}{self}\PY{p}{,}\PY{n}{n}\PY{o}{=}\PY{l+m+mi}{100}\PY{p}{,}\PY{n}{M}\PY{o}{=}\PY{l+m+mi}{5}\PY{p}{,}\PY{n}{nk}\PY{o}{=}\PY{l+m+mi}{1000}\PY{p}{,}\PY{n}{u0}\PY{o}{=}\PY{l+m+mi}{7}\PY{p}{,}\PY{n}{p}\PY{o}{=}\PY{l+m+mf}{0.5}\PY{p}{,}\PY{n}{Msig}\PY{o}{=}\PY{l+m+mf}{0.2}\PY{p}{)}\PY{p}{:}
                  \PY{n+nb+bp}{self}\PY{o}{.}\PY{n}{n}\PY{o}{=}\PY{n}{n}  \PY{c+c1}{\PYZsh{} spatial grid size.}
                  \PY{n+nb+bp}{self}\PY{o}{.}\PY{n}{M}\PY{o}{=}\PY{n}{M}    \PY{c+c1}{\PYZsh{} number of electrons injected per turn.}
                  \PY{n+nb+bp}{self}\PY{o}{.}\PY{n}{nk}\PY{o}{=}\PY{n}{nk} \PY{c+c1}{\PYZsh{} number of turns to simulate.}
                  \PY{n+nb+bp}{self}\PY{o}{.}\PY{n}{u0}\PY{o}{=}\PY{n}{u0}  \PY{c+c1}{\PYZsh{} threshold velocity.}
                  \PY{n+nb+bp}{self}\PY{o}{.}\PY{n}{p}\PY{o}{=}\PY{n}{p} \PY{c+c1}{\PYZsh{} probability that ionization will occur}
                  \PY{n+nb+bp}{self}\PY{o}{.}\PY{n}{Msig}\PY{o}{=}\PY{n}{Msig}
                  
              \PY{k}{def} \PY{n+nf}{create\PYZus{}vectors}\PY{p}{(}\PY{n+nb+bp}{self}\PY{p}{)}\PY{p}{:}
                  \PY{n}{dim}\PY{o}{=}\PY{n+nb+bp}{self}\PY{o}{.}\PY{n}{n}\PY{o}{*}\PY{n+nb+bp}{self}\PY{o}{.}\PY{n}{M} 
                  \PY{n+nb+bp}{self}\PY{o}{.}\PY{n}{xx}\PY{o}{=}\PY{n}{np}\PY{o}{.}\PY{n}{zeros}\PY{p}{(}\PY{n}{dim}\PY{p}{)} \PY{c+c1}{\PYZsh{} Electron position }
                  \PY{n+nb+bp}{self}\PY{o}{.}\PY{n}{u}\PY{o}{=}\PY{n}{np}\PY{o}{.}\PY{n}{zeros}\PY{p}{(}\PY{n}{dim}\PY{p}{)} \PY{c+c1}{\PYZsh{} Electron velocity}
                  \PY{n+nb+bp}{self}\PY{o}{.}\PY{n}{dx}\PY{o}{=}\PY{n}{np}\PY{o}{.}\PY{n}{zeros}\PY{p}{(}\PY{n}{dim}\PY{p}{)} \PY{c+c1}{\PYZsh{} Displacement in current turn}
                  \PY{n+nb+bp}{self}\PY{o}{.}\PY{n}{I}\PY{o}{=}\PY{p}{[}\PY{p}{]} \PY{c+c1}{\PYZsh{} Intensity of emitted light,}
                  \PY{n+nb+bp}{self}\PY{o}{.}\PY{n}{X}\PY{o}{=}\PY{p}{[}\PY{p}{]} \PY{c+c1}{\PYZsh{} Electron position}
                  \PY{n+nb+bp}{self}\PY{o}{.}\PY{n}{V}\PY{o}{=}\PY{p}{[}\PY{p}{]} \PY{c+c1}{\PYZsh{} Electron velocity}
                  
              \PY{k}{def} \PY{n+nf}{loop}\PY{p}{(}\PY{n+nb+bp}{self}\PY{p}{)}\PY{p}{:}
                  \PY{c+c1}{\PYZsh{} Find the electrons present in the chamber.}
                  \PY{n+nb+bp}{self}\PY{o}{.}\PY{n}{ii}\PY{o}{=}\PY{n}{np}\PY{o}{.}\PY{n}{where}\PY{p}{(}\PY{n+nb+bp}{self}\PY{o}{.}\PY{n}{xx}\PY{o}{\PYZgt{}}\PY{l+m+mi}{0}\PY{p}{)}\PY{p}{[}\PY{l+m+mi}{0}\PY{p}{]}
                  
                  \PY{c+c1}{\PYZsh{} Compute the displacement during this turn}
                  \PY{n+nb+bp}{self}\PY{o}{.}\PY{n}{dx}\PY{p}{[}\PY{n+nb+bp}{self}\PY{o}{.}\PY{n}{ii}\PY{p}{]}\PY{o}{=}\PY{n+nb+bp}{self}\PY{o}{.}\PY{n}{u}\PY{p}{[}\PY{n+nb+bp}{self}\PY{o}{.}\PY{n}{ii}\PY{p}{]}\PY{o}{+}\PY{l+m+mf}{0.5}
          
                  \PY{c+c1}{\PYZsh{} Advance the electron position and velocity for the turn.}
                  \PY{n+nb+bp}{self}\PY{o}{.}\PY{n}{xx}\PY{p}{[}\PY{n+nb+bp}{self}\PY{o}{.}\PY{n}{ii}\PY{p}{]}\PY{o}{+}\PY{o}{=}\PY{n+nb+bp}{self}\PY{o}{.}\PY{n}{dx}\PY{p}{[}\PY{n+nb+bp}{self}\PY{o}{.}\PY{n}{ii}\PY{p}{]}
                  \PY{n+nb+bp}{self}\PY{o}{.}\PY{n}{u}\PY{p}{[}\PY{n+nb+bp}{self}\PY{o}{.}\PY{n}{ii}\PY{p}{]}\PY{o}{+}\PY{o}{=}\PY{l+m+mi}{1}
                  
                  \PY{c+c1}{\PYZsh{} Determine which particles have hit the anode}
                  \PY{n+nb+bp}{self}\PY{o}{.}\PY{n}{hit\PYZus{}anode}\PY{o}{=}\PY{n}{np}\PY{o}{.}\PY{n}{where}\PY{p}{(}\PY{n+nb+bp}{self}\PY{o}{.}\PY{n}{xx}\PY{o}{\PYZgt{}}\PY{n+nb+bp}{self}\PY{o}{.}\PY{n}{n}\PY{p}{)}\PY{p}{[}\PY{l+m+mi}{0}\PY{p}{]}
                  \PY{n+nb+bp}{self}\PY{o}{.}\PY{n}{xx}\PY{p}{[}\PY{n+nb+bp}{self}\PY{o}{.}\PY{n}{hit\PYZus{}anode}\PY{p}{]}\PY{o}{=}\PY{l+m+mi}{0}
                  \PY{n+nb+bp}{self}\PY{o}{.}\PY{n}{u}\PY{p}{[}\PY{n+nb+bp}{self}\PY{o}{.}\PY{n}{hit\PYZus{}anode}\PY{p}{]}\PY{o}{=}\PY{l+m+mi}{0}
                  \PY{n+nb+bp}{self}\PY{o}{.}\PY{n}{dx}\PY{p}{[}\PY{n+nb+bp}{self}\PY{o}{.}\PY{n}{hit\PYZus{}anode}\PY{p}{]}\PY{o}{=}\PY{l+m+mi}{0}
                  
                  \PY{c+c1}{\PYZsh{} Find those electrons whose velocity is greater than or equal to the threshold.}
                  \PY{n+nb+bp}{self}\PY{o}{.}\PY{n}{kk}\PY{o}{=}\PY{n}{np}\PY{o}{.}\PY{n}{where}\PY{p}{(}\PY{n+nb+bp}{self}\PY{o}{.}\PY{n}{u}\PY{o}{\PYZgt{}}\PY{o}{=}\PY{n+nb+bp}{self}\PY{o}{.}\PY{n}{u0}\PY{p}{)}\PY{p}{[}\PY{l+m+mi}{0}\PY{p}{]}
                  \PY{c+c1}{\PYZsh{} Of these, which electrons are ionized}
                  \PY{n+nb+bp}{self}\PY{o}{.}\PY{n}{ll}\PY{o}{=}\PY{n}{np}\PY{o}{.}\PY{n}{where}\PY{p}{(}\PY{n}{np}\PY{o}{.}\PY{n}{random}\PY{o}{.}\PY{n}{rand}\PY{p}{(}\PY{n+nb}{len}\PY{p}{(}\PY{n+nb+bp}{self}\PY{o}{.}\PY{n}{kk}\PY{p}{)}\PY{p}{)}\PY{o}{\PYZlt{}}\PY{o}{=}\PY{n+nb+bp}{self}\PY{o}{.}\PY{n}{p}\PY{p}{)}\PY{p}{[}\PY{l+m+mi}{0}\PY{p}{]}
                  \PY{n+nb+bp}{self}\PY{o}{.}\PY{n}{kl}\PY{o}{=}\PY{n+nb+bp}{self}\PY{o}{.}\PY{n}{kk}\PY{p}{[}\PY{n+nb+bp}{self}\PY{o}{.}\PY{n}{ll}\PY{p}{]}
          
                  \PY{c+c1}{\PYZsh{} Reset the velocities of these electrons to zero (they suffered an inelastic collision)}
                  \PY{n+nb+bp}{self}\PY{o}{.}\PY{n}{u}\PY{p}{[}\PY{n+nb+bp}{self}\PY{o}{.}\PY{n}{kl}\PY{p}{]}\PY{o}{=}\PY{l+m+mi}{0}
                  \PY{c+c1}{\PYZsh{} The collision could have occurred at any point between the previous xi and the current xi}
                  \PY{n+nb+bp}{self}\PY{o}{.}\PY{n}{xx}\PY{p}{[}\PY{n+nb+bp}{self}\PY{o}{.}\PY{n}{kl}\PY{p}{]}\PY{o}{\PYZhy{}}\PY{o}{=}\PY{n+nb+bp}{self}\PY{o}{.}\PY{n}{dx}\PY{p}{[}\PY{n+nb+bp}{self}\PY{o}{.}\PY{n}{kl}\PY{p}{]}\PY{o}{*}\PY{n}{np}\PY{o}{.}\PY{n}{random}\PY{o}{.}\PY{n}{rand}\PY{p}{(}\PY{p}{)}
          
                  \PY{c+c1}{\PYZsh{} Excited atoms at this location resulted in emission from that point.}
                  \PY{n+nb+bp}{self}\PY{o}{.}\PY{n}{I}\PY{o}{.}\PY{n}{extend}\PY{p}{(}\PY{n+nb+bp}{self}\PY{o}{.}\PY{n}{xx}\PY{p}{[}\PY{n+nb+bp}{self}\PY{o}{.}\PY{n}{kl}\PY{p}{]}\PY{o}{.}\PY{n}{tolist}\PY{p}{(}\PY{p}{)}\PY{p}{)}
          
                  \PY{c+c1}{\PYZsh{} Inject M new electrons}
                  \PY{n}{m}\PY{o}{=} \PY{n+nb}{int}\PY{p}{(}\PY{n}{np}\PY{o}{.}\PY{n}{random}\PY{o}{.}\PY{n}{randn}\PY{p}{(}\PY{p}{)}\PY{o}{*}\PY{n+nb+bp}{self}\PY{o}{.}\PY{n}{Msig}\PY{o}{+}\PY{n+nb+bp}{self}\PY{o}{.}\PY{n}{M}\PY{p}{)}  \PY{c+c1}{\PYZsh{} actual number of electrons injected}
                  \PY{c+c1}{\PYZsh{} Add them to unused slots. Adding randomly}
                  \PY{n+nb+bp}{self}\PY{o}{.}\PY{n}{slots\PYZus{}to\PYZus{}add\PYZus{}to}\PY{o}{=}\PY{n}{np}\PY{o}{.}\PY{n}{where}\PY{p}{(}\PY{n+nb+bp}{self}\PY{o}{.}\PY{n}{xx}\PY{o}{==}\PY{l+m+mi}{0}\PY{p}{)}\PY{p}{[}\PY{l+m+mi}{0}\PY{p}{]}
          
                  \PY{k}{if} \PY{n+nb}{len}\PY{p}{(}\PY{n+nb+bp}{self}\PY{o}{.}\PY{n}{slots\PYZus{}to\PYZus{}add\PYZus{}to}\PY{p}{)}\PY{o}{\PYZgt{}}\PY{o}{=}\PY{n}{m}\PY{p}{:}
                      \PY{n}{random\PYZus{}start}\PY{o}{=}\PY{n}{np}\PY{o}{.}\PY{n}{random}\PY{o}{.}\PY{n}{randint}\PY{p}{(}\PY{n+nb}{len}\PY{p}{(}\PY{n+nb+bp}{self}\PY{o}{.}\PY{n}{slots\PYZus{}to\PYZus{}add\PYZus{}to}\PY{p}{)}\PY{p}{)}
                      \PY{n+nb+bp}{self}\PY{o}{.}\PY{n}{xx}\PY{p}{[}\PY{n+nb+bp}{self}\PY{o}{.}\PY{n}{slots\PYZus{}to\PYZus{}add\PYZus{}to}\PY{p}{[}\PY{n}{random\PYZus{}start}\PY{p}{:}\PY{n}{m}\PY{o}{+}\PY{n}{random\PYZus{}start}\PY{p}{]}\PY{p}{]}\PY{o}{=}\PY{l+m+mi}{1}
                      \PY{n+nb+bp}{self}\PY{o}{.}\PY{n}{u}\PY{p}{[}\PY{n+nb+bp}{self}\PY{o}{.}\PY{n}{slots\PYZus{}to\PYZus{}add\PYZus{}to}\PY{p}{[}\PY{n}{random\PYZus{}start}\PY{o}{\PYZhy{}}\PY{n}{m}\PY{p}{:}\PY{n}{random\PYZus{}start}\PY{p}{]}\PY{p}{]}\PY{o}{=}\PY{l+m+mi}{0}
                  \PY{k}{else}\PY{p}{:} \PY{c+c1}{\PYZsh{} If no free slots}
                      \PY{n+nb+bp}{self}\PY{o}{.}\PY{n}{xx}\PY{p}{[}\PY{n+nb+bp}{self}\PY{o}{.}\PY{n}{slots\PYZus{}to\PYZus{}add\PYZus{}to}\PY{p}{]}\PY{o}{=}\PY{l+m+mi}{1}
                      \PY{n+nb+bp}{self}\PY{o}{.}\PY{n}{u}\PY{p}{[}\PY{n+nb+bp}{self}\PY{o}{.}\PY{n}{slots\PYZus{}to\PYZus{}add\PYZus{}to}\PY{p}{]}\PY{o}{=}\PY{l+m+mi}{0}
                      
                  \PY{n+nb+bp}{self}\PY{o}{.}\PY{n}{existing\PYZus{}electrons}\PY{o}{=}\PY{n}{np}\PY{o}{.}\PY{n}{where}\PY{p}{(}\PY{n+nb+bp}{self}\PY{o}{.}\PY{n}{xx}\PY{o}{\PYZgt{}}\PY{l+m+mi}{0}\PY{p}{)}\PY{p}{[}\PY{l+m+mi}{0}\PY{p}{]}
                  \PY{n+nb+bp}{self}\PY{o}{.}\PY{n}{X}\PY{o}{.}\PY{n}{extend}\PY{p}{(}\PY{n+nb+bp}{self}\PY{o}{.}\PY{n}{xx}\PY{p}{[}\PY{n+nb+bp}{self}\PY{o}{.}\PY{n}{existing\PYZus{}electrons}\PY{p}{]}\PY{o}{.}\PY{n}{tolist}\PY{p}{(}\PY{p}{)}\PY{p}{)}
                  \PY{n+nb+bp}{self}\PY{o}{.}\PY{n}{V}\PY{o}{.}\PY{n}{extend}\PY{p}{(}\PY{n+nb+bp}{self}\PY{o}{.}\PY{n}{u}\PY{p}{[}\PY{n+nb+bp}{self}\PY{o}{.}\PY{n}{existing\PYZus{}electrons}\PY{p}{]}\PY{o}{.}\PY{n}{tolist}\PY{p}{(}\PY{p}{)}\PY{p}{)}
                  
              \PY{k}{def} \PY{n+nf}{run\PYZus{}loop}\PY{p}{(}\PY{n+nb+bp}{self}\PY{p}{)}\PY{p}{:}
                  \PY{p}{[}\PY{n+nb+bp}{self}\PY{o}{.}\PY{n}{loop}\PY{p}{(}\PY{p}{)} \PY{k}{for} \PY{n}{i} \PY{o+ow}{in} \PY{n+nb}{range}\PY{p}{(}\PY{n+nb+bp}{self}\PY{o}{.}\PY{n}{nk}\PY{p}{)}\PY{p}{]}
                  
              \PY{k}{def} \PY{n+nf}{plot\PYZus{}intensity}\PY{p}{(}\PY{n+nb+bp}{self}\PY{p}{)}\PY{p}{:}
                  \PY{n}{plt}\PY{o}{.}\PY{n}{hist}\PY{p}{(}\PY{n+nb+bp}{self}\PY{o}{.}\PY{n}{I}\PY{p}{,}\PY{n}{bins}\PY{o}{=}\PY{n}{np}\PY{o}{.}\PY{n}{arange}\PY{p}{(}\PY{l+m+mi}{1}\PY{p}{,}\PY{l+m+mi}{100}\PY{p}{)}\PY{p}{,}\PY{n}{ec}\PY{o}{=}\PY{l+s+s1}{\PYZsq{}}\PY{l+s+s1}{black}\PY{l+s+s1}{\PYZsq{}}\PY{p}{,}\PY{n}{alpha}\PY{o}{=}\PY{l+m+mf}{0.5}\PY{p}{)}
                  \PY{n}{plt}\PY{o}{.}\PY{n}{title}\PY{p}{(}\PY{l+s+s2}{\PYZdq{}}\PY{l+s+s2}{Light Intensity Histogram }\PY{l+s+s2}{\PYZdq{}}\PY{p}{)}
                  \PY{n}{plt}\PY{o}{.}\PY{n}{show}\PY{p}{(}\PY{p}{)}
              
              \PY{k}{def} \PY{n+nf}{plot\PYZus{}electron\PYZus{}density}\PY{p}{(}\PY{n+nb+bp}{self}\PY{p}{)}\PY{p}{:}
                  \PY{n}{plt}\PY{o}{.}\PY{n}{hist}\PY{p}{(}\PY{n+nb+bp}{self}\PY{o}{.}\PY{n}{X}\PY{p}{,}\PY{n}{bins}\PY{o}{=}\PY{n}{np}\PY{o}{.}\PY{n}{arange}\PY{p}{(}\PY{l+m+mi}{1}\PY{p}{,}\PY{l+m+mi}{100}\PY{p}{)}\PY{p}{,}\PY{n}{ec}\PY{o}{=}\PY{l+s+s1}{\PYZsq{}}\PY{l+s+s1}{black}\PY{l+s+s1}{\PYZsq{}}\PY{p}{,}\PY{n}{alpha}\PY{o}{=}\PY{l+m+mf}{0.5}\PY{p}{)}
                  \PY{n}{plt}\PY{o}{.}\PY{n}{title}\PY{p}{(}\PY{l+s+s2}{\PYZdq{}}\PY{l+s+s2}{Electron Density Histogram}\PY{l+s+s2}{\PYZdq{}}\PY{p}{)}
                  \PY{n}{plt}\PY{o}{.}\PY{n}{show}\PY{p}{(}\PY{p}{)}
              
              \PY{k}{def} \PY{n+nf}{plot\PYZus{}intensity\PYZus{}table}\PY{p}{(}\PY{n+nb+bp}{self}\PY{p}{)}\PY{p}{:}
                  \PY{k+kn}{import} \PY{n+nn}{pandas}
                  \PY{n}{a}\PY{p}{,}\PY{n}{bins}\PY{p}{,}\PY{n}{c}\PY{o}{=}\PY{n}{plt}\PY{o}{.}\PY{n}{hist}\PY{p}{(}\PY{n+nb+bp}{self}\PY{o}{.}\PY{n}{I}\PY{p}{,}\PY{n}{bins}\PY{o}{=}\PY{n}{np}\PY{o}{.}\PY{n}{arange}\PY{p}{(}\PY{l+m+mi}{1}\PY{p}{,}\PY{l+m+mi}{100}\PY{p}{)}\PY{p}{,}\PY{n}{ec}\PY{o}{=}\PY{l+s+s1}{\PYZsq{}}\PY{l+s+s1}{black}\PY{l+s+s1}{\PYZsq{}}\PY{p}{,}\PY{n}{alpha}\PY{o}{=}\PY{l+m+mf}{0.5}\PY{p}{)}
                  \PY{n}{xpos}\PY{o}{=}\PY{l+m+mf}{0.5}\PY{o}{*}\PY{p}{(}\PY{n}{bins}\PY{p}{[}\PY{l+m+mi}{0}\PY{p}{:}\PY{o}{\PYZhy{}}\PY{l+m+mi}{1}\PY{p}{]}\PY{o}{+}\PY{n}{bins}\PY{p}{[}\PY{l+m+mi}{1}\PY{p}{:}\PY{p}{]}\PY{p}{)}
                  \PY{n}{d}\PY{o}{=}\PY{p}{\PYZob{}}\PY{l+s+s1}{\PYZsq{}}\PY{l+s+s1}{Position}\PY{l+s+s1}{\PYZsq{}}\PY{p}{:}\PY{n}{xpos}\PY{p}{,}\PY{l+s+s1}{\PYZsq{}}\PY{l+s+s1}{Count}\PY{l+s+s1}{\PYZsq{}}\PY{p}{:}\PY{n}{a}\PY{p}{\PYZcb{}}
                  \PY{n}{p}\PY{o}{=}\PY{n}{pandas}\PY{o}{.}\PY{n}{DataFrame}\PY{p}{(}\PY{n}{data}\PY{o}{=}\PY{n}{d}\PY{p}{)}
                  \PY{n+nb}{print}\PY{p}{(}\PY{n}{p}\PY{p}{)}
              
              \PY{k}{def} \PY{n+nf}{plot\PYZus{}electron\PYZus{}phase\PYZus{}space}\PY{p}{(}\PY{n+nb+bp}{self}\PY{p}{)}\PY{p}{:}
                  \PY{n}{plt}\PY{o}{.}\PY{n}{plot}\PY{p}{(}\PY{n+nb+bp}{self}\PY{o}{.}\PY{n}{xx}\PY{p}{,}\PY{n+nb+bp}{self}\PY{o}{.}\PY{n}{u}\PY{p}{,}\PY{l+s+s1}{\PYZsq{}}\PY{l+s+s1}{x}\PY{l+s+s1}{\PYZsq{}}\PY{p}{)}
                  \PY{n}{plt}\PY{o}{.}\PY{n}{title}\PY{p}{(}\PY{l+s+s2}{\PYZdq{}}\PY{l+s+s2}{Electron Phase Space}\PY{l+s+s2}{\PYZdq{}}\PY{p}{)}
                  \PY{n}{plt}\PY{o}{.}\PY{n}{xlabel}\PY{p}{(}\PY{l+s+s2}{\PYZdq{}}\PY{l+s+s2}{Position}\PY{l+s+s2}{\PYZdq{}}\PY{p}{)}
                  \PY{n}{plt}\PY{o}{.}\PY{n}{figsize}\PY{o}{=}\PY{p}{(}\PY{l+m+mi}{15}\PY{p}{,}\PY{l+m+mi}{10}\PY{p}{)}
                  \PY{n}{plt}\PY{o}{.}\PY{n}{ylabel}\PY{p}{(}\PY{l+s+s2}{\PYZdq{}}\PY{l+s+s2}{Velocity}\PY{l+s+s2}{\PYZdq{}}\PY{p}{)}
                  \PY{n}{plt}\PY{o}{.}\PY{n}{show}\PY{p}{(}\PY{p}{)}
\end{Verbatim}


    The probability distribution for the displacement after collision is not
uniform in space despite being uniform in time. This can be corrected
using:

\[x=\frac{at^{2}}{2}\] (when u=0)

\[p_{X}(x)=p_{T}(t)*\frac {\partial t}{\partial x} \].

We also write a code block to utilise system arguments to alter various
paramters.

These are:

\begin{enumerate}
\def\labelenumi{\arabic{enumi}.}
\tightlist
\item
  `-n', ``--grid\_size'', help=``Grid Size'',default=100.
\item
  `-M',`--no\_of\_electrons', help=``Downscale ratio'', default=5.
\item
  `-nk', `--turns', help=``Number of iterations'',default=500.
\item
  `-u0', `--threshold\_velocity', help=``Threshold Velocity for
  electron'',default=500.
\item
  `-p', `--probability', help=``Probability of collision'',default=500.
\item
  `-Msig', `--variance', help=``Variance of probability distribution''.
\end{enumerate}

    \begin{Verbatim}[commandchars=\\\{\}]
{\color{incolor}In [{\color{incolor}397}]:} \PY{k+kn}{import} \PY{n+nn}{argparse} \PY{k}{as} \PY{n+nn}{ap}
          
          \PY{k}{if} \PY{n+nv+vm}{\PYZus{}\PYZus{}name\PYZus{}\PYZus{}}\PY{o}{==}\PY{l+s+s1}{\PYZsq{}}\PY{l+s+s1}{\PYZus{}\PYZus{}main\PYZus{}\PYZus{}}\PY{l+s+s1}{\PYZsq{}}\PY{p}{:}
              \PY{n}{parser} \PY{o}{=} \PY{n}{ap}\PY{o}{.}\PY{n}{ArgumentParser}\PY{p}{(}\PY{p}{)}
              \PY{n}{parser}\PY{o}{.}\PY{n}{add\PYZus{}argument}\PY{p}{(}\PY{l+s+s1}{\PYZsq{}}\PY{l+s+s1}{\PYZhy{}n}\PY{l+s+s1}{\PYZsq{}}\PY{p}{,} \PY{l+s+s2}{\PYZdq{}}\PY{l+s+s2}{\PYZhy{}\PYZhy{}grid\PYZus{}size}\PY{l+s+s2}{\PYZdq{}}\PY{p}{,} \PY{n}{help}\PY{o}{=}\PY{l+s+s2}{\PYZdq{}}\PY{l+s+s2}{Grid Size}\PY{l+s+s2}{\PYZdq{}}\PY{p}{,}\PY{n}{default}\PY{o}{=}\PY{l+m+mi}{100}\PY{p}{)}
              \PY{n}{parser}\PY{o}{.}\PY{n}{add\PYZus{}argument}\PY{p}{(}\PY{l+s+s1}{\PYZsq{}}\PY{l+s+s1}{\PYZhy{}M}\PY{l+s+s1}{\PYZsq{}}\PY{p}{,}\PY{l+s+s1}{\PYZsq{}}\PY{l+s+s1}{\PYZhy{}\PYZhy{}no\PYZus{}of\PYZus{}electrons}\PY{l+s+s1}{\PYZsq{}}\PY{p}{,} \PY{n}{help}\PY{o}{=}\PY{l+s+s2}{\PYZdq{}}\PY{l+s+s2}{Downscale ratio}\PY{l+s+s2}{\PYZdq{}}\PY{p}{,} \PY{n}{default}\PY{o}{=}\PY{l+m+mi}{5}\PY{p}{)}
              \PY{n}{parser}\PY{o}{.}\PY{n}{add\PYZus{}argument}\PY{p}{(}\PY{l+s+s1}{\PYZsq{}}\PY{l+s+s1}{\PYZhy{}nk}\PY{l+s+s1}{\PYZsq{}}\PY{p}{,} \PY{l+s+s1}{\PYZsq{}}\PY{l+s+s1}{\PYZhy{}\PYZhy{}turns}\PY{l+s+s1}{\PYZsq{}}\PY{p}{,} \PY{n}{help}\PY{o}{=}\PY{l+s+s2}{\PYZdq{}}\PY{l+s+s2}{Number of iterations}\PY{l+s+s2}{\PYZdq{}}\PY{p}{,}\PY{n}{default}\PY{o}{=}\PY{l+m+mi}{500}\PY{p}{)}
              \PY{n}{parser}\PY{o}{.}\PY{n}{add\PYZus{}argument}\PY{p}{(}\PY{l+s+s1}{\PYZsq{}}\PY{l+s+s1}{\PYZhy{}u0}\PY{l+s+s1}{\PYZsq{}}\PY{p}{,} \PY{l+s+s1}{\PYZsq{}}\PY{l+s+s1}{\PYZhy{}\PYZhy{}threshold\PYZus{}velocity}\PY{l+s+s1}{\PYZsq{}}\PY{p}{,} \PY{n}{help}\PY{o}{=}\PY{l+s+s2}{\PYZdq{}}\PY{l+s+s2}{Threshold Velocity for electron}\PY{l+s+s2}{\PYZdq{}}\PY{p}{,}\PY{n}{default}\PY{o}{=}\PY{l+m+mi}{7}\PY{p}{)}
              \PY{n}{parser}\PY{o}{.}\PY{n}{add\PYZus{}argument}\PY{p}{(}\PY{l+s+s1}{\PYZsq{}}\PY{l+s+s1}{\PYZhy{}p}\PY{l+s+s1}{\PYZsq{}}\PY{p}{,} \PY{l+s+s1}{\PYZsq{}}\PY{l+s+s1}{\PYZhy{}\PYZhy{}probability}\PY{l+s+s1}{\PYZsq{}}\PY{p}{,} \PY{n}{help}\PY{o}{=}\PY{l+s+s2}{\PYZdq{}}\PY{l+s+s2}{Probability of collision}\PY{l+s+s2}{\PYZdq{}}\PY{p}{,}\PY{n}{default}\PY{o}{=}\PY{l+m+mf}{0.5}\PY{p}{)}
              \PY{n}{parser}\PY{o}{.}\PY{n}{add\PYZus{}argument}\PY{p}{(}\PY{l+s+s1}{\PYZsq{}}\PY{l+s+s1}{\PYZhy{}Msig}\PY{l+s+s1}{\PYZsq{}}\PY{p}{,} \PY{l+s+s1}{\PYZsq{}}\PY{l+s+s1}{\PYZhy{}\PYZhy{}variance}\PY{l+s+s1}{\PYZsq{}}\PY{p}{,} \PY{n}{help}\PY{o}{=}\PY{l+s+s2}{\PYZdq{}}\PY{l+s+s2}{Variance of probability distribution }\PY{l+s+s2}{\PYZdq{}}\PY{p}{,}\PY{n}{default}\PY{o}{=}\PY{l+m+mf}{0.2}\PY{p}{)}
              \PY{n}{args} \PY{o}{=} \PY{n+nb}{vars}\PY{p}{(}\PY{n}{parser}\PY{o}{.}\PY{n}{parse\PYZus{}args}\PY{p}{(}\PY{p}{)}\PY{p}{)}
              \PY{n}{n}\PY{p}{,}\PY{n}{M}\PY{p}{,}\PY{n}{nk}\PY{p}{,}\PY{n}{u0}\PY{p}{,}\PY{n}{p}\PY{p}{,}\PY{n}{Msig}\PY{o}{=}\PY{p}{(}\PY{n}{args}\PY{o}{.}\PY{n}{grid\PYZus{}size}\PY{p}{,}\PY{n}{args}\PY{o}{.}\PY{n}{no\PYZus{}of\PYZus{}electrons}\PY{p}{,}\PY{n}{args}\PY{o}{.}\PY{n}{turns}\PY{p}{,}\PY{n}{args}\PY{o}{.}\PY{n}{threshold\PYZus{}velocity}\PY{p}{,}\PYZbs{}
                               \PY{n}{args}\PY{o}{.}\PY{n}{probability}\PY{p}{,}\PY{n}{args}\PY{o}{.}\PY{n}{variance}\PY{p}{)}
              \PY{n}{a}\PY{o}{=}\PY{n}{Simulate\PYZus{}tubelight}\PY{p}{(}\PY{n}{n}\PY{p}{,}\PY{n}{M}\PY{p}{,}\PY{n}{nk}\PY{p}{,}\PY{n}{u0}\PY{p}{,}\PY{n}{p}\PY{p}{,}\PY{n}{Msig}\PY{p}{)}
\end{Verbatim}


    \begin{Verbatim}[commandchars=\\\{\}]
usage: ipykernel\_launcher.py [-h] [-n GRID\_SIZE] [-M NO\_OF\_ELECTRONS]
                             [-nk TURNS] [-u0 THRESHOLD\_VELOCITY]
                             [-p PROBABILITY] [-Msig VARIANCE]
ipykernel\_launcher.py: error: unrecognized arguments: -f /Users/Ankivarun/Library/Jupyter/runtime/kernel-0af91722-e69b-439a-a845-875dd6231d43.json

    \end{Verbatim}

    \begin{Verbatim}[commandchars=\\\{\}]

        An exception has occurred, use \%tb to see the full traceback.


        SystemExit: 2


    \end{Verbatim}

    \begin{Verbatim}[commandchars=\\\{\}]
/Users/Ankivarun/anaconda3/envs/tf\_python3/lib/python3.6/site-packages/IPython/core/interactiveshell.py:2918: UserWarning: To exit: use 'exit', 'quit', or Ctrl-D.
  warn("To exit: use 'exit', 'quit', or Ctrl-D.", stacklevel=1)

    \end{Verbatim}

    \hypertarget{simulation-results}{%
\subsection{Simulation Results}\label{simulation-results}}

We plot the results for the values of:

\begin{enumerate}
\def\labelenumi{\arabic{enumi}.}
\tightlist
\item
  `-n', ``--grid\_size'', help=``Grid Size'', = 100.
\item
  `-M',`--no\_of\_electrons', help=``Downscale ratio'', = 5.
\item
  `-nk', `--turns', help=``Number of iterations'',= 500.
\item
  `-u0', `--threshold\_velocity', help=``Threshold Velocity for
  electron'',= 500.
\item
  `-p', `--probability', help=``Probability of collision'', = 500.
\item
  `-Msig', `--variance', help=``Variance of probability distribution''.
\end{enumerate}

The following plots are made:

\begin{enumerate}
\def\labelenumi{\arabic{enumi}.}
\tightlist
\item
  Electron Phase space plot (displacement - velocity)
\item
  Electron Density Histogram
\item
  Light Intensity Histogram
\item
  Intensity versus Displacement Table.
\end{enumerate}

    \begin{Verbatim}[commandchars=\\\{\}]
{\color{incolor}In [{\color{incolor}398}]:} \PY{n}{a}\PY{o}{=}\PY{n}{Simulate\PYZus{}tubelight}\PY{p}{(}\PY{p}{)}
          \PY{n}{a}\PY{o}{.}\PY{n}{run\PYZus{}loop}\PY{p}{(}\PY{p}{)}
\end{Verbatim}


    \begin{Verbatim}[commandchars=\\\{\}]
{\color{incolor}In [{\color{incolor}399}]:} \PY{n}{a}\PY{o}{.}\PY{n}{plot\PYZus{}electron\PYZus{}phase\PYZus{}space}\PY{p}{(}\PY{p}{)}
\end{Verbatim}


    \begin{center}
    \adjustimage{max size={0.9\linewidth}{0.9\paperheight}}{output_7_0.png}
    \end{center}
    { \hspace*{\fill} \\}
    
    \begin{Verbatim}[commandchars=\\\{\}]
{\color{incolor}In [{\color{incolor}400}]:} \PY{n}{a}\PY{o}{.}\PY{n}{plot\PYZus{}electron\PYZus{}density}\PY{p}{(}\PY{p}{)}
\end{Verbatim}


    \begin{center}
    \adjustimage{max size={0.9\linewidth}{0.9\paperheight}}{output_8_0.png}
    \end{center}
    { \hspace*{\fill} \\}
    
    \begin{Verbatim}[commandchars=\\\{\}]
{\color{incolor}In [{\color{incolor}401}]:} \PY{n}{a}\PY{o}{.}\PY{n}{plot\PYZus{}intensity}\PY{p}{(}\PY{p}{)}
\end{Verbatim}


    \begin{center}
    \adjustimage{max size={0.9\linewidth}{0.9\paperheight}}{output_9_0.png}
    \end{center}
    { \hspace*{\fill} \\}
    
    \begin{Verbatim}[commandchars=\\\{\}]
{\color{incolor}In [{\color{incolor}402}]:} \PY{n}{a}\PY{o}{.}\PY{n}{plot\PYZus{}intensity\PYZus{}table}\PY{p}{(}\PY{p}{)}
\end{Verbatim}


    \begin{Verbatim}[commandchars=\\\{\}]
    Count  Position
0     0.0       1.5
1     0.0       2.5
2     0.0       3.5
3     0.0       4.5
4     0.0       5.5
5     0.0       6.5
6     0.0       7.5
7     0.0       8.5
8     0.0       9.5
9     0.0      10.5
10    0.0      11.5
11    0.0      12.5
12    0.0      13.5
13    0.0      14.5
14    0.0      15.5
15    0.0      16.5
16    0.0      17.5
17    0.0      18.5
18  348.0      19.5
19  352.0      20.5
20  295.0      21.5
21  398.0      22.5
22  340.0      23.5
23  322.0      24.5
24  261.0      25.5
25  139.0      26.5
26  156.0      27.5
27  118.0      28.5
28  172.0      29.5
29  152.0      30.5
..    {\ldots}       {\ldots}
68  156.0      69.5
69  150.0      70.5
70  150.0      71.5
71  152.0      72.5
72  144.0      73.5
73  176.0      74.5
74  144.0      75.5
75  161.0      76.5
76  145.0      77.5
77  136.0      78.5
78  147.0      79.5
79  133.0      80.5
80  144.0      81.5
81  155.0      82.5
82  123.0      83.5
83  146.0      84.5
84  140.0      85.5
85  138.0      86.5
86  159.0      87.5
87  166.0      88.5
88  149.0      89.5
89  146.0      90.5
90  133.0      91.5
91  145.0      92.5
92  121.0      93.5
93   99.0      94.5
94   95.0      95.5
95   68.0      96.5
96   60.0      97.5
97   30.0      98.5

[98 rows x 2 columns]

    \end{Verbatim}

    \begin{center}
    \adjustimage{max size={0.9\linewidth}{0.9\paperheight}}{output_10_1.png}
    \end{center}
    { \hspace*{\fill} \\}
    
    \hypertarget{inference-and-discussion}{%
\subsection{Inference and Discussion}\label{inference-and-discussion}}

Firstly, the intensity histogram reveals that the electrons donot cause
excitation of atoms til they cross a particular threshold velocity, as
dictated by the nature of the gas used. Secondly, this gives rise to a
peak in intensity just after the first mean length. This is beaucse a
majority of electrons collide with atoms at this distance. Further this,
subsequent peaks do exist, but have larger spread and are less
prominent. We observer around 2 dark bands in this intensity profile.

The electron phase plots show the constant acceleration all electrons
initially undergo, and the subsequent random motion post collision. The
phase plots are nearly uniformly distributed in the middle portion of
the tubelight.

\hypertarget{altering-simulation-parameters}{%
\subsection{Altering Simulation
Parameters}\label{altering-simulation-parameters}}

We try out the following set of parameters:

\begin{enumerate}
\def\labelenumi{\arabic{enumi}.}
\tightlist
\item
  n=100, M = 5, nk =1000, u0=20, p=0.5,Msig=0.2 (larger threshold
  velocity).
\item
  n=100, M = 5, nk =1000, u0=7, p=0.1 ,Msig=0.2 (lower probabiltiy of
  collision).
\item
  n=100, M = 5, nk =1000, u0=7, p=0.1 ,Msig=4 (higher variance of
  randomness(normal variable)).
\end{enumerate}

    \begin{Verbatim}[commandchars=\\\{\}]
{\color{incolor}In [{\color{incolor}413}]:} \PY{n}{List\PYZus{}of\PYZus{}params}\PY{o}{=}\PY{p}{[}\PY{p}{(}\PY{l+m+mi}{100}\PY{p}{,} \PY{l+m+mi}{5}\PY{p}{,} \PY{l+m+mi}{1000}\PY{p}{,}\PY{l+m+mi}{12}\PY{p}{,}\PY{l+m+mf}{0.5}\PY{p}{,}\PY{l+m+mf}{0.2} \PY{p}{)}\PY{p}{,}\PY{p}{(}\PY{l+m+mi}{100}\PY{p}{,} \PY{l+m+mi}{5}\PY{p}{,} \PY{l+m+mi}{1000}\PY{p}{,}\PY{l+m+mi}{7}\PY{p}{,}\PY{l+m+mf}{0.1}\PY{p}{,}\PY{l+m+mf}{0.2} \PY{p}{)}\PY{p}{,}\PY{p}{(}\PY{l+m+mi}{100}\PY{p}{,} \PY{l+m+mi}{5}\PY{p}{,} \PY{l+m+mi}{1000}\PY{p}{,}\PY{l+m+mi}{7}\PY{p}{,}\PY{l+m+mf}{0.5}\PY{p}{,}\PY{l+m+mi}{4} \PY{p}{)}\PY{p}{]}
          
          \PY{k}{for} \PY{n}{params}\PY{p}{,}\PY{n}{i} \PY{o+ow}{in} \PY{n+nb}{zip}\PY{p}{(}\PY{n}{List\PYZus{}of\PYZus{}params}\PY{p}{,}\PY{n+nb}{range}\PY{p}{(}\PY{n+nb}{len}\PY{p}{(}\PY{n}{List\PYZus{}of\PYZus{}params}\PY{p}{)}\PY{p}{)}\PY{p}{)}\PY{p}{:}
              \PY{n}{n}\PY{p}{,}\PY{n}{M}\PY{p}{,}\PY{n}{nk}\PY{p}{,}\PY{n}{u0}\PY{p}{,}\PY{n}{p}\PY{p}{,}\PY{n}{Msig}\PY{o}{=}\PY{n}{params}
              \PY{n+nb}{print}\PY{p}{(}\PY{l+s+s2}{\PYZdq{}}\PY{l+s+se}{\PYZbs{}t}\PY{l+s+s2}{ Parameter Case }\PY{l+s+se}{\PYZbs{}t}\PY{l+s+s2}{\PYZdq{}}\PY{p}{,}\PY{n}{i}\PY{o}{+}\PY{l+m+mi}{1}\PY{p}{)}
              \PY{n}{a}\PY{o}{=}\PY{n}{Simulate\PYZus{}tubelight}\PY{p}{(}\PY{n}{n}\PY{p}{,}\PY{n}{M}\PY{p}{,}\PY{n}{nk}\PY{p}{,}\PY{n}{u0}\PY{p}{,}\PY{n}{p}\PY{p}{,}\PY{n}{Msig}\PY{p}{)}
              \PY{n}{a}\PY{o}{.}\PY{n}{run\PYZus{}loop}\PY{p}{(}\PY{p}{)}
              \PY{n}{a}\PY{o}{.}\PY{n}{plot\PYZus{}electron\PYZus{}density}\PY{p}{(}\PY{p}{)}
              \PY{n}{a}\PY{o}{.}\PY{n}{plot\PYZus{}electron\PYZus{}phase\PYZus{}space}\PY{p}{(}\PY{p}{)}
              \PY{n}{a}\PY{o}{.}\PY{n}{plot\PYZus{}intensity}\PY{p}{(}\PY{p}{)}
\end{Verbatim}


    \begin{Verbatim}[commandchars=\\\{\}]
	 Parameter Case 	 1

    \end{Verbatim}

    \begin{center}
    \adjustimage{max size={0.9\linewidth}{0.9\paperheight}}{output_12_1.png}
    \end{center}
    { \hspace*{\fill} \\}
    
    \begin{center}
    \adjustimage{max size={0.9\linewidth}{0.9\paperheight}}{output_12_2.png}
    \end{center}
    { \hspace*{\fill} \\}
    
    \begin{center}
    \adjustimage{max size={0.9\linewidth}{0.9\paperheight}}{output_12_3.png}
    \end{center}
    { \hspace*{\fill} \\}
    
    \begin{Verbatim}[commandchars=\\\{\}]
	 Parameter Case 	 2

    \end{Verbatim}

    \begin{center}
    \adjustimage{max size={0.9\linewidth}{0.9\paperheight}}{output_12_5.png}
    \end{center}
    { \hspace*{\fill} \\}
    
    \begin{center}
    \adjustimage{max size={0.9\linewidth}{0.9\paperheight}}{output_12_6.png}
    \end{center}
    { \hspace*{\fill} \\}
    
    \begin{center}
    \adjustimage{max size={0.9\linewidth}{0.9\paperheight}}{output_12_7.png}
    \end{center}
    { \hspace*{\fill} \\}
    
    \begin{Verbatim}[commandchars=\\\{\}]
	 Parameter Case 	 3

    \end{Verbatim}

    \begin{center}
    \adjustimage{max size={0.9\linewidth}{0.9\paperheight}}{output_12_9.png}
    \end{center}
    { \hspace*{\fill} \\}
    
    \begin{center}
    \adjustimage{max size={0.9\linewidth}{0.9\paperheight}}{output_12_10.png}
    \end{center}
    { \hspace*{\fill} \\}
    
    \begin{center}
    \adjustimage{max size={0.9\linewidth}{0.9\paperheight}}{output_12_11.png}
    \end{center}
    { \hspace*{\fill} \\}
    
    \hypertarget{results-of-altered-parameters}{%
\subsubsection{Results of Altered
Parameters}\label{results-of-altered-parameters}}

\begin{enumerate}
\def\labelenumi{\arabic{enumi}.}
\tightlist
\item
  Gives us a longer distance until first peak. If we cross the length of
  the tubelight, we see no emmision. The acceleration region of the
  phase space too has increased.
\item
  Lowers the intensity of the first peak. However, there are now more
  local maxima's in the intensity histogram.
\item
  Increases the overall intensity of light, while lowering those of
  minimae.
\end{enumerate}

    \hypertarget{conclusion}{%
\subsection{Conclusion}\label{conclusion}}

This week's assignment covers using python to simulate models for
various requirements. In this case,we utilsise it for simualting
electron motion in a tubelight, and hence find out the illumination at
different points. The existence of an initial peak, and those of dark
patches. In the subsequent sections, we also went over the effect of
changing various parameters including probability of collision,
threshold velocities and standard deviation.


    % Add a bibliography block to the postdoc
    
    
    
    \end{document}
