
% Default to the notebook output style

% Inherit from the specified cell style.
% * <varun19299@gmail.com> 2018-02-02T13:31:46.052Z:
%
% ^.

\documentclass[11pt]{article}

% Multicolumn Text
\usepackage{multicol}

\usepackage[T1]{fontenc}
% Nicer default font (+ math font) than Computer Modern for most use cases
\usepackage{mathpazo}

% Basic figure setup, for now with no caption control since it's done
% automatically by Pandoc (which extracts ![](path) syntax from Markdown).
\usepackage{graphicx}
% We will generate all images so they have a width \maxwidth. This means
% that they will get their normal width if they fit onto the page, but
% are scaled down if they would overflow the margins.
\makeatletter
\def\maxwidth{\ifdim\Gin@nat@width>\linewidth\linewidth
\else\Gin@nat@width\fi}
\makeatother
\let\Oldincludegraphics\includegraphics
% Set max figure width to be 80% of text width, for now hardcoded.
\renewcommand{\includegraphics}[1]{\Oldincludegraphics[width=.8\maxwidth]{#1}}
% Ensure that by default, figures have no caption (until we provide a
% proper Figure object with a Caption API and a way to capture that
% in the conversion process - todo).
\usepackage{caption}
\DeclareCaptionLabelFormat{nolabel}{}
\captionsetup{labelformat=nolabel}

\usepackage{adjustbox} % Used to constrain images to a maximum size
\usepackage{xcolor} % Allow colors to be defined
\usepackage{enumerate} % Needed for markdown enumerations to work
\usepackage{geometry} % Used to adjust the document margins
\usepackage{amsmath} % Equations
\usepackage{amssymb} % Equations
\usepackage{textcomp} % defines textquotesingle
% Hack from http://tex.stackexchange.com/a/47451/13684:
\AtBeginDocument{%
    \def\PYZsq{\textquotesingle}% Upright quotes in Pygmentized code
}
\usepackage{upquote} % Upright quotes for verbatim code
\usepackage{eurosym} % defines \euro
\usepackage[mathletters]{ucs} % Extended unicode (utf-8) support
\usepackage[utf8x]{inputenc} % Allow utf-8 characters in the tex document
\usepackage{fancyvrb} % verbatim replacement that allows latex
\usepackage{grffile} % extends the file name processing of package graphics
                     % to support a larger range
% The hyperref package gives us a pdf with properly built
% internal navigation ('pdf bookmarks' for the table of contents,
% internal cross-reference links, web links for URLs, etc.)
\usepackage{hyperref}
\usepackage{longtable} % longtable support required by pandoc >1.10
\usepackage{booktabs}  % table support for pandoc > 1.12.2
\usepackage[inline]{enumitem} % IRkernel/repr support (it uses the enumerate* environment)
\usepackage[normalem]{ulem} % ulem is needed to support strikethroughs (\sout)
                            % normalem makes italics be italics, not underlines




% Colors for the hyperref package
\definecolor{urlcolor}{rgb}{0,.145,.698}
\definecolor{linkcolor}{rgb}{.71,0.21,0.01}
\definecolor{citecolor}{rgb}{.12,.54,.11}

% ANSI colors
\definecolor{ansi-black}{HTML}{3E424D}
\definecolor{ansi-black-intense}{HTML}{282C36}
\definecolor{ansi-red}{HTML}{E75C58}
\definecolor{ansi-red-intense}{HTML}{B22B31}
\definecolor{ansi-green}{HTML}{00A250}
\definecolor{ansi-green-intense}{HTML}{007427}
\definecolor{ansi-yellow}{HTML}{DDB62B}
\definecolor{ansi-yellow-intense}{HTML}{B27D12}
\definecolor{ansi-blue}{HTML}{208FFB}
\definecolor{ansi-blue-intense}{HTML}{0065CA}
\definecolor{ansi-magenta}{HTML}{D160C4}
\definecolor{ansi-magenta-intense}{HTML}{A03196}
\definecolor{ansi-cyan}{HTML}{60C6C8}
\definecolor{ansi-cyan-intense}{HTML}{258F8F}
\definecolor{ansi-white}{HTML}{C5C1B4}
\definecolor{ansi-white-intense}{HTML}{A1A6B2}

% commands and environments needed by pandoc snippets
% extracted from the output of `pandoc -s`
\providecommand{\tightlist}{%
  \setlength{\itemsep}{0pt}\setlength{\parskip}{0pt}}
\DefineVerbatimEnvironment{Highlighting}{Verbatim}{commandchars=\\\{\}}
% Add ',fontsize=\small' for more characters per line
\newenvironment{Shaded}{}{}
\newcommand{\KeywordTok}[1]{\textcolor[rgb]{0.00,0.44,0.13}{\textbf{{#1}}}}
\newcommand{\DataTypeTok}[1]{\textcolor[rgb]{0.56,0.13,0.00}{{#1}}}
\newcommand{\DecValTok}[1]{\textcolor[rgb]{0.25,0.63,0.44}{{#1}}}
\newcommand{\BaseNTok}[1]{\textcolor[rgb]{0.25,0.63,0.44}{{#1}}}
\newcommand{\FloatTok}[1]{\textcolor[rgb]{0.25,0.63,0.44}{{#1}}}
\newcommand{\CharTok}[1]{\textcolor[rgb]{0.25,0.44,0.63}{{#1}}}
\newcommand{\StringTok}[1]{\textcolor[rgb]{0.25,0.44,0.63}{{#1}}}
\newcommand{\CommentTok}[1]{\textcolor[rgb]{0.38,0.63,0.69}{\textit{{#1}}}}
\newcommand{\OtherTok}[1]{\textcolor[rgb]{0.00,0.44,0.13}{{#1}}}
\newcommand{\AlertTok}[1]{\textcolor[rgb]{1.00,0.00,0.00}{\textbf{{#1}}}}
\newcommand{\FunctionTok}[1]{\textcolor[rgb]{0.02,0.16,0.49}{{#1}}}
\newcommand{\RegionMarkerTok}[1]{{#1}}
\newcommand{\ErrorTok}[1]{\textcolor[rgb]{1.00,0.00,0.00}{\textbf{{#1}}}}
\newcommand{\NormalTok}[1]{{#1}}

% Additional commands for more recent versions of Pandoc
\newcommand{\ConstantTok}[1]{\textcolor[rgb]{0.53,0.00,0.00}{{#1}}}
\newcommand{\SpecialCharTok}[1]{\textcolor[rgb]{0.25,0.44,0.63}{{#1}}}
\newcommand{\VerbatimStringTok}[1]{\textcolor[rgb]{0.25,0.44,0.63}{{#1}}}
\newcommand{\SpecialStringTok}[1]{\textcolor[rgb]{0.73,0.40,0.53}{{#1}}}
\newcommand{\ImportTok}[1]{{#1}}
\newcommand{\DocumentationTok}[1]{\textcolor[rgb]{0.73,0.13,0.13}{\textit{{#1}}}}
\newcommand{\AnnotationTok}[1]{\textcolor[rgb]{0.38,0.63,0.69}{\textbf{\textit{{#1}}}}}
\newcommand{\CommentVarTok}[1]{\textcolor[rgb]{0.38,0.63,0.69}{\textbf{\textit{{#1}}}}}
\newcommand{\VariableTok}[1]{\textcolor[rgb]{0.10,0.09,0.49}{{#1}}}
\newcommand{\ControlFlowTok}[1]{\textcolor[rgb]{0.00,0.44,0.13}{\textbf{{#1}}}}
\newcommand{\OperatorTok}[1]{\textcolor[rgb]{0.40,0.40,0.40}{{#1}}}
\newcommand{\BuiltInTok}[1]{{#1}}
\newcommand{\ExtensionTok}[1]{{#1}}
\newcommand{\PreprocessorTok}[1]{\textcolor[rgb]{0.74,0.48,0.00}{{#1}}}
\newcommand{\AttributeTok}[1]{\textcolor[rgb]{0.49,0.56,0.16}{{#1}}}
\newcommand{\InformationTok}[1]{\textcolor[rgb]{0.38,0.63,0.69}{\textbf{\textit{{#1}}}}}
\newcommand{\WarningTok}[1]{\textcolor[rgb]{0.38,0.63,0.69}{\textbf{\textit{{#1}}}}}


% Define a nice break command that doesn't care if a line doesn't already
% exist.
\def\br{\hspace*{\fill} \\* }
% Math Jax compatability definitions
\def\gt{>}
\def\lt{<}
% Document parameters
\title{Assignment\_2}




% Pygments definitions

\makeatletter
\def\PY@reset{\let\PY@it=\relax \let\PY@bf=\relax%
    \let\PY@ul=\relax \let\PY@tc=\relax%
    \let\PY@bc=\relax \let\PY@ff=\relax}
\def\PY@tok#1{\csname PY@tok@#1\endcsname}
\def\PY@toks#1+{\ifx\relax#1\empty\else%
    \PY@tok{#1}\expandafter\PY@toks\fi}
\def\PY@do#1{\PY@bc{\PY@tc{\PY@ul{%
    \PY@it{\PY@bf{\PY@ff{#1}}}}}}}
\def\PY#1#2{\PY@reset\PY@toks#1+\relax+\PY@do{#2}}

\expandafter\def\csname PY@tok@w\endcsname{\def\PY@tc##1{\textcolor[rgb]{0.73,0.73,0.73}{##1}}}
\expandafter\def\csname PY@tok@c\endcsname{\let\PY@it=\textit\def\PY@tc##1{\textcolor[rgb]{0.25,0.50,0.50}{##1}}}
\expandafter\def\csname PY@tok@cp\endcsname{\def\PY@tc##1{\textcolor[rgb]{0.74,0.48,0.00}{##1}}}
\expandafter\def\csname PY@tok@k\endcsname{\let\PY@bf=\textbf\def\PY@tc##1{\textcolor[rgb]{0.00,0.50,0.00}{##1}}}
\expandafter\def\csname PY@tok@kp\endcsname{\def\PY@tc##1{\textcolor[rgb]{0.00,0.50,0.00}{##1}}}
\expandafter\def\csname PY@tok@kt\endcsname{\def\PY@tc##1{\textcolor[rgb]{0.69,0.00,0.25}{##1}}}
\expandafter\def\csname PY@tok@o\endcsname{\def\PY@tc##1{\textcolor[rgb]{0.40,0.40,0.40}{##1}}}
\expandafter\def\csname PY@tok@ow\endcsname{\let\PY@bf=\textbf\def\PY@tc##1{\textcolor[rgb]{0.67,0.13,1.00}{##1}}}
\expandafter\def\csname PY@tok@nb\endcsname{\def\PY@tc##1{\textcolor[rgb]{0.00,0.50,0.00}{##1}}}
\expandafter\def\csname PY@tok@nf\endcsname{\def\PY@tc##1{\textcolor[rgb]{0.00,0.00,1.00}{##1}}}
\expandafter\def\csname PY@tok@nc\endcsname{\let\PY@bf=\textbf\def\PY@tc##1{\textcolor[rgb]{0.00,0.00,1.00}{##1}}}
\expandafter\def\csname PY@tok@nn\endcsname{\let\PY@bf=\textbf\def\PY@tc##1{\textcolor[rgb]{0.00,0.00,1.00}{##1}}}
\expandafter\def\csname PY@tok@ne\endcsname{\let\PY@bf=\textbf\def\PY@tc##1{\textcolor[rgb]{0.82,0.25,0.23}{##1}}}
\expandafter\def\csname PY@tok@nv\endcsname{\def\PY@tc##1{\textcolor[rgb]{0.10,0.09,0.49}{##1}}}
\expandafter\def\csname PY@tok@no\endcsname{\def\PY@tc##1{\textcolor[rgb]{0.53,0.00,0.00}{##1}}}
\expandafter\def\csname PY@tok@nl\endcsname{\def\PY@tc##1{\textcolor[rgb]{0.63,0.63,0.00}{##1}}}
\expandafter\def\csname PY@tok@ni\endcsname{\let\PY@bf=\textbf\def\PY@tc##1{\textcolor[rgb]{0.60,0.60,0.60}{##1}}}
\expandafter\def\csname PY@tok@na\endcsname{\def\PY@tc##1{\textcolor[rgb]{0.49,0.56,0.16}{##1}}}
\expandafter\def\csname PY@tok@nt\endcsname{\let\PY@bf=\textbf\def\PY@tc##1{\textcolor[rgb]{0.00,0.50,0.00}{##1}}}
\expandafter\def\csname PY@tok@nd\endcsname{\def\PY@tc##1{\textcolor[rgb]{0.67,0.13,1.00}{##1}}}
\expandafter\def\csname PY@tok@s\endcsname{\def\PY@tc##1{\textcolor[rgb]{0.73,0.13,0.13}{##1}}}
\expandafter\def\csname PY@tok@sd\endcsname{\let\PY@it=\textit\def\PY@tc##1{\textcolor[rgb]{0.73,0.13,0.13}{##1}}}
\expandafter\def\csname PY@tok@si\endcsname{\let\PY@bf=\textbf\def\PY@tc##1{\textcolor[rgb]{0.73,0.40,0.53}{##1}}}
\expandafter\def\csname PY@tok@se\endcsname{\let\PY@bf=\textbf\def\PY@tc##1{\textcolor[rgb]{0.73,0.40,0.13}{##1}}}
\expandafter\def\csname PY@tok@sr\endcsname{\def\PY@tc##1{\textcolor[rgb]{0.73,0.40,0.53}{##1}}}
\expandafter\def\csname PY@tok@ss\endcsname{\def\PY@tc##1{\textcolor[rgb]{0.10,0.09,0.49}{##1}}}
\expandafter\def\csname PY@tok@sx\endcsname{\def\PY@tc##1{\textcolor[rgb]{0.00,0.50,0.00}{##1}}}
\expandafter\def\csname PY@tok@m\endcsname{\def\PY@tc##1{\textcolor[rgb]{0.40,0.40,0.40}{##1}}}
\expandafter\def\csname PY@tok@gh\endcsname{\let\PY@bf=\textbf\def\PY@tc##1{\textcolor[rgb]{0.00,0.00,0.50}{##1}}}
\expandafter\def\csname PY@tok@gu\endcsname{\let\PY@bf=\textbf\def\PY@tc##1{\textcolor[rgb]{0.50,0.00,0.50}{##1}}}
\expandafter\def\csname PY@tok@gd\endcsname{\def\PY@tc##1{\textcolor[rgb]{0.63,0.00,0.00}{##1}}}
\expandafter\def\csname PY@tok@gi\endcsname{\def\PY@tc##1{\textcolor[rgb]{0.00,0.63,0.00}{##1}}}
\expandafter\def\csname PY@tok@gr\endcsname{\def\PY@tc##1{\textcolor[rgb]{1.00,0.00,0.00}{##1}}}
\expandafter\def\csname PY@tok@ge\endcsname{\let\PY@it=\textit}
\expandafter\def\csname PY@tok@gs\endcsname{\let\PY@bf=\textbf}
\expandafter\def\csname PY@tok@gp\endcsname{\let\PY@bf=\textbf\def\PY@tc##1{\textcolor[rgb]{0.00,0.00,0.50}{##1}}}
\expandafter\def\csname PY@tok@go\endcsname{\def\PY@tc##1{\textcolor[rgb]{0.53,0.53,0.53}{##1}}}
\expandafter\def\csname PY@tok@gt\endcsname{\def\PY@tc##1{\textcolor[rgb]{0.00,0.27,0.87}{##1}}}
\expandafter\def\csname PY@tok@err\endcsname{\def\PY@bc##1{\setlength{\fboxsep}{0pt}\fcolorbox[rgb]{1.00,0.00,0.00}{1,1,1}{\strut ##1}}}
\expandafter\def\csname PY@tok@kc\endcsname{\let\PY@bf=\textbf\def\PY@tc##1{\textcolor[rgb]{0.00,0.50,0.00}{##1}}}
\expandafter\def\csname PY@tok@kd\endcsname{\let\PY@bf=\textbf\def\PY@tc##1{\textcolor[rgb]{0.00,0.50,0.00}{##1}}}
\expandafter\def\csname PY@tok@kn\endcsname{\let\PY@bf=\textbf\def\PY@tc##1{\textcolor[rgb]{0.00,0.50,0.00}{##1}}}
\expandafter\def\csname PY@tok@kr\endcsname{\let\PY@bf=\textbf\def\PY@tc##1{\textcolor[rgb]{0.00,0.50,0.00}{##1}}}
\expandafter\def\csname PY@tok@bp\endcsname{\def\PY@tc##1{\textcolor[rgb]{0.00,0.50,0.00}{##1}}}
\expandafter\def\csname PY@tok@fm\endcsname{\def\PY@tc##1{\textcolor[rgb]{0.00,0.00,1.00}{##1}}}
\expandafter\def\csname PY@tok@vc\endcsname{\def\PY@tc##1{\textcolor[rgb]{0.10,0.09,0.49}{##1}}}
\expandafter\def\csname PY@tok@vg\endcsname{\def\PY@tc##1{\textcolor[rgb]{0.10,0.09,0.49}{##1}}}
\expandafter\def\csname PY@tok@vi\endcsname{\def\PY@tc##1{\textcolor[rgb]{0.10,0.09,0.49}{##1}}}
\expandafter\def\csname PY@tok@vm\endcsname{\def\PY@tc##1{\textcolor[rgb]{0.10,0.09,0.49}{##1}}}
\expandafter\def\csname PY@tok@sa\endcsname{\def\PY@tc##1{\textcolor[rgb]{0.73,0.13,0.13}{##1}}}
\expandafter\def\csname PY@tok@sb\endcsname{\def\PY@tc##1{\textcolor[rgb]{0.73,0.13,0.13}{##1}}}
\expandafter\def\csname PY@tok@sc\endcsname{\def\PY@tc##1{\textcolor[rgb]{0.73,0.13,0.13}{##1}}}
\expandafter\def\csname PY@tok@dl\endcsname{\def\PY@tc##1{\textcolor[rgb]{0.73,0.13,0.13}{##1}}}
\expandafter\def\csname PY@tok@s2\endcsname{\def\PY@tc##1{\textcolor[rgb]{0.73,0.13,0.13}{##1}}}
\expandafter\def\csname PY@tok@sh\endcsname{\def\PY@tc##1{\textcolor[rgb]{0.73,0.13,0.13}{##1}}}
\expandafter\def\csname PY@tok@s1\endcsname{\def\PY@tc##1{\textcolor[rgb]{0.73,0.13,0.13}{##1}}}
\expandafter\def\csname PY@tok@mb\endcsname{\def\PY@tc##1{\textcolor[rgb]{0.40,0.40,0.40}{##1}}}
\expandafter\def\csname PY@tok@mf\endcsname{\def\PY@tc##1{\textcolor[rgb]{0.40,0.40,0.40}{##1}}}
\expandafter\def\csname PY@tok@mh\endcsname{\def\PY@tc##1{\textcolor[rgb]{0.40,0.40,0.40}{##1}}}
\expandafter\def\csname PY@tok@mi\endcsname{\def\PY@tc##1{\textcolor[rgb]{0.40,0.40,0.40}{##1}}}
\expandafter\def\csname PY@tok@il\endcsname{\def\PY@tc##1{\textcolor[rgb]{0.40,0.40,0.40}{##1}}}
\expandafter\def\csname PY@tok@mo\endcsname{\def\PY@tc##1{\textcolor[rgb]{0.40,0.40,0.40}{##1}}}
\expandafter\def\csname PY@tok@ch\endcsname{\let\PY@it=\textit\def\PY@tc##1{\textcolor[rgb]{0.25,0.50,0.50}{##1}}}
\expandafter\def\csname PY@tok@cm\endcsname{\let\PY@it=\textit\def\PY@tc##1{\textcolor[rgb]{0.25,0.50,0.50}{##1}}}
\expandafter\def\csname PY@tok@cpf\endcsname{\let\PY@it=\textit\def\PY@tc##1{\textcolor[rgb]{0.25,0.50,0.50}{##1}}}
\expandafter\def\csname PY@tok@c1\endcsname{\let\PY@it=\textit\def\PY@tc##1{\textcolor[rgb]{0.25,0.50,0.50}{##1}}}
\expandafter\def\csname PY@tok@cs\endcsname{\let\PY@it=\textit\def\PY@tc##1{\textcolor[rgb]{0.25,0.50,0.50}{##1}}}

\def\PYZbs{\char`\\}
\def\PYZus{\char`\_}
\def\PYZob{\char`\{}
\def\PYZcb{\char`\}}
\def\PYZca{\char`\^}
\def\PYZam{\char`\&}
\def\PYZlt{\char`\<}
\def\PYZgt{\char`\>}
\def\PYZsh{\char`\#}
\def\PYZpc{\char`\%}
\def\PYZdl{\char`\$}
\def\PYZhy{\char`\-}
\def\PYZsq{\char`\'}
\def\PYZdq{\char`\"}
\def\PYZti{\char`\~}
% for compatibility with earlier versions
\def\PYZat{@}
\def\PYZlb{[}
\def\PYZrb{]}
\makeatother


% Exact colors from NB
\definecolor{incolor}{rgb}{0.0, 0.0, 0.5}
\definecolor{outcolor}{rgb}{0.545, 0.0, 0.0}




% Prevent overflowing lines due to hard-to-break entities
\sloppy
% Setup hyperref package
\hypersetup{
breaklinks=true,  % so long urls are correctly broken across lines
colorlinks=true,
urlcolor=urlcolor,
linkcolor=linkcolor,
citecolor=citecolor,
}
% Slightly bigger margins than the latex defaults

\geometry{verbose,tmargin=1in,bmargin=1in,lmargin=1in,rmargin=1in}



\begin{document}
\maketitle
\begin{center}
\emph{EE2703: Applied Programming} \\
\emph{Author: Varun Sundar, EE16B068} \\
\end{center}

\hypertarget{assignment-2}{%
\section{Introduction}\label{assignment-2}}
The assignment discusses various built in methods for integration and compares it to a common numerical technique, the trapezoidal rule. For the sake of this assignment, we consider the derivative $\arctan{x}$ function, $1/(1+t^{2})$.


\hypertarget{conventions}{%
\paragraph{Conventions}\label{conventions}}

\begin{enumerate}
\def\labelenumi{\arabic{enumi}.}
\tightlist
\item
  We are using Python 3, GCC for C
\item
  Underscore naming vs Camel Case
\item
  PEP 25 convention style.
\item
  Chosen to stick with importing \textit{matplotlib} and \textit{numpy} seperately, rather than use \textit{scipy's pylab}.
\end{enumerate}

\hypertarget{question-1}{%
\section{Question 1}\label{question-1}}

We define a vectorised version of our function under consideration.

\begin{Verbatim}[commandchars=\\\{\}]
{\color{incolor}In [{\color{incolor}4}]:} \PY{k+kn}{import} \PY{n+nn}{numpy} \PY{k}{as} \PY{n+nn}{np}
\end{Verbatim}

    \begin{Verbatim}[commandchars=\\\{\}]
{\color{incolor}In [{\color{incolor}5}]:} \PY{k}{def} \PY{n+nf}{f}\PY{p}{(}\PY{n}{t}\PY{p}{)}\PY{p}{:}
            \PY{c+c1}{\PYZsh{} t could be a vector}
            \PY{k}{return} \PY{l+m+mi}{1}\PY{o}{/}\PY{p}{(}\PY{l+m+mi}{1}\PY{o}{+}\PY{n}{np}\PY{o}{.}\PY{n}{square}\PY{p}{(}\PY{n}{t}\PY{p}{)}\PY{p}{)}
\end{Verbatim}


\hypertarget{question-2}{%
\section{Question 2}\label{question-2}}

Now, we assign a variable to a given evenly spaced array of values. Since 
\textit{numpy's} present implementation of \textit{linspace} contains no spacing argument, we choose to play nice with it. Rather calculate the requisite spacing as $\frac{Range}{Spacing} +1$.

    \begin{Verbatim}[commandchars=\\\{\}]
{\color{incolor}In [{\color{incolor}6}]:} \PY{c+c1}{\PYZsh{} Create a spacing of 0.1}
        \PY{n}{x}\PY{o}{=}\PY{n}{np}\PY{o}{.}\PY{n}{linspace}\PY{p}{(}\PY{l+m+mf}{0.0}\PY{p}{,} \PY{l+m+mf}{5.0}\PY{p}{,} \PY{n}{num}\PY{o}{=}\PY{l+m+mi}{51}\PY{p}{)}
\end{Verbatim}


    \hypertarget{question-3}{%
\section{Question 3}\label{question-3}}

We now plot our integrand. This has been done for a range of $0.0$ to $5.0$, uniformly sampled at $0.1$.

    \begin{Verbatim}[commandchars=\\\{\}]
{\color{incolor}In [{\color{incolor}7}]:} \PY{k+kn}{import} \PY{n+nn}{matplotlib}\PY{n+nn}{.}\PY{n+nn}{pyplot} \PY{k}{as} \PY{n+nn}{plt}
        \PY{o}{\PYZpc{}}\PY{k}{matplotlib} inline

        \PY{n}{plt}\PY{o}{.}\PY{n}{suptitle}\PY{p}{(}\PY{l+s+sa}{r}\PY{l+s+s1}{\PYZsq{}}\PY{l+s+s1}{Plot of \PYZdl{}1/(1+t\PYZca{}}\PY{l+s+si}{\PYZob{}2\PYZcb{}}\PY{l+s+s1}{)\PYZdl{}}\PY{l+s+s1}{\PYZsq{}}\PY{p}{,} \PY{n}{fontsize}\PY{o}{=}\PY{l+m+mi}{14}\PY{p}{,}\PY{n}{fontweight}\PY{o}{=}\PY{l+s+s2}{\PYZdq{}}\PY{l+s+s2}{bold}\PY{l+s+s2}{\PYZdq{}}\PY{p}{)}

        \PY{n}{plt}\PY{o}{.}\PY{n}{xlabel}\PY{p}{(}\PY{l+s+s1}{\PYZsq{}}\PY{l+s+s1}{\PYZdl{}x\PYZdl{}}\PY{l+s+s1}{\PYZsq{}}\PY{p}{,}\PY{n}{fontsize}\PY{o}{=}\PY{l+m+mi}{12}\PY{p}{)}
        \PY{n}{plt}\PY{o}{.}\PY{n}{ylabel}\PY{p}{(}\PY{l+s+s1}{\PYZsq{}}\PY{l+s+s1}{\PYZdl{}f(x)\PYZdl{}}\PY{l+s+s1}{\PYZsq{}}\PY{p}{,}\PY{n}{fontsize}\PY{o}{=}\PY{l+m+mi}{12}\PY{p}{)}

        \PY{n}{plt}\PY{o}{.}\PY{n}{plot}\PY{p}{(}\PY{n}{x}\PY{p}{,}\PY{n}{f}\PY{p}{(}\PY{n}{x}\PY{p}{)}\PY{p}{,}\PY{l+s+s1}{\PYZsq{}}\PY{l+s+s1}{ro}\PY{l+s+s1}{\PYZsq{}}\PY{p}{)}
        \PY{n}{plt}\PY{o}{.}\PY{n}{show}\PY{p}{(}\PY{p}{)}
\end{Verbatim}


    \begin{center}
    \adjustimage{max size={0.9\linewidth}{0.9\paperheight}}{Assignment_2_files/Assignment_2_7_0.png}
    \end{center}
    { \hspace*{\fill} \\}



%%%% Question 4
    \hypertarget{question-4}{%
\section{Question 4}\label{question-4}}

We now utilise \textit{scipy's} integration module for a quadratic approximated integration. Proceeding which we compare it to \textit{numpy's} arctan function.

A semilogy plot (log versus linear scale) has also been plotted depicting the error.

    \begin{Verbatim}[commandchars=\\\{\}]
{\color{incolor}In [{\color{incolor}8}]:} \PY{k+kn}{from} \PY{n+nn}{scipy}\PY{n+nn}{.}\PY{n+nn}{integrate} \PY{k}{import} \PY{n}{quad}
        quad\PY{o}{?}
\end{Verbatim}

Let's print out a few values of both our integrated output and the ground reality.

    \begin{Verbatim}[commandchars=\\\{\}]
{\color{incolor}In [{\color{incolor}102}]:} \PY{k+kn}{from} \PY{n+nn}{pprint} \PY{k}{import} \PY{n}{pprint}
          \PY{n}{integrated\PYZus{}arctan}\PY{o}{=}\PY{n}{np}\PY{o}{.}\PY{n}{array}\PY{p}{(}\PY{p}{[}\PY{n}{quad}\PY{p}{(}\PY{n}{f}\PY{p}{,}\PY{l+m+mi}{0}\PY{p}{,}\PY{n}{a}\PY{p}{)}\PY{p}{[}\PY{l+m+mi}{0}\PY{p}{]} \PY{k}{for} \PY{n}{a} \PY{o+ow}{in} \PY{n}{x}\PY{p}{]}\PY{p}{)}
          \PY{c+c1}{\PYZsh{} A Few values}
          \PY{n}{pprint}\PY{p}{(}\PY{n}{integrated\PYZus{}arctan}\PY{p}{[}\PY{p}{:}\PY{l+m+mi}{8}\PY{p}{]}\PY{p}{)}
\end{Verbatim}


    \begin{Verbatim}[commandchars=\\\{\}]
array([ 0.        ,  0.09966865,  0.19739556,  0.29145679,  0.38050638,
        0.46364761,  0.5404195 ,  0.61072596])

    \end{Verbatim}

    \begin{Verbatim}[commandchars=\\\{\}]
{\color{incolor}In [{\color{incolor}103}]:} \PY{n}{np\PYZus{}arctan}\PY{o}{=}\PY{n}{np}\PY{o}{.}\PY{n}{arctan}\PY{p}{(}\PY{n}{x}\PY{p}{)}
          \PY{c+c1}{\PYZsh{} A Few Values}
          \PY{n}{pprint}\PY{p}{(}\PY{n}{np\PYZus{}arctan}\PY{p}{[}\PY{p}{:}\PY{l+m+mi}{10}\PY{p}{]}\PY{p}{)}
\end{Verbatim}


    \begin{Verbatim}[commandchars=\\\{\}]
array([ 0.        ,  0.09966865,  0.19739556,  0.29145679,  0.38050638,
        0.46364761,  0.5404195 ,  0.61072596,  0.67474094,  0.7328151 ])

    \end{Verbatim}

    Using, \(\tan^{-1}{x}\) instead of \(\arctan{x}\)
\(\def\arctanAlt{\tan^{-1}}\)

    \begin{Verbatim}[commandchars=\\\{\}]
{\color{incolor}In [{\color{incolor}11}]:} \PY{n}{plt}\PY{o}{.}\PY{n}{plot}\PY{p}{(}\PY{n}{x}\PY{p}{,}\PY{n}{np\PYZus{}arctan}\PY{p}{,} \PY{l+s+s1}{\PYZsq{}}\PY{l+s+s1}{ro}\PY{l+s+s1}{\PYZsq{}}\PY{p}{,}\PY{n}{x}\PY{p}{,} \PY{n}{integrated\PYZus{}arctan}\PY{p}{,}\PY{l+s+s1}{\PYZsq{}}\PY{l+s+s1}{k\PYZhy{}}\PY{l+s+s1}{\PYZsq{}}\PY{p}{)}
         \PY{n}{plt}\PY{o}{.}\PY{n}{legend}\PY{p}{(}\PY{p}{(}\PY{l+s+s2}{\PYZdq{}}\PY{l+s+s2}{quad func}\PY{l+s+s2}{\PYZdq{}}\PY{p}{,}\PY{l+s+sa}{r}\PY{l+s+s2}{\PYZdq{}}\PY{l+s+s2}{\PYZdl{}}\PY{l+s+s2}{\PYZbs{}}\PY{l+s+s2}{tan\PYZca{}}\PY{l+s+s2}{\PYZob{}}\PY{l+s+s2}{\PYZhy{}1\PYZcb{}}\PY{l+s+si}{\PYZob{}x\PYZcb{}}\PY{l+s+s2}{\PYZdl{}}\PY{l+s+s2}{\PYZdq{}}\PY{p}{)}\PY{p}{)}
         \PY{n}{plt}\PY{o}{.}\PY{n}{xlabel}\PY{p}{(}\PY{l+s+s1}{\PYZsq{}}\PY{l+s+s1}{\PYZdl{}x\PYZdl{}}\PY{l+s+s1}{\PYZsq{}}\PY{p}{,}\PY{n}{fontsize}\PY{o}{=}\PY{l+m+mi}{12}\PY{p}{)}
         \PY{n}{plt}\PY{o}{.}\PY{n}{ylabel}\PY{p}{(}\PY{l+s+s1}{\PYZsq{}}\PY{l+s+s1}{\PYZdl{}}\PY{l+s+s1}{\PYZbs{}}\PY{l+s+s1}{int\PYZus{}}\PY{l+s+si}{\PYZob{}0\PYZcb{}}\PY{l+s+s1}{\PYZca{}}\PY{l+s+si}{\PYZob{}x\PYZcb{}}\PY{l+s+s1}{ dt}\PY{l+s+s1}{\PYZbs{}}\PY{l+s+s1}{,/(1+t\PYZca{}}\PY{l+s+si}{\PYZob{}2\PYZcb{}}\PY{l+s+s1}{)\PYZdl{}}\PY{l+s+s1}{\PYZsq{}}\PY{p}{,}\PY{n}{fontsize}\PY{o}{=}\PY{l+m+mi}{12}\PY{p}{)}
         \PY{n}{plt}\PY{o}{.}\PY{n}{show}\PY{p}{(}\PY{p}{)}
\end{Verbatim}


    \begin{center}
    \adjustimage{max size={0.9\linewidth}{0.9\paperheight}}{Assignment_2_files/Assignment_2_13_0.png}
    \end{center}
    { \hspace*{\fill} \\}

    \begin{Verbatim}[commandchars=\\\{\}]
{\color{incolor}In [{\color{incolor}12}]:} \PY{c+c1}{\PYZsh{} Set plot sizes}
         \PY{k+kn}{import} \PY{n+nn}{matplotlib}
         \PY{n}{matplotlib}\PY{o}{.}\PY{n}{rcParams}\PY{p}{[}\PY{l+s+s1}{\PYZsq{}}\PY{l+s+s1}{figure.figsize}\PY{l+s+s1}{\PYZsq{}}\PY{p}{]}\PY{o}{=}\PY{p}{[}\PY{l+m+mi}{8}\PY{p}{,}\PY{l+m+mi}{6}\PY{p}{]}
\end{Verbatim}


    \begin{Verbatim}[commandchars=\\\{\}]
{\color{incolor}In [{\color{incolor}13}]:} \PY{n}{plt}\PY{o}{.}\PY{n}{semilogy}\PY{p}{(}\PY{n}{x}\PY{p}{,}\PY{n}{np}\PY{o}{.}\PY{n}{abs}\PY{p}{(}\PY{n}{np\PYZus{}arctan} \PY{o}{\PYZhy{}} \PY{n}{integrated\PYZus{}arctan}\PY{p}{)}\PY{p}{,} \PY{l+s+s1}{\PYZsq{}}\PY{l+s+s1}{ro}\PY{l+s+s1}{\PYZsq{}}\PY{p}{)}
         \PY{n}{plt}\PY{o}{.}\PY{n}{xlabel}\PY{p}{(}\PY{l+s+s1}{\PYZsq{}}\PY{l+s+s1}{\PYZdl{}x\PYZdl{}}\PY{l+s+s1}{\PYZsq{}}\PY{p}{,}\PY{n}{fontsize}\PY{o}{=}\PY{l+m+mi}{12}\PY{p}{)}
         \PY{n}{plt}\PY{o}{.}\PY{n}{ylabel}\PY{p}{(}\PY{l+s+s2}{\PYZdq{}}\PY{l+s+s2}{Error}\PY{l+s+s2}{\PYZdq{}}\PY{p}{,}\PY{n}{fontsize}\PY{o}{=}\PY{l+m+mi}{12}\PY{p}{)}
         \PY{n}{plt}\PY{o}{.}\PY{n}{suptitle}\PY{p}{(}\PY{l+s+sa}{r}\PY{l+s+s1}{\PYZsq{}}\PY{l+s+s1}{Error in \PYZdl{}}\PY{l+s+s1}{\PYZbs{}}\PY{l+s+s1}{int\PYZus{}}\PY{l+s+si}{\PYZob{}0\PYZcb{}}\PY{l+s+s1}{\PYZca{}}\PY{l+s+si}{\PYZob{}x\PYZcb{}}\PY{l+s+s1}{dt}\PY{l+s+s1}{\PYZbs{}}\PY{l+s+s1}{,/(1+t\PYZca{}}\PY{l+s+si}{\PYZob{}2\PYZcb{}}\PY{l+s+s1}{)\PYZdl{}}\PY{l+s+s1}{\PYZsq{}}\PY{p}{,} \PY{n}{fontsize}\PY{o}{=}\PY{l+m+mi}{14}\PY{p}{,}\PY{n}{fontweight}\PY{o}{=}\PY{l+s+s2}{\PYZdq{}}\PY{l+s+s2}{bold}\PY{l+s+s2}{\PYZdq{}}\PY{p}{)}
         \PY{n}{plt}\PY{o}{.}\PY{n}{show}\PY{p}{(}\PY{p}{)}
\end{Verbatim}


    \begin{center}
    \adjustimage{max size={0.9\linewidth}{0.9\paperheight}}{Assignment_2_files/Assignment_2_15_0.png}
    \end{center}
    { \hspace*{\fill} \\}

    \hypertarget{question-5}{%
\section{Question 5}\label{question-5}}

We now perform a numerical method for integration; quadratic approximation via trapezoidal rule. It would be useful to compare this to the error obtained under \textit{scipy's} implementation and interpret the corresponding results.

Trapezoidal Rule:

$I_i=h * (\sum_{j=1}^{i-1} f(x_j)−2(f(x_1)+f(x_i)) $

for the $i$'th integral.

    \begin{Verbatim}[commandchars=\\\{\}]
{\color{incolor}In [{\color{incolor}28}]:} \PY{c+c1}{\PYZsh{} Define a single integral}
         \PY{k}{def} \PY{n+nf}{integral}\PY{p}{(}\PY{n}{f}\PY{p}{,}\PY{n}{x}\PY{p}{,}\PY{n}{i}\PY{p}{)}\PY{p}{:}
             \PY{c+c1}{\PYZsh{} f is the function}
             \PY{c+c1}{\PYZsh{} x is your array of values}
             \PY{c+c1}{\PYZsh{} i is the desired index (from 0 to n\PYZhy{}2)}
             \PY{c+c1}{\PYZsh{} h is the trapezoidal step}
             \PY{n}{h}\PY{o}{=}\PY{p}{(}\PY{n}{np}\PY{o}{.}\PY{n}{amax}\PY{p}{(}\PY{n}{x}\PY{p}{)}\PY{o}{\PYZhy{}}\PY{n}{np}\PY{o}{.}\PY{n}{amin}\PY{p}{(}\PY{n}{x}\PY{p}{)}\PY{p}{)}\PY{o}{/}\PY{n+nb}{len}\PY{p}{(}\PY{n}{x}\PY{p}{)}
             \PY{n}{j}\PY{o}{=}\PY{n}{i}\PY{o}{+}\PY{l+m+mi}{1}
             \PY{k}{return} \PY{n}{h}\PY{o}{*}\PY{p}{(}\PY{n}{np}\PY{o}{.}\PY{n}{sum}\PY{p}{(}\PY{n}{f}\PY{p}{(}\PY{n}{x}\PY{p}{[}\PY{p}{:}\PY{n}{j}\PY{p}{]}\PY{p}{)}\PY{p}{)}\PY{o}{\PYZhy{}}\PY{l+m+mi}{1}\PY{o}{/}\PY{l+m+mi}{2}\PY{o}{*}\PY{p}{(}\PY{n}{f}\PY{p}{(}\PY{n}{x}\PY{p}{[}\PY{l+m+mi}{0}\PY{p}{]}\PY{p}{)}\PY{o}{+}\PY{n}{f}\PY{p}{(}\PY{n}{x}\PY{p}{[}\PY{n}{j}\PY{p}{]}\PY{p}{)}\PY{p}{)}\PY{p}{)}

         \PY{n}{integral}\PY{p}{(}\PY{n}{f}\PY{p}{,}\PY{n}{x}\PY{p}{,}\PY{l+m+mi}{1}\PY{p}{)}
\end{Verbatim}


\begin{Verbatim}[commandchars=\\\{\}]
{\color{outcolor}Out[{\color{outcolor}28}]:} 0.098953899914878352
\end{Verbatim}

    A vectorised implementation \ldots{}

    \begin{Verbatim}[commandchars=\\\{\}]
{\color{incolor}In [{\color{incolor}48}]:} \PY{k}{def} \PY{n+nf}{integral\PYZus{}vectorised}\PY{p}{(}\PY{n}{f}\PY{p}{,}\PY{n}{x}\PY{p}{)}\PY{p}{:}
             \PY{c+c1}{\PYZsh{} f is the function}
             \PY{c+c1}{\PYZsh{} x is your array of values}
             \PY{c+c1}{\PYZsh{} i is the desired index (from 0 to n\PYZhy{}2)}
             \PY{c+c1}{\PYZsh{} h is the trapezoidal step}
             \PY{n}{h}\PY{o}{=}\PY{p}{(}\PY{n}{np}\PY{o}{.}\PY{n}{amax}\PY{p}{(}\PY{n}{x}\PY{p}{)}\PY{o}{\PYZhy{}}\PY{n}{np}\PY{o}{.}\PY{n}{amin}\PY{p}{(}\PY{n}{x}\PY{p}{)}\PY{p}{)}\PY{o}{/}\PY{p}{(}\PY{n+nb}{len}\PY{p}{(}\PY{n}{x}\PY{p}{)}\PY{o}{\PYZhy{}}\PY{l+m+mi}{1}\PY{p}{)}
             \PY{n}{n}\PY{o}{=}\PY{n+nb}{len}\PY{p}{(}\PY{n}{x}\PY{p}{)}
             \PY{k}{return} \PY{n}{np}\PY{o}{.}\PY{n}{array}\PY{p}{(}\PY{p}{[}\PY{n}{h}\PY{o}{*}\PY{p}{(}\PY{n}{np}\PY{o}{.}\PY{n}{cumsum}\PY{p}{(}\PY{n}{f}\PY{p}{(}\PY{n}{x}\PY{p}{)}\PY{p}{)}\PY{p}{[}\PY{n}{j}\PY{p}{]}\PY{o}{\PYZhy{}}\PY{l+m+mi}{1}\PY{o}{/}\PY{l+m+mi}{2}\PY{o}{*}\PY{p}{(}\PY{n}{f}\PY{p}{(}\PY{n}{x}\PY{p}{[}\PY{l+m+mi}{0}\PY{p}{]}\PY{p}{)}\PY{o}{+}\PY{n}{f}\PY{p}{(}\PY{n}{x}\PY{p}{[}\PY{n}{j}\PY{p}{]}\PY{p}{)}\PY{p}{)}\PY{p}{)} \PY{k}{for} \PY{n}{j} \PY{o+ow}{in} \PY{n+nb}{range}\PY{p}{(}\PY{n}{n}\PY{p}{)}\PY{p}{]}\PY{p}{)}

         \PY{n}{integral\PYZus{}vectorised}\PY{p}{(}\PY{n}{f}\PY{p}{,}\PY{n}{x}\PY{p}{)}
\end{Verbatim}


\begin{Verbatim}[commandchars=\\\{\}]
{\color{outcolor}Out[{\color{outcolor}48}]:} array([ 0.        ,  0.09950495,  0.19708682,  0.29103531,  0.38001031,
                 0.46311376,  0.53987847,  0.61020022,  0.67424507,  0.73235719,
                 0.7849815 ,  0.83260593,  0.87572217,  0.91480133,  0.95028059,
                 0.98255709,  1.01198665,  1.03888507,  1.06353099,  1.08616943,
                 1.10701542,  1.12625756,  1.14406135,  1.16057212,  1.17591769,
                 1.19021069,  1.20355055,  1.21602521,  1.22771268,  1.23868228,
                 1.24899578,  1.25870832,  1.26786925,  1.27652286,  1.28470897,
                 1.29246345,  1.29981869,  1.30680403,  1.31344605,  1.31976891,
                 1.3257946 ,  1.33154319,  1.337033  ,  1.34228082,  1.34730204,
                 1.35211077,  1.35672003,  1.36114179,  1.3653871 ,  1.36946616,
                 1.37338844])
\end{Verbatim}

    \begin{Verbatim}[commandchars=\\\{\}]
{\color{incolor}In [{\color{incolor}49}]:} \PY{n}{plt}\PY{o}{.}\PY{n}{plot}\PY{p}{(}\PY{n}{x}\PY{p}{,}\PY{n}{np\PYZus{}arctan}\PY{p}{,} \PY{l+s+s1}{\PYZsq{}}\PY{l+s+s1}{ro}\PY{l+s+s1}{\PYZsq{}}\PY{p}{,}\PY{n}{x}\PY{p}{,} \PY{n}{integrated\PYZus{}arctan}\PY{p}{,}\PY{l+s+s1}{\PYZsq{}}\PY{l+s+s1}{k\PYZhy{}}\PY{l+s+s1}{\PYZsq{}}\PY{p}{,}\PY{n}{x}\PY{p}{,}\PY{n}{integral\PYZus{}vectorised}\PY{p}{(}\PY{n}{f}\PY{p}{,}\PY{n}{x}\PY{p}{)}\PY{p}{,}\PY{l+s+s1}{\PYZsq{}}\PY{l+s+s1}{go}\PY{l+s+s1}{\PYZsq{}}\PY{p}{)}
         \PY{n}{plt}\PY{o}{.}\PY{n}{legend}\PY{p}{(}\PY{p}{(}\PY{l+s+s2}{\PYZdq{}}\PY{l+s+s2}{quad func}\PY{l+s+s2}{\PYZdq{}}\PY{p}{,}\PY{l+s+sa}{r}\PY{l+s+s2}{\PYZdq{}}\PY{l+s+s2}{\PYZdl{}}\PY{l+s+s2}{\PYZbs{}}\PY{l+s+s2}{tan\PYZca{}}\PY{l+s+s2}{\PYZob{}}\PY{l+s+s2}{\PYZhy{}1\PYZcb{}}\PY{l+s+si}{\PYZob{}x\PYZcb{}}\PY{l+s+s2}{\PYZdl{}}\PY{l+s+s2}{\PYZdq{}}\PY{p}{,}\PY{l+s+s2}{\PYZdq{}}\PY{l+s+s2}{Trapezoid rule}\PY{l+s+s2}{\PYZdq{}}\PY{p}{)}\PY{p}{)}
         \PY{n}{plt}\PY{o}{.}\PY{n}{xlabel}\PY{p}{(}\PY{l+s+s1}{\PYZsq{}}\PY{l+s+s1}{\PYZdl{}x\PYZdl{}}\PY{l+s+s1}{\PYZsq{}}\PY{p}{,}\PY{n}{fontsize}\PY{o}{=}\PY{l+m+mi}{12}\PY{p}{)}
         \PY{n}{plt}\PY{o}{.}\PY{n}{ylabel}\PY{p}{(}\PY{l+s+s1}{\PYZsq{}}\PY{l+s+s1}{\PYZdl{}}\PY{l+s+s1}{\PYZbs{}}\PY{l+s+s1}{int\PYZus{}}\PY{l+s+si}{\PYZob{}0\PYZcb{}}\PY{l+s+s1}{\PYZca{}}\PY{l+s+si}{\PYZob{}x\PYZcb{}}\PY{l+s+s1}{ dt}\PY{l+s+s1}{\PYZbs{}}\PY{l+s+s1}{,/(1+t\PYZca{}}\PY{l+s+si}{\PYZob{}2\PYZcb{}}\PY{l+s+s1}{)\PYZdl{}}\PY{l+s+s1}{\PYZsq{}}\PY{p}{,}\PY{n}{fontsize}\PY{o}{=}\PY{l+m+mi}{12}\PY{p}{)}
         \PY{n}{plt}\PY{o}{.}\PY{n}{show}\PY{p}{(}\PY{p}{)}
\end{Verbatim}


    \begin{center}
    \adjustimage{max size={0.9\linewidth}{0.9\paperheight}}{Assignment_2_files/Assignment_2_20_0.png}
    \end{center}
    { \hspace*{\fill} \\}

    Let's define a handy function for computing a cross error.

    \begin{Verbatim}[commandchars=\\\{\}]
{\color{incolor}In [{\color{incolor}50}]:} \PY{k}{def} \PY{n+nf}{error}\PY{p}{(}\PY{n}{u}\PY{p}{,}\PY{n}{v}\PY{p}{)}\PY{p}{:}
             \PY{c+c1}{\PYZsh{} Return maximum error between two numpy arrays}
             \PY{k}{return} \PY{n}{np}\PY{o}{.}\PY{n}{amax}\PY{p}{(}\PY{n}{np}\PY{o}{.}\PY{n}{abs}\PY{p}{(}\PY{n}{u}\PY{o}{\PYZhy{}}\PY{n}{v}\PY{p}{)}\PY{p}{)}
\end{Verbatim}


    And now use this to find out absolute errors.

    \begin{Verbatim}[commandchars=\\\{\}]
{\color{incolor}In [{\color{incolor}75}]:} \PY{n}{h\PYZus{}array}\PY{o}{=}\PY{p}{[}\PY{l+m+mi}{10}\PY{o}{*}\PY{o}{*}\PY{p}{(}\PY{o}{\PYZhy{}}\PY{n}{j}\PY{p}{)} \PY{k}{for} \PY{n}{j} \PY{o+ow}{in} \PY{n+nb}{range}\PY{p}{(}\PY{l+m+mi}{1}\PY{p}{,}\PY{l+m+mi}{5}\PY{p}{)}\PY{p}{]}
         \PY{n}{x\PYZus{}array}\PY{o}{=}\PY{p}{[}\PY{n}{np}\PY{o}{.}\PY{n}{linspace}\PY{p}{(}\PY{l+m+mf}{0.0}\PY{p}{,} \PY{l+m+mf}{5.0}\PY{p}{,} \PY{n}{num}\PY{o}{=}\PY{l+m+mi}{5}\PY{o}{/}\PY{p}{(}\PY{n}{h}\PY{p}{)}\PY{o}{+}\PY{l+m+mi}{1}\PY{p}{)} \PY{k}{for} \PY{n}{h} \PY{o+ow}{in} \PY{n}{h\PYZus{}array}\PY{p}{]}
         \PY{p}{[}\PY{n}{error}\PY{p}{(}\PY{n}{np}\PY{o}{.}\PY{n}{arctan}\PY{p}{(}\PY{n}{x}\PY{p}{)}\PY{p}{,}\PY{n}{integral\PYZus{}vectorised}\PY{p}{(}\PY{n}{f}\PY{p}{,}\PY{n}{x}\PY{p}{)}\PY{p}{)} \PY{k}{for} \PY{n}{x} \PY{o+ow}{in} \PY{n}{x\PYZus{}array} \PY{p}{]}
\end{Verbatim}


    \begin{Verbatim}[commandchars=\\\{\}]
/Users/Ankivarun/anaconda3/envs/tf\_python3/lib/python3.6/site-packages/ipykernel\_launcher.py:2: DeprecationWarning: object of type <class 'float'> cannot be safely interpreted as an integer.


    \end{Verbatim}

\begin{Verbatim}[commandchars=\\\{\}]
{\color{outcolor}Out[{\color{outcolor}75}]:} [0.00054103142650707703,
          5.4126137517540585e-06,
          5.4126577109236962e-08,
          5.4126414461563854e-10]
\end{Verbatim}

    We note that \(h<10^{-4}\) is gives us error tolerences
\(\delta < 10^{-8}\). So we ad-hoc start from \(h=10^{-2}\) and descend
to \(h=10^{-4}\); and note down requisite error.

    \begin{Verbatim}[commandchars=\\\{\}]
{\color{incolor}In [{\color{incolor}104}]:} \PY{c+c1}{\PYZsh{} A bottom threshold on the value of h}
          \PY{n}{h\PYZus{}sweet}\PY{o}{=}\PY{l+m+mi}{5}\PY{o}{*}\PY{l+m+mi}{10}\PY{o}{*}\PY{o}{*}\PY{p}{(}\PY{o}{\PYZhy{}}\PY{l+m+mi}{4}\PY{p}{)}
\end{Verbatim}


    \begin{Verbatim}[commandchars=\\\{\}]
{\color{incolor}In [{\color{incolor}83}]:} \PY{n}{i}\PY{o}{=}\PY{l+m+mi}{4}\PY{o}{*}\PY{l+m+mi}{10}\PY{o}{*}\PY{o}{*}\PY{p}{(}\PY{o}{\PYZhy{}}\PY{l+m+mi}{2}\PY{p}{)}
         \PY{n}{h\PYZus{}array}\PY{o}{=}\PY{p}{[}\PY{p}{]}
         \PY{k}{for} \PY{n}{j} \PY{o+ow}{in} \PY{n+nb}{range}\PY{p}{(}\PY{l+m+mi}{8}\PY{p}{)}\PY{p}{:}
             \PY{n}{h\PYZus{}array}\PY{o}{.}\PY{n}{append}\PY{p}{(}\PY{n}{i}\PY{o}{/}\PY{l+m+mi}{2}\PY{p}{)}
             \PY{n}{i}\PY{o}{=}\PY{n}{i}\PY{o}{/}\PY{l+m+mi}{2}
         \PY{n}{h\PYZus{}array}
\end{Verbatim}


\begin{Verbatim}[commandchars=\\\{\}]
{\color{outcolor}Out[{\color{outcolor}83}]:} [0.02, 0.01, 0.005, 0.0025, 0.00125, 0.000625, 0.0003125, 0.00015625]
\end{Verbatim}

    \begin{Verbatim}[commandchars=\\\{\}]
{\color{incolor}In [{\color{incolor}84}]:} \PY{n}{x\PYZus{}array}\PY{o}{=}\PY{p}{[}\PY{n}{np}\PY{o}{.}\PY{n}{linspace}\PY{p}{(}\PY{l+m+mf}{0.0}\PY{p}{,} \PY{l+m+mf}{5.0}\PY{p}{,} \PY{n}{num}\PY{o}{=}\PY{l+m+mi}{5}\PY{o}{/}\PY{p}{(}\PY{n}{h}\PY{p}{)}\PY{o}{+}\PY{l+m+mi}{1}\PY{p}{)} \PY{k}{for} \PY{n}{h} \PY{o+ow}{in} \PY{n}{h\PYZus{}array}\PY{p}{]}
         \PY{n}{error\PYZus{}exact\PYZus{}array}\PY{o}{=}\PY{p}{[}\PY{n}{error}\PY{p}{(}\PY{n}{np}\PY{o}{.}\PY{n}{arctan}\PY{p}{(}\PY{n}{x}\PY{p}{)}\PY{p}{,}\PY{n}{integral\PYZus{}vectorised}\PY{p}{(}\PY{n}{f}\PY{p}{,}\PY{n}{x}\PY{p}{)}\PY{p}{)} \PY{k}{for} \PY{n}{x} \PY{o+ow}{in} \PY{n}{x\PYZus{}array} \PY{p}{]}
         \PY{n}{error\PYZus{}exact\PYZus{}array}
\end{Verbatim}


    \begin{Verbatim}[commandchars=\\\{\}]
/Users/Ankivarun/anaconda3/envs/tf\_python3/lib/python3.6/site-packages/ipykernel\_launcher.py:1: DeprecationWarning: object of type <class 'float'> cannot be safely interpreted as an integer.
  """Entry point for launching an IPython kernel.

    \end{Verbatim}

\begin{Verbatim}[commandchars=\\\{\}]
{\color{outcolor}Out[{\color{outcolor}84}]:} [2.1650937672923476e-05,
          5.4126137517540585e-06,
          1.3531503878505546e-06,
          3.3829131440565874e-07,
          8.4572799208260108e-08,
          2.1143197970197036e-08,
          5.2857990207044736e-09,
          1.3214487282198206e-09]
\end{Verbatim}

    We define our \emph{estimated error} as follows:

For every integral with the trapezoidal width \(h\), consider one with
\(\frac{h}{2}\) to be the truth for the test.

To facilitate this, we define a function called \emph{cross-error} which
computes the corrresponding error. Obviously, we compared only sampled
values which are same.

    \begin{Verbatim}[commandchars=\\\{\}]
{\color{incolor}In [{\color{incolor}82}]:} \PY{k}{def} \PY{n+nf}{cross\PYZus{}error}\PY{p}{(}\PY{n}{h1}\PY{p}{,}\PY{n}{h2}\PY{p}{)}\PY{p}{:}
             \PY{c+c1}{\PYZsh{} Return between h1, h2}
             \PY{c+c1}{\PYZsh{} Assume h2 is h1/2}
             \PY{n}{h\PYZus{}array}\PY{o}{=}\PY{p}{[}\PY{n}{h1}\PY{p}{,}\PY{n}{h2}\PY{p}{]}
             \PY{n}{x1}\PY{p}{,}\PY{n}{x2}\PY{o}{=}\PY{p}{[}\PY{n}{np}\PY{o}{.}\PY{n}{linspace}\PY{p}{(}\PY{l+m+mf}{0.0}\PY{p}{,} \PY{l+m+mf}{5.0}\PY{p}{,} \PY{n}{num}\PY{o}{=}\PY{l+m+mi}{5}\PY{o}{/}\PY{p}{(}\PY{n}{h}\PY{p}{)}\PY{o}{+}\PY{l+m+mi}{1}\PY{p}{)} \PY{k}{for} \PY{n}{h} \PY{o+ow}{in} \PY{n}{h\PYZus{}array}\PY{p}{]}
             \PY{n}{i1}\PY{o}{=}\PY{n}{integral\PYZus{}vectorised}\PY{p}{(}\PY{n}{f}\PY{p}{,}\PY{n}{x1}\PY{p}{)}
             \PY{n}{i2}\PY{o}{=}\PY{n}{integral\PYZus{}vectorised}\PY{p}{(}\PY{n}{f}\PY{p}{,}\PY{n}{x2}\PY{p}{)}
             \PY{k}{return} \PY{n}{error}\PY{p}{(}\PY{n}{i1}\PY{p}{,}\PY{n}{i2}\PY{p}{[}\PY{p}{:}\PY{p}{:}\PY{l+m+mi}{2}\PY{p}{]}\PY{p}{)}

         \PY{n}{cross\PYZus{}error}\PY{p}{(}\PY{l+m+mf}{0.02}\PY{p}{,}\PY{l+m+mf}{0.01}\PY{p}{)}
\end{Verbatim}


    \begin{Verbatim}[commandchars=\\\{\}]
/Users/Ankivarun/anaconda3/envs/tf\_python3/lib/python3.6/site-packages/ipykernel\_launcher.py:5: DeprecationWarning: object of type <class 'float'> cannot be safely interpreted as an integer.
  """

    \end{Verbatim}

\begin{Verbatim}[commandchars=\\\{\}]
{\color{outcolor}Out[{\color{outcolor}82}]:} 1.6238323921169417e-05
\end{Verbatim}

    \begin{Verbatim}[commandchars=\\\{\}]
{\color{incolor}In [{\color{incolor}88}]:} \PY{n}{cross\PYZus{}tuples}\PY{o}{=}\PY{p}{[}\PY{p}{(}\PY{n}{h\PYZus{}array}\PY{p}{[}\PY{n}{i}\PY{p}{]}\PY{p}{,}\PY{n}{h\PYZus{}array}\PY{p}{[}\PY{n}{i}\PY{o}{+}\PY{l+m+mi}{1}\PY{p}{]}\PY{p}{)} \PY{k}{for} \PY{n}{i} \PY{o+ow}{in} \PY{n+nb}{range}\PY{p}{(}\PY{n+nb}{len}\PY{p}{(}\PY{n}{h\PYZus{}array}\PY{p}{)}\PY{o}{\PYZhy{}}\PY{l+m+mi}{1}\PY{p}{)}\PY{p}{]}
         \PY{n}{estimated\PYZus{}errors}\PY{o}{=}\PY{p}{[}\PY{n}{cross\PYZus{}error}\PY{p}{(}\PY{n}{h\PYZus{}tuple}\PY{p}{[}\PY{l+m+mi}{0}\PY{p}{]}\PY{p}{,}\PY{n}{h\PYZus{}tuple}\PY{p}{[}\PY{l+m+mi}{1}\PY{p}{]}\PY{p}{)} \PY{k}{for} \PY{n}{h\PYZus{}tuple} \PY{o+ow}{in} \PY{n}{cross\PYZus{}tuples}\PY{p}{]}
         \PY{k}{pass}
\end{Verbatim}


    \begin{Verbatim}[commandchars=\\\{\}]
/Users/Ankivarun/anaconda3/envs/tf\_python3/lib/python3.6/site-packages/ipykernel\_launcher.py:5: DeprecationWarning: object of type <class 'float'> cannot be safely interpreted as an integer.
  """

    \end{Verbatim}

    \begin{Verbatim}[commandchars=\\\{\}]
{\color{incolor}In [{\color{incolor}72}]:} \PY{n}{estimated\PYZus{}errors}
\end{Verbatim}


\begin{Verbatim}[commandchars=\\\{\}]
{\color{outcolor}Out[{\color{outcolor}72}]:} [1.6238323921169417e-05,
          4.0594678555327945e-06,
          1.0148632708650851e-06,
          2.5371851519739863e-07,
          6.3429601238063071e-08,
          1.5857398949492563e-08,
          3.9643504035069554e-09]
\end{Verbatim}

    \begin{Verbatim}[commandchars=\\\{\}]
{\color{incolor}In [{\color{incolor}74}]:} \PY{n}{error\PYZus{}exact\PYZus{}array}
\end{Verbatim}


\begin{Verbatim}[commandchars=\\\{\}]
{\color{outcolor}Out[{\color{outcolor}74}]:} [2.1650937672923476e-05,
          5.4126137517540585e-06,
          1.3531503878505546e-06,
          3.3829131440565874e-07,
          8.4572799208260108e-08,
          2.1143197970197036e-08,
          5.2857990207044736e-09,
          1.3214487282198206e-09]
\end{Verbatim}

    \begin{Verbatim}[commandchars=\\\{\}]
{\color{incolor}In [{\color{incolor}93}]:} \PY{c+c1}{\PYZsh{} Now to plot errors}
         \PY{n}{plt}\PY{o}{.}\PY{n}{loglog}\PY{p}{(}\PY{n}{h\PYZus{}array}\PY{p}{[}\PY{p}{:}\PY{l+m+mi}{7}\PY{p}{]}\PY{p}{,}\PY{n}{error\PYZus{}exact\PYZus{}array}\PY{p}{[}\PY{p}{:}\PY{l+m+mi}{7}\PY{p}{]}\PY{p}{,}\PY{l+s+s1}{\PYZsq{}}\PY{l+s+s1}{ro}\PY{l+s+s1}{\PYZsq{}}\PY{p}{,}\PY{n}{h\PYZus{}array}\PY{p}{[}\PY{p}{:}\PY{l+m+mi}{7}\PY{p}{]}\PY{p}{,}\PY{n}{estimated\PYZus{}errors}\PY{p}{[}\PY{p}{:}\PY{l+m+mi}{7}\PY{p}{]}\PY{p}{,}\PY{l+s+s1}{\PYZsq{}}\PY{l+s+s1}{+}\PY{l+s+s1}{\PYZsq{}}\PY{p}{)}
         \PY{n}{plt}\PY{o}{.}\PY{n}{xlabel}\PY{p}{(}\PY{l+s+s1}{\PYZsq{}}\PY{l+s+s1}{\PYZdl{}h\PYZdl{}}\PY{l+s+s1}{\PYZsq{}}\PY{p}{,}\PY{n}{fontsize}\PY{o}{=}\PY{l+m+mi}{12}\PY{p}{)}
         \PY{n}{plt}\PY{o}{.}\PY{n}{ylabel}\PY{p}{(}\PY{l+s+s2}{\PYZdq{}}\PY{l+s+s2}{Error}\PY{l+s+s2}{\PYZdq{}}\PY{p}{,}\PY{n}{fontsize}\PY{o}{=}\PY{l+m+mi}{12}\PY{p}{)}
         \PY{n}{plt}\PY{o}{.}\PY{n}{legend}\PY{p}{(}\PY{p}{(}\PY{l+s+s2}{\PYZdq{}}\PY{l+s+s2}{Exact Error}\PY{l+s+s2}{\PYZdq{}}\PY{p}{,}\PY{l+s+sa}{r}\PY{l+s+s2}{\PYZdq{}}\PY{l+s+s2}{Estimated Error}\PY{l+s+s2}{\PYZdq{}}\PY{p}{)}\PY{p}{)}
         \PY{n}{plt}\PY{o}{.}\PY{n}{suptitle}\PY{p}{(}\PY{l+s+sa}{r}\PY{l+s+s1}{\PYZsq{}}\PY{l+s+s1}{Error in \PYZdl{}}\PY{l+s+s1}{\PYZbs{}}\PY{l+s+s1}{int\PYZus{}}\PY{l+s+si}{\PYZob{}0\PYZcb{}}\PY{l+s+s1}{\PYZca{}}\PY{l+s+si}{\PYZob{}x\PYZcb{}}\PY{l+s+s1}{dt}\PY{l+s+s1}{\PYZbs{}}\PY{l+s+s1}{,/(1+t\PYZca{}}\PY{l+s+si}{\PYZob{}2\PYZcb{}}\PY{l+s+s1}{)\PYZdl{} measured}\PY{l+s+s1}{\PYZsq{}} \PY{p}{,} \PY{n}{fontsize}\PY{o}{=}\PY{l+m+mi}{14}\PY{p}{,}\PY{n}{fontweight}\PY{o}{=}\PY{l+s+s2}{\PYZdq{}}\PY{l+s+s2}{bold}\PY{l+s+s2}{\PYZdq{}}\PY{p}{)}
         \PY{n}{plt}\PY{o}{.}\PY{n}{show}\PY{p}{(}\PY{p}{)}
\end{Verbatim}


\begin{center}
\adjustimage{max size={0.9\linewidth}{0.9\paperheight}}{Assignment_2_files/Assignment_2_34_0.png}
\end{center}
{ \hspace*{\fill} \\}

\hypertarget{result}{%
\section{Results and Discussion}\label{res-disc}}

We have implemented a numerical method for integration : a quadratic approximation via the trapezoidal method. We also have tried out different spacing values to attain a given tolerance in error. \\ We note, under our given error tolerances, that the numerical approach converges to the ground truth for the function, $\arctan{x}$.

\end{document}
