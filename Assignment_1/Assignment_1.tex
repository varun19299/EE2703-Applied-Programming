
% Default to the notebook output style
% Inherit from the specified cell style.
\documentclass[11pt]{article}

\usepackage[T1]{fontenc}
% Nicer default font (+ math font) than Computer Modern for most use cases
\usepackage{mathpazo}

% Basic figure setup, for now with no caption control since it's done
% automatically by Pandoc (which extracts ![](path) syntax from Markdown).
\usepackage{graphicx}
% We will generate all images so they have a width \maxwidth. This means
% that they will get their normal width if they fit onto the page, but
% are scaled down if they would overflow the margins.
\makeatletter
\def\maxwidth{\ifdim\Gin@nat@width>\linewidth\linewidth
\else\Gin@nat@width\fi}
\makeatother
\let\Oldincludegraphics\includegraphics
% Set max figure width to be 80% of text width, for now hardcoded.
\renewcommand{\includegraphics}[1]{\Oldincludegraphics[width=.8\maxwidth]{#1}}
% Ensure that by default, figures have no caption (until we provide a
% proper Figure object with a Caption API and a way to capture that
% in the conversion process - todo).
\usepackage{caption}
\DeclareCaptionLabelFormat{nolabel}{}
\captionsetup{labelformat=nolabel}

\usepackage{adjustbox} % Used to constrain images to a maximum size
\usepackage{xcolor} % Allow colors to be defined
\usepackage{enumerate} % Needed for markdown enumerations to work
\usepackage{geometry} % Used to adjust the document margins
\usepackage{amsmath} % Equations
\usepackage{amssymb} % Equations
\usepackage{textcomp} % defines textquotesingle
% Hack from http://tex.stackexchange.com/a/47451/13684:
\AtBeginDocument{%
\def\PYZsq{\textquotesingle}% Upright quotes in Pygmentized code
}
\usepackage{upquote} % Upright quotes for verbatim code
\usepackage{eurosym} % defines \euro
\usepackage[mathletters]{ucs} % Extended unicode (utf-8) support
\usepackage[utf8x]{inputenc} % Allow utf-8 characters in the tex document
\usepackage{fancyvrb} % verbatim replacement that allows latex
\usepackage{grffile} % extends the file name processing of package graphics
% to support a larger range
% The hyperref package gives us a pdf with properly built
% internal navigation ('pdf bookmarks' for the table of contents,
% internal cross-reference links, web links for URLs, etc.)
\usepackage{hyperref}
\usepackage{longtable} % longtable support required by pandoc >1.10
\usepackage{booktabs}  % table support for pandoc > 1.12.2
\usepackage[inline]{enumitem} % IRkernel/repr support (it uses the enumerate* environment)
\usepackage[normalem]{ulem} % ulem is needed to support strikethroughs (\sout)
% normalem makes italics be italics, not underlines

% Set section counting to start from 0

\setcounter{section}{-1}


% Colors for the hyperref package
\definecolor{urlcolor}{rgb}{0,.145,.698}
\definecolor{linkcolor}{rgb}{.71,0.21,0.01}
\definecolor{citecolor}{rgb}{.12,.54,.11}

% ANSI colors
\definecolor{ansi-black}{HTML}{3E424D}
\definecolor{ansi-black-intense}{HTML}{282C36}
\definecolor{ansi-red}{HTML}{E75C58}
\definecolor{ansi-red-intense}{HTML}{B22B31}
\definecolor{ansi-green}{HTML}{00A250}
\definecolor{ansi-green-intense}{HTML}{007427}
\definecolor{ansi-yellow}{HTML}{DDB62B}
\definecolor{ansi-yellow-intense}{HTML}{B27D12}
\definecolor{ansi-blue}{HTML}{208FFB}
\definecolor{ansi-blue-intense}{HTML}{0065CA}
\definecolor{ansi-magenta}{HTML}{D160C4}
\definecolor{ansi-magenta-intense}{HTML}{A03196}
\definecolor{ansi-cyan}{HTML}{60C6C8}
\definecolor{ansi-cyan-intense}{HTML}{258F8F}
\definecolor{ansi-white}{HTML}{C5C1B4}
\definecolor{ansi-white-intense}{HTML}{A1A6B2}

% commands and environments needed by pandoc snippets
% extracted from the output of `pandoc -s`
\providecommand{\tightlist}{%
\setlength{\itemsep}{0pt}\setlength{\parskip}{0pt}}
\DefineVerbatimEnvironment{Highlighting}{Verbatim}{commandchars=\\\{\}}
% Add ',fontsize=\small' for more characters per line
\newenvironment{Shaded}{}{}
\newcommand{\KeywordTok}[1]{\textcolor[rgb]{0.00,0.44,0.13}{\textbf{{#1}}}}
\newcommand{\DataTypeTok}[1]{\textcolor[rgb]{0.56,0.13,0.00}{{#1}}}
\newcommand{\DecValTok}[1]{\textcolor[rgb]{0.25,0.63,0.44}{{#1}}}
\newcommand{\BaseNTok}[1]{\textcolor[rgb]{0.25,0.63,0.44}{{#1}}}
\newcommand{\FloatTok}[1]{\textcolor[rgb]{0.25,0.63,0.44}{{#1}}}
\newcommand{\CharTok}[1]{\textcolor[rgb]{0.25,0.44,0.63}{{#1}}}
\newcommand{\StringTok}[1]{\textcolor[rgb]{0.25,0.44,0.63}{{#1}}}
\newcommand{\CommentTok}[1]{\textcolor[rgb]{0.38,0.63,0.69}{\textit{{#1}}}}
\newcommand{\OtherTok}[1]{\textcolor[rgb]{0.00,0.44,0.13}{{#1}}}
\newcommand{\AlertTok}[1]{\textcolor[rgb]{1.00,0.00,0.00}{\textbf{{#1}}}}
\newcommand{\FunctionTok}[1]{\textcolor[rgb]{0.02,0.16,0.49}{{#1}}}
\newcommand{\RegionMarkerTok}[1]{{#1}}
\newcommand{\ErrorTok}[1]{\textcolor[rgb]{1.00,0.00,0.00}{\textbf{{#1}}}}
\newcommand{\NormalTok}[1]{{#1}}

% Additional commands for more recent versions of Pandoc
\newcommand{\ConstantTok}[1]{\textcolor[rgb]{0.53,0.00,0.00}{{#1}}}
\newcommand{\SpecialCharTok}[1]{\textcolor[rgb]{0.25,0.44,0.63}{{#1}}}
\newcommand{\VerbatimStringTok}[1]{\textcolor[rgb]{0.25,0.44,0.63}{{#1}}}
\newcommand{\SpecialStringTok}[1]{\textcolor[rgb]{0.73,0.40,0.53}{{#1}}}
\newcommand{\ImportTok}[1]{{#1}}
\newcommand{\DocumentationTok}[1]{\textcolor[rgb]{0.73,0.13,0.13}{\textit{{#1}}}}
\newcommand{\AnnotationTok}[1]{\textcolor[rgb]{0.38,0.63,0.69}{\textbf{\textit{{#1}}}}}
\newcommand{\CommentVarTok}[1]{\textcolor[rgb]{0.38,0.63,0.69}{\textbf{\textit{{#1}}}}}
\newcommand{\VariableTok}[1]{\textcolor[rgb]{0.10,0.09,0.49}{{#1}}}
\newcommand{\ControlFlowTok}[1]{\textcolor[rgb]{0.00,0.44,0.13}{\textbf{{#1}}}}
\newcommand{\OperatorTok}[1]{\textcolor[rgb]{0.40,0.40,0.40}{{#1}}}
\newcommand{\BuiltInTok}[1]{{#1}}
\newcommand{\ExtensionTok}[1]{{#1}}
\newcommand{\PreprocessorTok}[1]{\textcolor[rgb]{0.74,0.48,0.00}{{#1}}}
\newcommand{\AttributeTok}[1]{\textcolor[rgb]{0.49,0.56,0.16}{{#1}}}
\newcommand{\InformationTok}[1]{\textcolor[rgb]{0.38,0.63,0.69}{\textbf{\textit{{#1}}}}}
\newcommand{\WarningTok}[1]{\textcolor[rgb]{0.38,0.63,0.69}{\textbf{\textit{{#1}}}}}


% Define a nice break command that doesn't care if a line doesn't already
% exist.
\def\br{\hspace*{\fill} \\* }
% Math Jax compatability definitions
\def\gt{>}
\def\lt{<}
% Document parameters
\title{Assignment 1}
\author{Varun Sundar, EE16B068}


% Pygments definitions

\makeatletter
\def\PY@reset{\let\PY@it=\relax \let\PY@bf=\relax%
    \let\PY@ul=\relax \let\PY@tc=\relax%
    \let\PY@bc=\relax \let\PY@ff=\relax}
\def\PY@tok#1{\csname PY@tok@#1\endcsname}
\def\PY@toks#1+{\ifx\relax#1\empty\else%
    \PY@tok{#1}\expandafter\PY@toks\fi}
\def\PY@do#1{\PY@bc{\PY@tc{\PY@ul{%
    \PY@it{\PY@bf{\PY@ff{#1}}}}}}}
\def\PY#1#2{\PY@reset\PY@toks#1+\relax+\PY@do{#2}}

\expandafter\def\csname PY@tok@w\endcsname{\def\PY@tc##1{\textcolor[rgb]{0.73,0.73,0.73}{##1}}}
\expandafter\def\csname PY@tok@c\endcsname{\let\PY@it=\textit\def\PY@tc##1{\textcolor[rgb]{0.25,0.50,0.50}{##1}}}
\expandafter\def\csname PY@tok@cp\endcsname{\def\PY@tc##1{\textcolor[rgb]{0.74,0.48,0.00}{##1}}}
\expandafter\def\csname PY@tok@k\endcsname{\let\PY@bf=\textbf\def\PY@tc##1{\textcolor[rgb]{0.00,0.50,0.00}{##1}}}
\expandafter\def\csname PY@tok@kp\endcsname{\def\PY@tc##1{\textcolor[rgb]{0.00,0.50,0.00}{##1}}}
\expandafter\def\csname PY@tok@kt\endcsname{\def\PY@tc##1{\textcolor[rgb]{0.69,0.00,0.25}{##1}}}
\expandafter\def\csname PY@tok@o\endcsname{\def\PY@tc##1{\textcolor[rgb]{0.40,0.40,0.40}{##1}}}
\expandafter\def\csname PY@tok@ow\endcsname{\let\PY@bf=\textbf\def\PY@tc##1{\textcolor[rgb]{0.67,0.13,1.00}{##1}}}
\expandafter\def\csname PY@tok@nb\endcsname{\def\PY@tc##1{\textcolor[rgb]{0.00,0.50,0.00}{##1}}}
\expandafter\def\csname PY@tok@nf\endcsname{\def\PY@tc##1{\textcolor[rgb]{0.00,0.00,1.00}{##1}}}
\expandafter\def\csname PY@tok@nc\endcsname{\let\PY@bf=\textbf\def\PY@tc##1{\textcolor[rgb]{0.00,0.00,1.00}{##1}}}
\expandafter\def\csname PY@tok@nn\endcsname{\let\PY@bf=\textbf\def\PY@tc##1{\textcolor[rgb]{0.00,0.00,1.00}{##1}}}
\expandafter\def\csname PY@tok@ne\endcsname{\let\PY@bf=\textbf\def\PY@tc##1{\textcolor[rgb]{0.82,0.25,0.23}{##1}}}
\expandafter\def\csname PY@tok@nv\endcsname{\def\PY@tc##1{\textcolor[rgb]{0.10,0.09,0.49}{##1}}}
\expandafter\def\csname PY@tok@no\endcsname{\def\PY@tc##1{\textcolor[rgb]{0.53,0.00,0.00}{##1}}}
\expandafter\def\csname PY@tok@nl\endcsname{\def\PY@tc##1{\textcolor[rgb]{0.63,0.63,0.00}{##1}}}
\expandafter\def\csname PY@tok@ni\endcsname{\let\PY@bf=\textbf\def\PY@tc##1{\textcolor[rgb]{0.60,0.60,0.60}{##1}}}
\expandafter\def\csname PY@tok@na\endcsname{\def\PY@tc##1{\textcolor[rgb]{0.49,0.56,0.16}{##1}}}
\expandafter\def\csname PY@tok@nt\endcsname{\let\PY@bf=\textbf\def\PY@tc##1{\textcolor[rgb]{0.00,0.50,0.00}{##1}}}
\expandafter\def\csname PY@tok@nd\endcsname{\def\PY@tc##1{\textcolor[rgb]{0.67,0.13,1.00}{##1}}}
\expandafter\def\csname PY@tok@s\endcsname{\def\PY@tc##1{\textcolor[rgb]{0.73,0.13,0.13}{##1}}}
\expandafter\def\csname PY@tok@sd\endcsname{\let\PY@it=\textit\def\PY@tc##1{\textcolor[rgb]{0.73,0.13,0.13}{##1}}}
\expandafter\def\csname PY@tok@si\endcsname{\let\PY@bf=\textbf\def\PY@tc##1{\textcolor[rgb]{0.73,0.40,0.53}{##1}}}
\expandafter\def\csname PY@tok@se\endcsname{\let\PY@bf=\textbf\def\PY@tc##1{\textcolor[rgb]{0.73,0.40,0.13}{##1}}}
\expandafter\def\csname PY@tok@sr\endcsname{\def\PY@tc##1{\textcolor[rgb]{0.73,0.40,0.53}{##1}}}
\expandafter\def\csname PY@tok@ss\endcsname{\def\PY@tc##1{\textcolor[rgb]{0.10,0.09,0.49}{##1}}}
\expandafter\def\csname PY@tok@sx\endcsname{\def\PY@tc##1{\textcolor[rgb]{0.00,0.50,0.00}{##1}}}
\expandafter\def\csname PY@tok@m\endcsname{\def\PY@tc##1{\textcolor[rgb]{0.40,0.40,0.40}{##1}}}
\expandafter\def\csname PY@tok@gh\endcsname{\let\PY@bf=\textbf\def\PY@tc##1{\textcolor[rgb]{0.00,0.00,0.50}{##1}}}
\expandafter\def\csname PY@tok@gu\endcsname{\let\PY@bf=\textbf\def\PY@tc##1{\textcolor[rgb]{0.50,0.00,0.50}{##1}}}
\expandafter\def\csname PY@tok@gd\endcsname{\def\PY@tc##1{\textcolor[rgb]{0.63,0.00,0.00}{##1}}}
\expandafter\def\csname PY@tok@gi\endcsname{\def\PY@tc##1{\textcolor[rgb]{0.00,0.63,0.00}{##1}}}
\expandafter\def\csname PY@tok@gr\endcsname{\def\PY@tc##1{\textcolor[rgb]{1.00,0.00,0.00}{##1}}}
\expandafter\def\csname PY@tok@ge\endcsname{\let\PY@it=\textit}
\expandafter\def\csname PY@tok@gs\endcsname{\let\PY@bf=\textbf}
\expandafter\def\csname PY@tok@gp\endcsname{\let\PY@bf=\textbf\def\PY@tc##1{\textcolor[rgb]{0.00,0.00,0.50}{##1}}}
\expandafter\def\csname PY@tok@go\endcsname{\def\PY@tc##1{\textcolor[rgb]{0.53,0.53,0.53}{##1}}}
\expandafter\def\csname PY@tok@gt\endcsname{\def\PY@tc##1{\textcolor[rgb]{0.00,0.27,0.87}{##1}}}
\expandafter\def\csname PY@tok@err\endcsname{\def\PY@bc##1{\setlength{\fboxsep}{0pt}\fcolorbox[rgb]{1.00,0.00,0.00}{1,1,1}{\strut ##1}}}
\expandafter\def\csname PY@tok@kc\endcsname{\let\PY@bf=\textbf\def\PY@tc##1{\textcolor[rgb]{0.00,0.50,0.00}{##1}}}
\expandafter\def\csname PY@tok@kd\endcsname{\let\PY@bf=\textbf\def\PY@tc##1{\textcolor[rgb]{0.00,0.50,0.00}{##1}}}
\expandafter\def\csname PY@tok@kn\endcsname{\let\PY@bf=\textbf\def\PY@tc##1{\textcolor[rgb]{0.00,0.50,0.00}{##1}}}
\expandafter\def\csname PY@tok@kr\endcsname{\let\PY@bf=\textbf\def\PY@tc##1{\textcolor[rgb]{0.00,0.50,0.00}{##1}}}
\expandafter\def\csname PY@tok@bp\endcsname{\def\PY@tc##1{\textcolor[rgb]{0.00,0.50,0.00}{##1}}}
\expandafter\def\csname PY@tok@fm\endcsname{\def\PY@tc##1{\textcolor[rgb]{0.00,0.00,1.00}{##1}}}
\expandafter\def\csname PY@tok@vc\endcsname{\def\PY@tc##1{\textcolor[rgb]{0.10,0.09,0.49}{##1}}}
\expandafter\def\csname PY@tok@vg\endcsname{\def\PY@tc##1{\textcolor[rgb]{0.10,0.09,0.49}{##1}}}
\expandafter\def\csname PY@tok@vi\endcsname{\def\PY@tc##1{\textcolor[rgb]{0.10,0.09,0.49}{##1}}}
\expandafter\def\csname PY@tok@vm\endcsname{\def\PY@tc##1{\textcolor[rgb]{0.10,0.09,0.49}{##1}}}
\expandafter\def\csname PY@tok@sa\endcsname{\def\PY@tc##1{\textcolor[rgb]{0.73,0.13,0.13}{##1}}}
\expandafter\def\csname PY@tok@sb\endcsname{\def\PY@tc##1{\textcolor[rgb]{0.73,0.13,0.13}{##1}}}
\expandafter\def\csname PY@tok@sc\endcsname{\def\PY@tc##1{\textcolor[rgb]{0.73,0.13,0.13}{##1}}}
\expandafter\def\csname PY@tok@dl\endcsname{\def\PY@tc##1{\textcolor[rgb]{0.73,0.13,0.13}{##1}}}
\expandafter\def\csname PY@tok@s2\endcsname{\def\PY@tc##1{\textcolor[rgb]{0.73,0.13,0.13}{##1}}}
\expandafter\def\csname PY@tok@sh\endcsname{\def\PY@tc##1{\textcolor[rgb]{0.73,0.13,0.13}{##1}}}
\expandafter\def\csname PY@tok@s1\endcsname{\def\PY@tc##1{\textcolor[rgb]{0.73,0.13,0.13}{##1}}}
\expandafter\def\csname PY@tok@mb\endcsname{\def\PY@tc##1{\textcolor[rgb]{0.40,0.40,0.40}{##1}}}
\expandafter\def\csname PY@tok@mf\endcsname{\def\PY@tc##1{\textcolor[rgb]{0.40,0.40,0.40}{##1}}}
\expandafter\def\csname PY@tok@mh\endcsname{\def\PY@tc##1{\textcolor[rgb]{0.40,0.40,0.40}{##1}}}
\expandafter\def\csname PY@tok@mi\endcsname{\def\PY@tc##1{\textcolor[rgb]{0.40,0.40,0.40}{##1}}}
\expandafter\def\csname PY@tok@il\endcsname{\def\PY@tc##1{\textcolor[rgb]{0.40,0.40,0.40}{##1}}}
\expandafter\def\csname PY@tok@mo\endcsname{\def\PY@tc##1{\textcolor[rgb]{0.40,0.40,0.40}{##1}}}
\expandafter\def\csname PY@tok@ch\endcsname{\let\PY@it=\textit\def\PY@tc##1{\textcolor[rgb]{0.25,0.50,0.50}{##1}}}
\expandafter\def\csname PY@tok@cm\endcsname{\let\PY@it=\textit\def\PY@tc##1{\textcolor[rgb]{0.25,0.50,0.50}{##1}}}
\expandafter\def\csname PY@tok@cpf\endcsname{\let\PY@it=\textit\def\PY@tc##1{\textcolor[rgb]{0.25,0.50,0.50}{##1}}}
\expandafter\def\csname PY@tok@c1\endcsname{\let\PY@it=\textit\def\PY@tc##1{\textcolor[rgb]{0.25,0.50,0.50}{##1}}}
\expandafter\def\csname PY@tok@cs\endcsname{\let\PY@it=\textit\def\PY@tc##1{\textcolor[rgb]{0.25,0.50,0.50}{##1}}}

\def\PYZbs{\char`\\}
\def\PYZus{\char`\_}
\def\PYZob{\char`\{}
\def\PYZcb{\char`\}}
\def\PYZca{\char`\^}
\def\PYZam{\char`\&}
\def\PYZlt{\char`\<}
\def\PYZgt{\char`\>}
\def\PYZsh{\char`\#}
\def\PYZpc{\char`\%}
\def\PYZdl{\char`\$}
\def\PYZhy{\char`\-}
\def\PYZsq{\char`\'}
\def\PYZdq{\char`\"}
\def\PYZti{\char`\~}
% for compatibility with earlier versions
\def\PYZat{@}
\def\PYZlb{[}
\def\PYZrb{]}
\makeatother


% Exact colors from NB
\definecolor{incolor}{rgb}{0.0, 0.0, 0.5}
\definecolor{outcolor}{rgb}{0.545, 0.0, 0.0}




% Prevent overflowing lines due to hard-to-break entities
\sloppy
% Setup hyperref package
\hypersetup{
breaklinks=true,  % so long urls are correctly broken across lines
colorlinks=true,
urlcolor=urlcolor,
linkcolor=linkcolor,
citecolor=citecolor,
}
% Slightly bigger margins than the latex defaults

\geometry{verbose,tmargin=1in,bmargin=1in,lmargin=1in,rmargin=1in}


\begin{document}
\maketitle
\begin{center}
\emph{EE2703: Applied Programming}
\end{center}

\hypertarget{introduction}{%
\section{Introduction}\label{introduction}}

\hypertarget{conventions}{%
\paragraph{Conventions}\label{conventions}}

\begin{enumerate}
\def\labelenumi{\arabic{enumi}.}
\tightlist
\item
  We are using Python 3, GCC for C
\item
  Underscore naming vs Camel Case
\item
  PEP 25 style of coding (non-tensorflow)
\item
  All c code is executed in seperate c files, which may be found under
  respective folders
\end{enumerate}

\section{Question 1}\label{question-1}

Pythonic Code (Brute, Optimised):

    \begin{Verbatim}[commandchars=\\\{\}]
{\color{incolor}In [{\color{incolor}6}]:} \PY{o}{\PYZpc{}}\PY{k}{time}
        \PY{c+c1}{\PYZsh{} Brute}
        \PY{k+kn}{from} \PY{n+nn}{pprint} \PY{k}{import} \PY{n}{pprint}
        \PY{n}{n}\PY{o}{=}\PY{l+m+mi}{1}
        \PY{n}{n\PYZus{}old}\PY{o}{=}\PY{l+m+mi}{1}
        \PY{n}{L}\PY{o}{=}\PY{p}{[}\PY{p}{]}
        \PY{n}{L}\PY{o}{.}\PY{n}{append}\PY{p}{(}\PY{l+m+mi}{1}\PY{p}{)}
        \PY{n}{L}\PY{o}{.}\PY{n}{append}\PY{p}{(}\PY{l+m+mi}{2}\PY{p}{)}
        \PY{k}{for} \PY{n}{k} \PY{o+ow}{in} \PY{n+nb}{range}\PY{p}{(}\PY{l+m+mi}{3}\PY{p}{,}\PY{l+m+mi}{11}\PY{p}{)}\PY{p}{:}
            \PY{l+s+sd}{\PYZsq{}\PYZsq{}\PYZsq{}new=n+n\PYZus{}old}
        \PY{l+s+sd}{    n\PYZus{}old=n}
        \PY{l+s+sd}{    n=new\PYZsq{}\PYZsq{}\PYZsq{}}
            \PY{n}{n}\PY{p}{,}\PY{n}{n\PYZus{}old}\PY{o}{=}\PY{n}{n}\PY{o}{+}\PY{n}{n\PYZus{}old}\PY{p}{,}\PY{n}{n}
            \PY{n}{L}\PY{o}{.}\PY{n}{append}\PY{p}{(}\PY{n}{n}\PY{p}{)}
        \PY{n}{pprint}\PY{p}{(}\PY{n+nb}{dict}\PY{p}{(}\PY{n+nb}{zip}\PY{p}{(}\PY{n+nb}{range}\PY{p}{(}\PY{l+m+mi}{1}\PY{p}{,}\PY{l+m+mi}{11}\PY{p}{)}\PY{p}{,}\PY{n}{L}\PY{p}{[}\PY{p}{:}\PY{l+m+mi}{11}\PY{p}{]}\PY{p}{)}\PY{p}{)}\PY{p}{)}
\end{Verbatim}


    \begin{Verbatim}[commandchars=\\\{\}]
CPU times: user 2 µs, sys: 0 ns, total: 2 µs
Wall time: 5.72 µs
\{1: 1, 2: 2, 3: 2, 4: 3, 5: 5, 6: 8, 7: 13, 8: 21, 9: 34, 10: 55\}

    \end{Verbatim}

    \begin{Verbatim}[commandchars=\\\{\}]
{\color{incolor}In [{\color{incolor}9}]:} \PY{c+c1}{\PYZsh{} Optimised}
        \PY{o}{\PYZpc{}}\PY{k}{time}
        \PY{n}{series}\PY{o}{=}\PY{p}{[}\PY{p}{]}
        \PY{n}{series}\PY{o}{.}\PY{n}{append}\PY{p}{(}\PY{l+m+mi}{1}\PY{p}{)}
        \PY{n}{series}\PY{o}{.}\PY{n}{append}\PY{p}{(}\PY{l+m+mi}{1}\PY{p}{)}
        \PY{p}{[}\PY{n}{series}\PY{o}{.}\PY{n}{append}\PY{p}{(}\PY{n}{series}\PY{p}{[}\PY{n}{k}\PY{o}{\PYZhy{}}\PY{l+m+mi}{1}\PY{p}{]}\PY{o}{+}\PY{n}{series}\PY{p}{[}\PY{n}{k}\PY{o}{\PYZhy{}}\PY{l+m+mi}{2}\PY{p}{]}\PY{p}{)} \PY{k}{for} \PY{n}{k} \PY{o+ow}{in} \PY{n+nb}{range}\PY{p}{(}\PY{l+m+mi}{2}\PY{p}{,}\PY{l+m+mi}{100000}\PY{p}{)}\PY{p}{]}
        \PY{n}{series}\PY{o}{=}\PY{n+nb}{dict}\PY{p}{(}\PY{n+nb}{zip}\PY{p}{(}\PY{n+nb}{range}\PY{p}{(}\PY{l+m+mi}{1}\PY{p}{,}\PY{l+m+mi}{11}\PY{p}{)}\PY{p}{,}\PY{n}{series}\PY{p}{)}\PY{p}{)}
        \PY{n+nb}{print}\PY{p}{(}\PY{n}{series}\PY{p}{)}
\end{Verbatim}


    \begin{Verbatim}[commandchars=\\\{\}]
CPU times: user 4 µs, sys: 1 µs, total: 5 µs
Wall time: 9.06 µs
\{1: 1, 2: 1, 3: 2, 4: 3, 5: 5, 6: 8, 7: 13, 8: 21, 9: 34, 10: 55\}

    \end{Verbatim}

    \begin{Verbatim}[commandchars=\\\{\}]
{\color{incolor}In [{\color{incolor}4}]:} \PYZpc{}\PYZpc{}bash
        \PY{c+c1}{\PYZsh{} Fibonacci Series}
        gcc \PYZhy{}o c\PYZus{}1 c\PYZus{}1.c
        ./c\PYZus{}1
\end{Verbatim}


    \begin{Verbatim}[commandchars=\\\{\}]
1, 1
2, 1
3, 2
4, 3
5, 5
6, 8
7, 13
8, 21
9, 34
10, 55

    \end{Verbatim}

\section{Question 2}\label{question-2}

Pythonic And C methods

    \begin{Verbatim}[commandchars=\\\{\}]
{\color{incolor}In [{\color{incolor}8}]:} \PY{k+kn}{import} \PY{n+nn}{math}
        \PY{k+kn}{import} \PY{n+nn}{math}
        \PY{n}{pi}\PY{o}{=}\PY{n}{math}\PY{o}{.}\PY{n}{pi}
        \PY{n}{modf}\PY{o}{=}\PY{n}{math}\PY{o}{.}\PY{n}{modf}
        \PY{n}{n}\PY{o}{=}\PY{p}{[}\PY{l+m+mi}{0}\PY{p}{]}
        \PY{n}{n}\PY{p}{[}\PY{l+m+mi}{0}\PY{p}{]}\PY{o}{=}\PY{l+m+mf}{0.2}
        \PY{n}{alpha}\PY{o}{=}\PY{n}{pi}
        \PY{k}{for} \PY{n}{k} \PY{o+ow}{in} \PY{n+nb}{range}\PY{p}{(}\PY{l+m+mi}{1}\PY{p}{,}\PY{l+m+mi}{999}\PY{p}{)}\PY{p}{:}
            \PY{n}{n}\PY{o}{.}\PY{n}{append}\PY{p}{(}\PY{n}{modf}\PY{p}{(}\PY{p}{(}\PY{n}{n}\PY{p}{[}\PY{n}{k}\PY{o}{\PYZhy{}}\PY{l+m+mi}{1}\PY{p}{]}\PY{o}{+}\PY{n}{pi}\PY{p}{)}\PY{o}{*}\PY{l+m+mi}{100}\PY{p}{)}\PY{p}{[}\PY{l+m+mi}{0}\PY{p}{]}\PY{p}{)}

        \PY{c+c1}{\PYZsh{} Lets print only around 30 values to keep our report clean }
        \PY{n+nb}{print} \PY{p}{(}\PY{l+s+s2}{\PYZdq{}}\PY{l+s+s2}{Lenght of resultant series...}\PY{l+s+s2}{\PYZdq{}}\PY{p}{,}\PY{n+nb}{len}\PY{p}{(}\PY{n}{n}\PY{p}{)}\PY{p}{)}
        \PY{n}{pprint} \PY{p}{(}\PY{n}{n} \PY{p}{[}\PY{p}{:}\PY{l+m+mi}{30}\PY{p}{]}\PY{p}{)}
\end{Verbatim}


    \begin{Verbatim}[commandchars=\\\{\}]
Lenght of resultant series{\ldots} 999
[0.2,
 0.1592653589793258,
 0.08580125691190688,
 0.7393910501700134,
 0.09837037598066445,
 0.9963029570457707,
 0.7895610635563912,
 0.11537171461844764,
 0.6964368208240899,
 0.8029474413883122,
 0.4540094978105458,
 0.5602151400339039,
 0.18077936236971937,
 0.23720159595126233,
 0.8794249541055592,
 0.10176076953524671,
 0.3353423125039967,
 0.6934966093789967,
 0.5089262968789967,
 0.05189504687899671,
 0.3487700468789967,
 0.03627004687899671,
 0.7862700468789967,
 0.7862700468789967,
 0.7862700468789967,
 0.7862700468789967,
 0.7862700468789967,
 0.7862700468789967,
 0.7862700468789967,
 0.7862700468789967]

    \end{Verbatim}

    \begin{Verbatim}[commandchars=\\\{\}]
{\color{incolor}In [{\color{incolor}19}]:} \PY{o}{\PYZpc{}\PYZpc{}}\PY{k}{timeit}
         \PYZsh{} Optimised
         n=[0.2]
         [n.append(modf((n[k\PYZhy{}1]+pi)*100)[0]) for k in range(1,999)]
\end{Verbatim}


    \begin{Verbatim}[commandchars=\\\{\}]
308 µs ± 8.56 µs per loop (mean ± std. dev. of 7 runs, 1000 loops each)

    \end{Verbatim}

    \textbf{Now for the C implementation.}

\emph{Note that the corresponding file (here c\_2) maybe found under the
same directory.}

    \begin{Verbatim}[commandchars=\\\{\}]
{\color{incolor}In [{\color{incolor}22}]:} \PYZpc{}\PYZpc{}bash
         \PY{c+c1}{\PYZsh{} Random Series}
         gcc \PYZhy{}o c\PYZus{}2 c\PYZus{}2.c
         ./c\PYZus{}2
\end{Verbatim}


    \begin{Verbatim}[commandchars=\\\{\}]
0, 0.200000
1, 0.159265
2, 0.085801
3, 0.739391
4, 0.098370
5, 0.996303
6, 0.789561
7, 0.115372
8, 0.696437
9, 0.802947
10, 0.454009
11, 0.560215
12, 0.180779
13, 0.237202
14, 0.879425
15, 0.101761
16, 0.335342
17, 0.693497
18, 0.508926
19, 0.051895
20, 0.348770
21, 0.036270
22, 0.786270
23, 0.786270
24, 0.786270
25, 0.786270
26, 0.786270
27, 0.786270
28, 0.786270
29, 0.786270
30, 0.786270

    \end{Verbatim}

    \begin{Verbatim}[commandchars=\\\{\}]
c\_2.c:3:9: warning: 'M\_PI' macro redefined [-Wmacro-redefined]
\#define M\_PI acos(-1.0)
        \^{}
/Applications/Xcode.app/Contents/Developer/Platforms/MacOSX.platform/Developer/SDKs/MacOSX10.13.sdk/usr/include/math.h:703:9: note: previous definition is here
\#define M\_PI        3.14159265358979323846264338327950288   /* pi             */
        \^{}
1 warning generated.

    \end{Verbatim}

\section{Question 3}\label{question-3}

Comes under File IO. Evaluated code is Python.

File Name: the-hound-of-the-baskervilles.txt (Same Directory)

    \begin{Verbatim}[commandchars=\\\{\}]
{\color{incolor}In [{\color{incolor}10}]:} \PY{k+kn}{import} \PY{n+nn}{re}
         \PY{k+kn}{import} \PY{n+nn}{string}

         \PY{n}{filename}\PY{o}{=}\PY{l+s+s2}{\PYZdq{}}\PY{l+s+s2}{the\PYZhy{}hound\PYZhy{}of\PYZhy{}the\PYZhy{}baskervilles.txt}\PY{l+s+s2}{\PYZdq{}}
         \PY{k}{with} \PY{n+nb}{open}\PY{p}{(}\PY{n}{filename}\PY{p}{,}\PY{l+s+s2}{\PYZdq{}}\PY{l+s+s2}{r}\PY{l+s+s2}{\PYZdq{}}\PY{p}{)} \PY{k}{as} \PY{n}{f}\PY{p}{:}
             \PY{n}{contents}\PY{o}{=}\PY{n}{f}\PY{o}{.}\PY{n}{read}\PY{p}{(}\PY{p}{)}

         \PY{n}{regex} \PY{o}{=} \PY{n}{re}\PY{o}{.}\PY{n}{compile}\PY{p}{(}\PY{l+s+s1}{\PYZsq{}}\PY{l+s+s1}{[}\PY{l+s+si}{\PYZpc{}s}\PY{l+s+s1}{]}\PY{l+s+s1}{\PYZsq{}} \PY{o}{\PYZpc{}} \PY{n}{re}\PY{o}{.}\PY{n}{escape}\PY{p}{(}\PY{n}{string}\PY{o}{.}\PY{n}{punctuation}\PY{p}{)}\PY{p}{)}
         \PY{n}{contents}\PY{o}{=}\PY{n}{regex}\PY{o}{.}\PY{n}{sub}\PY{p}{(}\PY{l+s+s1}{\PYZsq{}}\PY{l+s+s1}{\PYZsq{}}\PY{p}{,} \PY{n}{contents}\PY{p}{)}

         \PY{n}{words}\PY{o}{=}\PY{n}{contents}\PY{o}{.}\PY{n}{split}\PY{p}{(}\PY{p}{)}
         \PY{n}{d}\PY{o}{=}\PY{p}{\PYZob{}}\PY{p}{\PYZcb{}} \PY{c+c1}{\PYZsh{} creates an empty dictionary}

         \PY{c+c1}{\PYZsh{} Update a word}
         \PY{k}{def} \PY{n+nf}{do}\PY{p}{(}\PY{n}{w}\PY{p}{)}\PY{p}{:}
             \PY{k}{global} \PY{n}{d}
             \PY{k}{if} \PY{n}{w} \PY{o+ow}{in} \PY{n}{d}\PY{p}{:} \PY{c+c1}{\PYZsh{} this also works, and is quite fast}
                 \PY{n}{d}\PY{p}{[}\PY{n}{w}\PY{p}{]}\PY{o}{+}\PY{o}{=}\PY{l+m+mi}{1}
             \PY{k}{else}\PY{p}{:}
                 \PY{n}{d}\PY{p}{[}\PY{n}{w}\PY{p}{]}\PY{o}{=}\PY{l+m+mi}{1}

         \PY{c+c1}{\PYZsh{} Because List comprehensions!}
         \PY{p}{[}\PY{n}{do}\PY{p}{(}\PY{n}{w}\PY{p}{)} \PY{k}{for} \PY{n}{w} \PY{o+ow}{in} \PY{n}{words}\PY{p}{]}
         \PY{n+nb}{print} \PY{p}{(}\PY{l+s+s2}{\PYZdq{}}\PY{l+s+s2}{We}\PY{l+s+s2}{\PYZsq{}}\PY{l+s+s2}{ve got our words}\PY{l+s+s2}{\PYZdq{}}\PY{p}{)}
\end{Verbatim}


    \begin{Verbatim}[commandchars=\\\{\}]
We've got our words

    \end{Verbatim}

    \begin{Verbatim}[commandchars=\\\{\}]
{\color{incolor}In [{\color{incolor}12}]:} \PY{c+c1}{\PYZsh{} Now lets display the count}
         \PY{c+c1}{\PYZsh{} Again lets keep our report clean so,display only top 40 most used words}
         \PY{p}{[}\PY{n+nb}{print}\PY{p}{(}\PY{n}{k}\PY{p}{,}\PY{l+s+s2}{\PYZdq{}}\PY{l+s+s2}{...}\PY{l+s+se}{\PYZbs{}t}\PY{l+s+s2}{\PYZdq{}}\PY{p}{,}\PY{n}{d}\PY{p}{[}\PY{n}{k}\PY{p}{]}\PY{p}{)} \PY{k}{for} \PY{n}{k} \PY{o+ow}{in} \PY{n+nb}{sorted}\PY{p}{(}\PY{n}{d}\PY{p}{,}\PY{n}{key}\PY{o}{=}\PY{n}{d}\PY{o}{.}\PY{n}{get}\PY{p}{,}\PY{n}{reverse}\PY{o}{=}\PY{k+kc}{True}\PY{p}{)}\PY{p}{[}\PY{p}{:}\PY{l+m+mi}{40}\PY{p}{]}\PY{p}{]}
         \PY{n+nb}{print}\PY{p}{(}\PY{p}{)}
\end{Verbatim}


    \begin{Verbatim}[commandchars=\\\{\}]
the {\ldots}	 3128
of {\ldots}	 1642
and {\ldots}	 1542
I {\ldots}	 1465
to {\ldots}	 1432
a {\ldots}	 1254
that {\ldots}	 1082
in {\ldots}	 881
was {\ldots}	 792
it {\ldots}	 784
you {\ldots}	 752
he {\ldots}	 718
his {\ldots}	 659
is {\ldots}	 621
have {\ldots}	 529
had {\ldots}	 501
with {\ldots}	 427
for {\ldots}	 420
my {\ldots}	 420
which {\ldots}	 419
as {\ldots}	 383
not {\ldots}	 365
we {\ldots}	 350
at {\ldots}	 345
be {\ldots}	 341
this {\ldots}	 332
me {\ldots}	 324
upon {\ldots}	 314
him {\ldots}	 313
from {\ldots}	 280
but {\ldots}	 262
The {\ldots}	 258
Sir {\ldots}	 248
said {\ldots}	 241
one {\ldots}	 227
on {\ldots}	 222
been {\ldots}	 221
so {\ldots}	 216
by {\ldots}	 215
all {\ldots}	 210


    \end{Verbatim}

    End of Assignment.


    % Add a bibliography block to the postdoc



    \end{document}
