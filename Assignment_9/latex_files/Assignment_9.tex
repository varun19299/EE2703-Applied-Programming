
% Default to the notebook output style

    


% Inherit from the specified cell style.




    
\documentclass[11pt]{article}

    
    
    \usepackage[T1]{fontenc}
    % Nicer default font (+ math font) than Computer Modern for most use cases
    \usepackage{mathpazo}

    % Basic figure setup, for now with no caption control since it's done
    % automatically by Pandoc (which extracts ![](path) syntax from Markdown).
    \usepackage{graphicx}
    % We will generate all images so they have a width \maxwidth. This means
    % that they will get their normal width if they fit onto the page, but
    % are scaled down if they would overflow the margins.
    \makeatletter
    \def\maxwidth{\ifdim\Gin@nat@width>\linewidth\linewidth
    \else\Gin@nat@width\fi}
    \makeatother
    \let\Oldincludegraphics\includegraphics
    % Set max figure width to be 80% of text width, for now hardcoded.
    \renewcommand{\includegraphics}[1]{\Oldincludegraphics[width=.8\maxwidth]{#1}}
    % Ensure that by default, figures have no caption (until we provide a
    % proper Figure object with a Caption API and a way to capture that
    % in the conversion process - todo).
    \usepackage{caption}
    \DeclareCaptionLabelFormat{nolabel}{}
    \captionsetup{labelformat=nolabel}

    \usepackage{adjustbox} % Used to constrain images to a maximum size 
    \usepackage{xcolor} % Allow colors to be defined
    \usepackage{enumerate} % Needed for markdown enumerations to work
    \usepackage{geometry} % Used to adjust the document margins
    \usepackage{amsmath} % Equations
    \usepackage{amssymb} % Equations
    \usepackage{textcomp} % defines textquotesingle
    % Hack from http://tex.stackexchange.com/a/47451/13684:
    \AtBeginDocument{%
        \def\PYZsq{\textquotesingle}% Upright quotes in Pygmentized code
    }
    \usepackage{upquote} % Upright quotes for verbatim code
    \usepackage{eurosym} % defines \euro
    \usepackage[mathletters]{ucs} % Extended unicode (utf-8) support
    \usepackage[utf8x]{inputenc} % Allow utf-8 characters in the tex document
    \usepackage{fancyvrb} % verbatim replacement that allows latex
    \usepackage{grffile} % extends the file name processing of package graphics 
                         % to support a larger range 
    % The hyperref package gives us a pdf with properly built
    % internal navigation ('pdf bookmarks' for the table of contents,
    % internal cross-reference links, web links for URLs, etc.)
    \usepackage{hyperref}
    \usepackage{longtable} % longtable support required by pandoc >1.10
    \usepackage{booktabs}  % table support for pandoc > 1.12.2
    \usepackage[inline]{enumitem} % IRkernel/repr support (it uses the enumerate* environment)
    \usepackage[normalem]{ulem} % ulem is needed to support strikethroughs (\sout)
                                % normalem makes italics be italics, not underlines
    

    
    
    % Colors for the hyperref package
    \definecolor{urlcolor}{rgb}{0,.145,.698}
    \definecolor{linkcolor}{rgb}{.71,0.21,0.01}
    \definecolor{citecolor}{rgb}{.12,.54,.11}

    % ANSI colors
    \definecolor{ansi-black}{HTML}{3E424D}
    \definecolor{ansi-black-intense}{HTML}{282C36}
    \definecolor{ansi-red}{HTML}{E75C58}
    \definecolor{ansi-red-intense}{HTML}{B22B31}
    \definecolor{ansi-green}{HTML}{00A250}
    \definecolor{ansi-green-intense}{HTML}{007427}
    \definecolor{ansi-yellow}{HTML}{DDB62B}
    \definecolor{ansi-yellow-intense}{HTML}{B27D12}
    \definecolor{ansi-blue}{HTML}{208FFB}
    \definecolor{ansi-blue-intense}{HTML}{0065CA}
    \definecolor{ansi-magenta}{HTML}{D160C4}
    \definecolor{ansi-magenta-intense}{HTML}{A03196}
    \definecolor{ansi-cyan}{HTML}{60C6C8}
    \definecolor{ansi-cyan-intense}{HTML}{258F8F}
    \definecolor{ansi-white}{HTML}{C5C1B4}
    \definecolor{ansi-white-intense}{HTML}{A1A6B2}

    % commands and environments needed by pandoc snippets
    % extracted from the output of `pandoc -s`
    \providecommand{\tightlist}{%
      \setlength{\itemsep}{0pt}\setlength{\parskip}{0pt}}
    \DefineVerbatimEnvironment{Highlighting}{Verbatim}{commandchars=\\\{\}}
    % Add ',fontsize=\small' for more characters per line
    \newenvironment{Shaded}{}{}
    \newcommand{\KeywordTok}[1]{\textcolor[rgb]{0.00,0.44,0.13}{\textbf{{#1}}}}
    \newcommand{\DataTypeTok}[1]{\textcolor[rgb]{0.56,0.13,0.00}{{#1}}}
    \newcommand{\DecValTok}[1]{\textcolor[rgb]{0.25,0.63,0.44}{{#1}}}
    \newcommand{\BaseNTok}[1]{\textcolor[rgb]{0.25,0.63,0.44}{{#1}}}
    \newcommand{\FloatTok}[1]{\textcolor[rgb]{0.25,0.63,0.44}{{#1}}}
    \newcommand{\CharTok}[1]{\textcolor[rgb]{0.25,0.44,0.63}{{#1}}}
    \newcommand{\StringTok}[1]{\textcolor[rgb]{0.25,0.44,0.63}{{#1}}}
    \newcommand{\CommentTok}[1]{\textcolor[rgb]{0.38,0.63,0.69}{\textit{{#1}}}}
    \newcommand{\OtherTok}[1]{\textcolor[rgb]{0.00,0.44,0.13}{{#1}}}
    \newcommand{\AlertTok}[1]{\textcolor[rgb]{1.00,0.00,0.00}{\textbf{{#1}}}}
    \newcommand{\FunctionTok}[1]{\textcolor[rgb]{0.02,0.16,0.49}{{#1}}}
    \newcommand{\RegionMarkerTok}[1]{{#1}}
    \newcommand{\ErrorTok}[1]{\textcolor[rgb]{1.00,0.00,0.00}{\textbf{{#1}}}}
    \newcommand{\NormalTok}[1]{{#1}}
    
    % Additional commands for more recent versions of Pandoc
    \newcommand{\ConstantTok}[1]{\textcolor[rgb]{0.53,0.00,0.00}{{#1}}}
    \newcommand{\SpecialCharTok}[1]{\textcolor[rgb]{0.25,0.44,0.63}{{#1}}}
    \newcommand{\VerbatimStringTok}[1]{\textcolor[rgb]{0.25,0.44,0.63}{{#1}}}
    \newcommand{\SpecialStringTok}[1]{\textcolor[rgb]{0.73,0.40,0.53}{{#1}}}
    \newcommand{\ImportTok}[1]{{#1}}
    \newcommand{\DocumentationTok}[1]{\textcolor[rgb]{0.73,0.13,0.13}{\textit{{#1}}}}
    \newcommand{\AnnotationTok}[1]{\textcolor[rgb]{0.38,0.63,0.69}{\textbf{\textit{{#1}}}}}
    \newcommand{\CommentVarTok}[1]{\textcolor[rgb]{0.38,0.63,0.69}{\textbf{\textit{{#1}}}}}
    \newcommand{\VariableTok}[1]{\textcolor[rgb]{0.10,0.09,0.49}{{#1}}}
    \newcommand{\ControlFlowTok}[1]{\textcolor[rgb]{0.00,0.44,0.13}{\textbf{{#1}}}}
    \newcommand{\OperatorTok}[1]{\textcolor[rgb]{0.40,0.40,0.40}{{#1}}}
    \newcommand{\BuiltInTok}[1]{{#1}}
    \newcommand{\ExtensionTok}[1]{{#1}}
    \newcommand{\PreprocessorTok}[1]{\textcolor[rgb]{0.74,0.48,0.00}{{#1}}}
    \newcommand{\AttributeTok}[1]{\textcolor[rgb]{0.49,0.56,0.16}{{#1}}}
    \newcommand{\InformationTok}[1]{\textcolor[rgb]{0.38,0.63,0.69}{\textbf{\textit{{#1}}}}}
    \newcommand{\WarningTok}[1]{\textcolor[rgb]{0.38,0.63,0.69}{\textbf{\textit{{#1}}}}}
    
    
    % Define a nice break command that doesn't care if a line doesn't already
    % exist.
    \def\br{\hspace*{\fill} \\* }
    % Math Jax compatability definitions
    \def\gt{>}
    \def\lt{<}
    % Document parameters
    \title{Assignment\_9}
    
    
    

    % Pygments definitions
    
\makeatletter
\def\PY@reset{\let\PY@it=\relax \let\PY@bf=\relax%
    \let\PY@ul=\relax \let\PY@tc=\relax%
    \let\PY@bc=\relax \let\PY@ff=\relax}
\def\PY@tok#1{\csname PY@tok@#1\endcsname}
\def\PY@toks#1+{\ifx\relax#1\empty\else%
    \PY@tok{#1}\expandafter\PY@toks\fi}
\def\PY@do#1{\PY@bc{\PY@tc{\PY@ul{%
    \PY@it{\PY@bf{\PY@ff{#1}}}}}}}
\def\PY#1#2{\PY@reset\PY@toks#1+\relax+\PY@do{#2}}

\expandafter\def\csname PY@tok@w\endcsname{\def\PY@tc##1{\textcolor[rgb]{0.73,0.73,0.73}{##1}}}
\expandafter\def\csname PY@tok@c\endcsname{\let\PY@it=\textit\def\PY@tc##1{\textcolor[rgb]{0.25,0.50,0.50}{##1}}}
\expandafter\def\csname PY@tok@cp\endcsname{\def\PY@tc##1{\textcolor[rgb]{0.74,0.48,0.00}{##1}}}
\expandafter\def\csname PY@tok@k\endcsname{\let\PY@bf=\textbf\def\PY@tc##1{\textcolor[rgb]{0.00,0.50,0.00}{##1}}}
\expandafter\def\csname PY@tok@kp\endcsname{\def\PY@tc##1{\textcolor[rgb]{0.00,0.50,0.00}{##1}}}
\expandafter\def\csname PY@tok@kt\endcsname{\def\PY@tc##1{\textcolor[rgb]{0.69,0.00,0.25}{##1}}}
\expandafter\def\csname PY@tok@o\endcsname{\def\PY@tc##1{\textcolor[rgb]{0.40,0.40,0.40}{##1}}}
\expandafter\def\csname PY@tok@ow\endcsname{\let\PY@bf=\textbf\def\PY@tc##1{\textcolor[rgb]{0.67,0.13,1.00}{##1}}}
\expandafter\def\csname PY@tok@nb\endcsname{\def\PY@tc##1{\textcolor[rgb]{0.00,0.50,0.00}{##1}}}
\expandafter\def\csname PY@tok@nf\endcsname{\def\PY@tc##1{\textcolor[rgb]{0.00,0.00,1.00}{##1}}}
\expandafter\def\csname PY@tok@nc\endcsname{\let\PY@bf=\textbf\def\PY@tc##1{\textcolor[rgb]{0.00,0.00,1.00}{##1}}}
\expandafter\def\csname PY@tok@nn\endcsname{\let\PY@bf=\textbf\def\PY@tc##1{\textcolor[rgb]{0.00,0.00,1.00}{##1}}}
\expandafter\def\csname PY@tok@ne\endcsname{\let\PY@bf=\textbf\def\PY@tc##1{\textcolor[rgb]{0.82,0.25,0.23}{##1}}}
\expandafter\def\csname PY@tok@nv\endcsname{\def\PY@tc##1{\textcolor[rgb]{0.10,0.09,0.49}{##1}}}
\expandafter\def\csname PY@tok@no\endcsname{\def\PY@tc##1{\textcolor[rgb]{0.53,0.00,0.00}{##1}}}
\expandafter\def\csname PY@tok@nl\endcsname{\def\PY@tc##1{\textcolor[rgb]{0.63,0.63,0.00}{##1}}}
\expandafter\def\csname PY@tok@ni\endcsname{\let\PY@bf=\textbf\def\PY@tc##1{\textcolor[rgb]{0.60,0.60,0.60}{##1}}}
\expandafter\def\csname PY@tok@na\endcsname{\def\PY@tc##1{\textcolor[rgb]{0.49,0.56,0.16}{##1}}}
\expandafter\def\csname PY@tok@nt\endcsname{\let\PY@bf=\textbf\def\PY@tc##1{\textcolor[rgb]{0.00,0.50,0.00}{##1}}}
\expandafter\def\csname PY@tok@nd\endcsname{\def\PY@tc##1{\textcolor[rgb]{0.67,0.13,1.00}{##1}}}
\expandafter\def\csname PY@tok@s\endcsname{\def\PY@tc##1{\textcolor[rgb]{0.73,0.13,0.13}{##1}}}
\expandafter\def\csname PY@tok@sd\endcsname{\let\PY@it=\textit\def\PY@tc##1{\textcolor[rgb]{0.73,0.13,0.13}{##1}}}
\expandafter\def\csname PY@tok@si\endcsname{\let\PY@bf=\textbf\def\PY@tc##1{\textcolor[rgb]{0.73,0.40,0.53}{##1}}}
\expandafter\def\csname PY@tok@se\endcsname{\let\PY@bf=\textbf\def\PY@tc##1{\textcolor[rgb]{0.73,0.40,0.13}{##1}}}
\expandafter\def\csname PY@tok@sr\endcsname{\def\PY@tc##1{\textcolor[rgb]{0.73,0.40,0.53}{##1}}}
\expandafter\def\csname PY@tok@ss\endcsname{\def\PY@tc##1{\textcolor[rgb]{0.10,0.09,0.49}{##1}}}
\expandafter\def\csname PY@tok@sx\endcsname{\def\PY@tc##1{\textcolor[rgb]{0.00,0.50,0.00}{##1}}}
\expandafter\def\csname PY@tok@m\endcsname{\def\PY@tc##1{\textcolor[rgb]{0.40,0.40,0.40}{##1}}}
\expandafter\def\csname PY@tok@gh\endcsname{\let\PY@bf=\textbf\def\PY@tc##1{\textcolor[rgb]{0.00,0.00,0.50}{##1}}}
\expandafter\def\csname PY@tok@gu\endcsname{\let\PY@bf=\textbf\def\PY@tc##1{\textcolor[rgb]{0.50,0.00,0.50}{##1}}}
\expandafter\def\csname PY@tok@gd\endcsname{\def\PY@tc##1{\textcolor[rgb]{0.63,0.00,0.00}{##1}}}
\expandafter\def\csname PY@tok@gi\endcsname{\def\PY@tc##1{\textcolor[rgb]{0.00,0.63,0.00}{##1}}}
\expandafter\def\csname PY@tok@gr\endcsname{\def\PY@tc##1{\textcolor[rgb]{1.00,0.00,0.00}{##1}}}
\expandafter\def\csname PY@tok@ge\endcsname{\let\PY@it=\textit}
\expandafter\def\csname PY@tok@gs\endcsname{\let\PY@bf=\textbf}
\expandafter\def\csname PY@tok@gp\endcsname{\let\PY@bf=\textbf\def\PY@tc##1{\textcolor[rgb]{0.00,0.00,0.50}{##1}}}
\expandafter\def\csname PY@tok@go\endcsname{\def\PY@tc##1{\textcolor[rgb]{0.53,0.53,0.53}{##1}}}
\expandafter\def\csname PY@tok@gt\endcsname{\def\PY@tc##1{\textcolor[rgb]{0.00,0.27,0.87}{##1}}}
\expandafter\def\csname PY@tok@err\endcsname{\def\PY@bc##1{\setlength{\fboxsep}{0pt}\fcolorbox[rgb]{1.00,0.00,0.00}{1,1,1}{\strut ##1}}}
\expandafter\def\csname PY@tok@kc\endcsname{\let\PY@bf=\textbf\def\PY@tc##1{\textcolor[rgb]{0.00,0.50,0.00}{##1}}}
\expandafter\def\csname PY@tok@kd\endcsname{\let\PY@bf=\textbf\def\PY@tc##1{\textcolor[rgb]{0.00,0.50,0.00}{##1}}}
\expandafter\def\csname PY@tok@kn\endcsname{\let\PY@bf=\textbf\def\PY@tc##1{\textcolor[rgb]{0.00,0.50,0.00}{##1}}}
\expandafter\def\csname PY@tok@kr\endcsname{\let\PY@bf=\textbf\def\PY@tc##1{\textcolor[rgb]{0.00,0.50,0.00}{##1}}}
\expandafter\def\csname PY@tok@bp\endcsname{\def\PY@tc##1{\textcolor[rgb]{0.00,0.50,0.00}{##1}}}
\expandafter\def\csname PY@tok@fm\endcsname{\def\PY@tc##1{\textcolor[rgb]{0.00,0.00,1.00}{##1}}}
\expandafter\def\csname PY@tok@vc\endcsname{\def\PY@tc##1{\textcolor[rgb]{0.10,0.09,0.49}{##1}}}
\expandafter\def\csname PY@tok@vg\endcsname{\def\PY@tc##1{\textcolor[rgb]{0.10,0.09,0.49}{##1}}}
\expandafter\def\csname PY@tok@vi\endcsname{\def\PY@tc##1{\textcolor[rgb]{0.10,0.09,0.49}{##1}}}
\expandafter\def\csname PY@tok@vm\endcsname{\def\PY@tc##1{\textcolor[rgb]{0.10,0.09,0.49}{##1}}}
\expandafter\def\csname PY@tok@sa\endcsname{\def\PY@tc##1{\textcolor[rgb]{0.73,0.13,0.13}{##1}}}
\expandafter\def\csname PY@tok@sb\endcsname{\def\PY@tc##1{\textcolor[rgb]{0.73,0.13,0.13}{##1}}}
\expandafter\def\csname PY@tok@sc\endcsname{\def\PY@tc##1{\textcolor[rgb]{0.73,0.13,0.13}{##1}}}
\expandafter\def\csname PY@tok@dl\endcsname{\def\PY@tc##1{\textcolor[rgb]{0.73,0.13,0.13}{##1}}}
\expandafter\def\csname PY@tok@s2\endcsname{\def\PY@tc##1{\textcolor[rgb]{0.73,0.13,0.13}{##1}}}
\expandafter\def\csname PY@tok@sh\endcsname{\def\PY@tc##1{\textcolor[rgb]{0.73,0.13,0.13}{##1}}}
\expandafter\def\csname PY@tok@s1\endcsname{\def\PY@tc##1{\textcolor[rgb]{0.73,0.13,0.13}{##1}}}
\expandafter\def\csname PY@tok@mb\endcsname{\def\PY@tc##1{\textcolor[rgb]{0.40,0.40,0.40}{##1}}}
\expandafter\def\csname PY@tok@mf\endcsname{\def\PY@tc##1{\textcolor[rgb]{0.40,0.40,0.40}{##1}}}
\expandafter\def\csname PY@tok@mh\endcsname{\def\PY@tc##1{\textcolor[rgb]{0.40,0.40,0.40}{##1}}}
\expandafter\def\csname PY@tok@mi\endcsname{\def\PY@tc##1{\textcolor[rgb]{0.40,0.40,0.40}{##1}}}
\expandafter\def\csname PY@tok@il\endcsname{\def\PY@tc##1{\textcolor[rgb]{0.40,0.40,0.40}{##1}}}
\expandafter\def\csname PY@tok@mo\endcsname{\def\PY@tc##1{\textcolor[rgb]{0.40,0.40,0.40}{##1}}}
\expandafter\def\csname PY@tok@ch\endcsname{\let\PY@it=\textit\def\PY@tc##1{\textcolor[rgb]{0.25,0.50,0.50}{##1}}}
\expandafter\def\csname PY@tok@cm\endcsname{\let\PY@it=\textit\def\PY@tc##1{\textcolor[rgb]{0.25,0.50,0.50}{##1}}}
\expandafter\def\csname PY@tok@cpf\endcsname{\let\PY@it=\textit\def\PY@tc##1{\textcolor[rgb]{0.25,0.50,0.50}{##1}}}
\expandafter\def\csname PY@tok@c1\endcsname{\let\PY@it=\textit\def\PY@tc##1{\textcolor[rgb]{0.25,0.50,0.50}{##1}}}
\expandafter\def\csname PY@tok@cs\endcsname{\let\PY@it=\textit\def\PY@tc##1{\textcolor[rgb]{0.25,0.50,0.50}{##1}}}

\def\PYZbs{\char`\\}
\def\PYZus{\char`\_}
\def\PYZob{\char`\{}
\def\PYZcb{\char`\}}
\def\PYZca{\char`\^}
\def\PYZam{\char`\&}
\def\PYZlt{\char`\<}
\def\PYZgt{\char`\>}
\def\PYZsh{\char`\#}
\def\PYZpc{\char`\%}
\def\PYZdl{\char`\$}
\def\PYZhy{\char`\-}
\def\PYZsq{\char`\'}
\def\PYZdq{\char`\"}
\def\PYZti{\char`\~}
% for compatibility with earlier versions
\def\PYZat{@}
\def\PYZlb{[}
\def\PYZrb{]}
\makeatother


    % Exact colors from NB
    \definecolor{incolor}{rgb}{0.0, 0.0, 0.5}
    \definecolor{outcolor}{rgb}{0.545, 0.0, 0.0}



    
    % Prevent overflowing lines due to hard-to-break entities
    \sloppy 
    % Setup hyperref package
    \hypersetup{
      breaklinks=true,  % so long urls are correctly broken across lines
      colorlinks=true,
      urlcolor=urlcolor,
      linkcolor=linkcolor,
      citecolor=citecolor,
      }
    % Slightly bigger margins than the latex defaults
    
    \geometry{verbose,tmargin=1in,bmargin=1in,lmargin=1in,rmargin=1in}
    
    

    \begin{document}
    
    
    \maketitle
    
    

    
    \emph{4th March 2018}\\
\emph{Author: Varun Sundar}\\
\emph{Course: EE2703}

\section{Abstract}\label{abstract}

This week's assignment covers the topic of Discrete Fourier Transform
(DFT) using python's fft library. The introduction serves as a write-up
on the various discrete transforms and implementation utility. We
further examine the DFT of various numeric arrays, while drawing
inferences from their analog counterparts.

\section{Introduction}\label{introduction}

The Discrete Fourier transform (DFT) is the digitsed analogue of the
analog Fourier Series.

A standard interpretation is to view it as the Discrete Time Fourier
Transform (DTFT), for periodic discrete signals.

Consider (DTFT),

\[ \begin{aligned} F(e^{j\theta})= \sum_{n=-\infty}^{\infty} f[n]e^{−jn\theta} \hfill \hfill \end{aligned} \]

now if,

\[ f[n+N] = f[n] \forall n \]

We notice that, this will converge to the Discrete Fourier Transform,
since:

\begin{itemize}
\item
  All the available information is contained within N samples.
\item
  \(X_{1/T}(f)\) converges to zero everywhere except integer multiples
  of \(\frac {1}{NT}\) and \(\frac {1}{NT}\) also called harmonic
  frequencies.
\item
  The DTFT is periodic, so the maximum number of unique harmonic
  amplitudes is N.
\end{itemize}

Introducing the notation \(\sum _{N}\) to represent a sum over any
n-sequence of length N, we can write:

\[ \begin{aligned}X_{1/T}\left({\frac {k}{NT}}\right)&=\sum _{m=-\infty }^{\infty }\left(\sum _{N}x[n-mN]\cdot e^{-i2\pi {\frac {k}{N}}(n-mN)}\right)\\&=\sum _{m=-\infty }^{\infty }\left(\sum _{N}x[n]\cdot e^{-i2\pi {\frac {k}{N}}n}\right)=T\underbrace {\left(\sum _{N}x(nT)\cdot e^{-i2\pi {\frac {k}{N}}n}\right)} _{X[k]\quad {\text{(DFT)}}}\cdot \left(\sum _{m=-\infty }^{\infty }1\right).\end{aligned} \]

\[ \begin{aligned}X_{1/T}\left({\frac {k}{NT}}\right)&=\sum _{m=-\infty }^{\infty }\left(\sum _{N}x[n-mN]\cdot e^{-i2\pi {\frac {k}{N}}(n-mN)}\right)\\&=\sum _{m=-\infty }^{\infty }\left(\sum _{N}x[n]\cdot e^{-i2\pi {\frac {k}{N}}n}\right)=T\underbrace {\left(\sum _{N}x(nT)\cdot e^{-i2\pi {\frac {k}{N}}n}\right)} _{X[k]\quad {\text{(DFT)}}}\cdot \left(\sum _{m=-\infty }^{\infty }1\right).\end{aligned} \]

Where, we have :

Discrete Fourier Transform:

\[ a_k = \frac{1}{N} \sum_{n=0}^{N-1} \tilde{x}[n]\exp(−i \frac{2\pi kn}{N}) \]

or

\[ \tilde{x}[n] = \frac{1}{N} \sum_{n=0}^{N-1} a_k \exp(−i \frac{2\pi kn}{N}) \]

Discrete Fourier Series: (of length \(N\))

\[ X_d (k) = \sum_{n=0}^{N-1} {x}[n]\exp(−i \frac{2\pi kn}{N})\] where k
goes from \([0...N-1]\)

or

\[ {x}[n] = \frac{1}{N} \sum_{n=0}^{N-1}X_d(k)\exp(−i \frac{2\pi kn}{N}) \]

where n goes from \([0...N-1]\)

Both the DFT and DFS may be seen as a sampled version of
\(X(e^{j\theta})\), albeit with different validity (DFS is valid only
for periodic \(x[n]\), but DFT holds for any finite \(x[n]\)).

\subsubsection{Using IFFT Shift, FFT
Shift}\label{using-ifft-shift-fft-shift}

These function can be used to change the domain of n and k to correct
the DFT of a numeric array. Given the options:

Following a MATLAB style convention,

\begin{itemize}
\item
  \texttt{fft(ifftshift(x))} is defined for \(n\) in
  \([-N/2...N/2-1\){]}, \(k\) in \([0...N-1]\).
\item
  \texttt{fftshift(fft(x))} is defined for \(n\) in \([0...N-1]\), \(k\)
  in \([-N/2...N/2-1]\) .
\item
  \texttt{fftshift(fft(fftshift(x)))} is defined for \(n\) in
  \([-N/2...N/2-1]\), \(k\) in \([-N/2...N/2-1]\).
\item
  \texttt{ifft(ifftshift(X))} is defined for \(n\) in \([0...N-1]\),
  \(k\) in \([-N/2...N/2-1]\).
\end{itemize}

Here, all functions are appropriately scoped from \emph{numpy's}
\emph{fft} module.

    \begin{Verbatim}[commandchars=\\\{\}]
{\color{incolor}In [{\color{incolor}66}]:} \PY{k+kn}{import} \PY{n+nn}{numpy} \PY{k}{as} \PY{n+nn}{np}
         \PY{k+kn}{import} \PY{n+nn}{scipy} \PY{k}{as} \PY{n+nn}{sp}
         \PY{k+kn}{import} \PY{n+nn}{matplotlib}\PY{n+nn}{.}\PY{n+nn}{pyplot} \PY{k}{as} \PY{n+nn}{plt}
\end{Verbatim}


    \begin{Verbatim}[commandchars=\\\{\}]
{\color{incolor}In [{\color{incolor}67}]:} \PY{k}{def} \PY{n+nf}{initialise\PYZus{}notebook}\PY{p}{(}\PY{p}{)}\PY{p}{:}
             \PY{n}{plt}\PY{o}{.}\PY{n}{style}\PY{o}{.}\PY{n}{use}\PY{p}{(}\PY{l+s+s1}{\PYZsq{}}\PY{l+s+s1}{ggplot}\PY{l+s+s1}{\PYZsq{}}\PY{p}{)}
             \PY{n}{plt}\PY{o}{.}\PY{n}{rcParams}\PY{p}{[}\PY{l+s+s1}{\PYZsq{}}\PY{l+s+s1}{font.family}\PY{l+s+s1}{\PYZsq{}}\PY{p}{]} \PY{o}{=} \PY{l+s+s1}{\PYZsq{}}\PY{l+s+s1}{serif}\PY{l+s+s1}{\PYZsq{}}
             \PY{n}{plt}\PY{o}{.}\PY{n}{rcParams}\PY{p}{[}\PY{l+s+s1}{\PYZsq{}}\PY{l+s+s1}{font.serif}\PY{l+s+s1}{\PYZsq{}}\PY{p}{]} \PY{o}{=} \PY{l+s+s1}{\PYZsq{}}\PY{l+s+s1}{Ubuntu}\PY{l+s+s1}{\PYZsq{}}
             \PY{n}{plt}\PY{o}{.}\PY{n}{rcParams}\PY{p}{[}\PY{l+s+s1}{\PYZsq{}}\PY{l+s+s1}{font.monospace}\PY{l+s+s1}{\PYZsq{}}\PY{p}{]} \PY{o}{=} \PY{l+s+s1}{\PYZsq{}}\PY{l+s+s1}{Ubuntu Mono}\PY{l+s+s1}{\PYZsq{}}
             \PY{n}{plt}\PY{o}{.}\PY{n}{rcParams}\PY{p}{[}\PY{l+s+s1}{\PYZsq{}}\PY{l+s+s1}{font.size}\PY{l+s+s1}{\PYZsq{}}\PY{p}{]} \PY{o}{=} \PY{l+m+mi}{10}
             \PY{n}{plt}\PY{o}{.}\PY{n}{rcParams}\PY{p}{[}\PY{l+s+s1}{\PYZsq{}}\PY{l+s+s1}{axes.labelsize}\PY{l+s+s1}{\PYZsq{}}\PY{p}{]} \PY{o}{=} \PY{l+m+mi}{10}
             \PY{n}{plt}\PY{o}{.}\PY{n}{rcParams}\PY{p}{[}\PY{l+s+s1}{\PYZsq{}}\PY{l+s+s1}{axes.labelweight}\PY{l+s+s1}{\PYZsq{}}\PY{p}{]} \PY{o}{=} \PY{l+s+s1}{\PYZsq{}}\PY{l+s+s1}{bold}\PY{l+s+s1}{\PYZsq{}}
             \PY{n}{plt}\PY{o}{.}\PY{n}{rcParams}\PY{p}{[}\PY{l+s+s1}{\PYZsq{}}\PY{l+s+s1}{axes.titlesize}\PY{l+s+s1}{\PYZsq{}}\PY{p}{]} \PY{o}{=} \PY{l+m+mi}{10}
             \PY{n}{plt}\PY{o}{.}\PY{n}{rcParams}\PY{p}{[}\PY{l+s+s1}{\PYZsq{}}\PY{l+s+s1}{xtick.labelsize}\PY{l+s+s1}{\PYZsq{}}\PY{p}{]} \PY{o}{=} \PY{l+m+mi}{8}
             \PY{n}{plt}\PY{o}{.}\PY{n}{rcParams}\PY{p}{[}\PY{l+s+s1}{\PYZsq{}}\PY{l+s+s1}{ytick.labelsize}\PY{l+s+s1}{\PYZsq{}}\PY{p}{]} \PY{o}{=} \PY{l+m+mi}{8}
             \PY{n}{plt}\PY{o}{.}\PY{n}{rcParams}\PY{p}{[}\PY{l+s+s1}{\PYZsq{}}\PY{l+s+s1}{legend.fontsize}\PY{l+s+s1}{\PYZsq{}}\PY{p}{]} \PY{o}{=} \PY{l+m+mi}{10}
             \PY{n}{plt}\PY{o}{.}\PY{n}{rcParams}\PY{p}{[}\PY{l+s+s1}{\PYZsq{}}\PY{l+s+s1}{figure.titlesize}\PY{l+s+s1}{\PYZsq{}}\PY{p}{]} \PY{o}{=} \PY{l+m+mi}{12}
             \PY{n}{plt}\PY{o}{.}\PY{n}{rcParams}\PY{p}{[}\PY{l+s+s1}{\PYZsq{}}\PY{l+s+s1}{figure.figsize}\PY{l+s+s1}{\PYZsq{}}\PY{p}{]} \PY{o}{=} \PY{l+m+mi}{25}\PY{p}{,} \PY{l+m+mi}{20}
             \PY{n}{plt}\PY{o}{.}\PY{n}{rcParams}\PY{p}{[}\PY{l+s+s1}{\PYZsq{}}\PY{l+s+s1}{figure.dpi}\PY{l+s+s1}{\PYZsq{}}\PY{p}{]} \PY{o}{=} \PY{l+m+mi}{150}
\end{Verbatim}


    \subsection{Numpy's FFT Module}\label{numpys-fft-module}

We use a series of examples to go over numpy's fft module.

\subsubsection{Random noise case}\label{random-noise-case}

    \begin{Verbatim}[commandchars=\\\{\}]
{\color{incolor}In [{\color{incolor}68}]:} \PY{n}{x}\PY{o}{=}\PY{n}{np}\PY{o}{.}\PY{n}{random}\PY{o}{.}\PY{n}{rand}\PY{p}{(}\PY{l+m+mi}{100}\PY{p}{)}
         \PY{n}{X}\PY{o}{=}\PY{n}{np}\PY{o}{.}\PY{n}{fft}\PY{o}{.}\PY{n}{fft}\PY{p}{(}\PY{n}{x}\PY{p}{)}
         \PY{n}{y}\PY{o}{=}\PY{n}{np}\PY{o}{.}\PY{n}{fft}\PY{o}{.}\PY{n}{ifft}\PY{p}{(}\PY{n}{X}\PY{p}{)}
         
         \PY{c+c1}{\PYZsh{} Numpy\PYZsq{}s Stack trick}
         \PY{n}{display}\PY{p}{(}\PY{n}{np}\PY{o}{.}\PY{n}{c\PYZus{}}\PY{p}{[}\PY{n}{x}\PY{p}{,}\PY{n}{y}\PY{p}{]}\PY{p}{[}\PY{p}{:}\PY{l+m+mi}{20}\PY{p}{]}\PY{p}{)}
         \PY{n+nb}{print}\PY{p}{(} \PY{n+nb}{abs}\PY{p}{(}\PY{n}{x}\PY{o}{\PYZhy{}}\PY{n}{y}\PY{p}{)}\PY{o}{.}\PY{n}{max}\PY{p}{(}\PY{p}{)}\PY{p}{)}
\end{Verbatim}


    
    \begin{verbatim}
array([[0.61593481+0.00000000e+00j, 0.61593481-1.33226763e-17j],
       [0.68246068+0.00000000e+00j, 0.68246068+1.99840144e-17j],
       [0.38321316+0.00000000e+00j, 0.38321316-4.66293670e-17j],
       [0.02191708+0.00000000e+00j, 0.02191708+1.15463195e-16j],
       [0.16709719+0.00000000e+00j, 0.16709719+3.46126074e-16j],
       [0.7461249 +0.00000000e+00j, 0.7461249 -2.81001382e-16j],
       [0.74901927+0.00000000e+00j, 0.74901927+1.20964430e-16j],
       [0.65138529+0.00000000e+00j, 0.65138529+2.01273673e-16j],
       [0.25579848+0.00000000e+00j, 0.25579848+1.66941654e-16j],
       [0.47021297+0.00000000e+00j, 0.47021297+2.20775046e-17j],
       [0.64664987+0.00000000e+00j, 0.64664987+2.57438244e-16j],
       [0.21417918+0.00000000e+00j, 0.21417918-1.89819690e-16j],
       [0.13172336+0.00000000e+00j, 0.13172336+1.74453650e-16j],
       [0.93757341+0.00000000e+00j, 0.93757341+9.55185020e-17j],
       [0.16118179+0.00000000e+00j, 0.16118179-5.21561203e-17j],
       [0.72331928+0.00000000e+00j, 0.72331928+8.06026128e-16j],
       [0.41785434+0.00000000e+00j, 0.41785434-1.50173434e-16j],
       [0.07359675+0.00000000e+00j, 0.07359675+2.45561879e-16j],
       [0.56543918+0.00000000e+00j, 0.56543918-4.28187720e-18j],
       [0.48550131+0.00000000e+00j, 0.48550131+1.86448151e-18j]])
    \end{verbatim}

    
    \begin{Verbatim}[commandchars=\\\{\}]
8.136363256089102e-16

    \end{Verbatim}

    We notice a few inaccuracies- in particular the imaginary part is not
always zero, due to the error in computing ifft().

    \subsubsection{The Sinusoid Case}\label{the-sinusoid-case}

Here, we attempt the DFT of a sinusoidal discrete array. As mentioned in
the assignment sheet, we examine the linear spacing to be followed, the
size of impulses to be plotted - among other details.

Note that,

\[ y = sin(x) = \frac{e^{jx} −e^{-jx}}{2j} \]

With the spectrum (continuous time),

\[ Y(\omega)= \frac{[\delta(\omega−1)−\delta(\omega+1)]}{2j} \]

    \begin{Verbatim}[commandchars=\\\{\}]
{\color{incolor}In [{\color{incolor}69}]:} \PY{n}{x}\PY{o}{=}\PY{n}{np}\PY{o}{.}\PY{n}{linspace}\PY{p}{(}\PY{l+m+mi}{0}\PY{p}{,}\PY{l+m+mi}{2}\PY{o}{*}\PY{n}{np}\PY{o}{.}\PY{n}{pi}\PY{p}{,}\PY{l+m+mi}{128}\PY{p}{)}
         \PY{n}{y}\PY{o}{=}\PY{n}{np}\PY{o}{.}\PY{n}{sin}\PY{p}{(}\PY{l+m+mi}{5}\PY{o}{*}\PY{n}{x}\PY{p}{)}
         \PY{n}{Y}\PY{o}{=}\PY{n}{np}\PY{o}{.}\PY{n}{fft}\PY{o}{.}\PY{n}{fft}\PY{p}{(}\PY{n}{y}\PY{p}{)}
         
         \PY{c+c1}{\PYZsh{} Some plotting fixes}
         \PY{n}{initialise\PYZus{}notebook}\PY{p}{(}\PY{p}{)}
         
         \PY{c+c1}{\PYZsh{} Matlab style subplotting}
         \PY{n}{plt}\PY{o}{.}\PY{n}{figure}\PY{p}{(}\PY{p}{)}
         \PY{n}{plt}\PY{o}{.}\PY{n}{subplot}\PY{p}{(}\PY{l+m+mi}{2}\PY{p}{,}\PY{l+m+mi}{1}\PY{p}{,}\PY{l+m+mi}{1}\PY{p}{)}
         \PY{n}{plt}\PY{o}{.}\PY{n}{plot}\PY{p}{(}\PY{n+nb}{abs}\PY{p}{(}\PY{n}{Y}\PY{p}{)}\PY{p}{,}\PY{n}{lw}\PY{o}{=}\PY{l+m+mi}{2}\PY{p}{)}
         \PY{n}{plt}\PY{o}{.}\PY{n}{grid}\PY{p}{(}\PY{k+kc}{True}\PY{p}{)}
         \PY{n}{plt}\PY{o}{.}\PY{n}{ylabel}\PY{p}{(}\PY{l+s+sa}{r}\PY{l+s+s2}{\PYZdq{}}\PY{l+s+s2}{\PYZdl{}|Y|\PYZdl{}}\PY{l+s+s2}{\PYZdq{}}\PY{p}{,}\PY{n}{size}\PY{o}{=}\PY{l+m+mi}{16}\PY{p}{)}
         \PY{n}{plt}\PY{o}{.}\PY{n}{title}\PY{p}{(}\PY{l+s+sa}{r}\PY{l+s+s2}{\PYZdq{}}\PY{l+s+s2}{Spectrum of \PYZdl{}}\PY{l+s+s2}{\PYZbs{}}\PY{l+s+s2}{sin(5t)\PYZdl{}}\PY{l+s+s2}{\PYZdq{}}\PY{p}{)}
         
         \PY{c+c1}{\PYZsh{} Plot two}
         \PY{n}{plt}\PY{o}{.}\PY{n}{subplot}\PY{p}{(}\PY{l+m+mi}{2}\PY{p}{,}\PY{l+m+mi}{1}\PY{p}{,}\PY{l+m+mi}{2}\PY{p}{)}
         \PY{n}{plt}\PY{o}{.}\PY{n}{plot}\PY{p}{(}\PY{n}{np}\PY{o}{.}\PY{n}{unwrap}\PY{p}{(}\PY{n}{np}\PY{o}{.}\PY{n}{angle}\PY{p}{(}\PY{n}{Y}\PY{p}{)}\PY{p}{)}\PY{p}{,}\PY{n}{lw}\PY{o}{=}\PY{l+m+mi}{2}\PY{p}{)}
         \PY{n}{plt}\PY{o}{.}\PY{n}{ylabel}\PY{p}{(}\PY{l+s+sa}{r}\PY{l+s+s2}{\PYZdq{}}\PY{l+s+s2}{Phase of \PYZdl{}Y\PYZdl{}}\PY{l+s+s2}{\PYZdq{}}\PY{p}{,}\PY{n}{size}\PY{o}{=}\PY{l+m+mi}{16}\PY{p}{)}
         \PY{n}{plt}\PY{o}{.}\PY{n}{xlabel}\PY{p}{(}\PY{l+s+sa}{r}\PY{l+s+s2}{\PYZdq{}}\PY{l+s+s2}{\PYZdl{}k\PYZdl{}}\PY{l+s+s2}{\PYZdq{}}\PY{p}{,}\PY{n}{size}\PY{o}{=}\PY{l+m+mi}{16}\PY{p}{)}
         \PY{n}{plt}\PY{o}{.}\PY{n}{grid}\PY{p}{(}\PY{k+kc}{True}\PY{p}{)}
         \PY{n}{plt}\PY{o}{.}\PY{n}{show}\PY{p}{(}\PY{p}{)}
\end{Verbatim}


    \begin{Verbatim}[commandchars=\\\{\}]
/Users/Ankivarun/anaconda3/envs/tf\_python3/lib/python3.6/site-packages/matplotlib/font\_manager.py:1316: UserWarning: findfont: Font family ['serif'] not found. Falling back to DejaVu Sans
  (prop.get\_family(), self.defaultFamily[fontext]))

    \end{Verbatim}

    \begin{center}
    \adjustimage{max size={0.9\linewidth}{0.9\paperheight}}{output_7_1.png}
    \end{center}
    { \hspace*{\fill} \\}
    
    There are a couple of issues that need fixing here,

\begin{itemize}
\tightlist
\item
  Firstly, the position of the spikes needs to be corrected. This can be
  done by fft shift.
\item
  The magnitude of the spikes also needs to be corrected (from 64 to
  0.5). This may be done by dividing by the sample rate.
\item
  The frequency axis needs to be in place. This is due to the duplicacy
  of \(0\) and \(2\pi\).
\end{itemize}

    \begin{Verbatim}[commandchars=\\\{\}]
{\color{incolor}In [{\color{incolor}70}]:} \PY{n}{x}\PY{o}{=}\PY{n}{np}\PY{o}{.}\PY{n}{linspace}\PY{p}{(}\PY{l+m+mi}{0}\PY{p}{,}\PY{l+m+mi}{2}\PY{o}{*}\PY{n}{np}\PY{o}{.}\PY{n}{pi}\PY{p}{,}\PY{l+m+mi}{129}\PY{p}{)}\PY{p}{;}\PY{n}{x}\PY{o}{=}\PY{n}{x}\PY{p}{[}\PY{p}{:}\PY{o}{\PYZhy{}}\PY{l+m+mi}{1}\PY{p}{]}
         \PY{n}{y}\PY{o}{=}\PY{n}{np}\PY{o}{.}\PY{n}{sin}\PY{p}{(}\PY{l+m+mi}{5}\PY{o}{*}\PY{n}{x}\PY{p}{)}
         \PY{n}{Y}\PY{o}{=}\PY{n}{np}\PY{o}{.}\PY{n}{fft}\PY{o}{.}\PY{n}{fftshift}\PY{p}{(}\PY{n}{np}\PY{o}{.}\PY{n}{fft}\PY{o}{.}\PY{n}{fft}\PY{p}{(}\PY{n}{y}\PY{p}{)}\PY{p}{)}\PY{o}{/}\PY{l+m+mf}{128.0}
         \PY{n}{w}\PY{o}{=}\PY{n}{np}\PY{o}{.}\PY{n}{linspace}\PY{p}{(}\PY{o}{\PYZhy{}}\PY{l+m+mi}{64}\PY{p}{,}\PY{l+m+mi}{63}\PY{p}{,}\PY{l+m+mi}{128}\PY{p}{)}
         
         \PY{c+c1}{\PYZsh{} Figure 1}
         \PY{n}{plt}\PY{o}{.}\PY{n}{figure}\PY{p}{(}\PY{p}{)}
         \PY{n}{plt}\PY{o}{.}\PY{n}{subplot}\PY{p}{(}\PY{l+m+mi}{2}\PY{p}{,}\PY{l+m+mi}{1}\PY{p}{,}\PY{l+m+mi}{1}\PY{p}{)}
         \PY{n}{plt}\PY{o}{.}\PY{n}{plot}\PY{p}{(}\PY{n}{w}\PY{p}{,}\PY{n+nb}{abs}\PY{p}{(}\PY{n}{Y}\PY{p}{)}\PY{p}{,}\PY{n}{lw}\PY{o}{=}\PY{l+m+mi}{2}\PY{p}{)}
         \PY{n}{plt}\PY{o}{.}\PY{n}{xlim}\PY{p}{(}\PY{p}{[}\PY{o}{\PYZhy{}}\PY{l+m+mi}{10}\PY{p}{,}\PY{l+m+mi}{10}\PY{p}{]}\PY{p}{)}
         \PY{n}{plt}\PY{o}{.}\PY{n}{ylabel}\PY{p}{(}\PY{l+s+sa}{r}\PY{l+s+s2}{\PYZdq{}}\PY{l+s+s2}{\PYZdl{}|Y|\PYZdl{}}\PY{l+s+s2}{\PYZdq{}}\PY{p}{,}\PY{n}{size}\PY{o}{=}\PY{l+m+mi}{16}\PY{p}{)}
         \PY{n}{plt}\PY{o}{.}\PY{n}{title}\PY{p}{(}\PY{l+s+sa}{r}\PY{l+s+s2}{\PYZdq{}}\PY{l+s+s2}{Spectrum of \PYZdl{}}\PY{l+s+s2}{\PYZbs{}}\PY{l+s+s2}{sin(5t)\PYZdl{}}\PY{l+s+s2}{\PYZdq{}}\PY{p}{)}
         \PY{n}{plt}\PY{o}{.}\PY{n}{grid}\PY{p}{(}\PY{k+kc}{True}\PY{p}{)}
         
         \PY{c+c1}{\PYZsh{} Figure 2}
         \PY{n}{plt}\PY{o}{.}\PY{n}{subplot}\PY{p}{(}\PY{l+m+mi}{2}\PY{p}{,}\PY{l+m+mi}{1}\PY{p}{,}\PY{l+m+mi}{2}\PY{p}{)}
         
         \PY{c+c1}{\PYZsh{} We choose to supress points with no amplitude (below a threshold)}
         \PY{c+c1}{\PYZsh{}plt.plot(w,np.angle(Y),\PYZsq{}ro\PYZsq{},lw=2)}
         \PY{n}{ii}\PY{o}{=}\PY{n}{np}\PY{o}{.}\PY{n}{where}\PY{p}{(}\PY{n}{np}\PY{o}{.}\PY{n}{abs}\PY{p}{(}\PY{n}{Y}\PY{p}{)}\PY{o}{\PYZgt{}}\PY{l+m+mf}{1e\PYZhy{}3}\PY{p}{)}
         \PY{n}{plt}\PY{o}{.}\PY{n}{plot}\PY{p}{(}\PY{n}{w}\PY{p}{[}\PY{n}{ii}\PY{p}{]}\PY{p}{,}\PY{n}{np}\PY{o}{.}\PY{n}{angle}\PY{p}{(}\PY{n}{Y}\PY{p}{[}\PY{n}{ii}\PY{p}{]}\PY{p}{)}\PY{p}{,}\PY{l+s+s1}{\PYZsq{}}\PY{l+s+s1}{go}\PY{l+s+s1}{\PYZsq{}}\PY{p}{,}\PY{n}{lw}\PY{o}{=}\PY{l+m+mi}{2}\PY{p}{)}
         \PY{n}{plt}\PY{o}{.}\PY{n}{xlim}\PY{p}{(}\PY{p}{[}\PY{o}{\PYZhy{}}\PY{l+m+mi}{10}\PY{p}{,}\PY{l+m+mi}{10}\PY{p}{]}\PY{p}{)}
         \PY{n}{plt}\PY{o}{.}\PY{n}{ylabel}\PY{p}{(}\PY{l+s+sa}{r}\PY{l+s+s2}{\PYZdq{}}\PY{l+s+s2}{Phase of \PYZdl{}Y\PYZdl{}}\PY{l+s+s2}{\PYZdq{}}\PY{p}{,}\PY{n}{size}\PY{o}{=}\PY{l+m+mi}{16}\PY{p}{)}
         \PY{n}{plt}\PY{o}{.}\PY{n}{xlabel}\PY{p}{(}\PY{l+s+sa}{r}\PY{l+s+s2}{\PYZdq{}}\PY{l+s+s2}{\PYZdl{}k\PYZdl{}}\PY{l+s+s2}{\PYZdq{}}\PY{p}{,}\PY{n}{size}\PY{o}{=}\PY{l+m+mi}{16}\PY{p}{)}
         \PY{n}{plt}\PY{o}{.}\PY{n}{grid}\PY{p}{(}\PY{k+kc}{True}\PY{p}{)}
         
         \PY{c+c1}{\PYZsh{} Display}
         \PY{n}{plt}\PY{o}{.}\PY{n}{show}\PY{p}{(}\PY{p}{)}
\end{Verbatim}


    \begin{Verbatim}[commandchars=\\\{\}]
/Users/Ankivarun/anaconda3/envs/tf\_python3/lib/python3.6/site-packages/matplotlib/font\_manager.py:1316: UserWarning: findfont: Font family ['serif'] not found. Falling back to DejaVu Sans
  (prop.get\_family(), self.defaultFamily[fontext]))

    \end{Verbatim}

    \begin{center}
    \adjustimage{max size={0.9\linewidth}{0.9\paperheight}}{output_9_1.png}
    \end{center}
    { \hspace*{\fill} \\}
    
    \subsubsection{Amplitude Modulation
Case}\label{amplitude-modulation-case}

Here we examine an amplitude modulated signal,
\[ f(t) = (1+0.1cos(t))cos(10t) \]

We expect a shifted set of spikes, with a main impulse and two side
impulses on each side. The main impulse would have an amplitude of 0.5,
whereas the shifted ones would have an amplitude of 0.025, considering
that,

\[ 0.1 cos(10t) cos(t) = 0.05 (cos 11t + cos 9t ) \] \[
= 0.025 (e^{11tj} +e^{9tj} +e^{−11tj} +e^{−9tj})
\]

    \begin{Verbatim}[commandchars=\\\{\}]
{\color{incolor}In [{\color{incolor}71}]:} \PY{n}{t}\PY{o}{=}\PY{n}{np}\PY{o}{.}\PY{n}{linspace}\PY{p}{(}\PY{l+m+mi}{0}\PY{p}{,}\PY{l+m+mi}{2}\PY{o}{*}\PY{n}{np}\PY{o}{.}\PY{n}{pi}\PY{p}{,}\PY{l+m+mi}{129}\PY{p}{)}\PY{p}{;}\PY{n}{t}\PY{o}{=}\PY{n}{t}\PY{p}{[}\PY{p}{:}\PY{o}{\PYZhy{}}\PY{l+m+mi}{1}\PY{p}{]}
         \PY{n}{y}\PY{o}{=}\PY{p}{(}\PY{l+m+mi}{1}\PY{o}{+}\PY{l+m+mf}{0.1}\PY{o}{*}\PY{n}{np}\PY{o}{.}\PY{n}{cos}\PY{p}{(}\PY{n}{t}\PY{p}{)}\PY{p}{)}\PY{o}{*}\PY{n}{np}\PY{o}{.}\PY{n}{cos}\PY{p}{(}\PY{l+m+mi}{10}\PY{o}{*}\PY{n}{t}\PY{p}{)}
         \PY{n}{Y}\PY{o}{=}\PY{n}{np}\PY{o}{.}\PY{n}{fft}\PY{o}{.}\PY{n}{fftshift}\PY{p}{(}\PY{n}{np}\PY{o}{.}\PY{n}{fft}\PY{o}{.}\PY{n}{fft}\PY{p}{(}\PY{n}{y}\PY{p}{)}\PY{p}{)}\PY{o}{/}\PY{l+m+mf}{128.0}
         \PY{n}{w}\PY{o}{=}\PY{n}{np}\PY{o}{.}\PY{n}{linspace}\PY{p}{(}\PY{o}{\PYZhy{}}\PY{l+m+mi}{64}\PY{p}{,}\PY{l+m+mi}{63}\PY{p}{,}\PY{l+m+mi}{128}\PY{p}{)}
         
         \PY{c+c1}{\PYZsh{} Figure 1}
         \PY{n}{plt}\PY{o}{.}\PY{n}{figure}\PY{p}{(}\PY{p}{)}
         \PY{n}{plt}\PY{o}{.}\PY{n}{subplot}\PY{p}{(}\PY{l+m+mi}{2}\PY{p}{,}\PY{l+m+mi}{1}\PY{p}{,}\PY{l+m+mi}{1}\PY{p}{)}
         \PY{n}{plt}\PY{o}{.}\PY{n}{plot}\PY{p}{(}\PY{n}{w}\PY{p}{,}\PY{n+nb}{abs}\PY{p}{(}\PY{n}{Y}\PY{p}{)}\PY{p}{,}\PY{n}{lw}\PY{o}{=}\PY{l+m+mi}{2}\PY{p}{)}
         \PY{n}{plt}\PY{o}{.}\PY{n}{xlim}\PY{p}{(}\PY{p}{[}\PY{o}{\PYZhy{}}\PY{l+m+mi}{15}\PY{p}{,}\PY{l+m+mi}{15}\PY{p}{]}\PY{p}{)}
         \PY{n}{plt}\PY{o}{.}\PY{n}{ylabel}\PY{p}{(}\PY{l+s+sa}{r}\PY{l+s+s2}{\PYZdq{}}\PY{l+s+s2}{\PYZdl{}|Y|\PYZdl{}}\PY{l+s+s2}{\PYZdq{}}\PY{p}{,}\PY{n}{size}\PY{o}{=}\PY{l+m+mi}{16}\PY{p}{)}
         \PY{n}{plt}\PY{o}{.}\PY{n}{title}\PY{p}{(}\PY{l+s+sa}{r}\PY{l+s+s2}{\PYZdq{}}\PY{l+s+s2}{Spectrum of \PYZdl{}}\PY{l+s+s2}{\PYZbs{}}\PY{l+s+s2}{left(1+0.1}\PY{l+s+s2}{\PYZbs{}}\PY{l+s+s2}{cos}\PY{l+s+s2}{\PYZbs{}}\PY{l+s+s2}{left(t}\PY{l+s+s2}{\PYZbs{}}\PY{l+s+s2}{right)}\PY{l+s+s2}{\PYZbs{}}\PY{l+s+s2}{right)}\PY{l+s+s2}{\PYZbs{}}\PY{l+s+s2}{cos}\PY{l+s+s2}{\PYZbs{}}\PY{l+s+s2}{left(10t}\PY{l+s+s2}{\PYZbs{}}\PY{l+s+s2}{right)\PYZdl{}}\PY{l+s+s2}{\PYZdq{}}\PY{p}{)}
         \PY{n}{plt}\PY{o}{.}\PY{n}{grid}\PY{p}{(}\PY{k+kc}{True}\PY{p}{)}
          
         \PY{c+c1}{\PYZsh{} Figure 2}
         \PY{n}{plt}\PY{o}{.}\PY{n}{subplot}\PY{p}{(}\PY{l+m+mi}{2}\PY{p}{,}\PY{l+m+mi}{1}\PY{p}{,}\PY{l+m+mi}{2}\PY{p}{)}
         \PY{c+c1}{\PYZsh{}plt.plot(w,np.angle(Y),\PYZsq{}ro\PYZsq{},lw=2)}
         \PY{n}{ii}\PY{o}{=}\PY{n}{np}\PY{o}{.}\PY{n}{where}\PY{p}{(}\PY{n}{np}\PY{o}{.}\PY{n}{abs}\PY{p}{(}\PY{n}{Y}\PY{p}{)}\PY{o}{\PYZgt{}}\PY{l+m+mf}{1e\PYZhy{}3}\PY{p}{)}
         \PY{n}{plt}\PY{o}{.}\PY{n}{plot}\PY{p}{(}\PY{n}{w}\PY{p}{[}\PY{n}{ii}\PY{p}{]}\PY{p}{,}\PY{n}{np}\PY{o}{.}\PY{n}{angle}\PY{p}{(}\PY{n}{Y}\PY{p}{[}\PY{n}{ii}\PY{p}{]}\PY{p}{)}\PY{p}{,}\PY{l+s+s1}{\PYZsq{}}\PY{l+s+s1}{go}\PY{l+s+s1}{\PYZsq{}}\PY{p}{,}\PY{n}{lw}\PY{o}{=}\PY{l+m+mi}{2}\PY{p}{)}
         \PY{n}{plt}\PY{o}{.}\PY{n}{xlim}\PY{p}{(}\PY{p}{[}\PY{o}{\PYZhy{}}\PY{l+m+mi}{15}\PY{p}{,}\PY{l+m+mi}{15}\PY{p}{]}\PY{p}{)}
         \PY{n}{plt}\PY{o}{.}\PY{n}{ylim}\PY{p}{(}\PY{p}{[}\PY{o}{\PYZhy{}}\PY{l+m+mi}{3}\PY{p}{,}\PY{l+m+mi}{3}\PY{p}{]}\PY{p}{)}
         \PY{n}{plt}\PY{o}{.}\PY{n}{ylabel}\PY{p}{(}\PY{l+s+sa}{r}\PY{l+s+s2}{\PYZdq{}}\PY{l+s+s2}{Phase of \PYZdl{}Y\PYZdl{}}\PY{l+s+s2}{\PYZdq{}}\PY{p}{,}\PY{n}{size}\PY{o}{=}\PY{l+m+mi}{16}\PY{p}{)}
         \PY{n}{plt}\PY{o}{.}\PY{n}{xlabel}\PY{p}{(}\PY{l+s+sa}{r}\PY{l+s+s2}{\PYZdq{}}\PY{l+s+s2}{\PYZdl{}}\PY{l+s+s2}{\PYZbs{}}\PY{l+s+s2}{omega\PYZdl{}}\PY{l+s+s2}{\PYZdq{}}\PY{p}{,}\PY{n}{size}\PY{o}{=}\PY{l+m+mi}{16}\PY{p}{)}
         \PY{n}{plt}\PY{o}{.}\PY{n}{grid}\PY{p}{(}\PY{k+kc}{True}\PY{p}{)}
         \PY{n}{plt}\PY{o}{.}\PY{n}{show}\PY{p}{(}\PY{p}{)}
\end{Verbatim}


    \begin{Verbatim}[commandchars=\\\{\}]
/Users/Ankivarun/anaconda3/envs/tf\_python3/lib/python3.6/site-packages/matplotlib/font\_manager.py:1316: UserWarning: findfont: Font family ['serif'] not found. Falling back to DejaVu Sans
  (prop.get\_family(), self.defaultFamily[fontext]))

    \end{Verbatim}

    \begin{center}
    \adjustimage{max size={0.9\linewidth}{0.9\paperheight}}{output_11_1.png}
    \end{center}
    { \hspace*{\fill} \\}
    
    Notice that in order to see the two side bands, we will have to improve
the frequency resolution. Note that this will involve reducing the time
resolution, owing to their inverse dependence.

Since we wish to keep the number of samples more or less consistent, we
choose to broaden the time space instead.

    \begin{Verbatim}[commandchars=\\\{\}]
{\color{incolor}In [{\color{incolor}72}]:} \PY{n}{t}\PY{o}{=}\PY{n}{np}\PY{o}{.}\PY{n}{linspace}\PY{p}{(}\PY{o}{\PYZhy{}}\PY{l+m+mi}{4}\PY{o}{*}\PY{n}{np}\PY{o}{.}\PY{n}{pi}\PY{p}{,}\PY{l+m+mi}{4}\PY{o}{*}\PY{n}{np}\PY{o}{.}\PY{n}{pi}\PY{p}{,}\PY{l+m+mi}{513}\PY{p}{)}\PY{p}{;}\PY{n}{t}\PY{o}{=}\PY{n}{t}\PY{p}{[}\PY{p}{:}\PY{o}{\PYZhy{}}\PY{l+m+mi}{1}\PY{p}{]}
         \PY{n}{y}\PY{o}{=}\PY{p}{(}\PY{l+m+mi}{1}\PY{o}{+}\PY{l+m+mf}{0.1}\PY{o}{*}\PY{n}{np}\PY{o}{.}\PY{n}{cos}\PY{p}{(}\PY{n}{t}\PY{p}{)}\PY{p}{)}\PY{o}{*}\PY{n}{np}\PY{o}{.}\PY{n}{cos}\PY{p}{(}\PY{l+m+mi}{10}\PY{o}{*}\PY{n}{t}\PY{p}{)}
         \PY{n}{Y}\PY{o}{=}\PY{n}{np}\PY{o}{.}\PY{n}{fft}\PY{o}{.}\PY{n}{fftshift}\PY{p}{(}\PY{n}{np}\PY{o}{.}\PY{n}{fft}\PY{o}{.}\PY{n}{fft}\PY{p}{(}\PY{n}{y}\PY{p}{)}\PY{p}{)}\PY{o}{/}\PY{l+m+mf}{512.0}
         \PY{n}{w}\PY{o}{=}\PY{n}{np}\PY{o}{.}\PY{n}{linspace}\PY{p}{(}\PY{o}{\PYZhy{}}\PY{l+m+mi}{64}\PY{p}{,}\PY{l+m+mi}{63}\PY{p}{,}\PY{l+m+mi}{513}\PY{p}{)}\PY{p}{;}\PY{n}{w}\PY{o}{=}\PY{n}{w}\PY{p}{[}\PY{p}{:}\PY{o}{\PYZhy{}}\PY{l+m+mi}{1}\PY{p}{]}
         
         \PY{c+c1}{\PYZsh{} Figure 1}
         \PY{n}{plt}\PY{o}{.}\PY{n}{figure}\PY{p}{(}\PY{p}{)}
         \PY{n}{plt}\PY{o}{.}\PY{n}{subplot}\PY{p}{(}\PY{l+m+mi}{2}\PY{p}{,}\PY{l+m+mi}{1}\PY{p}{,}\PY{l+m+mi}{1}\PY{p}{)}
         \PY{n}{plt}\PY{o}{.}\PY{n}{plot}\PY{p}{(}\PY{n}{w}\PY{p}{,}\PY{n+nb}{abs}\PY{p}{(}\PY{n}{Y}\PY{p}{)}\PY{p}{,}\PY{n}{lw}\PY{o}{=}\PY{l+m+mi}{2}\PY{p}{)}
         \PY{n}{plt}\PY{o}{.}\PY{n}{xlim}\PY{p}{(}\PY{p}{[}\PY{o}{\PYZhy{}}\PY{l+m+mi}{15}\PY{p}{,}\PY{l+m+mi}{15}\PY{p}{]}\PY{p}{)}
         \PY{n}{plt}\PY{o}{.}\PY{n}{ylabel}\PY{p}{(}\PY{l+s+sa}{r}\PY{l+s+s2}{\PYZdq{}}\PY{l+s+s2}{\PYZdl{}|Y|\PYZdl{}}\PY{l+s+s2}{\PYZdq{}}\PY{p}{,}\PY{n}{size}\PY{o}{=}\PY{l+m+mi}{16}\PY{p}{)}
         \PY{n}{plt}\PY{o}{.}\PY{n}{title}\PY{p}{(}\PY{l+s+sa}{r}\PY{l+s+s2}{\PYZdq{}}\PY{l+s+s2}{Spectrum of \PYZdl{}}\PY{l+s+s2}{\PYZbs{}}\PY{l+s+s2}{left(1+0.1}\PY{l+s+s2}{\PYZbs{}}\PY{l+s+s2}{cos}\PY{l+s+s2}{\PYZbs{}}\PY{l+s+s2}{left(t}\PY{l+s+s2}{\PYZbs{}}\PY{l+s+s2}{right)}\PY{l+s+s2}{\PYZbs{}}\PY{l+s+s2}{right)}\PY{l+s+s2}{\PYZbs{}}\PY{l+s+s2}{cos}\PY{l+s+s2}{\PYZbs{}}\PY{l+s+s2}{left(10t}\PY{l+s+s2}{\PYZbs{}}\PY{l+s+s2}{right)\PYZdl{}}\PY{l+s+s2}{\PYZdq{}}\PY{p}{)}
         \PY{n}{plt}\PY{o}{.}\PY{n}{grid}\PY{p}{(}\PY{k+kc}{True}\PY{p}{)}
          
         \PY{c+c1}{\PYZsh{} Figure 2}
         \PY{n}{plt}\PY{o}{.}\PY{n}{subplot}\PY{p}{(}\PY{l+m+mi}{2}\PY{p}{,}\PY{l+m+mi}{1}\PY{p}{,}\PY{l+m+mi}{2}\PY{p}{)}
         \PY{c+c1}{\PYZsh{}plt.plot(w,np.angle(Y),\PYZsq{}ro\PYZsq{},lw=2)}
         \PY{n}{ii}\PY{o}{=}\PY{n}{np}\PY{o}{.}\PY{n}{where}\PY{p}{(}\PY{n}{np}\PY{o}{.}\PY{n}{abs}\PY{p}{(}\PY{n}{Y}\PY{p}{)}\PY{o}{\PYZgt{}}\PY{l+m+mf}{1e\PYZhy{}3}\PY{p}{)}
         \PY{n}{plt}\PY{o}{.}\PY{n}{plot}\PY{p}{(}\PY{n}{w}\PY{p}{[}\PY{n}{ii}\PY{p}{]}\PY{p}{,}\PY{n}{np}\PY{o}{.}\PY{n}{angle}\PY{p}{(}\PY{n}{Y}\PY{p}{[}\PY{n}{ii}\PY{p}{]}\PY{p}{)}\PY{p}{,}\PY{l+s+s1}{\PYZsq{}}\PY{l+s+s1}{go}\PY{l+s+s1}{\PYZsq{}}\PY{p}{,}\PY{n}{lw}\PY{o}{=}\PY{l+m+mi}{2}\PY{p}{)}
         \PY{n}{plt}\PY{o}{.}\PY{n}{xlim}\PY{p}{(}\PY{p}{[}\PY{o}{\PYZhy{}}\PY{l+m+mi}{15}\PY{p}{,}\PY{l+m+mi}{15}\PY{p}{]}\PY{p}{)}
         \PY{n}{plt}\PY{o}{.}\PY{n}{ylim}\PY{p}{(}\PY{p}{[}\PY{o}{\PYZhy{}}\PY{l+m+mi}{3}\PY{p}{,}\PY{l+m+mi}{3}\PY{p}{]}\PY{p}{)}
         \PY{n}{plt}\PY{o}{.}\PY{n}{ylabel}\PY{p}{(}\PY{l+s+sa}{r}\PY{l+s+s2}{\PYZdq{}}\PY{l+s+s2}{Phase of \PYZdl{}Y\PYZdl{}}\PY{l+s+s2}{\PYZdq{}}\PY{p}{,}\PY{n}{size}\PY{o}{=}\PY{l+m+mi}{16}\PY{p}{)}
         \PY{n}{plt}\PY{o}{.}\PY{n}{xlabel}\PY{p}{(}\PY{l+s+sa}{r}\PY{l+s+s2}{\PYZdq{}}\PY{l+s+s2}{\PYZdl{}}\PY{l+s+s2}{\PYZbs{}}\PY{l+s+s2}{omega\PYZdl{}}\PY{l+s+s2}{\PYZdq{}}\PY{p}{,}\PY{n}{size}\PY{o}{=}\PY{l+m+mi}{16}\PY{p}{)}
         \PY{n}{plt}\PY{o}{.}\PY{n}{grid}\PY{p}{(}\PY{k+kc}{True}\PY{p}{)}
         \PY{n}{plt}\PY{o}{.}\PY{n}{show}\PY{p}{(}\PY{p}{)}
\end{Verbatim}


    \begin{Verbatim}[commandchars=\\\{\}]
/Users/Ankivarun/anaconda3/envs/tf\_python3/lib/python3.6/site-packages/matplotlib/font\_manager.py:1316: UserWarning: findfont: Font family ['serif'] not found. Falling back to DejaVu Sans
  (prop.get\_family(), self.defaultFamily[fontext]))

    \end{Verbatim}

    \begin{center}
    \adjustimage{max size={0.9\linewidth}{0.9\paperheight}}{output_13_1.png}
    \end{center}
    { \hspace*{\fill} \\}
    
    \subsubsection{Writting a class to handle
these}\label{writting-a-class-to-handle-these}

Since the assignment involves examining the DFT in various cases of
functions, we might as well wrap all of these into a single class.

    \begin{Verbatim}[commandchars=\\\{\}]
{\color{incolor}In [{\color{incolor}73}]:} \PY{k}{class} \PY{n+nc}{DFT\PYZus{}assgn}\PY{p}{(}\PY{n+nb}{object}\PY{p}{)}\PY{p}{:}
             \PY{k}{def} \PY{n+nf}{\PYZus{}\PYZus{}init\PYZus{}\PYZus{}}\PY{p}{(}\PY{n+nb+bp}{self}\PY{p}{)}\PY{p}{:}
                 \PY{k}{pass}
             \PY{k}{def} \PY{n+nf}{plot\PYZus{}fft}\PY{p}{(}\PY{n+nb+bp}{self}\PY{p}{,}\PY{n}{y}\PY{p}{,}\PY{n}{title}\PY{p}{,}\PY{n}{samples}\PY{o}{=}\PY{l+m+mi}{512}\PY{p}{,}\PY{n}{sup}\PY{o}{=}\PY{l+m+mf}{1e\PYZhy{}3}\PY{p}{,}\PY{n}{freq\PYZus{}max}\PY{o}{=}\PY{l+m+mi}{64}\PY{p}{,}\PY{n}{x\PYZus{}lim}\PY{o}{=}\PY{l+m+mi}{15}\PY{p}{,}\PY{n}{y\PYZus{}freq\PYZus{}lim}\PY{o}{=}\PY{l+m+mi}{3}\PY{p}{,}\PY{n}{verbose}\PY{o}{=}\PY{k+kc}{False}\PY{p}{,}\PY{n}{funky}\PY{o}{=}\PY{k+kc}{False}\PY{p}{,}\PY{n}{return\PYZus{}val}\PY{o}{=}\PY{k+kc}{False}\PY{p}{)}\PY{p}{:}
                 \PY{c+c1}{\PYZsh{} y is the array DFT is to be performed on}
                 \PY{n}{Y}\PY{o}{=}\PY{n}{np}\PY{o}{.}\PY{n}{fft}\PY{o}{.}\PY{n}{fftshift}\PY{p}{(}\PY{n}{np}\PY{o}{.}\PY{n}{fft}\PY{o}{.}\PY{n}{fft}\PY{p}{(}\PY{n}{y}\PY{p}{)}\PY{p}{)}\PY{o}{/}\PY{p}{(}\PY{n}{samples}\PY{p}{)}
                 \PY{k}{if} \PY{n}{funky}\PY{p}{:}
                     \PY{n}{Y}\PY{o}{=}\PY{n}{np}\PY{o}{.}\PY{n}{fft}\PY{o}{.}\PY{n}{fftshift}\PY{p}{(}\PY{n}{np}\PY{o}{.}\PY{n}{fft}\PY{o}{.}\PY{n}{fft}\PY{p}{(}\PY{n}{np}\PY{o}{.}\PY{n}{fft}\PY{o}{.}\PY{n}{ifftshift}\PY{p}{(}\PY{n}{y}\PY{p}{)}\PY{p}{)}\PY{p}{)}\PY{o}{/}\PY{p}{(}\PY{n}{samples}\PY{p}{)}
                 \PY{n}{w}\PY{o}{=}\PY{n}{np}\PY{o}{.}\PY{n}{linspace}\PY{p}{(}\PY{o}{\PYZhy{}}\PY{n}{freq\PYZus{}max}\PY{p}{,}\PY{n}{freq\PYZus{}max}\PY{o}{\PYZhy{}}\PY{l+m+mi}{1}\PY{p}{,}\PY{n}{samples}\PY{o}{+}\PY{l+m+mi}{1}\PY{p}{)}\PY{p}{;}\PY{n}{w}\PY{o}{=}\PY{n}{w}\PY{p}{[}\PY{p}{:}\PY{o}{\PYZhy{}}\PY{l+m+mi}{1}\PY{p}{]}
         
                 \PY{c+c1}{\PYZsh{} Figure 1}
                 \PY{n}{plt}\PY{o}{.}\PY{n}{figure}\PY{p}{(}\PY{p}{)}
                 \PY{n}{plt}\PY{o}{.}\PY{n}{subplot}\PY{p}{(}\PY{l+m+mi}{2}\PY{p}{,}\PY{l+m+mi}{1}\PY{p}{,}\PY{l+m+mi}{1}\PY{p}{)}
                 \PY{n}{plt}\PY{o}{.}\PY{n}{plot}\PY{p}{(}\PY{n}{w}\PY{p}{,}\PY{n+nb}{abs}\PY{p}{(}\PY{n}{Y}\PY{p}{)}\PY{p}{,}\PY{n}{lw}\PY{o}{=}\PY{l+m+mi}{2}\PY{p}{)}
                 \PY{n}{plt}\PY{o}{.}\PY{n}{xlim}\PY{p}{(}\PY{p}{[}\PY{o}{\PYZhy{}}\PY{n}{x\PYZus{}lim}\PY{p}{,}\PY{n}{x\PYZus{}lim}\PY{p}{]}\PY{p}{)}
                 \PY{n}{plt}\PY{o}{.}\PY{n}{ylabel}\PY{p}{(}\PY{l+s+sa}{r}\PY{l+s+s2}{\PYZdq{}}\PY{l+s+s2}{\PYZdl{}|Y|\PYZdl{}}\PY{l+s+s2}{\PYZdq{}}\PY{p}{,}\PY{n}{size}\PY{o}{=}\PY{l+m+mi}{16}\PY{p}{)}
                 \PY{n}{plt}\PY{o}{.}\PY{n}{title}\PY{p}{(}\PY{n}{title}\PY{p}{)}
                 \PY{n}{plt}\PY{o}{.}\PY{n}{grid}\PY{p}{(}\PY{k+kc}{True}\PY{p}{)}
         
                 \PY{c+c1}{\PYZsh{} Figure 2}
                 \PY{n}{plt}\PY{o}{.}\PY{n}{subplot}\PY{p}{(}\PY{l+m+mi}{2}\PY{p}{,}\PY{l+m+mi}{1}\PY{p}{,}\PY{l+m+mi}{2}\PY{p}{)}
                 \PY{c+c1}{\PYZsh{}plt.plot(w,np.angle(Y),\PYZsq{}ro\PYZsq{},lw=2)}
                 \PY{n}{ii}\PY{o}{=}\PY{n}{np}\PY{o}{.}\PY{n}{where}\PY{p}{(}\PY{n}{np}\PY{o}{.}\PY{n}{abs}\PY{p}{(}\PY{n}{Y}\PY{p}{)}\PY{o}{\PYZgt{}}\PY{n}{sup}\PY{p}{)}
                 \PY{k}{if} \PY{n}{verbose}\PY{p}{:}
                     \PY{n}{display}\PY{p}{(}\PY{n}{Y}\PY{p}{[}\PY{n}{ii}\PY{p}{]}\PY{p}{)}
                 \PY{n}{plt}\PY{o}{.}\PY{n}{plot}\PY{p}{(}\PY{n}{w}\PY{p}{[}\PY{n}{ii}\PY{p}{]}\PY{p}{,}\PY{n}{np}\PY{o}{.}\PY{n}{angle}\PY{p}{(}\PY{n}{Y}\PY{p}{[}\PY{n}{ii}\PY{p}{]}\PY{p}{)}\PY{p}{,}\PY{l+s+s1}{\PYZsq{}}\PY{l+s+s1}{go}\PY{l+s+s1}{\PYZsq{}}\PY{p}{,}\PY{n}{lw}\PY{o}{=}\PY{l+m+mi}{2}\PY{p}{)}
                 \PY{n}{plt}\PY{o}{.}\PY{n}{xlim}\PY{p}{(}\PY{p}{[}\PY{o}{\PYZhy{}}\PY{n}{x\PYZus{}lim}\PY{p}{,}\PY{n}{x\PYZus{}lim}\PY{p}{]}\PY{p}{)}
                 \PY{n}{plt}\PY{o}{.}\PY{n}{ylim}\PY{p}{(}\PY{p}{[}\PY{o}{\PYZhy{}}\PY{n}{y\PYZus{}freq\PYZus{}lim}\PY{p}{,}\PY{n}{y\PYZus{}freq\PYZus{}lim}\PY{p}{]}\PY{p}{)}
                 \PY{n}{plt}\PY{o}{.}\PY{n}{ylabel}\PY{p}{(}\PY{l+s+sa}{r}\PY{l+s+s2}{\PYZdq{}}\PY{l+s+s2}{Phase of \PYZdl{}Y\PYZdl{}}\PY{l+s+s2}{\PYZdq{}}\PY{p}{,}\PY{n}{size}\PY{o}{=}\PY{l+m+mi}{16}\PY{p}{)}
                 \PY{n}{plt}\PY{o}{.}\PY{n}{xlabel}\PY{p}{(}\PY{l+s+sa}{r}\PY{l+s+s2}{\PYZdq{}}\PY{l+s+s2}{\PYZdl{}}\PY{l+s+s2}{\PYZbs{}}\PY{l+s+s2}{omega\PYZdl{}}\PY{l+s+s2}{\PYZdq{}}\PY{p}{,}\PY{n}{size}\PY{o}{=}\PY{l+m+mi}{16}\PY{p}{)}
                 \PY{n}{plt}\PY{o}{.}\PY{n}{grid}\PY{p}{(}\PY{k+kc}{True}\PY{p}{)}
                 \PY{n}{plt}\PY{o}{.}\PY{n}{show}\PY{p}{(}\PY{p}{)}
                 
                 \PY{k}{if} \PY{n}{return\PYZus{}val}\PY{p}{:}
                     \PY{k}{return} \PY{p}{(}\PY{n}{Y}\PY{p}{,}\PY{n}{w}\PY{p}{)}
\end{Verbatim}


    We test this on the previous example.

    \begin{Verbatim}[commandchars=\\\{\}]
{\color{incolor}In [{\color{incolor}74}]:} \PY{n}{a}\PY{o}{=}\PY{n}{DFT\PYZus{}assgn}\PY{p}{(}\PY{p}{)}
         \PY{n}{t}\PY{o}{=}\PY{n}{np}\PY{o}{.}\PY{n}{linspace}\PY{p}{(}\PY{o}{\PYZhy{}}\PY{l+m+mi}{4}\PY{o}{*}\PY{n}{np}\PY{o}{.}\PY{n}{pi}\PY{p}{,}\PY{l+m+mi}{4}\PY{o}{*}\PY{n}{np}\PY{o}{.}\PY{n}{pi}\PY{p}{,}\PY{l+m+mi}{513}\PY{p}{)}\PY{p}{;}\PY{n}{t}\PY{o}{=}\PY{n}{t}\PY{p}{[}\PY{p}{:}\PY{o}{\PYZhy{}}\PY{l+m+mi}{1}\PY{p}{]}
         \PY{n}{y}\PY{o}{=}\PY{p}{(}\PY{l+m+mi}{1}\PY{o}{+}\PY{l+m+mf}{0.1}\PY{o}{*}\PY{n}{np}\PY{o}{.}\PY{n}{cos}\PY{p}{(}\PY{n}{t}\PY{p}{)}\PY{p}{)}\PY{o}{*}\PY{n}{np}\PY{o}{.}\PY{n}{cos}\PY{p}{(}\PY{l+m+mi}{10}\PY{o}{*}\PY{n}{t}\PY{p}{)}
         \PY{n}{a}\PY{o}{.}\PY{n}{plot\PYZus{}fft}\PY{p}{(}\PY{n}{y}\PY{p}{,}\PY{l+s+sa}{r}\PY{l+s+s2}{\PYZdq{}}\PY{l+s+s2}{Spectrum of \PYZdl{}}\PY{l+s+s2}{\PYZbs{}}\PY{l+s+s2}{left(1+0.1}\PY{l+s+s2}{\PYZbs{}}\PY{l+s+s2}{cos}\PY{l+s+s2}{\PYZbs{}}\PY{l+s+s2}{left(t}\PY{l+s+s2}{\PYZbs{}}\PY{l+s+s2}{right)}\PY{l+s+s2}{\PYZbs{}}\PY{l+s+s2}{right)}\PY{l+s+s2}{\PYZbs{}}\PY{l+s+s2}{cos}\PY{l+s+s2}{\PYZbs{}}\PY{l+s+s2}{left(10t}\PY{l+s+s2}{\PYZbs{}}\PY{l+s+s2}{right)\PYZdl{}}\PY{l+s+s2}{\PYZdq{}}\PY{p}{)}
\end{Verbatim}


    \begin{Verbatim}[commandchars=\\\{\}]
/Users/Ankivarun/anaconda3/envs/tf\_python3/lib/python3.6/site-packages/matplotlib/font\_manager.py:1316: UserWarning: findfont: Font family ['serif'] not found. Falling back to DejaVu Sans
  (prop.get\_family(), self.defaultFamily[fontext]))

    \end{Verbatim}

    \begin{center}
    \adjustimage{max size={0.9\linewidth}{0.9\paperheight}}{output_17_1.png}
    \end{center}
    { \hspace*{\fill} \\}
    
    \subsection{Assignment Questions}\label{assignment-questions}

For the sake of brevity, we shall use the defined class. If the default
parameters donot give satisfactory or intended results, we supply
suitable arguments. This has been mentioned as and when necessary.

\subsubsection{Sinusoids}\label{sinusoids}

We examine \(sin^3t\) and \(cos^3t\).

Note that, \[ sin^3t=\frac{3sin(t)-sin(3t)}{4} \]

and

\[ cos^3t=\frac{3cos(t)+cos(3t)}{4} \]

We therefore expect 4 impulses in both cases, with one pair having
thrice the magintude of the other. The phase plots in the two cases will
however differ, with the sinusoids having alternative \(\pm \pi/2\).

    \begin{Verbatim}[commandchars=\\\{\}]
{\color{incolor}In [{\color{incolor}75}]:} \PY{n}{t}\PY{o}{=}\PY{n}{np}\PY{o}{.}\PY{n}{linspace}\PY{p}{(}\PY{o}{\PYZhy{}}\PY{l+m+mi}{4}\PY{o}{*}\PY{n}{np}\PY{o}{.}\PY{n}{pi}\PY{p}{,}\PY{l+m+mi}{4}\PY{o}{*}\PY{n}{np}\PY{o}{.}\PY{n}{pi}\PY{p}{,}\PY{l+m+mi}{513}\PY{p}{)}\PY{p}{;}\PY{n}{t}\PY{o}{=}\PY{n}{t}\PY{p}{[}\PY{p}{:}\PY{o}{\PYZhy{}}\PY{l+m+mi}{1}\PY{p}{]}
         \PY{n}{y}\PY{o}{=}\PY{n}{np}\PY{o}{.}\PY{n}{sin}\PY{p}{(}\PY{n}{t}\PY{p}{)}\PY{o}{*}\PY{o}{*}\PY{l+m+mi}{3}
         \PY{n}{a}\PY{o}{.}\PY{n}{plot\PYZus{}fft}\PY{p}{(}\PY{n}{y}\PY{p}{,}\PY{l+s+sa}{r}\PY{l+s+s2}{\PYZdq{}}\PY{l+s+s2}{Spectrum of \PYZdl{}sin\PYZca{}}\PY{l+s+si}{\PYZob{}3\PYZcb{}}\PY{l+s+s2}{t\PYZdl{}}\PY{l+s+s2}{\PYZdq{}}\PY{p}{)}
\end{Verbatim}


    \begin{Verbatim}[commandchars=\\\{\}]
/Users/Ankivarun/anaconda3/envs/tf\_python3/lib/python3.6/site-packages/matplotlib/font\_manager.py:1316: UserWarning: findfont: Font family ['serif'] not found. Falling back to DejaVu Sans
  (prop.get\_family(), self.defaultFamily[fontext]))

    \end{Verbatim}

    \begin{center}
    \adjustimage{max size={0.9\linewidth}{0.9\paperheight}}{output_19_1.png}
    \end{center}
    { \hspace*{\fill} \\}
    
    \begin{Verbatim}[commandchars=\\\{\}]
{\color{incolor}In [{\color{incolor}76}]:} \PY{n}{t}\PY{o}{=}\PY{n}{np}\PY{o}{.}\PY{n}{linspace}\PY{p}{(}\PY{o}{\PYZhy{}}\PY{l+m+mi}{4}\PY{o}{*}\PY{n}{np}\PY{o}{.}\PY{n}{pi}\PY{p}{,}\PY{l+m+mi}{4}\PY{o}{*}\PY{n}{np}\PY{o}{.}\PY{n}{pi}\PY{p}{,}\PY{l+m+mi}{513}\PY{p}{)}\PY{p}{;}\PY{n}{t}\PY{o}{=}\PY{n}{t}\PY{p}{[}\PY{p}{:}\PY{o}{\PYZhy{}}\PY{l+m+mi}{1}\PY{p}{]}
         \PY{n}{y}\PY{o}{=}\PY{n}{np}\PY{o}{.}\PY{n}{cos}\PY{p}{(}\PY{n}{t}\PY{p}{)}\PY{o}{*}\PY{o}{*}\PY{l+m+mi}{3}
         \PY{n}{a}\PY{o}{.}\PY{n}{plot\PYZus{}fft}\PY{p}{(}\PY{n}{y}\PY{p}{,}\PY{l+s+sa}{r}\PY{l+s+s2}{\PYZdq{}}\PY{l+s+s2}{Spectrum of \PYZdl{}cos\PYZca{}}\PY{l+s+si}{\PYZob{}3\PYZcb{}}\PY{l+s+s2}{t\PYZdl{}}\PY{l+s+s2}{\PYZdq{}}\PY{p}{)}
\end{Verbatim}


    \begin{Verbatim}[commandchars=\\\{\}]
/Users/Ankivarun/anaconda3/envs/tf\_python3/lib/python3.6/site-packages/matplotlib/font\_manager.py:1316: UserWarning: findfont: Font family ['serif'] not found. Falling back to DejaVu Sans
  (prop.get\_family(), self.defaultFamily[fontext]))

    \end{Verbatim}

    \begin{center}
    \adjustimage{max size={0.9\linewidth}{0.9\paperheight}}{output_20_1.png}
    \end{center}
    { \hspace*{\fill} \\}
    
    \subsubsection{Frequency Modulation}\label{frequency-modulation}

Here, we examine the spectrum of \(cos(20t+5cos(t))\).

In order to view the whole spectrum, the frequency x limits are altered
to 35 from 15.

Note that,

\[ cos(20t+5cos(t))=\Re(\exp({20t + 5cos(t)}))\]
\[= \Re(\exp({j20t)})\sum_{k=-\infty}^{\infty}J_{k}({\beta})e^{jkt-\pi/2})\]
\[= \sum_{k=-\infty}^{\infty}J_{k}({\beta})\cos({kt+20t-\pi/2}) \]
\[= \sum_{k=-\infty}^{\infty}J_{k}({\beta})\sin({kt+20t}) \]

Here, \[ \beta=5 \]

So, \[= \sum_{k=-\infty}^{\infty}J_{k}({5})\sin({kt+20t}) \]

Also, by the Laurent expansion defenition for Bessel Functions,
\[ \exp({\frac{\beta(z-1/z)}{2}})= \sum_{k=-\infty}^{\infty}J_{k}({\beta})z^{k}\]

\paragraph{References}\label{references}

\begin{enumerate}
\def\labelenumi{\arabic{enumi}.}
\tightlist
\item
  https://www.dsprelated.com/freebooks/mdft/Sinusoidal\_Frequency\_Modulation\_FM.html
\end{enumerate}

    \begin{Verbatim}[commandchars=\\\{\}]
{\color{incolor}In [{\color{incolor}77}]:} \PY{n}{t}\PY{o}{=}\PY{n}{np}\PY{o}{.}\PY{n}{linspace}\PY{p}{(}\PY{o}{\PYZhy{}}\PY{l+m+mi}{4}\PY{o}{*}\PY{n}{np}\PY{o}{.}\PY{n}{pi}\PY{p}{,}\PY{l+m+mi}{4}\PY{o}{*}\PY{n}{np}\PY{o}{.}\PY{n}{pi}\PY{p}{,}\PY{l+m+mi}{2049}\PY{p}{)}\PY{p}{;}\PY{n}{t}\PY{o}{=}\PY{n}{t}\PY{p}{[}\PY{p}{:}\PY{o}{\PYZhy{}}\PY{l+m+mi}{1}\PY{p}{]}
         \PY{n}{y}\PY{o}{=}\PY{n}{np}\PY{o}{.}\PY{n}{cos}\PY{p}{(}\PY{l+m+mi}{20}\PY{o}{*}\PY{n}{t}\PY{o}{+}\PY{l+m+mi}{5}\PY{o}{*}\PY{n}{np}\PY{o}{.}\PY{n}{cos}\PY{p}{(}\PY{n}{t}\PY{p}{)}\PY{p}{)}
         \PY{n}{a}\PY{o}{.}\PY{n}{plot\PYZus{}fft}\PY{p}{(}\PY{n}{y}\PY{p}{,}\PY{l+s+sa}{r}\PY{l+s+s2}{\PYZdq{}}\PY{l+s+s2}{Spectrum of FM signal}\PY{l+s+s2}{\PYZdq{}}\PY{p}{,}\PY{n}{samples}\PY{o}{=}\PY{l+m+mi}{2048}\PY{p}{,}\PY{n}{x\PYZus{}lim}\PY{o}{=}\PY{l+m+mi}{10}\PY{p}{,}\PY{n}{y\PYZus{}freq\PYZus{}lim}\PY{o}{=}\PY{l+m+mi}{4}\PY{p}{)}
\end{Verbatim}


    \begin{Verbatim}[commandchars=\\\{\}]
/Users/Ankivarun/anaconda3/envs/tf\_python3/lib/python3.6/site-packages/matplotlib/font\_manager.py:1316: UserWarning: findfont: Font family ['serif'] not found. Falling back to DejaVu Sans
  (prop.get\_family(), self.defaultFamily[fontext]))

    \end{Verbatim}

    \begin{center}
    \adjustimage{max size={0.9\linewidth}{0.9\paperheight}}{output_22_1.png}
    \end{center}
    { \hspace*{\fill} \\}
    
    We notice the similarity with the Bessel expansion by sampling the same.
A continuous version of the same is shown below.

    \begin{Verbatim}[commandchars=\\\{\}]
{\color{incolor}In [{\color{incolor}78}]:} \PY{c+c1}{\PYZsh{} Figure 1}
         \PY{n}{plt}\PY{o}{.}\PY{n}{figure}\PY{p}{(}\PY{p}{)}
         \PY{k+kn}{from} \PY{n+nn}{scipy}\PY{n+nn}{.}\PY{n+nn}{special} \PY{k}{import} \PY{n}{jv} \PY{k}{as} \PY{n}{Bess}
         \PY{n}{w}\PY{o}{=}\PY{n}{np}\PY{o}{.}\PY{n}{linspace}\PY{p}{(}\PY{o}{\PYZhy{}}\PY{l+m+mi}{100}\PY{p}{,}\PY{l+m+mi}{100}\PY{p}{,}\PY{l+m+mi}{1025}\PY{p}{)}
         \PY{n}{w}\PY{o}{=}\PY{n}{w}\PY{p}{[}\PY{p}{:}\PY{o}{\PYZhy{}}\PY{l+m+mi}{1}\PY{p}{]}
         \PY{n}{p}\PY{o}{=}\PY{n}{np}\PY{o}{.}\PY{n}{abs}\PY{p}{(}\PY{n}{Bess}\PY{p}{(}\PY{l+m+mi}{5}\PY{p}{,}\PY{n}{w}\PY{p}{)}\PY{p}{)}
         \PY{n}{p}\PY{p}{[}\PY{l+m+mi}{74}\PY{p}{:}\PY{p}{]}\PY{o}{+}\PY{o}{=}\PY{n}{p}\PY{p}{[}\PY{p}{:}\PY{o}{\PYZhy{}}\PY{l+m+mi}{74}\PY{p}{]}
         \PY{n}{p}\PY{p}{[}\PY{n}{np}\PY{o}{.}\PY{n}{where}\PY{p}{(}\PY{n}{p}\PY{o}{\PYZlt{}}\PY{l+m+mf}{1e\PYZhy{}3}\PY{p}{)}\PY{p}{]}\PY{o}{=}\PY{l+m+mi}{0}
         \PY{n}{plt}\PY{o}{.}\PY{n}{plot}\PY{p}{(}\PY{n}{w}\PY{p}{,}\PY{n}{p}\PY{p}{,}\PY{n}{lw}\PY{o}{=}\PY{l+m+mi}{2}\PY{p}{)}
         \PY{n}{plt}\PY{o}{.}\PY{n}{grid}\PY{p}{(}\PY{k+kc}{True}\PY{p}{)}
         \PY{n}{plt}\PY{o}{.}\PY{n}{show}\PY{p}{(}\PY{p}{)}
\end{Verbatim}


    \begin{Verbatim}[commandchars=\\\{\}]
/Users/Ankivarun/anaconda3/envs/tf\_python3/lib/python3.6/site-packages/matplotlib/font\_manager.py:1316: UserWarning: findfont: Font family ['serif'] not found. Falling back to DejaVu Sans
  (prop.get\_family(), self.defaultFamily[fontext]))

    \end{Verbatim}

    \begin{center}
    \adjustimage{max size={0.9\linewidth}{0.9\paperheight}}{output_24_1.png}
    \end{center}
    { \hspace*{\fill} \\}
    
    \subsection{Gaussian case}\label{gaussian-case}

The Gaussian \(e^{−t^2/2}\) is not ``bandlimited'' in frequency. We want
to get its spectrum accurate to 6 digits, by different time ranges, and
see what gets us a frequency domain that is so accurate.

A couple of subtilities involved here:

\begin{itemize}
\tightlist
\item
  The standard fftgauss = fftshift(fft(gauss)); fails here, since we get
  negative real values.
\item
  The argument \emph{funky} alters this to fftgauss =
  fftshift(fft(ifftshift(gauss)))
\end{itemize}

This is an analogous problem in the time domain, where the FFT method
warps around the wromg intial value of time, thereby introducing a phase
error. Ifft\_shift undoes this, centring our gaussian at 0.

\paragraph{References}\label{references}

\begin{enumerate}
\def\labelenumi{\arabic{enumi}.}
\tightlist
\item
  https://in.mathworks.com/matlabcentral/answers/40257-gaussian-fft
\end{enumerate}

    \begin{Verbatim}[commandchars=\\\{\}]
{\color{incolor}In [{\color{incolor}79}]:} \PY{n}{t}\PY{o}{=}\PY{n}{np}\PY{o}{.}\PY{n}{linspace}\PY{p}{(}\PY{o}{\PYZhy{}}\PY{l+m+mi}{100}\PY{o}{*}\PY{n}{np}\PY{o}{.}\PY{n}{pi}\PY{p}{,}\PY{l+m+mi}{100}\PY{o}{*}\PY{n}{np}\PY{o}{.}\PY{n}{pi}\PY{p}{,}\PY{l+m+mi}{1025}\PY{p}{)}\PY{p}{;}\PY{n}{t}\PY{o}{=}\PY{n}{t}\PY{p}{[}\PY{p}{:}\PY{o}{\PYZhy{}}\PY{l+m+mi}{1}\PY{p}{]}
         \PY{n}{y}\PY{o}{=}\PY{n}{np}\PY{o}{.}\PY{n}{exp}\PY{p}{(}\PY{o}{\PYZhy{}}\PY{n}{t}\PY{o}{*}\PY{o}{*}\PY{l+m+mi}{2}\PY{o}{/}\PY{l+m+mi}{2}\PY{p}{)}
         \PY{n}{a}\PY{o}{.}\PY{n}{plot\PYZus{}fft}\PY{p}{(}\PY{n}{y}\PY{p}{,}\PY{l+s+sa}{r}\PY{l+s+s2}{\PYZdq{}}\PY{l+s+s2}{Spectrum of Gaussian,\PYZdl{}e\PYZca{}}\PY{l+s+s2}{\PYZob{}}\PY{l+s+s2}{−t\PYZca{}2/2\PYZcb{}\PYZdl{}  }\PY{l+s+s2}{\PYZdq{}}\PY{p}{,}\PY{n}{samples}\PY{o}{=}\PY{l+m+mi}{1024}\PY{p}{,}\PY{n}{sup}\PY{o}{=}\PY{l+m+mf}{1e\PYZhy{}6}\PY{p}{,}\PY{n}{freq\PYZus{}max}\PY{o}{=}\PY{l+m+mi}{300}\PY{p}{,}\PY{n}{x\PYZus{}lim}\PY{o}{=}\PY{l+m+mi}{300}\PY{p}{,}\PY{n}{y\PYZus{}freq\PYZus{}lim}\PY{o}{=}\PY{l+m+mi}{4}\PY{p}{,}\PY{n}{verbose}\PY{o}{=}\PY{k+kc}{False}\PY{p}{,}\PY{n}{funky}\PY{o}{=}\PY{k+kc}{True}\PY{p}{)}
\end{Verbatim}


    \begin{Verbatim}[commandchars=\\\{\}]
/Users/Ankivarun/anaconda3/envs/tf\_python3/lib/python3.6/site-packages/matplotlib/font\_manager.py:1316: UserWarning: findfont: Font family ['serif'] not found. Falling back to DejaVu Sans
  (prop.get\_family(), self.defaultFamily[fontext]))

    \end{Verbatim}

    \begin{center}
    \adjustimage{max size={0.9\linewidth}{0.9\paperheight}}{output_26_1.png}
    \end{center}
    { \hspace*{\fill} \\}
    
    \subsubsection{Converging to a given error
tolerance}\label{converging-to-a-given-error-tolerance}

Here, we vary the windowing region and compare the output to a Gaussian:

\[ X(\omega)= \sqrt{2\pi }e^{-\omega^{2}/2} \]

We try symmetric values of \(t\), ie.. \((-t_o,t_o)\).

    \begin{Verbatim}[commandchars=\\\{\}]
{\color{incolor}In [{\color{incolor}80}]:} \PY{c+c1}{\PYZsh{}\PYZsh{}\PYZsh{} Symmetric time span limits}
         \PY{n}{t\PYZus{}s}\PY{o}{=}\PY{n}{np}\PY{o}{.}\PY{n}{linspace}\PY{p}{(}\PY{l+m+mi}{10}\PY{p}{,}\PY{l+m+mi}{100}\PY{p}{,}\PY{n}{num}\PY{o}{=}\PY{l+m+mi}{4}\PY{p}{)}
         \PY{n}{samples\PYZus{}s}\PY{o}{=}\PY{n}{np}\PY{o}{.}\PY{n}{ones}\PY{p}{(}\PY{n+nb}{len}\PY{p}{(}\PY{n}{t\PYZus{}s}\PY{p}{)}\PY{p}{)}\PY{o}{*}\PY{l+m+mi}{4097}
         \PY{n}{Y\PYZus{}w\PYZus{}list}\PY{o}{=}\PY{p}{[}\PY{p}{]}
         
         \PY{k}{for} \PY{n}{i} \PY{o+ow}{in} \PY{n+nb}{range}\PY{p}{(}\PY{n+nb}{len}\PY{p}{(}\PY{n}{t\PYZus{}s}\PY{p}{)}\PY{p}{)}\PY{p}{:}
             \PY{n}{t}\PY{o}{=}\PY{n}{np}\PY{o}{.}\PY{n}{linspace}\PY{p}{(}\PY{o}{\PYZhy{}}\PY{n}{t\PYZus{}s}\PY{p}{[}\PY{n}{i}\PY{p}{]}\PY{o}{*}\PY{n}{np}\PY{o}{.}\PY{n}{pi}\PY{p}{,}\PY{n}{t\PYZus{}s}\PY{p}{[}\PY{n}{i}\PY{p}{]}\PY{o}{*}\PY{n}{np}\PY{o}{.}\PY{n}{pi}\PY{p}{,}\PY{n}{samples\PYZus{}s}\PY{p}{[}\PY{n}{i}\PY{p}{]}\PY{p}{)}\PY{p}{;}\PY{n}{t}\PY{o}{=}\PY{n}{t}\PY{p}{[}\PY{p}{:}\PY{o}{\PYZhy{}}\PY{l+m+mi}{1}\PY{p}{]}
             \PY{n}{y}\PY{o}{=}\PY{n}{np}\PY{o}{.}\PY{n}{exp}\PY{p}{(}\PY{o}{\PYZhy{}}\PY{n}{t}\PY{o}{*}\PY{o}{*}\PY{l+m+mi}{2}\PY{o}{/}\PY{l+m+mi}{2}\PY{p}{)}
             \PY{n+nb}{print} \PY{p}{(}\PY{l+s+s2}{\PYZdq{}}\PY{l+s+s2}{Case t\PYZus{}i = }\PY{l+s+se}{\PYZbs{}t}\PY{l+s+s2}{\PYZdq{}}\PY{p}{,}\PY{n}{t\PYZus{}s}\PY{p}{[}\PY{n}{i}\PY{p}{]}\PY{p}{)}
             \PY{n}{Y}\PY{p}{,}\PY{n}{w}\PY{o}{=}\PY{n}{a}\PY{o}{.}\PY{n}{plot\PYZus{}fft}\PY{p}{(}\PY{n}{y}\PY{p}{,}\PY{l+s+sa}{r}\PY{l+s+s2}{\PYZdq{}}\PY{l+s+s2}{Spectrum of Gaussian,\PYZdl{}e\PYZca{}}\PY{l+s+s2}{\PYZob{}}\PY{l+s+s2}{−t\PYZca{}2/2\PYZcb{}\PYZdl{}}\PY{l+s+s2}{\PYZdq{}}\PY{p}{,}\PY{n}{samples}\PY{o}{=}\PY{n}{samples\PYZus{}s}\PY{p}{[}\PY{n}{i}\PY{p}{]}\PY{o}{\PYZhy{}}\PY{l+m+mi}{1}\PY{p}{,}\PY{n}{sup}\PY{o}{=}\PY{l+m+mf}{1e\PYZhy{}6}\PY{p}{,}\PY{n}{freq\PYZus{}max}\PY{o}{=}\PY{l+m+mi}{300}\PY{p}{,}\PY{n}{x\PYZus{}lim}\PY{o}{=}\PY{l+m+mi}{100}\PY{p}{,}\PY{n}{y\PYZus{}freq\PYZus{}lim}\PY{o}{=}\PY{l+m+mi}{4}\PY{p}{,}\PYZbs{}
                        \PY{n}{verbose}\PY{o}{=}\PY{k+kc}{False}\PY{p}{,}\PY{n}{funky}\PY{o}{=}\PY{k+kc}{True}\PY{p}{,}\PY{n}{return\PYZus{}val}\PY{o}{=}\PY{k+kc}{True}\PY{p}{)}
             \PY{n}{Y\PYZus{}w\PYZus{}list}\PY{o}{.}\PY{n}{append}\PY{p}{(}\PY{p}{(}\PY{n}{Y}\PY{p}{,}\PY{n}{w}\PY{p}{)}\PY{p}{)}
             \PY{n}{gauss}\PY{o}{=}\PY{l+m+mi}{1}\PY{o}{/}\PY{n}{np}\PY{o}{.}\PY{n}{sqrt}\PY{p}{(}\PY{l+m+mi}{2}\PY{o}{*}\PY{n}{np}\PY{o}{.}\PY{n}{pi}\PY{p}{)}\PY{o}{*}\PY{n}{np}\PY{o}{.}\PY{n}{exp}\PY{p}{(}\PY{o}{\PYZhy{}}\PY{l+m+mi}{4}\PY{o}{*}\PY{n}{w}\PY{o}{*}\PY{o}{*}\PY{l+m+mi}{2}\PY{o}{/}\PY{n}{t\PYZus{}s}\PY{p}{[}\PY{n}{i}\PY{p}{]}\PY{o}{*}\PY{o}{*}\PY{l+m+mi}{2}\PY{p}{)}
             \PY{n+nb}{print}\PY{p}{(}\PY{n+nb}{len}\PY{p}{(}\PY{n}{Y}\PY{p}{)}\PY{p}{)}
             \PY{n+nb}{print}\PY{p}{(}\PY{n+nb}{len}\PY{p}{(}\PY{n}{w}\PY{p}{)}\PY{p}{)}
             \PY{k}{if} \PY{n}{i}\PY{o}{\PYZgt{}}\PY{o}{=}\PY{l+m+mi}{1}\PY{p}{:}
                 \PY{n+nb}{print}\PY{p}{(}\PY{l+s+s2}{\PYZdq{}}\PY{l+s+s2}{Max Error is }\PY{l+s+se}{\PYZbs{}t}\PY{l+s+s2}{\PYZdq{}}\PY{p}{,}\PY{n}{np}\PY{o}{.}\PY{n}{max}\PY{p}{(}\PY{n}{np}\PY{o}{.}\PY{n}{abs}\PY{p}{(}\PY{n}{Y}\PY{o}{*}\PY{n}{t\PYZus{}s}\PY{p}{[}\PY{n}{i}\PY{p}{]}\PY{o}{\PYZhy{}}\PY{n}{Y\PYZus{}w\PYZus{}list}\PY{p}{[}\PY{o}{\PYZhy{}}\PY{l+m+mi}{2}\PY{p}{]}\PY{p}{[}\PY{l+m+mi}{0}\PY{p}{]}\PY{o}{*}\PY{n}{t\PYZus{}s}\PY{p}{[}\PY{n}{i}\PY{o}{\PYZhy{}}\PY{l+m+mi}{1}\PY{p}{]}\PY{p}{)}\PY{o}{/}\PY{n}{samples\PYZus{}s}\PY{p}{[}\PY{n}{i}\PY{p}{]}\PY{p}{)}\PY{p}{)} \PY{c+c1}{\PYZsh{}Y\PYZus{}w\PYZus{}list[\PYZhy{}2][0]*t\PYZus{}s[i\PYZhy{}1]}
             \PY{c+c1}{\PYZsh{}plt.plot(w,gauss)}
\end{Verbatim}


    \begin{Verbatim}[commandchars=\\\{\}]
/Users/Ankivarun/anaconda3/envs/tf\_python3/lib/python3.6/site-packages/ipykernel\_launcher.py:7: DeprecationWarning: object of type <class 'numpy.float64'> cannot be safely interpreted as an integer.
  import sys
/Users/Ankivarun/anaconda3/envs/tf\_python3/lib/python3.6/site-packages/ipykernel\_launcher.py:9: DeprecationWarning: object of type <class 'numpy.float64'> cannot be safely interpreted as an integer.
  if \_\_name\_\_ == '\_\_main\_\_':
/Users/Ankivarun/anaconda3/envs/tf\_python3/lib/python3.6/site-packages/matplotlib/font\_manager.py:1316: UserWarning: findfont: Font family ['serif'] not found. Falling back to DejaVu Sans
  (prop.get\_family(), self.defaultFamily[fontext]))

    \end{Verbatim}

    \begin{Verbatim}[commandchars=\\\{\}]
Case t\_i = 	 10.0

    \end{Verbatim}

    \begin{center}
    \adjustimage{max size={0.9\linewidth}{0.9\paperheight}}{output_28_2.png}
    \end{center}
    { \hspace*{\fill} \\}
    
    \begin{Verbatim}[commandchars=\\\{\}]
4096
4096
Case t\_i = 	 40.0

    \end{Verbatim}

    \begin{center}
    \adjustimage{max size={0.9\linewidth}{0.9\paperheight}}{output_28_4.png}
    \end{center}
    { \hspace*{\fill} \\}
    
    \begin{Verbatim}[commandchars=\\\{\}]
4096
4096
Max Error is 	 7.58677257986226e-05
Case t\_i = 	 70.0

    \end{Verbatim}

    \begin{center}
    \adjustimage{max size={0.9\linewidth}{0.9\paperheight}}{output_28_6.png}
    \end{center}
    { \hspace*{\fill} \\}
    
    \begin{Verbatim}[commandchars=\\\{\}]
4096
4096
Max Error is 	 3.811429339574343e-05
Case t\_i = 	 100.0

    \end{Verbatim}

    \begin{center}
    \adjustimage{max size={0.9\linewidth}{0.9\paperheight}}{output_28_8.png}
    \end{center}
    { \hspace*{\fill} \\}
    
    \begin{Verbatim}[commandchars=\\\{\}]
4096
4096
Max Error is 	 2.502413944628378e-05

    \end{Verbatim}

    Hence, a time-span of \((-100,100)*2\pi\) gives us the required cross
error rate.

    \section{Results and Discussion}\label{results-and-discussion}

We have examined the DFT's of sinusoids, amplitude modulated signals,
frequency modulated modulated signals, exponentiated sinusoids, and
gaussians. In each case, we utilise sampling of the DTFT to explain the
spectrum or have used the discretisation of the continuous array the
signal is derived from. We noted the need for using \texttt{ifftshift}
and \texttt{fftshift} to undistort even signals, and in cases where
phase has been distorted.

\subsection{A Note on the relationship between DTFT and
DFT}\label{a-note-on-the-relationship-between-dtft-and-dft}

While it may inherently seem straightforward that a DFT is the sampled
verison of a DTFT, there are a few catches.

We will need to truncate the DTFT to a finite range of N samples,
without arbitrary loss of power (no constant functions or deltas).

Next, we need to determine if the range of \(x[n]\) should be
\([0...N-1]\) or \([-N/2...N/2-1]\). That will determine if ifftshift
needs to be used.

Note that the continuous \(X (\omega)\) is now discretized or sampled.
Locations of samples are either \(\omega ∈ 2\pi [0...N − 1]\) or
\(\omega ∈ 2\pi [ N / 2...N / 2 − 1]\) depending of whether fftshift has
been used.

This also causes ambiguity in the inverse, due to the dual range for
\(\omega\). Observing that DFT and DFS are closely related, the inverse
results in a periodic construct for \(\tilde{x}[n]\), from which we
recover \(x[n]\)

This creates an ambiguity as to whether \(x[n]\) is defined for \(n\) in
\([0...N-1]\) or \([-N/2...N/2-1]\), hence creating a need for
\texttt{iiftshift}.


    % Add a bibliography block to the postdoc
    
    
    
    \end{document}
