
% Default to the notebook output style

    


% Inherit from the specified cell style.




    
\documentclass[11pt]{article}

    
    
    \usepackage[T1]{fontenc}
    % Nicer default font (+ math font) than Computer Modern for most use cases
    \usepackage{mathpazo}

    % Basic figure setup, for now with no caption control since it's done
    % automatically by Pandoc (which extracts ![](path) syntax from Markdown).
    \usepackage{graphicx}
    % We will generate all images so they have a width \maxwidth. This means
    % that they will get their normal width if they fit onto the page, but
    % are scaled down if they would overflow the margins.
    \makeatletter
    \def\maxwidth{\ifdim\Gin@nat@width>\linewidth\linewidth
    \else\Gin@nat@width\fi}
    \makeatother
    \let\Oldincludegraphics\includegraphics
    % Set max figure width to be 80% of text width, for now hardcoded.
    \renewcommand{\includegraphics}[1]{\Oldincludegraphics[width=.8\maxwidth]{#1}}
    % Ensure that by default, figures have no caption (until we provide a
    % proper Figure object with a Caption API and a way to capture that
    % in the conversion process - todo).
    \usepackage{caption}
    \DeclareCaptionLabelFormat{nolabel}{}
    \captionsetup{labelformat=nolabel}

    \usepackage{adjustbox} % Used to constrain images to a maximum size 
    \usepackage{xcolor} % Allow colors to be defined
    \usepackage{enumerate} % Needed for markdown enumerations to work
    \usepackage{geometry} % Used to adjust the document margins
    \usepackage{amsmath} % Equations
    \usepackage{amssymb} % Equations
    \usepackage{textcomp} % defines textquotesingle
    % Hack from http://tex.stackexchange.com/a/47451/13684:
    \AtBeginDocument{%
        \def\PYZsq{\textquotesingle}% Upright quotes in Pygmentized code
    }
    \usepackage{upquote} % Upright quotes for verbatim code
    \usepackage{eurosym} % defines \euro
    \usepackage[mathletters]{ucs} % Extended unicode (utf-8) support
    \usepackage[utf8x]{inputenc} % Allow utf-8 characters in the tex document
    \usepackage{fancyvrb} % verbatim replacement that allows latex
    \usepackage{grffile} % extends the file name processing of package graphics 
                         % to support a larger range 
    % The hyperref package gives us a pdf with properly built
    % internal navigation ('pdf bookmarks' for the table of contents,
    % internal cross-reference links, web links for URLs, etc.)
    \usepackage{hyperref}
    \usepackage{longtable} % longtable support required by pandoc >1.10
    \usepackage{booktabs}  % table support for pandoc > 1.12.2
    \usepackage[inline]{enumitem} % IRkernel/repr support (it uses the enumerate* environment)
    \usepackage[normalem]{ulem} % ulem is needed to support strikethroughs (\sout)
                                % normalem makes italics be italics, not underlines
    

    
    
    % Colors for the hyperref package
    \definecolor{urlcolor}{rgb}{0,.145,.698}
    \definecolor{linkcolor}{rgb}{.71,0.21,0.01}
    \definecolor{citecolor}{rgb}{.12,.54,.11}

    % ANSI colors
    \definecolor{ansi-black}{HTML}{3E424D}
    \definecolor{ansi-black-intense}{HTML}{282C36}
    \definecolor{ansi-red}{HTML}{E75C58}
    \definecolor{ansi-red-intense}{HTML}{B22B31}
    \definecolor{ansi-green}{HTML}{00A250}
    \definecolor{ansi-green-intense}{HTML}{007427}
    \definecolor{ansi-yellow}{HTML}{DDB62B}
    \definecolor{ansi-yellow-intense}{HTML}{B27D12}
    \definecolor{ansi-blue}{HTML}{208FFB}
    \definecolor{ansi-blue-intense}{HTML}{0065CA}
    \definecolor{ansi-magenta}{HTML}{D160C4}
    \definecolor{ansi-magenta-intense}{HTML}{A03196}
    \definecolor{ansi-cyan}{HTML}{60C6C8}
    \definecolor{ansi-cyan-intense}{HTML}{258F8F}
    \definecolor{ansi-white}{HTML}{C5C1B4}
    \definecolor{ansi-white-intense}{HTML}{A1A6B2}

    % commands and environments needed by pandoc snippets
    % extracted from the output of `pandoc -s`
    \providecommand{\tightlist}{%
      \setlength{\itemsep}{0pt}\setlength{\parskip}{0pt}}
    \DefineVerbatimEnvironment{Highlighting}{Verbatim}{commandchars=\\\{\}}
    % Add ',fontsize=\small' for more characters per line
    \newenvironment{Shaded}{}{}
    \newcommand{\KeywordTok}[1]{\textcolor[rgb]{0.00,0.44,0.13}{\textbf{{#1}}}}
    \newcommand{\DataTypeTok}[1]{\textcolor[rgb]{0.56,0.13,0.00}{{#1}}}
    \newcommand{\DecValTok}[1]{\textcolor[rgb]{0.25,0.63,0.44}{{#1}}}
    \newcommand{\BaseNTok}[1]{\textcolor[rgb]{0.25,0.63,0.44}{{#1}}}
    \newcommand{\FloatTok}[1]{\textcolor[rgb]{0.25,0.63,0.44}{{#1}}}
    \newcommand{\CharTok}[1]{\textcolor[rgb]{0.25,0.44,0.63}{{#1}}}
    \newcommand{\StringTok}[1]{\textcolor[rgb]{0.25,0.44,0.63}{{#1}}}
    \newcommand{\CommentTok}[1]{\textcolor[rgb]{0.38,0.63,0.69}{\textit{{#1}}}}
    \newcommand{\OtherTok}[1]{\textcolor[rgb]{0.00,0.44,0.13}{{#1}}}
    \newcommand{\AlertTok}[1]{\textcolor[rgb]{1.00,0.00,0.00}{\textbf{{#1}}}}
    \newcommand{\FunctionTok}[1]{\textcolor[rgb]{0.02,0.16,0.49}{{#1}}}
    \newcommand{\RegionMarkerTok}[1]{{#1}}
    \newcommand{\ErrorTok}[1]{\textcolor[rgb]{1.00,0.00,0.00}{\textbf{{#1}}}}
    \newcommand{\NormalTok}[1]{{#1}}
    
    % Additional commands for more recent versions of Pandoc
    \newcommand{\ConstantTok}[1]{\textcolor[rgb]{0.53,0.00,0.00}{{#1}}}
    \newcommand{\SpecialCharTok}[1]{\textcolor[rgb]{0.25,0.44,0.63}{{#1}}}
    \newcommand{\VerbatimStringTok}[1]{\textcolor[rgb]{0.25,0.44,0.63}{{#1}}}
    \newcommand{\SpecialStringTok}[1]{\textcolor[rgb]{0.73,0.40,0.53}{{#1}}}
    \newcommand{\ImportTok}[1]{{#1}}
    \newcommand{\DocumentationTok}[1]{\textcolor[rgb]{0.73,0.13,0.13}{\textit{{#1}}}}
    \newcommand{\AnnotationTok}[1]{\textcolor[rgb]{0.38,0.63,0.69}{\textbf{\textit{{#1}}}}}
    \newcommand{\CommentVarTok}[1]{\textcolor[rgb]{0.38,0.63,0.69}{\textbf{\textit{{#1}}}}}
    \newcommand{\VariableTok}[1]{\textcolor[rgb]{0.10,0.09,0.49}{{#1}}}
    \newcommand{\ControlFlowTok}[1]{\textcolor[rgb]{0.00,0.44,0.13}{\textbf{{#1}}}}
    \newcommand{\OperatorTok}[1]{\textcolor[rgb]{0.40,0.40,0.40}{{#1}}}
    \newcommand{\BuiltInTok}[1]{{#1}}
    \newcommand{\ExtensionTok}[1]{{#1}}
    \newcommand{\PreprocessorTok}[1]{\textcolor[rgb]{0.74,0.48,0.00}{{#1}}}
    \newcommand{\AttributeTok}[1]{\textcolor[rgb]{0.49,0.56,0.16}{{#1}}}
    \newcommand{\InformationTok}[1]{\textcolor[rgb]{0.38,0.63,0.69}{\textbf{\textit{{#1}}}}}
    \newcommand{\WarningTok}[1]{\textcolor[rgb]{0.38,0.63,0.69}{\textbf{\textit{{#1}}}}}
    
    
    % Define a nice break command that doesn't care if a line doesn't already
    % exist.
    \def\br{\hspace*{\fill} \\* }
    % Math Jax compatability definitions
    \def\gt{>}
    \def\lt{<}
    % Document parameters
    \title{Assignment\_7}
    
    
    

    % Pygments definitions
    
\makeatletter
\def\PY@reset{\let\PY@it=\relax \let\PY@bf=\relax%
    \let\PY@ul=\relax \let\PY@tc=\relax%
    \let\PY@bc=\relax \let\PY@ff=\relax}
\def\PY@tok#1{\csname PY@tok@#1\endcsname}
\def\PY@toks#1+{\ifx\relax#1\empty\else%
    \PY@tok{#1}\expandafter\PY@toks\fi}
\def\PY@do#1{\PY@bc{\PY@tc{\PY@ul{%
    \PY@it{\PY@bf{\PY@ff{#1}}}}}}}
\def\PY#1#2{\PY@reset\PY@toks#1+\relax+\PY@do{#2}}

\expandafter\def\csname PY@tok@w\endcsname{\def\PY@tc##1{\textcolor[rgb]{0.73,0.73,0.73}{##1}}}
\expandafter\def\csname PY@tok@c\endcsname{\let\PY@it=\textit\def\PY@tc##1{\textcolor[rgb]{0.25,0.50,0.50}{##1}}}
\expandafter\def\csname PY@tok@cp\endcsname{\def\PY@tc##1{\textcolor[rgb]{0.74,0.48,0.00}{##1}}}
\expandafter\def\csname PY@tok@k\endcsname{\let\PY@bf=\textbf\def\PY@tc##1{\textcolor[rgb]{0.00,0.50,0.00}{##1}}}
\expandafter\def\csname PY@tok@kp\endcsname{\def\PY@tc##1{\textcolor[rgb]{0.00,0.50,0.00}{##1}}}
\expandafter\def\csname PY@tok@kt\endcsname{\def\PY@tc##1{\textcolor[rgb]{0.69,0.00,0.25}{##1}}}
\expandafter\def\csname PY@tok@o\endcsname{\def\PY@tc##1{\textcolor[rgb]{0.40,0.40,0.40}{##1}}}
\expandafter\def\csname PY@tok@ow\endcsname{\let\PY@bf=\textbf\def\PY@tc##1{\textcolor[rgb]{0.67,0.13,1.00}{##1}}}
\expandafter\def\csname PY@tok@nb\endcsname{\def\PY@tc##1{\textcolor[rgb]{0.00,0.50,0.00}{##1}}}
\expandafter\def\csname PY@tok@nf\endcsname{\def\PY@tc##1{\textcolor[rgb]{0.00,0.00,1.00}{##1}}}
\expandafter\def\csname PY@tok@nc\endcsname{\let\PY@bf=\textbf\def\PY@tc##1{\textcolor[rgb]{0.00,0.00,1.00}{##1}}}
\expandafter\def\csname PY@tok@nn\endcsname{\let\PY@bf=\textbf\def\PY@tc##1{\textcolor[rgb]{0.00,0.00,1.00}{##1}}}
\expandafter\def\csname PY@tok@ne\endcsname{\let\PY@bf=\textbf\def\PY@tc##1{\textcolor[rgb]{0.82,0.25,0.23}{##1}}}
\expandafter\def\csname PY@tok@nv\endcsname{\def\PY@tc##1{\textcolor[rgb]{0.10,0.09,0.49}{##1}}}
\expandafter\def\csname PY@tok@no\endcsname{\def\PY@tc##1{\textcolor[rgb]{0.53,0.00,0.00}{##1}}}
\expandafter\def\csname PY@tok@nl\endcsname{\def\PY@tc##1{\textcolor[rgb]{0.63,0.63,0.00}{##1}}}
\expandafter\def\csname PY@tok@ni\endcsname{\let\PY@bf=\textbf\def\PY@tc##1{\textcolor[rgb]{0.60,0.60,0.60}{##1}}}
\expandafter\def\csname PY@tok@na\endcsname{\def\PY@tc##1{\textcolor[rgb]{0.49,0.56,0.16}{##1}}}
\expandafter\def\csname PY@tok@nt\endcsname{\let\PY@bf=\textbf\def\PY@tc##1{\textcolor[rgb]{0.00,0.50,0.00}{##1}}}
\expandafter\def\csname PY@tok@nd\endcsname{\def\PY@tc##1{\textcolor[rgb]{0.67,0.13,1.00}{##1}}}
\expandafter\def\csname PY@tok@s\endcsname{\def\PY@tc##1{\textcolor[rgb]{0.73,0.13,0.13}{##1}}}
\expandafter\def\csname PY@tok@sd\endcsname{\let\PY@it=\textit\def\PY@tc##1{\textcolor[rgb]{0.73,0.13,0.13}{##1}}}
\expandafter\def\csname PY@tok@si\endcsname{\let\PY@bf=\textbf\def\PY@tc##1{\textcolor[rgb]{0.73,0.40,0.53}{##1}}}
\expandafter\def\csname PY@tok@se\endcsname{\let\PY@bf=\textbf\def\PY@tc##1{\textcolor[rgb]{0.73,0.40,0.13}{##1}}}
\expandafter\def\csname PY@tok@sr\endcsname{\def\PY@tc##1{\textcolor[rgb]{0.73,0.40,0.53}{##1}}}
\expandafter\def\csname PY@tok@ss\endcsname{\def\PY@tc##1{\textcolor[rgb]{0.10,0.09,0.49}{##1}}}
\expandafter\def\csname PY@tok@sx\endcsname{\def\PY@tc##1{\textcolor[rgb]{0.00,0.50,0.00}{##1}}}
\expandafter\def\csname PY@tok@m\endcsname{\def\PY@tc##1{\textcolor[rgb]{0.40,0.40,0.40}{##1}}}
\expandafter\def\csname PY@tok@gh\endcsname{\let\PY@bf=\textbf\def\PY@tc##1{\textcolor[rgb]{0.00,0.00,0.50}{##1}}}
\expandafter\def\csname PY@tok@gu\endcsname{\let\PY@bf=\textbf\def\PY@tc##1{\textcolor[rgb]{0.50,0.00,0.50}{##1}}}
\expandafter\def\csname PY@tok@gd\endcsname{\def\PY@tc##1{\textcolor[rgb]{0.63,0.00,0.00}{##1}}}
\expandafter\def\csname PY@tok@gi\endcsname{\def\PY@tc##1{\textcolor[rgb]{0.00,0.63,0.00}{##1}}}
\expandafter\def\csname PY@tok@gr\endcsname{\def\PY@tc##1{\textcolor[rgb]{1.00,0.00,0.00}{##1}}}
\expandafter\def\csname PY@tok@ge\endcsname{\let\PY@it=\textit}
\expandafter\def\csname PY@tok@gs\endcsname{\let\PY@bf=\textbf}
\expandafter\def\csname PY@tok@gp\endcsname{\let\PY@bf=\textbf\def\PY@tc##1{\textcolor[rgb]{0.00,0.00,0.50}{##1}}}
\expandafter\def\csname PY@tok@go\endcsname{\def\PY@tc##1{\textcolor[rgb]{0.53,0.53,0.53}{##1}}}
\expandafter\def\csname PY@tok@gt\endcsname{\def\PY@tc##1{\textcolor[rgb]{0.00,0.27,0.87}{##1}}}
\expandafter\def\csname PY@tok@err\endcsname{\def\PY@bc##1{\setlength{\fboxsep}{0pt}\fcolorbox[rgb]{1.00,0.00,0.00}{1,1,1}{\strut ##1}}}
\expandafter\def\csname PY@tok@kc\endcsname{\let\PY@bf=\textbf\def\PY@tc##1{\textcolor[rgb]{0.00,0.50,0.00}{##1}}}
\expandafter\def\csname PY@tok@kd\endcsname{\let\PY@bf=\textbf\def\PY@tc##1{\textcolor[rgb]{0.00,0.50,0.00}{##1}}}
\expandafter\def\csname PY@tok@kn\endcsname{\let\PY@bf=\textbf\def\PY@tc##1{\textcolor[rgb]{0.00,0.50,0.00}{##1}}}
\expandafter\def\csname PY@tok@kr\endcsname{\let\PY@bf=\textbf\def\PY@tc##1{\textcolor[rgb]{0.00,0.50,0.00}{##1}}}
\expandafter\def\csname PY@tok@bp\endcsname{\def\PY@tc##1{\textcolor[rgb]{0.00,0.50,0.00}{##1}}}
\expandafter\def\csname PY@tok@fm\endcsname{\def\PY@tc##1{\textcolor[rgb]{0.00,0.00,1.00}{##1}}}
\expandafter\def\csname PY@tok@vc\endcsname{\def\PY@tc##1{\textcolor[rgb]{0.10,0.09,0.49}{##1}}}
\expandafter\def\csname PY@tok@vg\endcsname{\def\PY@tc##1{\textcolor[rgb]{0.10,0.09,0.49}{##1}}}
\expandafter\def\csname PY@tok@vi\endcsname{\def\PY@tc##1{\textcolor[rgb]{0.10,0.09,0.49}{##1}}}
\expandafter\def\csname PY@tok@vm\endcsname{\def\PY@tc##1{\textcolor[rgb]{0.10,0.09,0.49}{##1}}}
\expandafter\def\csname PY@tok@sa\endcsname{\def\PY@tc##1{\textcolor[rgb]{0.73,0.13,0.13}{##1}}}
\expandafter\def\csname PY@tok@sb\endcsname{\def\PY@tc##1{\textcolor[rgb]{0.73,0.13,0.13}{##1}}}
\expandafter\def\csname PY@tok@sc\endcsname{\def\PY@tc##1{\textcolor[rgb]{0.73,0.13,0.13}{##1}}}
\expandafter\def\csname PY@tok@dl\endcsname{\def\PY@tc##1{\textcolor[rgb]{0.73,0.13,0.13}{##1}}}
\expandafter\def\csname PY@tok@s2\endcsname{\def\PY@tc##1{\textcolor[rgb]{0.73,0.13,0.13}{##1}}}
\expandafter\def\csname PY@tok@sh\endcsname{\def\PY@tc##1{\textcolor[rgb]{0.73,0.13,0.13}{##1}}}
\expandafter\def\csname PY@tok@s1\endcsname{\def\PY@tc##1{\textcolor[rgb]{0.73,0.13,0.13}{##1}}}
\expandafter\def\csname PY@tok@mb\endcsname{\def\PY@tc##1{\textcolor[rgb]{0.40,0.40,0.40}{##1}}}
\expandafter\def\csname PY@tok@mf\endcsname{\def\PY@tc##1{\textcolor[rgb]{0.40,0.40,0.40}{##1}}}
\expandafter\def\csname PY@tok@mh\endcsname{\def\PY@tc##1{\textcolor[rgb]{0.40,0.40,0.40}{##1}}}
\expandafter\def\csname PY@tok@mi\endcsname{\def\PY@tc##1{\textcolor[rgb]{0.40,0.40,0.40}{##1}}}
\expandafter\def\csname PY@tok@il\endcsname{\def\PY@tc##1{\textcolor[rgb]{0.40,0.40,0.40}{##1}}}
\expandafter\def\csname PY@tok@mo\endcsname{\def\PY@tc##1{\textcolor[rgb]{0.40,0.40,0.40}{##1}}}
\expandafter\def\csname PY@tok@ch\endcsname{\let\PY@it=\textit\def\PY@tc##1{\textcolor[rgb]{0.25,0.50,0.50}{##1}}}
\expandafter\def\csname PY@tok@cm\endcsname{\let\PY@it=\textit\def\PY@tc##1{\textcolor[rgb]{0.25,0.50,0.50}{##1}}}
\expandafter\def\csname PY@tok@cpf\endcsname{\let\PY@it=\textit\def\PY@tc##1{\textcolor[rgb]{0.25,0.50,0.50}{##1}}}
\expandafter\def\csname PY@tok@c1\endcsname{\let\PY@it=\textit\def\PY@tc##1{\textcolor[rgb]{0.25,0.50,0.50}{##1}}}
\expandafter\def\csname PY@tok@cs\endcsname{\let\PY@it=\textit\def\PY@tc##1{\textcolor[rgb]{0.25,0.50,0.50}{##1}}}

\def\PYZbs{\char`\\}
\def\PYZus{\char`\_}
\def\PYZob{\char`\{}
\def\PYZcb{\char`\}}
\def\PYZca{\char`\^}
\def\PYZam{\char`\&}
\def\PYZlt{\char`\<}
\def\PYZgt{\char`\>}
\def\PYZsh{\char`\#}
\def\PYZpc{\char`\%}
\def\PYZdl{\char`\$}
\def\PYZhy{\char`\-}
\def\PYZsq{\char`\'}
\def\PYZdq{\char`\"}
\def\PYZti{\char`\~}
% for compatibility with earlier versions
\def\PYZat{@}
\def\PYZlb{[}
\def\PYZrb{]}
\makeatother


    % Exact colors from NB
    \definecolor{incolor}{rgb}{0.0, 0.0, 0.5}
    \definecolor{outcolor}{rgb}{0.545, 0.0, 0.0}



    
    % Prevent overflowing lines due to hard-to-break entities
    \sloppy 
    % Setup hyperref package
    \hypersetup{
      breaklinks=true,  % so long urls are correctly broken across lines
      colorlinks=true,
      urlcolor=urlcolor,
      linkcolor=linkcolor,
      citecolor=citecolor,
      }
    % Slightly bigger margins than the latex defaults
    
    \geometry{verbose,tmargin=1in,bmargin=1in,lmargin=1in,rmargin=1in}
    
    

    \begin{document}
    
    
    \maketitle
    
    

    
    \hypertarget{assignment-7}{%
\section{Assignment 7}\label{assignment-7}}

\emph{EE2703: Applied Programming\\
Author: Varun Sundar, EE16B068}

\hypertarget{abstract}{%
\section{Abstract}\label{abstract}}

This weeks assignment covers the usage of numpy polynomial libraries as
well as scipy's signal processing module to analyse LTI signals.

\hypertarget{introduction}{%
\section{Introduction}\label{introduction}}

In particular, we shall utilise continuous time Laplace transforms. This
is done for the case of driver oscillators, coupled spring mass systems
and an instance of second order low pass systems.

Scipy's signal processing libraries allow us to

Conventions:

\begin{enumerate}
\def\labelenumi{\arabic{enumi}.}
\tightlist
\item
  We are using Python 3, GCC for C
\item
  Underscore naming vs Camel Case
\item
  PEP 25 convention style.
\end{enumerate}

    \begin{Verbatim}[commandchars=\\\{\}]
{\color{incolor}In [{\color{incolor}32}]:} \PY{k+kn}{import} \PY{n+nn}{numpy} \PY{k}{as} \PY{n+nn}{np}
         \PY{k+kn}{import} \PY{n+nn}{matplotlib}\PY{n+nn}{.}\PY{n+nn}{pyplot} \PY{k}{as} \PY{n+nn}{plt}
         \PY{k+kn}{import} \PY{n+nn}{scipy}\PY{n+nn}{.}\PY{n+nn}{signal} \PY{k}{as} \PY{n+nn}{sp}
         
         \PY{c+c1}{\PYZsh{}\PYZpc{}matplotlib nbagg}
         \PY{c+c1}{\PYZsh{}For Interactive Plots}
         
         \PY{o}{\PYZpc{}}\PY{k}{matplotlib} inline
\end{Verbatim}


    \begin{Verbatim}[commandchars=\\\{\}]
{\color{incolor}In [{\color{incolor}3}]:} \PY{k}{class} \PY{n+nc}{Spring}\PY{p}{(}\PY{n+nb}{object}\PY{p}{)}\PY{p}{:}
            \PY{l+s+sd}{\PYZsq{}\PYZsq{}\PYZsq{}}
        \PY{l+s+sd}{    Solve for a spring mass system of nth order}
        \PY{l+s+sd}{    Given differential polynomial and laplace domain representation of forcing function.}
        \PY{l+s+sd}{    \PYZsq{}\PYZsq{}\PYZsq{}}
            \PY{k}{def} \PY{n+nf}{\PYZus{}\PYZus{}init\PYZus{}\PYZus{}}\PY{p}{(}\PY{n+nb+bp}{self}\PY{p}{,}\PY{n}{D}\PY{p}{,}\PY{n}{f}\PY{p}{,}\PY{n}{initial}\PY{o}{=}\PY{k+kc}{None}\PY{p}{,}\PY{n}{degree}\PY{o}{=}\PY{l+m+mi}{2}\PY{p}{)}\PY{p}{:}
                \PY{c+c1}{\PYZsh{} D is the D operator coefficients}
                \PY{c+c1}{\PYZsh{} f is the Laplace transform of the forcing function [Num,Denom]}
                \PY{n+nb+bp}{self}\PY{o}{.}\PY{n}{D}\PY{o}{=}\PY{n}{np}\PY{o}{.}\PY{n}{poly1d}\PY{p}{(}\PY{n}{D}\PY{p}{)}
                \PY{n+nb+bp}{self}\PY{o}{.}\PY{n}{f}\PY{o}{=}\PY{n}{f}
                \PY{n+nb+bp}{self}\PY{o}{.}\PY{n}{degree}\PY{o}{=}\PY{n}{degree}
                \PY{n+nb+bp}{self}\PY{o}{.}\PY{n}{set\PYZus{}initial\PYZus{}conditions}\PY{p}{(}\PY{n}{initial}\PY{p}{)}
        
            \PY{k}{def} \PY{n+nf}{set\PYZus{}initial\PYZus{}conditions}\PY{p}{(}\PY{n+nb+bp}{self}\PY{p}{,}\PY{n}{L}\PY{o}{=}\PY{k+kc}{None}\PY{p}{)}\PY{p}{:}
                
                \PY{c+c1}{\PYZsh{} L being the list of initial conditions}
                
                \PY{k}{if} \PY{o+ow}{not} \PY{n}{L}\PY{p}{:}
                    \PY{n+nb+bp}{self}\PY{o}{.}\PY{n}{initial\PYZus{}conditions}\PY{o}{=}\PY{n}{np}\PY{o}{.}\PY{n}{zeros}\PY{p}{(}\PY{n+nb+bp}{self}\PY{o}{.}\PY{n}{degree}\PY{p}{)}
                \PY{k}{else}\PY{p}{:}
                    \PY{k}{if} \PY{n+nb}{len}\PY{p}{(}\PY{n}{L}\PY{p}{)}\PY{o}{==}\PY{n+nb+bp}{self}\PY{o}{.}\PY{n}{degree}\PY{p}{:}
                        \PY{n+nb+bp}{self}\PY{o}{.}\PY{n}{initial\PYZus{}conditions}\PY{o}{=}\PY{n}{L} \PY{c+c1}{\PYZsh{} x\PYZus{}\PYZob{}\PYZsq{}\PYZcb{},x\PYZus{}\PYZob{}\PYZsq{}\PYZsq{}\PYZcb{},x\PYZus{}\PYZob{}\PYZsq{}\PYZsq{}\PYZsq{}\PYZcb{},...}
                    \PY{k}{else}\PY{p}{:}
                        \PY{k}{raise} \PY{n+ne}{ValueError}\PY{p}{(}\PY{l+s+s1}{\PYZsq{}}\PY{l+s+s1}{Represents a list of coefficients with a mismatch to degree}\PY{l+s+s1}{\PYZsq{}}\PY{p}{)}
                        
            \PY{k}{def} \PY{n+nf}{\PYZus{}\PYZus{}print\PYZus{}\PYZus{}}\PY{p}{(}\PY{n+nb+bp}{self}\PY{p}{)}\PY{p}{:}
                \PY{k}{return} \PY{n+nb}{str}\PY{p}{(}\PY{n}{np}\PY{o}{.}\PY{n}{poly1d}\PY{p}{(}\PY{n+nb+bp}{self}\PY{o}{.}\PY{n}{D}\PY{p}{)}\PY{p}{)}\PY{o}{+}\PY{l+s+s2}{\PYZdq{}}\PY{l+s+s2}{ = }\PY{l+s+s2}{\PYZdq{}} \PY{o}{+} \PY{n+nb}{str}\PY{p}{(}\PY{n+nb+bp}{self}\PY{o}{.}\PY{n}{f}\PY{p}{)}
            
            \PY{n+nd}{@staticmethod}
            \PY{k}{def} \PY{n+nf}{cosine\PYZus{}forcing}\PY{p}{(}\PY{n}{omega}\PY{p}{,}\PY{n}{sigma}\PY{p}{,}\PY{n}{bode}\PY{o}{=}\PY{k+kc}{False}\PY{p}{)}\PY{p}{:}
                \PY{c+c1}{\PYZsh{} Generate the laplace domain representation for a cosine forcing function.}
                \PY{n}{Num}\PY{o}{=}\PY{n}{np}\PY{o}{.}\PY{n}{poly1d}\PY{p}{(}\PY{p}{[}\PY{l+m+mi}{1}\PY{p}{,}\PY{n}{sigma}\PY{p}{]}\PY{p}{)}
                \PY{n}{Denom}\PY{o}{=}\PY{n}{Num}\PY{o}{*}\PY{n}{Num}\PY{o}{+} \PY{n}{np}\PY{o}{.}\PY{n}{poly1d}\PY{p}{(}\PY{p}{[}\PY{n}{omega}\PY{o}{*}\PY{o}{*}\PY{l+m+mi}{2}\PY{p}{]}\PY{p}{)}
                \PY{n}{f}\PY{o}{=} \PY{n}{sp}\PY{o}{.}\PY{n}{lti}\PY{p}{(}\PY{n}{Num}\PY{p}{,}\PY{n}{Denom}\PY{p}{)}    \PY{c+c1}{\PYZsh{} Num, Denominator}
                
                \PY{k}{if} \PY{n}{bode}\PY{p}{:}
                    \PY{n}{w}\PY{p}{,}\PY{n}{S}\PY{p}{,}\PY{n}{phi}\PY{o}{=}\PY{n}{f}\PY{o}{.}\PY{n}{bode}\PY{p}{(}\PY{p}{)}
                    \PY{n}{fig}\PY{p}{,}\PY{p}{(}\PY{n}{ax1}\PY{p}{,}\PY{n}{ax2}\PY{p}{)}\PY{o}{=}\PY{n}{plt}\PY{o}{.}\PY{n}{subplots}\PY{p}{(}\PY{l+m+mi}{2}\PY{p}{,}\PY{l+m+mi}{1}\PY{p}{,}\PY{n}{sharex}\PY{o}{=}\PY{k+kc}{True}\PY{p}{)}
                    \PY{n}{ax1}\PY{o}{.}\PY{n}{semilogx}\PY{p}{(}\PY{n}{w}\PY{p}{,}\PY{n}{S}\PY{p}{)}
                    \PY{n}{ax2}\PY{o}{.}\PY{n}{semilogx}\PY{p}{(}\PY{n}{w}\PY{p}{,}\PY{n}{phi}\PY{p}{)}
                \PY{k}{return} \PY{p}{[}\PY{n}{Num}\PY{p}{,}\PY{n}{Denom}\PY{p}{]}
            
            \PY{k}{def} \PY{n+nf}{solve}\PY{p}{(}\PY{n+nb+bp}{self}\PY{p}{,}\PY{n}{time}\PY{p}{,}\PY{n}{bode}\PY{o}{=}\PY{k+kc}{False}\PY{p}{)}\PY{p}{:}
                \PY{n+nb+bp}{self}\PY{o}{.}\PY{n}{initial\PYZus{}conditions\PYZus{}polynomial}\PY{o}{=}\PY{n}{np}\PY{o}{.}\PY{n}{poly1d}\PY{p}{(}\PY{n}{np}\PY{o}{.}\PY{n}{multiply}\PY{p}{(}\PY{n+nb+bp}{self}\PY{o}{.}\PY{n}{initial\PYZus{}conditions}\PY{p}{,}\PY{n}{np}\PY{o}{.}\PY{n}{array}\PY{p}{(}\PY{n+nb+bp}{self}\PY{o}{.}\PY{n}{D}\PY{p}{)}\PY{p}{[}\PY{p}{:}\PY{o}{\PYZhy{}}\PY{l+m+mi}{1}\PY{p}{]}\PY{p}{)}\PY{p}{)}
                \PY{c+c1}{\PYZsh{}self.H=sp.lti(self.f[0],self.f[1]*self.D) }
                \PY{n+nb+bp}{self}\PY{o}{.}\PY{n}{H}\PY{o}{=}\PY{n}{sp}\PY{o}{.}\PY{n}{lti}\PY{p}{(}\PY{n+nb+bp}{self}\PY{o}{.}\PY{n}{f}\PY{p}{[}\PY{l+m+mi}{0}\PY{p}{]}\PY{o}{+}\PY{n+nb+bp}{self}\PY{o}{.}\PY{n}{initial\PYZus{}conditions\PYZus{}polynomial}\PY{o}{*}\PY{n+nb+bp}{self}\PY{o}{.}\PY{n}{f}\PY{p}{[}\PY{l+m+mi}{1}\PY{p}{]}\PY{p}{,}\PY{n+nb+bp}{self}\PY{o}{.}\PY{n}{f}\PY{p}{[}\PY{l+m+mi}{1}\PY{p}{]}\PY{o}{*}\PY{n+nb+bp}{self}\PY{o}{.}\PY{n}{D}\PY{p}{)}
                
                \PY{k}{if} \PY{n}{bode}\PY{p}{:}
                    \PY{n}{w}\PY{p}{,}\PY{n}{S}\PY{p}{,}\PY{n}{phi}\PY{o}{=}\PY{n+nb+bp}{self}\PY{o}{.}\PY{n}{H}\PY{o}{.}\PY{n}{bode}\PY{p}{(}\PY{p}{)}
                    \PY{n}{fig}\PY{p}{,}\PY{p}{(}\PY{n}{ax1}\PY{p}{,}\PY{n}{ax2}\PY{p}{)}\PY{o}{=}\PY{n}{plt}\PY{o}{.}\PY{n}{subplots}\PY{p}{(}\PY{l+m+mi}{2}\PY{p}{,}\PY{l+m+mi}{1}\PY{p}{)}
                    \PY{n}{ax1}\PY{o}{.}\PY{n}{semilogx}\PY{p}{(}\PY{n}{w}\PY{p}{,}\PY{n}{S}\PY{p}{)}
                    \PY{n}{ax1}\PY{o}{.}\PY{n}{set\PYZus{}title}\PY{p}{(}\PY{l+s+s2}{\PYZdq{}}\PY{l+s+s2}{Magnitude plot}\PY{l+s+s2}{\PYZdq{}}\PY{p}{)}
                    \PY{n}{ax2}\PY{o}{.}\PY{n}{semilogx}\PY{p}{(}\PY{n}{w}\PY{p}{,}\PY{n}{phi}\PY{p}{)}
                    \PY{n}{ax2}\PY{o}{.}\PY{n}{set\PYZus{}title}\PY{p}{(}\PY{l+s+s2}{\PYZdq{}}\PY{l+s+s2}{Phase plot}\PY{l+s+s2}{\PYZdq{}}\PY{p}{)}
                    \PY{n}{fig}\PY{o}{.}\PY{n}{tight\PYZus{}layout}\PY{p}{(}\PY{p}{)}
                    \PY{n}{plt}\PY{o}{.}\PY{n}{show}\PY{p}{(}\PY{p}{)}
                
                \PY{n+nb}{print}\PY{p}{(}\PY{n+nb+bp}{self}\PY{o}{.}\PY{n}{H}\PY{p}{)}
                \PY{n}{t}\PY{p}{,}\PY{n}{x}\PY{o}{=}\PY{n}{sp}\PY{o}{.}\PY{n}{impulse}\PY{p}{(}\PY{n+nb+bp}{self}\PY{o}{.}\PY{n}{H}\PY{p}{,}\PY{k+kc}{None}\PY{p}{,} \PY{n}{time}\PY{p}{)}
                \PY{n}{plt}\PY{o}{.}\PY{n}{title}\PY{p}{(}\PY{l+s+s2}{\PYZdq{}}\PY{l+s+s2}{Response with time}\PY{l+s+s2}{\PYZdq{}}\PY{p}{)}
                \PY{n}{plt}\PY{o}{.}\PY{n}{plot}\PY{p}{(}\PY{n}{t}\PY{p}{,}\PY{n}{x}\PY{p}{)}
                \PY{n}{plt}\PY{o}{.}\PY{n}{show}\PY{p}{(}\PY{p}{)}
\end{Verbatim}


    \begin{Verbatim}[commandchars=\\\{\}]
{\color{incolor}In [{\color{incolor}4}]:} \PY{c+c1}{\PYZsh{} Example of cosine forcing Bode plot}
        \PY{n}{Spring}\PY{o}{.}\PY{n}{cosine\PYZus{}forcing}\PY{p}{(}\PY{l+m+mf}{1.5}\PY{p}{,}\PY{l+m+mf}{0.5}\PY{p}{,}\PY{l+m+mi}{1}\PY{p}{)}
\end{Verbatim}


\begin{Verbatim}[commandchars=\\\{\}]
{\color{outcolor}Out[{\color{outcolor}4}]:} [poly1d([ 1. ,  0.5]), poly1d([ 1. ,  1. ,  2.5])]
\end{Verbatim}
            
    \begin{center}
    \adjustimage{max size={0.9\linewidth}{0.9\paperheight}}{output_3_1.png}
    \end{center}
    { \hspace*{\fill} \\}
    
    \hypertarget{assignment-questions}{%
\section{Assignment Questions}\label{assignment-questions}}

\hypertarget{problem-1}{%
\subsection{Problem 1}\label{problem-1}}

We solve for the time response of a spring satisfying
\[  \frac{d^2 x}{dt^2}+2.25x= f(t) \]

Where,

\[ f(t) =cos(1.5t)e^{−0.5t}u_0(t) \]

and \(f(t)\) has a laplace transform,

\[ F(s)= \frac{s+0.5}{(s + 0.5)2 + 2.25} \].

Also, initial conditions are intial rest.

    \begin{Verbatim}[commandchars=\\\{\}]
{\color{incolor}In [{\color{incolor}30}]:} \PY{n}{a}\PY{o}{=}\PY{n}{Spring}\PY{p}{(}\PY{p}{[}\PY{l+m+mi}{1}\PY{p}{,}\PY{l+m+mi}{0}\PY{p}{,}\PY{l+m+mf}{2.25}\PY{p}{]}\PY{p}{,}\PY{n}{Spring}\PY{o}{.}\PY{n}{cosine\PYZus{}forcing}\PY{p}{(}\PY{l+m+mf}{1.5}\PY{p}{,}\PY{l+m+mi}{5}\PY{p}{)}\PY{p}{)}
         \PY{n}{a}\PY{o}{.}\PY{n}{solve}\PY{p}{(}\PY{n}{np}\PY{o}{.}\PY{n}{linspace}\PY{p}{(}\PY{l+m+mi}{0}\PY{p}{,}\PY{l+m+mi}{50}\PY{p}{,}\PY{l+m+mi}{2000}\PY{p}{)}\PY{p}{)}
\end{Verbatim}


    \begin{Verbatim}[commandchars=\\\{\}]
TransferFunctionContinuous(
array([ 1.,  5.]),
array([  1.    ,  10.    ,  29.5   ,  22.5   ,  61.3125]),
dt: None
)

    \end{Verbatim}

    \begin{center}
    \adjustimage{max size={0.9\linewidth}{0.9\paperheight}}{output_5_1.png}
    \end{center}
    { \hspace*{\fill} \\}
    
    We notice that the steady state response is a decaying sinusoid
function. This makes intuitive sense, according to Bode plot
inerpretations. Also, note that the decay is hard to observe, unless the
timespan is broadened.

    \hypertarget{problem-2}{%
\subsection{Problem 2}\label{problem-2}}

Now done with a much smaller decay,

\[ f(t) =cos(1.5t)e^{−0.05t}u_0(t) \]

    \begin{Verbatim}[commandchars=\\\{\}]
{\color{incolor}In [{\color{incolor}6}]:} \PY{n}{a}\PY{o}{=}\PY{n}{Spring}\PY{p}{(}\PY{p}{[}\PY{l+m+mi}{1}\PY{p}{,}\PY{l+m+mi}{0}\PY{p}{,}\PY{l+m+mf}{2.25}\PY{p}{]}\PY{p}{,}\PY{n}{Spring}\PY{o}{.}\PY{n}{cosine\PYZus{}forcing}\PY{p}{(}\PY{l+m+mf}{1.5}\PY{p}{,}\PY{l+m+mf}{0.05}\PY{p}{)}\PY{p}{)}
        \PY{n}{a}\PY{o}{.}\PY{n}{solve}\PY{p}{(}\PY{n}{np}\PY{o}{.}\PY{n}{linspace}\PY{p}{(}\PY{l+m+mi}{0}\PY{p}{,}\PY{l+m+mi}{50}\PY{p}{,}\PY{l+m+mi}{1001}\PY{p}{)}\PY{p}{)}
\end{Verbatim}


    \begin{Verbatim}[commandchars=\\\{\}]
TransferFunctionContinuous(
array([ 1.  ,  0.05]),
array([ 1.      ,  0.1     ,  4.5025  ,  0.225   ,  5.068125]),
dt: None
)

    \end{Verbatim}

    \begin{center}
    \adjustimage{max size={0.9\linewidth}{0.9\paperheight}}{output_8_1.png}
    \end{center}
    { \hspace*{\fill} \\}
    
    Here, the amplitude (at moderate time) is much larger owing to a much
smaller decay.

\hypertarget{problem-3}{%
\subsection{Problem 3}\label{problem-3}}

Now, we vary the frequency of the cosine in \(f(t)\) from 1.4 to 1.6 in
steps of 0.05 keeping the exponent as \(exp(−0.05t)\) and plot the
resulting responses.

We notice that resonance occurs at 1.5 radians/sec, which is the natural
frequency of this under-damped spring mass system.

    \begin{Verbatim}[commandchars=\\\{\}]
{\color{incolor}In [{\color{incolor}7}]:} \PY{n}{frequencies}\PY{o}{=}\PY{n}{np}\PY{o}{.}\PY{n}{linspace}\PY{p}{(}\PY{l+m+mf}{1.4}\PY{p}{,}\PY{l+m+mf}{1.6}\PY{p}{,}\PY{l+m+mi}{5}\PY{p}{)}
        \PY{k}{for} \PY{n}{freq} \PY{o+ow}{in} \PY{n}{frequencies}\PY{p}{:}
            \PY{n+nb}{print} \PY{p}{(}\PY{l+s+s2}{\PYZdq{}}\PY{l+s+s2}{Frequency is }\PY{l+s+se}{\PYZbs{}t}\PY{l+s+s2}{\PYZdq{}}\PY{p}{,} \PY{n}{freq}\PY{p}{)}
            \PY{n}{a}\PY{o}{=}\PY{n}{Spring}\PY{p}{(}\PY{p}{[}\PY{l+m+mi}{1}\PY{p}{,}\PY{l+m+mi}{0}\PY{p}{,}\PY{l+m+mf}{2.25}\PY{p}{]}\PY{p}{,}\PY{n}{Spring}\PY{o}{.}\PY{n}{cosine\PYZus{}forcing}\PY{p}{(}\PY{n}{freq}\PY{p}{,}\PY{l+m+mf}{0.05}\PY{p}{)}\PY{p}{)}
            \PY{n}{a}\PY{o}{.}\PY{n}{solve}\PY{p}{(}\PY{n}{np}\PY{o}{.}\PY{n}{linspace}\PY{p}{(}\PY{l+m+mi}{0}\PY{p}{,}\PY{l+m+mi}{150}\PY{p}{,}\PY{l+m+mi}{1001}\PY{p}{)}\PY{p}{)}
\end{Verbatim}


    \begin{Verbatim}[commandchars=\\\{\}]
Frequency is 	 1.4
TransferFunctionContinuous(
array([ 1.  ,  0.05]),
array([ 1.      ,  0.1     ,  4.2125  ,  0.225   ,  4.415625]),
dt: None
)

    \end{Verbatim}

    \begin{center}
    \adjustimage{max size={0.9\linewidth}{0.9\paperheight}}{output_10_1.png}
    \end{center}
    { \hspace*{\fill} \\}
    
    \begin{Verbatim}[commandchars=\\\{\}]
Frequency is 	 1.45
TransferFunctionContinuous(
array([ 1.  ,  0.05]),
array([ 1.     ,  0.1    ,  4.355  ,  0.225  ,  4.73625]),
dt: None
)

    \end{Verbatim}

    \begin{center}
    \adjustimage{max size={0.9\linewidth}{0.9\paperheight}}{output_10_3.png}
    \end{center}
    { \hspace*{\fill} \\}
    
    \begin{Verbatim}[commandchars=\\\{\}]
Frequency is 	 1.5
TransferFunctionContinuous(
array([ 1.  ,  0.05]),
array([ 1.      ,  0.1     ,  4.5025  ,  0.225   ,  5.068125]),
dt: None
)

    \end{Verbatim}

    \begin{center}
    \adjustimage{max size={0.9\linewidth}{0.9\paperheight}}{output_10_5.png}
    \end{center}
    { \hspace*{\fill} \\}
    
    \begin{Verbatim}[commandchars=\\\{\}]
Frequency is 	 1.55
TransferFunctionContinuous(
array([ 1.  ,  0.05]),
array([ 1.     ,  0.1    ,  4.655  ,  0.225  ,  5.41125]),
dt: None
)

    \end{Verbatim}

    \begin{center}
    \adjustimage{max size={0.9\linewidth}{0.9\paperheight}}{output_10_7.png}
    \end{center}
    { \hspace*{\fill} \\}
    
    \begin{Verbatim}[commandchars=\\\{\}]
Frequency is 	 1.6
TransferFunctionContinuous(
array([ 1.  ,  0.05]),
array([ 1.      ,  0.1     ,  4.8125  ,  0.225   ,  5.765625]),
dt: None
)

    \end{Verbatim}

    \begin{center}
    \adjustimage{max size={0.9\linewidth}{0.9\paperheight}}{output_10_9.png}
    \end{center}
    { \hspace*{\fill} \\}
    
    \hypertarget{problem-4}{%
\subsection{Problem 4}\label{problem-4}}

We solve for the coupled differential system.

\[\ddot{x}+(x−y) = 0\] \[\ddot{y}+2(y−x) = 0\]

with \[x(0) = 1, \dot{x}(0) = y(0) = \dot{y}(0) = 0\]

This is done by substituting for y from the first equation into the
second and obtaining a fourth order equation. We solve for its time
evolution, and from it obtain \(x(t)\) and \(y(t)\) for \(0 ≤ t ≤ 20\).

Our implementation of the Spring mass system allows us to solve constant
coefficient ODE's, so we utlise the same. We can also use coupled ODE
packages under scipy as an alternative, which would be recommended for
more complicated coupled harmonic equations.

    \begin{Verbatim}[commandchars=\\\{\}]
{\color{incolor}In [{\color{incolor}8}]:} \PY{c+c1}{\PYZsh{} For x, x\PYZus{}4+x\PYZus{}23=0}
        \PY{n}{a}\PY{o}{=}\PY{n}{Spring}\PY{p}{(}\PY{p}{[}\PY{l+m+mi}{1}\PY{p}{,}\PY{l+m+mi}{0}\PY{p}{,}\PY{l+m+mi}{3}\PY{p}{,}\PY{l+m+mi}{0}\PY{p}{,}\PY{l+m+mi}{0}\PY{p}{]}\PY{p}{,}\PY{p}{[}\PY{l+m+mi}{0}\PY{p}{,}\PY{l+m+mi}{1}\PY{p}{]}\PY{p}{,}\PY{n}{initial}\PY{o}{=}\PY{p}{[}\PY{l+m+mi}{1}\PY{p}{,}\PY{l+m+mi}{0}\PY{p}{,}\PY{o}{\PYZhy{}}\PY{l+m+mi}{1}\PY{p}{,}\PY{l+m+mi}{0}\PY{p}{]}\PY{p}{,}\PY{n}{degree}\PY{o}{=}\PY{l+m+mi}{4}\PY{p}{)}
        \PY{n}{a}\PY{o}{.}\PY{n}{solve}\PY{p}{(}\PY{n}{np}\PY{o}{.}\PY{n}{linspace}\PY{p}{(}\PY{l+m+mi}{0}\PY{p}{,}\PY{l+m+mi}{20}\PY{p}{,}\PY{l+m+mi}{1000}\PY{p}{)}\PY{p}{)}
        
        \PY{o}{\PYZpc{}}\PY{k}{matplotlib} inline
\end{Verbatim}


    \begin{Verbatim}[commandchars=\\\{\}]
TransferFunctionContinuous(
array([ 1.,  0., -3.,  0.]),
array([ 1.,  0.,  3.,  0.,  0.]),
dt: None
)

    \end{Verbatim}

    \begin{center}
    \adjustimage{max size={0.9\linewidth}{0.9\paperheight}}{output_12_1.png}
    \end{center}
    { \hspace*{\fill} \\}
    
    \begin{Verbatim}[commandchars=\\\{\}]
{\color{incolor}In [{\color{incolor}9}]:} \PY{c+c1}{\PYZsh{} For y, y\PYZus{}4+y\PYZus{}2*3=0}
        \PY{n}{a}\PY{o}{=}\PY{n}{Spring}\PY{p}{(}\PY{p}{[}\PY{l+m+mi}{1}\PY{p}{,}\PY{l+m+mi}{0}\PY{p}{,}\PY{l+m+mi}{3}\PY{p}{,}\PY{l+m+mi}{0}\PY{p}{,}\PY{l+m+mi}{0}\PY{p}{]}\PY{p}{,}\PY{p}{[}\PY{l+m+mi}{0}\PY{p}{,}\PY{l+m+mi}{1}\PY{p}{]}\PY{p}{,}\PY{n}{initial}\PY{o}{=}\PY{p}{[}\PY{l+m+mi}{0}\PY{p}{,}\PY{l+m+mi}{0}\PY{p}{,}\PY{l+m+mi}{2}\PY{p}{,}\PY{l+m+mi}{0}\PY{p}{]}\PY{p}{,}\PY{n}{degree}\PY{o}{=}\PY{l+m+mi}{4}\PY{p}{)}
        \PY{n}{a}\PY{o}{.}\PY{n}{solve}\PY{p}{(}\PY{n}{np}\PY{o}{.}\PY{n}{linspace}\PY{p}{(}\PY{l+m+mi}{0}\PY{p}{,}\PY{l+m+mi}{20}\PY{p}{,}\PY{l+m+mi}{1000}\PY{p}{)}\PY{p}{)}
\end{Verbatim}


    \begin{Verbatim}[commandchars=\\\{\}]
TransferFunctionContinuous(
array([ 6.,  0.]),
array([ 1.,  0.,  3.,  0.,  0.]),
dt: None
)

    \end{Verbatim}

    \begin{center}
    \adjustimage{max size={0.9\linewidth}{0.9\paperheight}}{output_13_1.png}
    \end{center}
    { \hspace*{\fill} \\}
    
    We observe that the outputs are sinusoids, and \(90^{\circ}\) out of
phase. This can be interpreted in terms of a double mass, single spring
system, with the extension in the spring alternating in phase.

\hypertarget{problem-5}{%
\subsection{Problem 5}\label{problem-5}}

We obtain the magnitude and phase response of the steady state transfer
function of a low-pass network, second order network.

    \begin{Verbatim}[commandchars=\\\{\}]
{\color{incolor}In [{\color{incolor}10}]:} \PY{n}{H}\PY{o}{=}\PY{n}{sp}\PY{o}{.}\PY{n}{lti}\PY{p}{(}\PY{l+m+mi}{1}\PY{p}{,}\PY{n}{np}\PY{o}{.}\PY{n}{poly1d}\PY{p}{(}\PY{p}{[}\PY{l+m+mf}{1e\PYZhy{}12}\PY{p}{,}\PY{l+m+mf}{1e\PYZhy{}4}\PY{p}{,}\PY{l+m+mi}{1}\PY{p}{]}\PY{p}{)}\PY{p}{)}
         \PY{n+nb}{print}\PY{p}{(}\PY{n}{H}\PY{p}{)}
         \PY{n}{w}\PY{p}{,}\PY{n}{S}\PY{p}{,}\PY{n}{phi}\PY{o}{=}\PY{n}{H}\PY{o}{.}\PY{n}{bode}\PY{p}{(}\PY{p}{)}
         \PY{n}{fig}\PY{p}{,}\PY{p}{(}\PY{n}{ax1}\PY{p}{,}\PY{n}{ax2}\PY{p}{)}\PY{o}{=}\PY{n}{plt}\PY{o}{.}\PY{n}{subplots}\PY{p}{(}\PY{l+m+mi}{2}\PY{p}{,}\PY{l+m+mi}{1}\PY{p}{)}
         \PY{n}{ax1}\PY{o}{.}\PY{n}{semilogx}\PY{p}{(}\PY{n}{w}\PY{p}{,}\PY{n}{S}\PY{p}{)}
         \PY{n}{ax1}\PY{o}{.}\PY{n}{set\PYZus{}title}\PY{p}{(}\PY{l+s+s2}{\PYZdq{}}\PY{l+s+s2}{Magnitude plot}\PY{l+s+s2}{\PYZdq{}}\PY{p}{)}
         \PY{n}{ax2}\PY{o}{.}\PY{n}{semilogx}\PY{p}{(}\PY{n}{w}\PY{p}{,}\PY{n}{phi}\PY{p}{)}
         \PY{n}{ax2}\PY{o}{.}\PY{n}{set\PYZus{}title}\PY{p}{(}\PY{l+s+s2}{\PYZdq{}}\PY{l+s+s2}{Phase plot}\PY{l+s+s2}{\PYZdq{}}\PY{p}{)}
         \PY{n}{fig}\PY{o}{.}\PY{n}{tight\PYZus{}layout}\PY{p}{(}\PY{p}{)}
         \PY{n}{plt}\PY{o}{.}\PY{n}{show}\PY{p}{(}\PY{p}{)}
\end{Verbatim}


    \begin{Verbatim}[commandchars=\\\{\}]
TransferFunctionContinuous(
array([  1.00000000e+12]),
array([  1.00000000e+00,   1.00000000e+08,   1.00000000e+12]),
dt: None
)

    \end{Verbatim}

    \begin{center}
    \adjustimage{max size={0.9\linewidth}{0.9\paperheight}}{output_15_1.png}
    \end{center}
    { \hspace*{\fill} \\}
    
    \hypertarget{problem-6}{%
\subsection{Problem 6}\label{problem-6}}

With regard to the previous two port, we obtain the output for

\[ v_i(t) = cos (10^3t)u(t)−cos(10^6t)u(t)\]\\
Obtain the output voltage \(v_0(t)\) by defining the transfer function
as a system and obtaining the output using \emph{signal.lsim}. We
observe the output for a small timescale and a large timescale of
\(30\mu s\) and \(30m s\) respectively.

    \begin{Verbatim}[commandchars=\\\{\}]
{\color{incolor}In [{\color{incolor}11}]:} \PY{o}{\PYZpc{}}\PY{k}{matplotlib} inline
         \PY{n}{t}\PY{o}{=}\PY{n}{np}\PY{o}{.}\PY{n}{linspace}\PY{p}{(}\PY{l+m+mi}{0}\PY{p}{,}\PY{l+m+mf}{30e\PYZhy{}6}\PY{p}{,}\PY{l+m+mi}{1000}\PY{p}{)}
         \PY{n}{f}\PY{o}{=}\PY{n}{np}\PY{o}{.}\PY{n}{cos}\PY{p}{(}\PY{l+m+mf}{1e3}\PY{o}{*}\PY{n}{t}\PY{p}{)}\PY{o}{\PYZhy{}}\PY{n}{np}\PY{o}{.}\PY{n}{cos}\PY{p}{(}\PY{l+m+mf}{1e6}\PY{o}{*}\PY{n}{t}\PY{p}{)}
         \PY{n}{t}\PY{p}{,}\PY{n}{y}\PY{p}{,}\PY{n}{svec}\PY{o}{=}\PY{n}{sp}\PY{o}{.}\PY{n}{lsim}\PY{p}{(}\PY{n}{H}\PY{p}{,}\PY{n}{f}\PY{p}{,}\PY{n}{t}\PY{p}{)}
         \PY{n}{plt}\PY{o}{.}\PY{n}{figsize}\PY{o}{=}\PY{p}{(}\PY{l+m+mi}{20}\PY{p}{,}\PY{l+m+mi}{20}\PY{p}{)}
         \PY{n}{plt}\PY{o}{.}\PY{n}{plot}\PY{p}{(}\PY{n}{t}\PY{p}{,}\PY{n}{y}\PY{p}{)}
\end{Verbatim}


\begin{Verbatim}[commandchars=\\\{\}]
{\color{outcolor}Out[{\color{outcolor}11}]:} [<matplotlib.lines.Line2D at 0x10e5ca6d8>]
\end{Verbatim}
            
    \begin{center}
    \adjustimage{max size={0.9\linewidth}{0.9\paperheight}}{output_17_1.png}
    \end{center}
    { \hspace*{\fill} \\}
    
    \begin{Verbatim}[commandchars=\\\{\}]
{\color{incolor}In [{\color{incolor}31}]:} \PY{o}{\PYZpc{}}\PY{k}{matplotlib} inline
         \PY{n}{t}\PY{o}{=}\PY{n}{np}\PY{o}{.}\PY{n}{linspace}\PY{p}{(}\PY{l+m+mi}{0}\PY{p}{,}\PY{l+m+mf}{30e\PYZhy{}3}\PY{p}{,}\PY{l+m+mi}{10000}\PY{p}{)}
         \PY{n}{f}\PY{o}{=}\PY{n}{np}\PY{o}{.}\PY{n}{cos}\PY{p}{(}\PY{l+m+mf}{1e3}\PY{o}{*}\PY{n}{t}\PY{p}{)}\PY{o}{\PYZhy{}}\PY{n}{np}\PY{o}{.}\PY{n}{cos}\PY{p}{(}\PY{l+m+mf}{1e6}\PY{o}{*}\PY{n}{t}\PY{p}{)}
         \PY{n}{t}\PY{p}{,}\PY{n}{y}\PY{p}{,}\PY{n}{svec}\PY{o}{=}\PY{n}{sp}\PY{o}{.}\PY{n}{lsim}\PY{p}{(}\PY{n}{H}\PY{p}{,}\PY{n}{f}\PY{p}{,}\PY{n}{t}\PY{p}{)}
         \PY{n}{plt}\PY{o}{.}\PY{n}{figsize}\PY{o}{=}\PY{p}{(}\PY{l+m+mi}{20}\PY{p}{,}\PY{l+m+mi}{20}\PY{p}{)}
         \PY{n}{plt}\PY{o}{.}\PY{n}{plot}\PY{p}{(}\PY{n}{t}\PY{p}{,}\PY{n}{y}\PY{p}{)}
\end{Verbatim}


\begin{Verbatim}[commandchars=\\\{\}]
{\color{outcolor}Out[{\color{outcolor}31}]:} [<matplotlib.lines.Line2D at 0x1c111f4240>]
\end{Verbatim}
            
    \begin{center}
    \adjustimage{max size={0.9\linewidth}{0.9\paperheight}}{output_18_1.png}
    \end{center}
    { \hspace*{\fill} \\}
    
    Here, the high frequency component is diminshed at steady state, and the
low pass filter allows only the harmonic of \(1k Hz\) through. This is
seen in the large timespan. The smaller timespan shows the slight
pertubation caused by the diminished high frequency component.

This is similar to a linear assumption made for many pulse width
modulation systems, including DC-DC converters; where we approximate a
duty cycle modulated pulse train by its averaged step response.

\hypertarget{results-and-discussion}{%
\section{Results and Discussion}\label{results-and-discussion}}

Analysing LTI systems and approximate linear behaviour models are
crucial in many aspects of engineering. Here, we have explored the usage
of built-in python libraries under numpy's polynomial library, scipy's
signal processing and linear simulation modules.

This allows for convinient analysis of many systems, including spring
systems, electrical filters, coupled bodies,\ldots{}etc. We have covered
a few as a part of this assignment.


    % Add a bibliography block to the postdoc
    
    
    
    \end{document}
