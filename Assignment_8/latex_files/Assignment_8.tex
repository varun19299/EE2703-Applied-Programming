
% Default to the notebook output style

    


% Inherit from the specified cell style.




    
\documentclass[10pt,notitlepage,onecolumn,aps,pra]{revtex4-1}

    
    
\usepackage[T1]{fontenc}
\usepackage{graphicx}
% We will generate all images so they have a width \maxwidth. This means
% that they will get their normal width if they fit onto the page, but
% are scaled down if they would overflow the margins.
\makeatletter
\def\maxwidth{\ifdim\Gin@nat@width>\linewidth\linewidth
\else\Gin@nat@width\fi}
\makeatother
\let\Oldincludegraphics\includegraphics
% Set max figure width to be 80% of text width, for now hardcoded.
\renewcommand{\includegraphics}[1]{\Oldincludegraphics[width=.8\maxwidth]{#1}}
% Ensure that by default, figures have no caption (until we provide a
% proper Figure object with a Caption API and a way to capture that
% in the conversion process - todo).
\usepackage{caption}
\DeclareCaptionLabelFormat{nolabel}{}
\captionsetup{labelformat=nolabel}

\usepackage{adjustbox} % Used to constrain images to a maximum size
\usepackage{xcolor} % Allow colors to be defined
\usepackage{enumerate} % Needed for markdown enumerations to work
\usepackage{geometry} % Used to adjust the document margins
\usepackage{amsmath} % Equations
\usepackage{amssymb} % Equations
\usepackage{textcomp} % defines textquotesingle
% Hack from http://tex.stackexchange.com/a/47451/13684:
\AtBeginDocument{%
    \def\PYZsq{\textquotesingle}% Upright quotes in Pygmentized code
}
\usepackage{upquote} % Upright quotes for verbatim code
\usepackage{eurosym} % defines \euro
\usepackage[mathletters]{ucs} % Extended unicode (utf-8) support
\usepackage[utf8x]{inputenc} % Allow utf-8 characters in the tex document
\usepackage{fancyvrb} % verbatim replacement that allows latex
\usepackage{grffile} % extends the file name processing of package graphics
                     % to support a larger range
% The hyperref package gives us a pdf with properly built
% internal navigation ('pdf bookmarks' for the table of contents,
% internal cross-reference links, web links for URLs, etc.)
\usepackage{hyperref}
\usepackage{booktabs}  % table support for pandoc > 1.12.2
\usepackage[inline]{enumitem} % IRkernel/repr support (it uses the enumerate* environment)
\usepackage[normalem]{ulem} % ulem is needed to support strikethroughs (\sout)
                            % normalem makes italics be italics, not underlines
\usepackage{braket}


    
    
    % Colors for the hyperref package
    \definecolor{urlcolor}{rgb}{0,.145,.698}
    \definecolor{linkcolor}{rgb}{.71,0.21,0.01}
    \definecolor{citecolor}{rgb}{.12,.54,.11}

    % ANSI colors
    \definecolor{ansi-black}{HTML}{3E424D}
    \definecolor{ansi-black-intense}{HTML}{282C36}
    \definecolor{ansi-red}{HTML}{E75C58}
    \definecolor{ansi-red-intense}{HTML}{B22B31}
    \definecolor{ansi-green}{HTML}{00A250}
    \definecolor{ansi-green-intense}{HTML}{007427}
    \definecolor{ansi-yellow}{HTML}{DDB62B}
    \definecolor{ansi-yellow-intense}{HTML}{B27D12}
    \definecolor{ansi-blue}{HTML}{208FFB}
    \definecolor{ansi-blue-intense}{HTML}{0065CA}
    \definecolor{ansi-magenta}{HTML}{D160C4}
    \definecolor{ansi-magenta-intense}{HTML}{A03196}
    \definecolor{ansi-cyan}{HTML}{60C6C8}
    \definecolor{ansi-cyan-intense}{HTML}{258F8F}
    \definecolor{ansi-white}{HTML}{C5C1B4}
    \definecolor{ansi-white-intense}{HTML}{A1A6B2}

    % commands and environments needed by pandoc snippets
    % extracted from the output of `pandoc -s`
    \providecommand{\tightlist}{%
      \setlength{\itemsep}{0pt}\setlength{\parskip}{0pt}}
    \DefineVerbatimEnvironment{Highlighting}{Verbatim}{commandchars=\\\{\}}
    % Add ',fontsize=\small' for more characters per line
    \newenvironment{Shaded}{}{}
    \newcommand{\KeywordTok}[1]{\textcolor[rgb]{0.00,0.44,0.13}{\textbf{{#1}}}}
    \newcommand{\DataTypeTok}[1]{\textcolor[rgb]{0.56,0.13,0.00}{{#1}}}
    \newcommand{\DecValTok}[1]{\textcolor[rgb]{0.25,0.63,0.44}{{#1}}}
    \newcommand{\BaseNTok}[1]{\textcolor[rgb]{0.25,0.63,0.44}{{#1}}}
    \newcommand{\FloatTok}[1]{\textcolor[rgb]{0.25,0.63,0.44}{{#1}}}
    \newcommand{\CharTok}[1]{\textcolor[rgb]{0.25,0.44,0.63}{{#1}}}
    \newcommand{\StringTok}[1]{\textcolor[rgb]{0.25,0.44,0.63}{{#1}}}
    \newcommand{\CommentTok}[1]{\textcolor[rgb]{0.38,0.63,0.69}{\textit{{#1}}}}
    \newcommand{\OtherTok}[1]{\textcolor[rgb]{0.00,0.44,0.13}{{#1}}}
    \newcommand{\AlertTok}[1]{\textcolor[rgb]{1.00,0.00,0.00}{\textbf{{#1}}}}
    \newcommand{\FunctionTok}[1]{\textcolor[rgb]{0.02,0.16,0.49}{{#1}}}
    \newcommand{\RegionMarkerTok}[1]{{#1}}
    \newcommand{\ErrorTok}[1]{\textcolor[rgb]{1.00,0.00,0.00}{\textbf{{#1}}}}
    \newcommand{\NormalTok}[1]{{#1}}
    
    % Additional commands for more recent versions of Pandoc
    \newcommand{\ConstantTok}[1]{\textcolor[rgb]{0.53,0.00,0.00}{{#1}}}
    \newcommand{\SpecialCharTok}[1]{\textcolor[rgb]{0.25,0.44,0.63}{{#1}}}
    \newcommand{\VerbatimStringTok}[1]{\textcolor[rgb]{0.25,0.44,0.63}{{#1}}}
    \newcommand{\SpecialStringTok}[1]{\textcolor[rgb]{0.73,0.40,0.53}{{#1}}}
    \newcommand{\ImportTok}[1]{{#1}}
    \newcommand{\DocumentationTok}[1]{\textcolor[rgb]{0.73,0.13,0.13}{\textit{{#1}}}}
    \newcommand{\AnnotationTok}[1]{\textcolor[rgb]{0.38,0.63,0.69}{\textbf{\textit{{#1}}}}}
    \newcommand{\CommentVarTok}[1]{\textcolor[rgb]{0.38,0.63,0.69}{\textbf{\textit{{#1}}}}}
    \newcommand{\VariableTok}[1]{\textcolor[rgb]{0.10,0.09,0.49}{{#1}}}
    \newcommand{\ControlFlowTok}[1]{\textcolor[rgb]{0.00,0.44,0.13}{\textbf{{#1}}}}
    \newcommand{\OperatorTok}[1]{\textcolor[rgb]{0.40,0.40,0.40}{{#1}}}
    \newcommand{\BuiltInTok}[1]{{#1}}
    \newcommand{\ExtensionTok}[1]{{#1}}
    \newcommand{\PreprocessorTok}[1]{\textcolor[rgb]{0.74,0.48,0.00}{{#1}}}
    \newcommand{\AttributeTok}[1]{\textcolor[rgb]{0.49,0.56,0.16}{{#1}}}
    \newcommand{\InformationTok}[1]{\textcolor[rgb]{0.38,0.63,0.69}{\textbf{\textit{{#1}}}}}
    \newcommand{\WarningTok}[1]{\textcolor[rgb]{0.38,0.63,0.69}{\textbf{\textit{{#1}}}}}
    
    
    % Define a nice break command that doesn't care if a line doesn't already
    % exist.
    \def\br{\hspace*{\fill} \\* }
    % Math Jax compatability definitions
    \def\gt{>}
    \def\lt{<}
    \let\Oldtex\TeX
    \let\Oldlatex\LaTeX
    \renewcommand{\TeX}{\textrm{\Oldtex}}
    \renewcommand{\LaTeX}{\textrm{\Oldlatex}}
    % Document parameters
    % Document title
    
    
    
    

    % Pygments definitions
    
\makeatletter
\def\PY@reset{\let\PY@it=\relax \let\PY@bf=\relax%
    \let\PY@ul=\relax \let\PY@tc=\relax%
    \let\PY@bc=\relax \let\PY@ff=\relax}
\def\PY@tok#1{\csname PY@tok@#1\endcsname}
\def\PY@toks#1+{\ifx\relax#1\empty\else%
    \PY@tok{#1}\expandafter\PY@toks\fi}
\def\PY@do#1{\PY@bc{\PY@tc{\PY@ul{%
    \PY@it{\PY@bf{\PY@ff{#1}}}}}}}
\def\PY#1#2{\PY@reset\PY@toks#1+\relax+\PY@do{#2}}

\expandafter\def\csname PY@tok@gd\endcsname{\def\PY@tc##1{\textcolor[rgb]{0.63,0.00,0.00}{##1}}}
\expandafter\def\csname PY@tok@gu\endcsname{\let\PY@bf=\textbf\def\PY@tc##1{\textcolor[rgb]{0.50,0.00,0.50}{##1}}}
\expandafter\def\csname PY@tok@gt\endcsname{\def\PY@tc##1{\textcolor[rgb]{0.00,0.27,0.87}{##1}}}
\expandafter\def\csname PY@tok@gs\endcsname{\let\PY@bf=\textbf}
\expandafter\def\csname PY@tok@gr\endcsname{\def\PY@tc##1{\textcolor[rgb]{1.00,0.00,0.00}{##1}}}
\expandafter\def\csname PY@tok@cm\endcsname{\let\PY@it=\textit\def\PY@tc##1{\textcolor[rgb]{0.25,0.50,0.50}{##1}}}
\expandafter\def\csname PY@tok@vg\endcsname{\def\PY@tc##1{\textcolor[rgb]{0.10,0.09,0.49}{##1}}}
\expandafter\def\csname PY@tok@vi\endcsname{\def\PY@tc##1{\textcolor[rgb]{0.10,0.09,0.49}{##1}}}
\expandafter\def\csname PY@tok@vm\endcsname{\def\PY@tc##1{\textcolor[rgb]{0.10,0.09,0.49}{##1}}}
\expandafter\def\csname PY@tok@mh\endcsname{\def\PY@tc##1{\textcolor[rgb]{0.40,0.40,0.40}{##1}}}
\expandafter\def\csname PY@tok@cs\endcsname{\let\PY@it=\textit\def\PY@tc##1{\textcolor[rgb]{0.25,0.50,0.50}{##1}}}
\expandafter\def\csname PY@tok@ge\endcsname{\let\PY@it=\textit}
\expandafter\def\csname PY@tok@vc\endcsname{\def\PY@tc##1{\textcolor[rgb]{0.10,0.09,0.49}{##1}}}
\expandafter\def\csname PY@tok@il\endcsname{\def\PY@tc##1{\textcolor[rgb]{0.40,0.40,0.40}{##1}}}
\expandafter\def\csname PY@tok@go\endcsname{\def\PY@tc##1{\textcolor[rgb]{0.53,0.53,0.53}{##1}}}
\expandafter\def\csname PY@tok@cp\endcsname{\def\PY@tc##1{\textcolor[rgb]{0.74,0.48,0.00}{##1}}}
\expandafter\def\csname PY@tok@gi\endcsname{\def\PY@tc##1{\textcolor[rgb]{0.00,0.63,0.00}{##1}}}
\expandafter\def\csname PY@tok@gh\endcsname{\let\PY@bf=\textbf\def\PY@tc##1{\textcolor[rgb]{0.00,0.00,0.50}{##1}}}
\expandafter\def\csname PY@tok@ni\endcsname{\let\PY@bf=\textbf\def\PY@tc##1{\textcolor[rgb]{0.60,0.60,0.60}{##1}}}
\expandafter\def\csname PY@tok@nl\endcsname{\def\PY@tc##1{\textcolor[rgb]{0.63,0.63,0.00}{##1}}}
\expandafter\def\csname PY@tok@nn\endcsname{\let\PY@bf=\textbf\def\PY@tc##1{\textcolor[rgb]{0.00,0.00,1.00}{##1}}}
\expandafter\def\csname PY@tok@no\endcsname{\def\PY@tc##1{\textcolor[rgb]{0.53,0.00,0.00}{##1}}}
\expandafter\def\csname PY@tok@na\endcsname{\def\PY@tc##1{\textcolor[rgb]{0.49,0.56,0.16}{##1}}}
\expandafter\def\csname PY@tok@nb\endcsname{\def\PY@tc##1{\textcolor[rgb]{0.00,0.50,0.00}{##1}}}
\expandafter\def\csname PY@tok@nc\endcsname{\let\PY@bf=\textbf\def\PY@tc##1{\textcolor[rgb]{0.00,0.00,1.00}{##1}}}
\expandafter\def\csname PY@tok@nd\endcsname{\def\PY@tc##1{\textcolor[rgb]{0.67,0.13,1.00}{##1}}}
\expandafter\def\csname PY@tok@ne\endcsname{\let\PY@bf=\textbf\def\PY@tc##1{\textcolor[rgb]{0.82,0.25,0.23}{##1}}}
\expandafter\def\csname PY@tok@nf\endcsname{\def\PY@tc##1{\textcolor[rgb]{0.00,0.00,1.00}{##1}}}
\expandafter\def\csname PY@tok@si\endcsname{\let\PY@bf=\textbf\def\PY@tc##1{\textcolor[rgb]{0.73,0.40,0.53}{##1}}}
\expandafter\def\csname PY@tok@s2\endcsname{\def\PY@tc##1{\textcolor[rgb]{0.73,0.13,0.13}{##1}}}
\expandafter\def\csname PY@tok@nt\endcsname{\let\PY@bf=\textbf\def\PY@tc##1{\textcolor[rgb]{0.00,0.50,0.00}{##1}}}
\expandafter\def\csname PY@tok@nv\endcsname{\def\PY@tc##1{\textcolor[rgb]{0.10,0.09,0.49}{##1}}}
\expandafter\def\csname PY@tok@s1\endcsname{\def\PY@tc##1{\textcolor[rgb]{0.73,0.13,0.13}{##1}}}
\expandafter\def\csname PY@tok@dl\endcsname{\def\PY@tc##1{\textcolor[rgb]{0.73,0.13,0.13}{##1}}}
\expandafter\def\csname PY@tok@ch\endcsname{\let\PY@it=\textit\def\PY@tc##1{\textcolor[rgb]{0.25,0.50,0.50}{##1}}}
\expandafter\def\csname PY@tok@m\endcsname{\def\PY@tc##1{\textcolor[rgb]{0.40,0.40,0.40}{##1}}}
\expandafter\def\csname PY@tok@gp\endcsname{\let\PY@bf=\textbf\def\PY@tc##1{\textcolor[rgb]{0.00,0.00,0.50}{##1}}}
\expandafter\def\csname PY@tok@sh\endcsname{\def\PY@tc##1{\textcolor[rgb]{0.73,0.13,0.13}{##1}}}
\expandafter\def\csname PY@tok@ow\endcsname{\let\PY@bf=\textbf\def\PY@tc##1{\textcolor[rgb]{0.67,0.13,1.00}{##1}}}
\expandafter\def\csname PY@tok@sx\endcsname{\def\PY@tc##1{\textcolor[rgb]{0.00,0.50,0.00}{##1}}}
\expandafter\def\csname PY@tok@bp\endcsname{\def\PY@tc##1{\textcolor[rgb]{0.00,0.50,0.00}{##1}}}
\expandafter\def\csname PY@tok@c1\endcsname{\let\PY@it=\textit\def\PY@tc##1{\textcolor[rgb]{0.25,0.50,0.50}{##1}}}
\expandafter\def\csname PY@tok@fm\endcsname{\def\PY@tc##1{\textcolor[rgb]{0.00,0.00,1.00}{##1}}}
\expandafter\def\csname PY@tok@o\endcsname{\def\PY@tc##1{\textcolor[rgb]{0.40,0.40,0.40}{##1}}}
\expandafter\def\csname PY@tok@kc\endcsname{\let\PY@bf=\textbf\def\PY@tc##1{\textcolor[rgb]{0.00,0.50,0.00}{##1}}}
\expandafter\def\csname PY@tok@c\endcsname{\let\PY@it=\textit\def\PY@tc##1{\textcolor[rgb]{0.25,0.50,0.50}{##1}}}
\expandafter\def\csname PY@tok@mf\endcsname{\def\PY@tc##1{\textcolor[rgb]{0.40,0.40,0.40}{##1}}}
\expandafter\def\csname PY@tok@err\endcsname{\def\PY@bc##1{\setlength{\fboxsep}{0pt}\fcolorbox[rgb]{1.00,0.00,0.00}{1,1,1}{\strut ##1}}}
\expandafter\def\csname PY@tok@mb\endcsname{\def\PY@tc##1{\textcolor[rgb]{0.40,0.40,0.40}{##1}}}
\expandafter\def\csname PY@tok@ss\endcsname{\def\PY@tc##1{\textcolor[rgb]{0.10,0.09,0.49}{##1}}}
\expandafter\def\csname PY@tok@sr\endcsname{\def\PY@tc##1{\textcolor[rgb]{0.73,0.40,0.53}{##1}}}
\expandafter\def\csname PY@tok@mo\endcsname{\def\PY@tc##1{\textcolor[rgb]{0.40,0.40,0.40}{##1}}}
\expandafter\def\csname PY@tok@kd\endcsname{\let\PY@bf=\textbf\def\PY@tc##1{\textcolor[rgb]{0.00,0.50,0.00}{##1}}}
\expandafter\def\csname PY@tok@mi\endcsname{\def\PY@tc##1{\textcolor[rgb]{0.40,0.40,0.40}{##1}}}
\expandafter\def\csname PY@tok@kn\endcsname{\let\PY@bf=\textbf\def\PY@tc##1{\textcolor[rgb]{0.00,0.50,0.00}{##1}}}
\expandafter\def\csname PY@tok@cpf\endcsname{\let\PY@it=\textit\def\PY@tc##1{\textcolor[rgb]{0.25,0.50,0.50}{##1}}}
\expandafter\def\csname PY@tok@kr\endcsname{\let\PY@bf=\textbf\def\PY@tc##1{\textcolor[rgb]{0.00,0.50,0.00}{##1}}}
\expandafter\def\csname PY@tok@s\endcsname{\def\PY@tc##1{\textcolor[rgb]{0.73,0.13,0.13}{##1}}}
\expandafter\def\csname PY@tok@kp\endcsname{\def\PY@tc##1{\textcolor[rgb]{0.00,0.50,0.00}{##1}}}
\expandafter\def\csname PY@tok@w\endcsname{\def\PY@tc##1{\textcolor[rgb]{0.73,0.73,0.73}{##1}}}
\expandafter\def\csname PY@tok@kt\endcsname{\def\PY@tc##1{\textcolor[rgb]{0.69,0.00,0.25}{##1}}}
\expandafter\def\csname PY@tok@sc\endcsname{\def\PY@tc##1{\textcolor[rgb]{0.73,0.13,0.13}{##1}}}
\expandafter\def\csname PY@tok@sb\endcsname{\def\PY@tc##1{\textcolor[rgb]{0.73,0.13,0.13}{##1}}}
\expandafter\def\csname PY@tok@sa\endcsname{\def\PY@tc##1{\textcolor[rgb]{0.73,0.13,0.13}{##1}}}
\expandafter\def\csname PY@tok@k\endcsname{\let\PY@bf=\textbf\def\PY@tc##1{\textcolor[rgb]{0.00,0.50,0.00}{##1}}}
\expandafter\def\csname PY@tok@se\endcsname{\let\PY@bf=\textbf\def\PY@tc##1{\textcolor[rgb]{0.73,0.40,0.13}{##1}}}
\expandafter\def\csname PY@tok@sd\endcsname{\let\PY@it=\textit\def\PY@tc##1{\textcolor[rgb]{0.73,0.13,0.13}{##1}}}

\def\PYZbs{\char`\\}
\def\PYZus{\char`\_}
\def\PYZob{\char`\{}
\def\PYZcb{\char`\}}
\def\PYZca{\char`\^}
\def\PYZam{\char`\&}
\def\PYZlt{\char`\<}
\def\PYZgt{\char`\>}
\def\PYZsh{\char`\#}
\def\PYZpc{\char`\%}
\def\PYZdl{\char`\$}
\def\PYZhy{\char`\-}
\def\PYZsq{\char`\'}
\def\PYZdq{\char`\"}
\def\PYZti{\char`\~}
% for compatibility with earlier versions
\def\PYZat{@}
\def\PYZlb{[}
\def\PYZrb{]}
\makeatother


    % Exact colors from NB
    \definecolor{incolor}{rgb}{0.0, 0.0, 0.5}
    \definecolor{outcolor}{rgb}{0.545, 0.0, 0.0}



    
    % Prevent overflowing lines due to hard-to-break entities
    \sloppy 
    % Setup hyperref package
    \hypersetup{
      breaklinks=true,  % so long urls are correctly broken across lines
      colorlinks=true,
      urlcolor=urlcolor,
      linkcolor=linkcolor,
      citecolor=citecolor,
      }
    % Slightly bigger margins than the latex defaults
    
    \geometry{verbose,tmargin=1in,bmargin=1in,lmargin=1in,rmargin=1in}
    
    

    \begin{document}
    
    
    \title{Assignment\_8}\author{Michael Goerz}

\date{\today}
\maketitle


    
    

    
    \hypertarget{assignment-8}{%
\section{Assignment 8}\label{assignment-8}}

\emph{EE2703: Applied Programming Author: Varun Sundar, EE16B068}

\hypertarget{abstract}{%
\section{Abstract}\label{abstract}}

This week's assignment involves the analysis of active filter's using
laplce transforms. Python's symbolic solving library, sympy, comes off
as a particularly nifty tool to handle our requirements in solving
Modified Nodal Analysis (MNA) equations. Besides this the library also
includes useful classes to handle the simulation and response to inputs.

Coupled with \emph{scipy's signal} module, we are able to analyse both
Sallen-Key and Raunch filters, two common active filter topologies to
attain second order filters with just a single operational amplifier.

\hypertarget{introduction}{%
\section{Introduction}\label{introduction}}

Reference to Horowitz and Hill (Active Filters) :
\href{https://artofelectronics.net/}{link}

We begin by understanding the usage of sympy in defining, evaluating and
plotting symbolic equations. Besides this, we also examine the use case
of matrix equation for MNA solutions. With this, we start analysing the
case of an active low pass filter.

Conventions: 1. We are using Python 3, GCC for C 2. Underscore naming vs
Camel Case 3. PEP 25 convention style.
\begin{Verbatim}[commandchars=\\\{\}]
{\color{incolor}In [{\color{incolor}216}]:} \PY{k+kn}{import} \PY{n+nn}{sympy}
          \PY{k+kn}{from} \PY{n+nn}{IPython}\PY{n+nn}{.}\PY{n+nn}{display} \PY{k}{import} \PY{n}{display}
          \PY{k+kn}{import} \PY{n+nn}{numpy} \PY{k}{as} \PY{n+nn}{np}
          \PY{k+kn}{import} \PY{n+nn}{matplotlib}\PY{n+nn}{.}\PY{n+nn}{pyplot} \PY{k}{as} \PY{n+nn}{plt}
          
          \PY{k+kn}{import} \PY{n+nn}{scipy}\PY{n+nn}{.}\PY{n+nn}{signal} \PY{k}{as} \PY{n+nn}{sp}
          
          \PY{c+c1}{\PYZsh{} Use either}
          
          \PY{n}{sympy}\PY{o}{.}\PY{n}{init\PYZus{}session}
          \PY{c+c1}{\PYZsh{}sympy.init\PYZus{}printing(use\PYZus{}unicode=True)}
          \PY{n}{sympy}\PY{o}{.}\PY{n}{init\PYZus{}printing}\PY{p}{(}\PY{n}{use\PYZus{}latex}\PY{o}{=}\PY{l+s+s1}{\PYZsq{}}\PY{l+s+s1}{mathjax}\PY{l+s+s1}{\PYZsq{}}\PY{p}{)}
\end{Verbatim}

    Here, we use sympy's lambdify method to plot the Bode Plot of the
transfer function, \[ H(s) = 1/(s^3+2*s^2+2s+1) \]
\begin{Verbatim}[commandchars=\\\{\}]
{\color{incolor}In [{\color{incolor}217}]:} \PY{n}{s}\PY{o}{=}\PY{n}{sympy}\PY{o}{.}\PY{n}{symbols}\PY{p}{(}\PY{l+s+s1}{\PYZsq{}}\PY{l+s+s1}{s}\PY{l+s+s1}{\PYZsq{}}\PY{p}{)}
          \PY{n}{h}\PY{o}{=}\PY{l+m+mi}{1}\PY{o}{/}\PY{p}{(}\PY{n}{s}\PY{o}{*}\PY{o}{*}\PY{l+m+mi}{3}\PY{o}{+}\PY{l+m+mi}{2}\PY{o}{*}\PY{n}{s}\PY{o}{*}\PY{o}{*}\PY{l+m+mi}{2}\PY{o}{+}\PY{l+m+mi}{2}\PY{o}{*}\PY{n}{s}\PY{o}{+}\PY{l+m+mi}{1}\PY{p}{)}
          \PY{n}{w}\PY{o}{=}\PY{n}{np}\PY{o}{.}\PY{n}{logspace}\PY{p}{(}\PY{o}{\PYZhy{}}\PY{l+m+mi}{1}\PY{p}{,}\PY{l+m+mi}{1}\PY{p}{,}\PY{l+m+mi}{21}\PY{p}{)}
          \PY{n}{ss}\PY{o}{=}\PY{l+m+mi}{1}\PY{n}{j}\PY{o}{*}\PY{n}{w}
          \PY{n}{f}\PY{o}{=}\PY{n}{sympy}\PY{o}{.}\PY{n}{lambdify}\PY{p}{(}\PY{n}{s}\PY{p}{,}\PY{n}{h}\PY{p}{,}\PY{l+s+s2}{\PYZdq{}}\PY{l+s+s2}{numpy}\PY{l+s+s2}{\PYZdq{}}\PY{p}{)}
          
          \PY{n}{plt}\PY{o}{.}\PY{n}{loglog}\PY{p}{(}\PY{n}{w}\PY{p}{,}\PY{n}{np}\PY{o}{.}\PY{n}{abs}\PY{p}{(}\PY{n}{f}\PY{p}{(}\PY{n}{ss}\PY{p}{)}\PY{p}{)}\PY{p}{)}
          \PY{n}{plt}\PY{o}{.}\PY{n}{grid}\PY{p}{(}\PY{k+kc}{True}\PY{p}{)}
          \PY{n}{plt}\PY{o}{.}\PY{n}{show}\PY{p}{(}\PY{p}{)}
\end{Verbatim}

    \begin{center}
    \adjustimage{max size={0.9\linewidth}{0.9\paperheight}}{Assignment_8_files/Assignment_8_3_0.png}
    \end{center}
    { \hspace*{\fill} \\}
    
    We solve for an active low pass filter (Sallen-Key). This particular
active filter is known for being a single op-amp based low pass filter.

The MNA (Nodal) Equations may be recast as:

\[
\begin{gather}
\begin{bmatrix}
   0& 0& 1& -1/G\\   
   −\frac{1}{1+s R_2 C_2}& 1& 0& 0&\\
   0& −G& G& 1&\\
   -1/R_1-1/R_2-sC_1& 1/R_2& 0& sC_1&
   \end{bmatrix}
   *
   \begin{bmatrix} V_1 \\ V_p \\ V_m\\ V_o \end{bmatrix}
   =
   \begin{bmatrix} 0\\ 0\\ 0\\ V_i(s)/R \end{bmatrix}
\end{gather}
\]

Note that the lab assignment sheet used,

\[ V_o=G(V_p −V_m) \]

incorrectly, since the op-amp is a large gain device. This should be
replaced by:

\[ V_o=G(V_p) \] which correctly replaces the intended gain block.
\begin{Verbatim}[commandchars=\\\{\}]
{\color{incolor}In [{\color{incolor}218}]:} \PY{k}{def} \PY{n+nf}{lowpass}\PY{p}{(}\PY{n}{R1}\PY{p}{,}\PY{n}{R2}\PY{p}{,}\PY{n}{C1}\PY{p}{,}\PY{n}{C2}\PY{p}{,}\PY{n}{G}\PY{p}{,}\PY{n}{Vi}\PY{p}{)}\PY{p}{:}
              \PY{n}{s}\PY{o}{=}\PY{n}{sympy}\PY{o}{.}\PY{n}{symbols}\PY{p}{(}\PY{l+s+s1}{\PYZsq{}}\PY{l+s+s1}{s}\PY{l+s+s1}{\PYZsq{}}\PY{p}{)}
              \PY{n}{A}\PY{o}{=}\PY{n}{sympy}\PY{o}{.}\PY{n}{Matrix}\PY{p}{(}\PY{p}{[}\PY{p}{[}\PY{l+m+mi}{0}\PY{p}{,}\PY{l+m+mi}{0}\PY{p}{,}\PY{l+m+mi}{1}\PY{p}{,}\PY{o}{\PYZhy{}}\PY{l+m+mi}{1}\PY{o}{/}\PY{n}{G}\PY{p}{]}\PY{p}{,}\PY{p}{[}\PY{o}{\PYZhy{}}\PY{l+m+mi}{1}\PY{o}{/}\PY{p}{(}\PY{l+m+mi}{1}\PY{o}{+}\PY{n}{s}\PY{o}{*}\PY{n}{R2}\PY{o}{*}\PY{n}{C2}\PY{p}{)}\PY{p}{,}\PY{l+m+mi}{1}\PY{p}{,}\PY{l+m+mi}{0}\PY{p}{,}\PY{l+m+mi}{0}\PY{p}{]}\PY{p}{,} \PYZbs{}
              \PY{p}{[}\PY{l+m+mi}{0}\PY{p}{,}\PY{o}{\PYZhy{}}\PY{n}{G}\PY{p}{,}\PY{l+m+mi}{0}\PY{p}{,}\PY{l+m+mi}{1}\PY{p}{]}\PY{p}{,}\PY{p}{[}\PY{o}{\PYZhy{}}\PY{l+m+mi}{1}\PY{o}{/}\PY{n}{R1}\PY{o}{\PYZhy{}}\PY{l+m+mi}{1}\PY{o}{/}\PY{n}{R2}\PY{o}{\PYZhy{}}\PY{n}{s}\PY{o}{*}\PY{n}{C1}\PY{p}{,}\PY{l+m+mi}{1}\PY{o}{/}\PY{n}{R2}\PY{p}{,}\PY{l+m+mi}{0}\PY{p}{,}\PY{n}{s}\PY{o}{*}\PY{n}{C1}\PY{p}{]}\PY{p}{]}\PY{p}{)}
              \PY{n}{b}\PY{o}{=}\PY{n}{sympy}\PY{o}{.}\PY{n}{Matrix}\PY{p}{(}\PY{p}{[}\PY{l+m+mi}{0}\PY{p}{,}\PY{l+m+mi}{0}\PY{p}{,}\PY{l+m+mi}{0}\PY{p}{,}\PY{n}{Vi}\PY{o}{/}\PY{n}{R1}\PY{p}{]}\PY{p}{)}
              \PY{n}{V}\PY{o}{=}\PY{n}{A}\PY{o}{.}\PY{n}{inv}\PY{p}{(}\PY{p}{)}\PY{o}{*}\PY{n}{b}
              \PY{k}{return} \PY{p}{(}\PY{n}{A}\PY{p}{,}\PY{n}{b}\PY{p}{,}\PY{n}{V}\PY{p}{)}
\end{Verbatim}
\begin{Verbatim}[commandchars=\\\{\}]
{\color{incolor}In [{\color{incolor}219}]:} \PY{n}{A}\PY{p}{,}\PY{n}{b}\PY{p}{,}\PY{n}{V}\PY{o}{=}\PY{n}{lowpass}\PY{p}{(}\PY{l+m+mi}{10000}\PY{p}{,}\PY{l+m+mi}{10000}\PY{p}{,}\PY{l+m+mf}{1e\PYZhy{}9}\PY{p}{,}\PY{l+m+mf}{1e\PYZhy{}9}\PY{p}{,}\PY{l+m+mf}{1.586}\PY{p}{,}\PY{l+m+mi}{1}\PY{p}{)}
          
          \PY{n+nb}{print} \PY{p}{(}\PY{l+s+s1}{\PYZsq{}}\PY{l+s+s1}{G=1000}\PY{l+s+s1}{\PYZsq{}}\PY{p}{)}
          \PY{n}{Vo}\PY{o}{=}\PY{n}{V}\PY{p}{[}\PY{l+m+mi}{3}\PY{p}{]}
          \PY{c+c1}{\PYZsh{} Computed inverse}
          \PY{n+nb}{print} \PY{p}{(}\PY{l+s+s2}{\PYZdq{}}\PY{l+s+s2}{Computed Inverse}\PY{l+s+s2}{\PYZdq{}}\PY{p}{)}
          \PY{n}{display}\PY{p}{(}\PY{n}{Vo}\PY{p}{)}
          \PY{c+c1}{\PYZsh{} Simplify existing expression}
          \PY{n+nb}{print} \PY{p}{(}\PY{l+s+s2}{\PYZdq{}}\PY{l+s+s2}{Simplified Expression}\PY{l+s+s2}{\PYZdq{}}\PY{p}{)}
          \PY{n}{Vo}\PY{o}{=}\PY{n}{sympy}\PY{o}{.}\PY{n}{simplify}\PY{p}{(}\PY{n}{Vo}\PY{p}{)}
          \PY{n}{display}\PY{p}{(}\PY{n}{Vo}\PY{p}{)}
          
          \PY{n}{s}\PY{o}{=}\PY{n}{sympy}\PY{o}{.}\PY{n}{symbols}\PY{p}{(}\PY{l+s+s1}{\PYZsq{}}\PY{l+s+s1}{s}\PY{l+s+s1}{\PYZsq{}}\PY{p}{)}
          \PY{n}{display}\PY{p}{(}\PY{p}{(}\PY{l+m+mi}{1}\PY{o}{/}\PY{n}{Vo}\PY{p}{)}\PY{o}{.}\PY{n}{coeff}\PY{p}{(}\PY{n}{s}\PY{p}{)}\PY{p}{)}
          
          \PY{n}{a}\PY{o}{=}\PY{n}{sympy}\PY{o}{.}\PY{n}{Poly}\PY{p}{(}\PY{l+m+mi}{1}\PY{o}{/}\PY{n}{Vo}\PY{p}{,}\PY{n}{s}\PY{p}{)}
          \PY{n}{display}\PY{p}{(}\PY{n}{a}\PY{o}{.}\PY{n}{all\PYZus{}coeffs}\PY{p}{(}\PY{p}{)}\PY{p}{)}
          
          \PY{n}{w}\PY{o}{=}\PY{n}{np}\PY{o}{.}\PY{n}{logspace}\PY{p}{(}\PY{l+m+mi}{0}\PY{p}{,}\PY{l+m+mi}{8}\PY{p}{,}\PY{l+m+mi}{801}\PY{p}{)}
          \PY{n}{ss}\PY{o}{=}\PY{l+m+mi}{1}\PY{n}{j}\PY{o}{*}\PY{n}{w}
          \PY{n}{hf}\PY{o}{=}\PY{n}{sympy}\PY{o}{.}\PY{n}{lambdify}\PY{p}{(}\PY{n}{s}\PY{p}{,}\PY{n}{Vo}\PY{p}{,}\PY{l+s+s2}{\PYZdq{}}\PY{l+s+s2}{numpy}\PY{l+s+s2}{\PYZdq{}}\PY{p}{)}
          \PY{n}{v}\PY{o}{=}\PY{n}{hf}\PY{p}{(}\PY{n}{ss}\PY{p}{)}
          
          \PY{n}{plt}\PY{o}{.}\PY{n}{loglog}\PY{p}{(}\PY{n}{w}\PY{p}{,}\PY{n+nb}{abs}\PY{p}{(}\PY{n}{v}\PY{p}{)}\PY{p}{,}\PY{n}{lw}\PY{o}{=}\PY{l+m+mi}{2}\PY{p}{)}
          \PY{n}{plt}\PY{o}{.}\PY{n}{grid}\PY{p}{(}\PY{k+kc}{True}\PY{p}{)}
          \PY{n}{plt}\PY{o}{.}\PY{n}{show}\PY{p}{(}\PY{p}{)}
\end{Verbatim}

    \begin{Verbatim}[commandchars=\\\{\}]
G=1000
Computed Inverse

    \end{Verbatim}

    $$\frac{0.0001586}{\left(1.0 \cdot 10^{-5} s + 1\right) \left(- 1.0 \cdot 10^{-9} s + \frac{1.586 \cdot 10^{-9} s}{1.0 \cdot 10^{-5} s + 1} - 0.0002 + \frac{0.0001}{1.0 \cdot 10^{-5} s + 1}\right)}$$

    
    \begin{Verbatim}[commandchars=\\\{\}]
Simplified Expression

    \end{Verbatim}

    $$- \frac{0.0001586}{1.0 \cdot 10^{-14} s^{2} + 1.414 \cdot 10^{-9} s + 0.0001}$$

    
    $$-8.91551071878941 \cdot 10^{-6}$$

    
    $$\left [ -6.30517023959647 \cdot 10^{-11}, \quad -8.91551071878941 \cdot 10^{-6}, \quad -0.630517023959647\right ]$$

    
    \begin{center}
    \adjustimage{max size={0.9\linewidth}{0.9\paperheight}}{Assignment_8_files/Assignment_8_6_6.png}
    \end{center}
    { \hspace*{\fill} \\}
    
    \hypertarget{assignment-questions}{%
\section{Assignment Questions}\label{assignment-questions}}

\hypertarget{question-1}{%
\subsection{Question 1}\label{question-1}}

To obtain the step response of the circuit, we do so in the laplace
domain and then convert it to the time domain.
\begin{Verbatim}[commandchars=\\\{\}]
{\color{incolor}In [{\color{incolor}220}]:} \PY{n}{H}\PY{o}{=}\PY{n}{Vo}\PY{o}{*}\PY{l+m+mi}{1}\PY{o}{/}\PY{n}{s}
          \PY{n}{display}\PY{p}{(}\PY{n}{H}\PY{p}{)}
          \PY{n}{H}\PY{o}{=}\PY{n}{sympy}\PY{o}{.}\PY{n}{simplify}\PY{p}{(}\PY{n}{H}\PY{p}{)}
          \PY{n+nb}{print} \PY{p}{(}\PY{l+s+s2}{\PYZdq{}}\PY{l+s+s2}{ Simplified Expression }\PY{l+s+s2}{\PYZdq{}}\PY{p}{)}
          \PY{n}{display}\PY{p}{(}\PY{n}{H}\PY{p}{)}
\end{Verbatim}

    $$- \frac{0.0001586}{s \left(1.0 \cdot 10^{-14} s^{2} + 1.414 \cdot 10^{-9} s + 0.0001\right)}$$

    
    \begin{Verbatim}[commandchars=\\\{\}]
 Simplified Expression

    \end{Verbatim}

    $$- \frac{0.0001586}{s \left(1.0 \cdot 10^{-14} s^{2} + 1.414 \cdot 10^{-9} s + 0.0001\right)}$$

    
    We compute the time-domain response of the active low pass filter.
\begin{Verbatim}[commandchars=\\\{\}]
{\color{incolor}In [{\color{incolor}237}]:} \PY{n}{A}\PY{p}{,}\PY{n}{b}\PY{p}{,}\PY{n}{V}\PY{o}{=}\PY{n}{lowpass}\PY{p}{(}\PY{l+m+mi}{10000}\PY{p}{,}\PY{l+m+mi}{10000}\PY{p}{,}\PY{l+m+mf}{1e\PYZhy{}9}\PY{p}{,}\PY{l+m+mf}{1e\PYZhy{}9}\PY{p}{,}\PY{l+m+mf}{1.586}\PY{p}{,}\PY{l+m+mi}{1}\PY{p}{)}
          
          \PY{n+nb}{print} \PY{p}{(}\PY{l+s+s1}{\PYZsq{}}\PY{l+s+s1}{G=1000}\PY{l+s+s1}{\PYZsq{}}\PY{p}{)}
          \PY{n}{Vo}\PY{o}{=}\PY{n}{V}\PY{p}{[}\PY{l+m+mi}{3}\PY{p}{]}
          \PY{n}{Vo}\PY{o}{=}\PY{n}{sympy}\PY{o}{.}\PY{n}{simplify}\PY{p}{(}\PY{n}{Vo}\PY{p}{)}
          \PY{n}{display}\PY{p}{(}\PY{n}{Vo}\PY{p}{)}
          
          \PY{n}{s}\PY{p}{,}\PY{n}{t}\PY{o}{=}\PY{n}{sympy}\PY{o}{.}\PY{n}{symbols}\PY{p}{(}\PY{l+s+s2}{\PYZdq{}}\PY{l+s+s2}{s t}\PY{l+s+s2}{\PYZdq{}}\PY{p}{)}
          \PY{n}{t}\PY{o}{=}\PY{n}{sympy}\PY{o}{.}\PY{n}{Symbol}\PY{p}{(}\PY{l+s+s2}{\PYZdq{}}\PY{l+s+s2}{t}\PY{l+s+s2}{\PYZdq{}}\PY{p}{,}\PY{n}{positive}\PY{o}{=}\PY{k+kc}{True}\PY{p}{)}
          \PY{n}{n}\PY{p}{,}\PY{n}{d} \PY{o}{=} \PY{n}{sympy}\PY{o}{.}\PY{n}{fraction}\PY{p}{(}\PY{n}{Vo}\PY{p}{)}
          \PY{n}{n\PYZus{}sp}\PY{p}{,}\PY{n}{d\PYZus{}sp}\PY{o}{=}\PY{p}{(}\PY{n}{np}\PY{o}{.}\PY{n}{array}\PY{p}{(}\PY{n}{sympy}\PY{o}{.}\PY{n}{Poly}\PY{p}{(}\PY{n}{j}\PY{p}{,}\PY{n}{s}\PY{p}{)}\PY{o}{.}\PY{n}{all\PYZus{}coeffs}\PY{p}{(}\PY{p}{)}\PY{p}{,}\PY{n}{dtype}\PY{o}{=}\PY{n+nb}{float}\PY{p}{)} \PY{k}{for} \PY{n}{j} \PY{o+ow}{in} \PY{p}{(}\PY{n}{n}\PY{p}{,}\PY{n}{d}\PY{p}{)}\PY{p}{)}
          
          \PY{n+nb}{print}\PY{p}{(}\PY{n}{n\PYZus{}sp}\PY{p}{,}\PY{n}{d\PYZus{}sp}\PY{p}{)}
          \PY{n}{ts}\PY{o}{=}\PY{n}{np}\PY{o}{.}\PY{n}{linspace}\PY{p}{(}\PY{l+m+mi}{0}\PY{p}{,}\PY{l+m+mf}{0.001}\PY{p}{,}\PY{l+m+mi}{8001}\PY{p}{)}
          \PY{n}{t}\PY{p}{,}\PY{n}{x}\PY{p}{,}\PY{n}{svec}\PY{o}{=}\PY{n}{sp}\PY{o}{.}\PY{n}{lsim}\PY{p}{(}\PY{n}{sp}\PY{o}{.}\PY{n}{lti}\PY{p}{(}\PY{n}{n\PYZus{}sp}\PY{p}{,}\PY{n}{d\PYZus{}sp}\PY{p}{)}\PY{p}{,}\PY{n}{np}\PY{o}{.}\PY{n}{ones}\PY{p}{(}\PY{n+nb}{len}\PY{p}{(}\PY{n}{ts}\PY{p}{)}\PY{p}{)}\PY{p}{,}\PY{n}{ts}\PY{p}{)}
          \PY{c+c1}{\PYZsh{} Plot the absolute step response}
          \PY{n}{plt}\PY{o}{.}\PY{n}{plot}\PY{p}{(}\PY{n}{t}\PY{p}{,}\PY{n}{np}\PY{o}{.}\PY{n}{abs}\PY{p}{(}\PY{n}{x}\PY{p}{)}\PY{p}{,}\PY{n}{lw}\PY{o}{=}\PY{l+m+mi}{2}\PY{p}{)}
          \PY{n}{plt}\PY{o}{.}\PY{n}{grid}\PY{p}{(}\PY{k+kc}{True}\PY{p}{)}
          \PY{n}{plt}\PY{o}{.}\PY{n}{show}\PY{p}{(}\PY{p}{)}
\end{Verbatim}

    \begin{Verbatim}[commandchars=\\\{\}]
G=1000

    \end{Verbatim}

    $$- \frac{0.0001586}{1.0 \cdot 10^{-14} s^{2} + 1.414 \cdot 10^{-9} s + 0.0001}$$

    
    \begin{Verbatim}[commandchars=\\\{\}]
[-0.0001586] [  1.00000000e-14   1.41400000e-09   1.00000000e-04]

    \end{Verbatim}

    \begin{center}
    \adjustimage{max size={0.9\linewidth}{0.9\paperheight}}{Assignment_8_files/Assignment_8_10_3.png}
    \end{center}
    { \hspace*{\fill} \\}
    
    \hypertarget{question-2}{%
\subsection{Question 2}\label{question-2}}

Here, the input is\\
\$ v\_i(t) = (sin(2000\pi t)+cos (2*10\^{}6 \pi t))u(t)\$ Volts

We need to determine the output voltage \(v_0(t)\).
\begin{Verbatim}[commandchars=\\\{\}]
{\color{incolor}In [{\color{incolor}222}]:} \PY{n}{n}\PY{p}{,}\PY{n}{d} \PY{o}{=} \PY{n}{sympy}\PY{o}{.}\PY{n}{fraction}\PY{p}{(}\PY{n}{Vo}\PY{p}{)}
          \PY{n}{n\PYZus{}sp}\PY{p}{,}\PY{n}{d\PYZus{}sp}\PY{o}{=}\PY{p}{(}\PY{n}{np}\PY{o}{.}\PY{n}{array}\PY{p}{(}\PY{n}{sympy}\PY{o}{.}\PY{n}{Poly}\PY{p}{(}\PY{n}{j}\PY{p}{,}\PY{n}{s}\PY{p}{)}\PY{o}{.}\PY{n}{all\PYZus{}coeffs}\PY{p}{(}\PY{p}{)}\PY{p}{,}\PY{n}{dtype}\PY{o}{=}\PY{n+nb}{float}\PY{p}{)} \PY{k}{for} \PY{n}{j} \PY{o+ow}{in} \PY{p}{(}\PY{n}{n}\PY{p}{,}\PY{n}{d}\PY{p}{)}\PY{p}{)}
          
          \PY{c+c1}{\PYZsh{} Function to simulate}
          \PY{n}{ts}\PY{o}{=}\PY{n}{np}\PY{o}{.}\PY{n}{linspace}\PY{p}{(}\PY{l+m+mi}{0}\PY{p}{,}\PY{l+m+mf}{0.01}\PY{p}{,}\PY{l+m+mi}{8001}\PY{p}{)}
          \PY{n}{vi}\PY{o}{=} \PY{n}{np}\PY{o}{.}\PY{n}{sin}\PY{p}{(}\PY{l+m+mi}{2000}\PY{o}{*}\PY{n}{np}\PY{o}{.}\PY{n}{pi}\PY{o}{*}\PY{n}{ts}\PY{p}{)}\PY{o}{+}\PY{n}{np}\PY{o}{.}\PY{n}{cos}\PY{p}{(}\PY{l+m+mi}{2}\PY{o}{*}\PY{l+m+mi}{10}\PY{o}{*}\PY{o}{*}\PY{l+m+mi}{6}\PY{o}{*}\PY{n}{np}\PY{o}{.}\PY{n}{pi}\PY{o}{*}\PY{n}{ts}\PY{p}{)}
          
          \PY{n}{t}\PY{p}{,}\PY{n}{x}\PY{p}{,}\PY{n}{svec}\PY{o}{=}\PY{n}{sp}\PY{o}{.}\PY{n}{lsim}\PY{p}{(}\PY{n}{sp}\PY{o}{.}\PY{n}{lti}\PY{p}{(}\PY{n}{n\PYZus{}sp}\PY{p}{,}\PY{n}{d\PYZus{}sp}\PY{p}{)}\PY{p}{,}\PY{n}{vi}\PY{p}{,}\PY{n}{ts}\PY{p}{)}
          \PY{c+c1}{\PYZsh{} Plot the lamdified values}
          \PY{n}{plt}\PY{o}{.}\PY{n}{plot}\PY{p}{(}\PY{n}{t}\PY{p}{,}\PY{n}{x}\PY{p}{,}\PY{n}{lw}\PY{o}{=}\PY{l+m+mi}{2}\PY{p}{)}
          \PY{n}{plt}\PY{o}{.}\PY{n}{grid}\PY{p}{(}\PY{k+kc}{True}\PY{p}{)}
          \PY{n}{plt}\PY{o}{.}\PY{n}{show}\PY{p}{(}\PY{p}{)}
\end{Verbatim}

    \begin{center}
    \adjustimage{max size={0.9\linewidth}{0.9\paperheight}}{Assignment_8_files/Assignment_8_12_0.png}
    \end{center}
    { \hspace*{\fill} \\}
    
    Given that this is a low pass filter, we expect attenuated sinusoids as
outputs (which is indeed the case). Also, note that the component due to
a frequency of \(10^6 Hz\) is nearly attenutated out.

To visualise this, we vary the frequency of the cosine component from
\(10^3\) to \(10^4\) Hertz.
\begin{Verbatim}[commandchars=\\\{\}]
{\color{incolor}In [{\color{incolor}223}]:} \PY{n}{n}\PY{p}{,}\PY{n}{d} \PY{o}{=} \PY{n}{sympy}\PY{o}{.}\PY{n}{fraction}\PY{p}{(}\PY{n}{Vo}\PY{p}{)}
          \PY{n}{n\PYZus{}sp}\PY{p}{,}\PY{n}{d\PYZus{}sp}\PY{o}{=}\PY{p}{(}\PY{n}{np}\PY{o}{.}\PY{n}{array}\PY{p}{(}\PY{n}{sympy}\PY{o}{.}\PY{n}{Poly}\PY{p}{(}\PY{n}{j}\PY{p}{,}\PY{n}{s}\PY{p}{)}\PY{o}{.}\PY{n}{all\PYZus{}coeffs}\PY{p}{(}\PY{p}{)}\PY{p}{,}\PY{n}{dtype}\PY{o}{=}\PY{n+nb}{float}\PY{p}{)} \PY{k}{for} \PY{n}{j} \PY{o+ow}{in} \PY{p}{(}\PY{n}{n}\PY{p}{,}\PY{n}{d}\PY{p}{)}\PY{p}{)}
          
          \PY{n}{n}\PY{o}{=}\PY{p}{[}\PY{l+m+mi}{3}\PY{p}{,}\PY{l+m+mi}{4}\PY{p}{,}\PY{l+m+mi}{5}\PY{p}{,}\PY{l+m+mi}{6}\PY{p}{,}\PY{l+m+mi}{7}\PY{p}{]}
          \PY{k}{for} \PY{n}{i} \PY{o+ow}{in} \PY{n}{n}\PY{p}{:}
              \PY{c+c1}{\PYZsh{} Function to simulate}
              \PY{n}{ts}\PY{o}{=}\PY{n}{np}\PY{o}{.}\PY{n}{linspace}\PY{p}{(}\PY{l+m+mi}{0}\PY{p}{,}\PY{l+m+mf}{0.01}\PY{p}{,}\PY{l+m+mi}{8001}\PY{p}{)}
              \PY{n}{vi}\PY{o}{=} \PY{n}{np}\PY{o}{.}\PY{n}{sin}\PY{p}{(}\PY{l+m+mi}{2000}\PY{o}{*}\PY{n}{np}\PY{o}{.}\PY{n}{pi}\PY{o}{*}\PY{n}{ts}\PY{p}{)}\PY{o}{+}\PY{n}{np}\PY{o}{.}\PY{n}{cos}\PY{p}{(}\PY{l+m+mi}{2}\PY{o}{*}\PY{l+m+mi}{10}\PY{o}{*}\PY{o}{*}\PY{n}{i}\PY{o}{*}\PY{n}{np}\PY{o}{.}\PY{n}{pi}\PY{o}{*}\PY{n}{ts}\PY{p}{)}
          
              \PY{n}{t}\PY{p}{,}\PY{n}{x}\PY{p}{,}\PY{n}{svec}\PY{o}{=}\PY{n}{sp}\PY{o}{.}\PY{n}{lsim}\PY{p}{(}\PY{n}{sp}\PY{o}{.}\PY{n}{lti}\PY{p}{(}\PY{n}{n\PYZus{}sp}\PY{p}{,}\PY{n}{d\PYZus{}sp}\PY{p}{)}\PY{p}{,}\PY{n}{vi}\PY{p}{,}\PY{n}{ts}\PY{p}{)}
              \PY{c+c1}{\PYZsh{} Plot the lamdified values}
              \PY{n}{plt}\PY{o}{.}\PY{n}{plot}\PY{p}{(}\PY{n}{t}\PY{p}{,}\PY{n}{x}\PY{p}{,}\PY{n}{lw}\PY{o}{=}\PY{l+m+mi}{2}\PY{p}{)}
              \PY{n}{plt}\PY{o}{.}\PY{n}{grid}\PY{p}{(}\PY{k+kc}{True}\PY{p}{)}
              \PY{n}{plt}\PY{o}{.}\PY{n}{title}\PY{p}{(}\PY{l+s+s2}{\PYZdq{}}\PY{l+s+s2}{At Frequency of Cosine Component = 10\PYZca{}}\PY{l+s+s2}{\PYZdq{}}\PY{o}{+}\PY{n+nb}{str}\PY{p}{(}\PY{n}{i}\PY{p}{)}\PY{p}{)}
              \PY{n}{plt}\PY{o}{.}\PY{n}{show}\PY{p}{(}\PY{p}{)}
\end{Verbatim}

    \begin{center}
    \adjustimage{max size={0.9\linewidth}{0.9\paperheight}}{Assignment_8_files/Assignment_8_14_0.png}
    \end{center}
    { \hspace*{\fill} \\}
    
    \begin{center}
    \adjustimage{max size={0.9\linewidth}{0.9\paperheight}}{Assignment_8_files/Assignment_8_14_1.png}
    \end{center}
    { \hspace*{\fill} \\}
    
    \begin{center}
    \adjustimage{max size={0.9\linewidth}{0.9\paperheight}}{Assignment_8_files/Assignment_8_14_2.png}
    \end{center}
    { \hspace*{\fill} \\}
    
    \begin{center}
    \adjustimage{max size={0.9\linewidth}{0.9\paperheight}}{Assignment_8_files/Assignment_8_14_3.png}
    \end{center}
    { \hspace*{\fill} \\}
    
    \begin{center}
    \adjustimage{max size={0.9\linewidth}{0.9\paperheight}}{Assignment_8_files/Assignment_8_14_4.png}
    \end{center}
    { \hspace*{\fill} \\}
    
    \hypertarget{question-3}{%
\subsection{Question 3}\label{question-3}}

We now solve for an active low pass filter (Rauch). This particular
active filter is known for being a single op-amp based high pass filter.

The MNA (Nodal) Equations may be recast as:

\[
\begin{gather}
\begin{bmatrix}
   s(C_1+C_2) + 1/R_1& 0& -sC_2& -1/R_1\\   
   0& G& 0 &-1\\
   -sC_2& 0& 1/R_3 + sC_2& 0&\\
   0& -G& G& 1&
   \end{bmatrix}
   *
   \begin{bmatrix} V_1 \\ V_p \\ V_m\\ V_o \end{bmatrix}
   =
   \begin{bmatrix} sC_1V_i\\ 0\\ 0\\ 0 \end{bmatrix}
\end{gather}
\]
\begin{Verbatim}[commandchars=\\\{\}]
{\color{incolor}In [{\color{incolor}224}]:} \PY{k}{def} \PY{n+nf}{highpass}\PY{p}{(}\PY{n}{R1}\PY{p}{,}\PY{n}{R2}\PY{p}{,}\PY{n}{C1}\PY{p}{,}\PY{n}{C2}\PY{p}{,}\PY{n}{G}\PY{p}{,}\PY{n}{Vi}\PY{p}{)}\PY{p}{:}
              \PY{n}{s}\PY{o}{=}\PY{n}{sympy}\PY{o}{.}\PY{n}{symbols}\PY{p}{(}\PY{l+s+s1}{\PYZsq{}}\PY{l+s+s1}{s}\PY{l+s+s1}{\PYZsq{}}\PY{p}{)}
              \PY{n}{A}\PY{o}{=}\PY{n}{sympy}\PY{o}{.}\PY{n}{Matrix}\PY{p}{(}\PY{p}{[}\PY{p}{[}\PY{n}{s}\PY{o}{*}\PY{p}{(}\PY{n}{C1}\PY{o}{+}\PY{n}{C2}\PY{p}{)}\PY{o}{+}\PY{l+m+mi}{1}\PY{o}{/}\PY{n}{R1}\PY{p}{,}\PY{l+m+mi}{0}\PY{p}{,}\PY{o}{\PYZhy{}}\PY{n}{s}\PY{o}{*}\PY{n}{C2}\PY{p}{,}\PY{o}{\PYZhy{}}\PY{l+m+mi}{1}\PY{o}{/}\PY{n}{R1}\PY{p}{]}\PY{p}{,}\PY{p}{[}\PY{l+m+mi}{0}\PY{p}{,}\PY{n}{G}\PY{p}{,}\PY{l+m+mi}{0}\PY{p}{,}\PY{o}{\PYZhy{}}\PY{l+m+mi}{1}\PY{p}{]}\PY{p}{,} \PYZbs{}
              \PY{p}{[}\PY{o}{\PYZhy{}}\PY{n}{s}\PY{o}{*}\PY{n}{C2}\PY{p}{,}\PY{l+m+mi}{0}\PY{p}{,}\PY{l+m+mi}{1}\PY{o}{/}\PY{n}{R2}\PY{o}{+}\PY{n}{s}\PY{o}{*}\PY{n}{C2}\PY{p}{,}\PY{l+m+mi}{0}\PY{p}{]}\PY{p}{,}\PY{p}{[}\PY{l+m+mi}{0}\PY{p}{,}\PY{l+m+mi}{0}\PY{p}{,}\PY{o}{\PYZhy{}}\PY{n}{G}\PY{p}{,}\PY{l+m+mi}{1}\PY{p}{]}\PY{p}{]}\PY{p}{)}
              \PY{n}{b}\PY{o}{=}\PY{n}{sympy}\PY{o}{.}\PY{n}{Matrix}\PY{p}{(}\PY{p}{[}\PY{n}{Vi}\PY{o}{*}\PY{n}{s}\PY{o}{*}\PY{n}{C1}\PY{p}{,}\PY{l+m+mi}{0}\PY{p}{,}\PY{l+m+mi}{0}\PY{p}{,}\PY{l+m+mi}{0}\PY{p}{]}\PY{p}{)}
              \PY{n}{V}\PY{o}{=}\PY{n}{A}\PY{o}{.}\PY{n}{inv}\PY{p}{(}\PY{p}{)}\PY{o}{*}\PY{n}{b}
              \PY{k}{return} \PY{p}{(}\PY{n}{A}\PY{p}{,}\PY{n}{b}\PY{p}{,}\PY{n}{V}\PY{p}{)}
\end{Verbatim}
\begin{Verbatim}[commandchars=\\\{\}]
{\color{incolor}In [{\color{incolor}225}]:} \PY{n}{A}\PY{p}{,}\PY{n}{b}\PY{p}{,}\PY{n}{V}\PY{o}{=}\PY{n}{highpass}\PY{p}{(}\PY{l+m+mi}{10000}\PY{p}{,}\PY{l+m+mi}{10000}\PY{p}{,}\PY{l+m+mf}{1e\PYZhy{}9}\PY{p}{,}\PY{l+m+mf}{1e\PYZhy{}9}\PY{p}{,}\PY{l+m+mf}{1.586}\PY{p}{,}\PY{l+m+mi}{1}\PY{p}{)}
          
          \PY{n+nb}{print} \PY{p}{(}\PY{l+s+s1}{\PYZsq{}}\PY{l+s+s1}{G=1000}\PY{l+s+s1}{\PYZsq{}}\PY{p}{)}
          \PY{n}{Vo}\PY{o}{=}\PY{n}{V}\PY{p}{[}\PY{l+m+mi}{3}\PY{p}{]}
          \PY{n+nb}{print}\PY{p}{(}\PY{l+s+s2}{\PYZdq{}}\PY{l+s+s2}{Vo (transfer function)}\PY{l+s+s2}{\PYZdq{}}\PY{p}{)}
          \PY{n}{display}\PY{p}{(}\PY{n}{Vo}\PY{p}{)}
          \PY{n}{Vo}\PY{o}{=}\PY{n}{sympy}\PY{o}{.}\PY{n}{simplify}\PY{p}{(}\PY{n}{Vo}\PY{p}{)}
          \PY{n+nb}{print}\PY{p}{(}\PY{l+s+s2}{\PYZdq{}}\PY{l+s+s2}{Vo (transfer function) Simplified}\PY{l+s+s2}{\PYZdq{}}\PY{p}{)}
          \PY{n}{display}\PY{p}{(}\PY{n}{Vo}\PY{p}{)}
          \PY{n}{w}\PY{o}{=}\PY{n}{np}\PY{o}{.}\PY{n}{logspace}\PY{p}{(}\PY{l+m+mi}{0}\PY{p}{,}\PY{l+m+mi}{8}\PY{p}{,}\PY{l+m+mi}{801}\PY{p}{)}
          \PY{n}{ss}\PY{o}{=}\PY{l+m+mi}{1}\PY{n}{j}\PY{o}{*}\PY{n}{w}
          \PY{n}{hf}\PY{o}{=}\PY{n}{sympy}\PY{o}{.}\PY{n}{lambdify}\PY{p}{(}\PY{n}{s}\PY{p}{,}\PY{n}{Vo}\PY{p}{,}\PY{l+s+s2}{\PYZdq{}}\PY{l+s+s2}{numpy}\PY{l+s+s2}{\PYZdq{}}\PY{p}{)}
          \PY{n}{v}\PY{o}{=}\PY{n}{hf}\PY{p}{(}\PY{n}{ss}\PY{p}{)}
          
          \PY{n}{plt}\PY{o}{.}\PY{n}{loglog}\PY{p}{(}\PY{n}{w}\PY{p}{,}\PY{n+nb}{abs}\PY{p}{(}\PY{n}{v}\PY{p}{)}\PY{p}{,}\PY{n}{lw}\PY{o}{=}\PY{l+m+mi}{2}\PY{p}{)}
          \PY{n}{plt}\PY{o}{.}\PY{n}{grid}\PY{p}{(}\PY{k+kc}{True}\PY{p}{)}
          \PY{n}{plt}\PY{o}{.}\PY{n}{show}\PY{p}{(}\PY{p}{)}
\end{Verbatim}

    \begin{Verbatim}[commandchars=\\\{\}]
G=1000
Vo (transfer function)

    \end{Verbatim}

    $$- \frac{1.586 \cdot 10^{-18} s^{2}}{1.0 \cdot 10^{-18} s^{2} + 1.586 \cdot 10^{-13} s - \left(1.0 \cdot 10^{-9} s + 0.0001\right) \left(2.0 \cdot 10^{-9} s + 0.0001\right)}$$

    
    \begin{Verbatim}[commandchars=\\\{\}]
Vo (transfer function) Simplified

    \end{Verbatim}

    $$\frac{1.586 \cdot 10^{-18} s^{2}}{1.0 \cdot 10^{-18} s^{2} + 1.41400000000001 \cdot 10^{-13} s + 1.0 \cdot 10^{-8}}$$

    
    \begin{center}
    \adjustimage{max size={0.9\linewidth}{0.9\paperheight}}{Assignment_8_files/Assignment_8_17_4.png}
    \end{center}
    { \hspace*{\fill} \\}
    
    Clearly, the Rauch filter acts as a second order high pass filter.

\hypertarget{question-4}{%
\subsection{Question 4}\label{question-4}}

We now obtain the response of the circuit to a damped sinusoid.

Let us start with a damping coefficient of \(0.01 s^{-1}\), and a
frequency of \(10^{3} Hertz\), and sweep the frequency.
\begin{Verbatim}[commandchars=\\\{\}]
{\color{incolor}In [{\color{incolor}226}]:} \PY{n}{n}\PY{p}{,}\PY{n}{d} \PY{o}{=} \PY{n}{sympy}\PY{o}{.}\PY{n}{fraction}\PY{p}{(}\PY{n}{Vo}\PY{p}{)}
          \PY{n}{n\PYZus{}sp}\PY{p}{,}\PY{n}{d\PYZus{}sp}\PY{o}{=}\PY{p}{(}\PY{n}{np}\PY{o}{.}\PY{n}{array}\PY{p}{(}\PY{n}{sympy}\PY{o}{.}\PY{n}{Poly}\PY{p}{(}\PY{n}{j}\PY{p}{,}\PY{n}{s}\PY{p}{)}\PY{o}{.}\PY{n}{all\PYZus{}coeffs}\PY{p}{(}\PY{p}{)}\PY{p}{,}\PY{n}{dtype}\PY{o}{=}\PY{n+nb}{float}\PY{p}{)} \PY{k}{for} \PY{n}{j} \PY{o+ow}{in} \PY{p}{(}\PY{n}{n}\PY{p}{,}\PY{n}{d}\PY{p}{)}\PY{p}{)}
          
          \PY{n}{n}\PY{o}{=}\PY{p}{[}\PY{l+m+mi}{3}\PY{p}{,}\PY{l+m+mi}{4}\PY{p}{,}\PY{l+m+mi}{5}\PY{p}{,}\PY{l+m+mi}{6}\PY{p}{,}\PY{l+m+mi}{7}\PY{p}{]}
          \PY{k}{for} \PY{n}{i} \PY{o+ow}{in} \PY{n}{n}\PY{p}{:}
              \PY{c+c1}{\PYZsh{} Function to simulate}
              \PY{n}{ts}\PY{o}{=}\PY{n}{np}\PY{o}{.}\PY{n}{linspace}\PY{p}{(}\PY{l+m+mi}{0}\PY{p}{,}\PY{l+m+mf}{0.15}\PY{o}{*}\PY{o}{*}\PY{n}{i}\PY{p}{,}\PY{l+m+mi}{8001}\PY{p}{)}
              \PY{n}{vi}\PY{o}{=} \PY{n}{np}\PY{o}{.}\PY{n}{exp}\PY{p}{(}\PY{o}{\PYZhy{}}\PY{l+m+mf}{0.05}\PY{o}{*}\PY{n}{ts}\PY{p}{)}\PY{o}{*}\PY{n}{np}\PY{o}{.}\PY{n}{sin}\PY{p}{(}\PY{l+m+mi}{2}\PY{o}{*}\PY{l+m+mi}{10}\PY{o}{*}\PY{o}{*}\PY{n}{i}\PY{o}{*}\PY{n}{np}\PY{o}{.}\PY{n}{pi}\PY{o}{*}\PY{n}{ts}\PY{p}{)}
          
              \PY{n}{t}\PY{p}{,}\PY{n}{x}\PY{p}{,}\PY{n}{svec}\PY{o}{=}\PY{n}{sp}\PY{o}{.}\PY{n}{lsim}\PY{p}{(}\PY{n}{sp}\PY{o}{.}\PY{n}{lti}\PY{p}{(}\PY{n}{n\PYZus{}sp}\PY{p}{,}\PY{n}{d\PYZus{}sp}\PY{p}{)}\PY{p}{,}\PY{n}{vi}\PY{p}{,}\PY{n}{ts}\PY{p}{)}
              \PY{n}{plt}\PY{o}{.}\PY{n}{title}\PY{p}{(}\PY{l+s+s2}{\PYZdq{}}\PY{l+s+s2}{At Frequency of Sine Component = 10\PYZca{}}\PY{l+s+s2}{\PYZdq{}}\PY{o}{+}\PY{n+nb}{str}\PY{p}{(}\PY{n}{i}\PY{p}{)}\PY{p}{)}
              \PY{c+c1}{\PYZsh{} Plot the lamdified values}
              \PY{n}{plt}\PY{o}{.}\PY{n}{plot}\PY{p}{(}\PY{n}{t}\PY{p}{,}\PY{n}{x}\PY{p}{,}\PY{n}{lw}\PY{o}{=}\PY{l+m+mi}{2}\PY{p}{)}
              \PY{n}{plt}\PY{o}{.}\PY{n}{grid}\PY{p}{(}\PY{k+kc}{True}\PY{p}{)}
              \PY{n}{plt}\PY{o}{.}\PY{n}{show}\PY{p}{(}\PY{p}{)}
\end{Verbatim}

    \begin{center}
    \adjustimage{max size={0.9\linewidth}{0.9\paperheight}}{Assignment_8_files/Assignment_8_19_0.png}
    \end{center}
    { \hspace*{\fill} \\}
    
    \begin{center}
    \adjustimage{max size={0.9\linewidth}{0.9\paperheight}}{Assignment_8_files/Assignment_8_19_1.png}
    \end{center}
    { \hspace*{\fill} \\}
    
    \begin{center}
    \adjustimage{max size={0.9\linewidth}{0.9\paperheight}}{Assignment_8_files/Assignment_8_19_2.png}
    \end{center}
    { \hspace*{\fill} \\}
    
    \begin{center}
    \adjustimage{max size={0.9\linewidth}{0.9\paperheight}}{Assignment_8_files/Assignment_8_19_3.png}
    \end{center}
    { \hspace*{\fill} \\}
    
    \begin{center}
    \adjustimage{max size={0.9\linewidth}{0.9\paperheight}}{Assignment_8_files/Assignment_8_19_4.png}
    \end{center}
    { \hspace*{\fill} \\}
    
    Note that for brevity, we have changed the time-axis as the frequency
keeps increasing. More importantly, notice that the gain of the circuit
increases with frequency, a characteristic of high pass filters.

\hypertarget{question-5}{%
\subsection{Question 5}\label{question-5}}

We now obtain the response of the circuit to a unit step function.
\begin{Verbatim}[commandchars=\\\{\}]
{\color{incolor}In [{\color{incolor}227}]:} \PY{n}{s}\PY{p}{,}\PY{n}{t}\PY{o}{=}\PY{n}{sympy}\PY{o}{.}\PY{n}{symbols}\PY{p}{(}\PY{l+s+s2}{\PYZdq{}}\PY{l+s+s2}{s t}\PY{l+s+s2}{\PYZdq{}}\PY{p}{)}
          \PY{n}{t}\PY{o}{=}\PY{n}{sympy}\PY{o}{.}\PY{n}{Symbol}\PY{p}{(}\PY{l+s+s2}{\PYZdq{}}\PY{l+s+s2}{t}\PY{l+s+s2}{\PYZdq{}}\PY{p}{,}\PY{n}{positive}\PY{o}{=}\PY{k+kc}{True}\PY{p}{)}
          \PY{n}{n}\PY{p}{,}\PY{n}{d} \PY{o}{=} \PY{n}{sympy}\PY{o}{.}\PY{n}{fraction}\PY{p}{(}\PY{n}{Vo}\PY{p}{)}
          \PY{n}{n\PYZus{}sp}\PY{p}{,}\PY{n}{d\PYZus{}sp}\PY{o}{=}\PY{p}{(}\PY{n}{np}\PY{o}{.}\PY{n}{array}\PY{p}{(}\PY{n}{sympy}\PY{o}{.}\PY{n}{Poly}\PY{p}{(}\PY{n}{j}\PY{p}{,}\PY{n}{s}\PY{p}{)}\PY{o}{.}\PY{n}{all\PYZus{}coeffs}\PY{p}{(}\PY{p}{)}\PY{p}{,}\PY{n}{dtype}\PY{o}{=}\PY{n+nb}{float}\PY{p}{)} \PY{k}{for} \PY{n}{j} \PY{o+ow}{in} \PY{p}{(}\PY{n}{n}\PY{p}{,}\PY{n}{d}\PY{p}{)}\PY{p}{)}
          
          \PY{n+nb}{print}\PY{p}{(}\PY{n}{n\PYZus{}sp}\PY{p}{,}\PY{n}{d\PYZus{}sp}\PY{p}{)}
          \PY{n}{ts}\PY{o}{=}\PY{n}{np}\PY{o}{.}\PY{n}{linspace}\PY{p}{(}\PY{l+m+mi}{0}\PY{p}{,}\PY{l+m+mf}{0.001}\PY{p}{,}\PY{l+m+mi}{8001}\PY{p}{)}
          \PY{n}{t}\PY{p}{,}\PY{n}{x}\PY{p}{,}\PY{n}{svec}\PY{o}{=}\PY{n}{sp}\PY{o}{.}\PY{n}{lsim}\PY{p}{(}\PY{n}{sp}\PY{o}{.}\PY{n}{lti}\PY{p}{(}\PY{n}{n\PYZus{}sp}\PY{p}{,}\PY{n}{d\PYZus{}sp}\PY{p}{)}\PY{p}{,}\PY{n}{np}\PY{o}{.}\PY{n}{ones}\PY{p}{(}\PY{n+nb}{len}\PY{p}{(}\PY{n}{ts}\PY{p}{)}\PY{p}{)}\PY{p}{,}\PY{n}{ts}\PY{p}{)}
          \PY{c+c1}{\PYZsh{} Plot the lamdified values}
          \PY{n}{plt}\PY{o}{.}\PY{n}{plot}\PY{p}{(}\PY{n}{t}\PY{p}{,}\PY{n}{x}\PY{p}{,}\PY{n}{lw}\PY{o}{=}\PY{l+m+mi}{2}\PY{p}{)}
          \PY{n}{plt}\PY{o}{.}\PY{n}{grid}\PY{p}{(}\PY{k+kc}{True}\PY{p}{)}
          \PY{n}{plt}\PY{o}{.}\PY{n}{show}\PY{p}{(}\PY{p}{)}
\end{Verbatim}

    \begin{Verbatim}[commandchars=\\\{\}]
[  1.58600000e-18   0.00000000e+00   0.00000000e+00] [  1.00000000e-18
1.41400000e-13   1.00000000e-08]

    \end{Verbatim}

    \begin{center}
    \adjustimage{max size={0.9\linewidth}{0.9\paperheight}}{Assignment_8_files/Assignment_8_21_1.png}
    \end{center}
    { \hspace*{\fill} \\}
    
    \hypertarget{results-and-discussion}{%
\section{Results and Discussion}\label{results-and-discussion}}

We have analysed the working of both Sallen-Key (low pass) and Rauch
(high pass) filters. This has been done by studying the step response as
well as reponse to sinusoids.

In the former case, we noted the following:

\begin{enumerate}
\def\labelenumi{\arabic{enumi}.}
\item
  The low pass filter reponse has an initial value greater than the
  steady state response, which indicates a quality factor \((Q>1)\).
\item
  Similarly, the high pass filter reponse has an initial value lesser
  than the steady state response, which indicates a quality factor
  \((Q>1)\).
\item
  In the case of a low pass filter, we noticed that higher harmonics get
  attenuated as compared to lower harmonics. The converse happens in
  case of a high pass filter.
\end{enumerate}

On the programming paradigm perspective, we observe that while scipy
provides a rather good amount of dexterity with symbolic expressions,
the lack of compatibility with scipy poses a major rethink in problem
solving approaches. There is certain amount of cross usage aided by
lambda objects, however, these are costly solutions.

In our case, we use to strength the fact that the transfer functions are
rational and second order. This helps us extract coefficients and
pipeline efficiently into the signal processing library of scipy.


    % Add a bibliography block to the postdoc
    
    
\bibliography{references}

    
    \end{document}
